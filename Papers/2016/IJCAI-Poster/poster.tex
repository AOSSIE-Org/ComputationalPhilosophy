\documentclass[final,hyperref={pdfpagelabels=false},xcolor=dvipsnames]{beamer}
\mode<presentation>
  {
  %  \usetheme{Berlin}
  \usetheme{FUBerlin}
  \usecolortheme{orchid}
%\usetheme{TUGraz}
  }
  \usepackage{times}
  \usepackage{amsmath,amsthm, amssymb, latexsym, fancyvrb}
  \boldmath
  \usepackage[english]{babel}
  \usepackage[latin1]{inputenc}
  \usepackage[orientation=portrait,size=a0,scale=1.4,debug]{beamerposter}


  \usepackage{stmaryrd}
  \newcommand{\typearrow}{\shortrightarrow}
  \newcommand{\itype}{\iota}
  \newcommand{\utype}{u}
  \newcommand{\proptype}{\tau}
  \newcommand{\exInW}{eiw}
  \newcommand{\chriscite}[1]{\textcolor{darkgray}{\textit{\scriptsize Reading:
        #1}}}
  \newcommand{\qcl}{QCL}

  \newcommand{\holpp}{\textcolor{brown}{H}}
  \newcommand{\holp}{\textcolor{brown}{$\holpp$}}
  \newcommand{\holpLeo}{\textcolor{brown}{$\holpp_{L}$}}
  \newcommand{\holpSatallax}{\textcolor{brown}{$\holpp_{S}$}}
  \newcommand{\holpIsabelle}{\textcolor{brown}{$\holpp_{I}$}}
  \newcommand{\holpNitpick}{\textcolor{brown}{$\holpp_{N}$}}
  \newcommand{\holpAgsyhol}{\textcolor{brown}{$\holpp_{A}$}}
  \newcommand{\holpLSA}{\textcolor{brown}{$\holpp_{A,L,S}$}}
  \newcommand{\holpLS}{\textcolor{brown}{$\holpp_{L,S}$}}
  \newcommand{\holpLA}{\textcolor{brown}{$\holpp_{A,L}$}}
  \newcommand{\holpSA}{\textcolor{brown}{$\holpp_{A,S}$}}

  %%%%%%%%%%%%%%%%%%%%%%%%%%%%%%%%%%%%%%%%%%%%%%%%%%%%%%%%%%%%%%%%%%%%%%%%%%%%%%%%%5
  \graphicspath{{figures/}}
  \title[Automating Quantified Conditional Logics in HOL]{Automating Quantified Conditional Logics in HOL}
  \author[Benzm�ller]{Christoph Benzm�ller}
%  \institute[Freie Universit�t Berlin]{Freie Universit�t Berlin,   Germany}
  %\date{Jul. 31th, 2007}


  %%%%%%%%%%%%%%%%%%%%%%%%%%%%%%%%%%%%%%%%%%%%%%%%%%%%%%%%%%%%%%%%%%%%%%%%%%%%%%%%%5
  \begin{document}
  \begin{frame}{} 
  \vskip-3ex
    \begin{columns}[t]
      \begin{column}{.32\linewidth}
        \begin{block}{\large Abstract} \footnotesize
          A notion of \textbf{quantified conditional logics (QCLs)} is
          provided that includes quantification over
          \textbf{individual and propositional variables}. The former
          is supported with respect to \textbf{constant and variable
            domain semantics}. In addition, a \textbf{sound and
            complete embedding of this framework in classical
            higher-order logic (HOL)} is presented. Using prominent examples
          from the literature it is demonstrated how this embedding
          enables \textbf{effective automation of reasoning} within
          (object-level) and about (meta-level) quantified conditional
          logics with \textbf{off-the-shelf higher-order theorem
            provers and model finders}.
        \end{block}
        \begin{block}{\large Overall Motivation and Contribution}
          \footnotesize QCLs are very expressive non-classical logics;
          they have many applications; no provers have been
          available so far.  However,
          \begin{itemize}
           \item \textbf{\textcolor{blue}{QCLs are fragments of HOL (with Henkin semantics)}}
           \item \textbf{\textcolor{blue}{and they can easily be automated as such,}}
           \item \textbf{\textcolor{blue}{they inherit important meta- resp. proof-theoretical
             properties (cut-elimination, compactness, etc.), and}}
           \item \textbf{\textcolor{blue}{they can easily be combined with other logics in 
             HOL.}}
           \end{itemize}
           {This reserach is part of a larger project which
             takes HOL as starting point for studying classical and
             non-classical logics and their combinations.} \\[-1.5ex] \ \hfill
 \chriscite{[Benzm\"uller'13]}
        \vskip.1ex
        \end{block}
      \end{column}
      \hfill
      \begin{column}{.32\linewidth}
        \begin{block}{\large Quantified Conditional Logics (QCLs)}
          \footnotesize  
          \colorbox{GreenYellow}{
            \begin{minipage}{.93\textwidth} 
              \begin{align*}
                \textcolor{red}{\varphi},\textcolor{red}{\psi} ::= \quad & \textcolor{red}{P} \mid \textcolor{red}{k(X^1,\ldots,X^n)} \mid \textcolor{red}{\neg \varphi} \mid \textcolor{red}{\varphi \vee \psi} \mid \textcolor{red}{\varphi \Rightarrow \psi} \mid \\
                & \textcolor{red}{\forall^{co} X \varphi} \mid \textcolor{red}{\forall^{va} X \varphi} \mid  \textcolor{red}{\forall P \varphi}  \\[-1em]
              \end{align*}
            \end{minipage}
          }
          \vskip1.5ex
          \textbf{Interpretation}: $M = \langle S,f,D,D',Q,I
          \rangle$ where $S$ is a set of 'worlds',
          $f: S \times 2^{S} \mapsto 2^{S}$ is the selection function,
          $D\not=\emptyset$ is a set of \emph{individuals} (constant
          domain), $D'$ is a function that assigns a
          subset $D'(w)\not=\emptyset$ of $D$ to each world $w$
          (varying domains), $Q\not=\emptyset$ is a collection of subsets
          of $W$ (prop.~domain), and $I$ is an interpretation function
          s.t. for each predicate symbol $k$, $I(k,w) \subseteq
          D^n$. \\
          \vskip1ex
          \textbf{Satisfiability} of $\textcolor{red}{\varphi}$ (denoted as $M,g,s
          \models \textcolor{red}{\varphi}$) for an interpretation $M$, a world $s \in
          S$, and a variable assignment $g = (g^{i},g^{p})$:
          \begin{align*}
            M,g,s \models\;& \textcolor{red}{k(X^1,...\,,X^n)} \text{\;iff\;} \langle g^{i}(\textcolor{red}{X^1}),...\,, g^{i}(\textcolor{red}{X^n}) \rangle \in I(k,w)\\
            M,g,s \models\;& \textcolor{red}{P}  \text{\;iff\;} s \in g^{p}(\textcolor{red}{P})
            \\
            M,g,s \models\;& \textcolor{red}{\neg \varphi} \text{\;iff\;} M,g,s \not\models \textcolor{red}{\varphi} \, (\text{that is, not } M,g,s \models \textcolor{red}{\varphi})
            \\
            M,g,s \models\;& \textcolor{red}{\varphi \vee \psi} \text{\;iff\;} M,g,s \models \textcolor{red}{\varphi} \text{ or } M,g,s \models \textcolor{red}{\psi}
            \\
            M,g,s \models\;& \textcolor{red}{\forall^{co} X \varphi} \text{\;iff\;} M,([d/X]g^{i},g^{p}), s \models \textcolor{red}{\varphi} \text{ for all } d \in D
            \\
            M,g,s \models\;& \textcolor{red}{\forall^{va} X \varphi} \text{\;iff\;} M,([d/X]g^{i},g^{p}), s \models \textcolor{red}{\varphi} \text{ for all } d \in D'(s)
            \\
            M,g,s \models\;& \textcolor{red}{\forall P \varphi} \text{\;iff\;} M,(g^{i},[p/P]g^{p}),s \models \textcolor{red}{\varphi} \text{ for all } p \in Q
            \\
            M,g,s \models\;& \textcolor{red}{\varphi \Rightarrow \psi} \text{\;iff\;}
            M,g,t \models \textcolor{red}{\psi} \text{ for all } t \in S \text{ s.t.}  t \in f(s,[\textcolor{red}{\varphi}]) \\ & \text{ where } [\textcolor{red}{\varphi}] = \{u \mid M,g,u \models \textcolor{red}{\varphi}\}
          \end{align*}
          $M \models^{\qcl} \textcolor{red}{\varphi}$ iff $M,g,s \models \textcolor{red}{\varphi}$ for all $s$, $g$. $\models \textcolor{red}{\varphi}$
          iff $M \models^{\qcl} \textcolor{red}{\varphi}$ for all
          $M$.\\
          \vskip1ex
          \chriscite{[Stalnaker'68],[Delgrande'98]}
        \end{block}
      \end{column}
      \hfill
      \begin{column}{.32\linewidth}
        \begin{block}{\large Classical Higher-order Logic (HOL)}
          \footnotesize

          Types \hfill ${\alpha,\beta} ::= {\iota}
          \textcolor{gray}{\,(worlds)}\mid
          {\mu} \textcolor{gray}{\,(indiv.)} \mid {o}
          \textcolor{gray}{\,(Booleans)}\mid {\alpha \rightarrow
            \beta}$ \\[1em]
          \colorbox{GreenYellow}{
          \begin{minipage}{.93\textwidth} 
          \begin{align*}
            \textcolor{blue}{s},\textcolor{blue}{t} ::=  & \textcolor{blue}{c_{\alpha}} \mid \textcolor{blue}{X_{\alpha}} \mid \textcolor{blue}{(\lambda X_{\alpha}s_{\beta})_{\alpha \typearrow \beta}} \mid 
            \textcolor{blue}{(s_{\alpha \typearrow \beta}\, t_\alpha)_{\beta}} \mid \\
            & \textcolor{blue}{(\neg_{o \typearrow o}\;s_{o})_o} \mid \textcolor{blue}{(s_o \vee_{o\typearrow o \typearrow o} t_o)_o}
            \mid \textcolor{blue}{(\Pi_{(\alpha \typearrow o)\typearrow o}\; s_{\alpha
              \typearrow o})_o}  \\[-1em]
          \end{align*}
          Note:~Binder~notation~\textcolor{blue}{$\forall X_\alpha
            t_o$}~as~syntactic~sugar~for~\textcolor{blue}{$\Pi_{(\alpha
              \typearrow o)\typearrow o} \lambda X_\alpha t_o$}\, 
        \end{minipage}} \vskip2.5ex \textbf{Frame}: collection
      $\{D_{\alpha}\}_{\alpha \in {T}}$ s.t. $D_{o} = \{T,F\}$,
      $D_\itype \not= \emptyset$ and $D_\utype \not= \emptyset$
      arbitrary, and $D_{\alpha \typearrow \beta}$ are collections of
      total functions from
      $D_{\alpha}$ to $D_{\beta}$.\\
      \vskip1.5ex \textbf{Interpretation}: Tuple $\langle
      \{D_{\alpha}\}_{\alpha \in {T}}, I \rangle$ where
      $\{D_{\alpha}\}_{\alpha \in {T}}$ is a frame and where function
      $I$ maps each typed constant symbol $\textcolor{blue}{c_{\alpha}}$ to an
      appropriate element of $D_{\alpha}$, called the
      \emph{denotation} of $c_{\alpha}$. The denotations of
      $\textcolor{blue}{\neg},\textcolor{blue}{\vee}$ and $\textcolor{blue}{\Pi_{(\alpha \typearrow o)\typearrow o}}$ are
      always chosen as usual.\\
      \vskip1.5ex 
      An interpretation is a \textbf{Henkin model} iff there
      is a valuation function ${V}$ s.t.  ${V}(\phi,\textcolor{blue}{s_{\alpha}}) \in
      D_{\alpha}$ for each variable assignment $\phi$ and term
      $\textcolor{blue}{s_{\alpha}}$, and the following conditions are satisfied:
      ${V}(\phi,\textcolor{blue}{X_{\alpha}}) = \phi(\textcolor{blue}{X_\alpha})$, ${V}(\phi,\textcolor{blue}{c_{\alpha}}) =
      I(\textcolor{blue}{c_{\alpha}})$, ${V}(\phi,\textcolor{blue}{l_{\alpha \typearrow \beta}\,
      r_{\alpha}}) = ({V}(\phi,\textcolor{blue}{l_{\alpha \typearrow \beta}})
      {V}(\phi,\textcolor{blue}{r_{\alpha}}))$, and ${V}(\phi,\textcolor{blue}{\lambda
      X_{\alpha}s_{\beta}})$ represents the function from $D_{\alpha}$
      into $D_{\beta}$ whose value for each argument $z \in
      D_{\alpha}$ is ${V}(\phi[z/\textcolor{blue}{X_{\alpha}}], \textcolor{blue}{s_\beta})$. % , where
%       $\phi[z/X_\alpha]$ is that assignment
%       s.t. $\phi[z/X_\alpha](X_\alpha) = z$ and
%       $\phi[z/X_\alpha]Y_\beta = \phi Y_\beta$ when $Y_\beta
%       \not=X_\alpha$.
      If an interpretation is an {Henkin model} the function ${V}$ is
      uniquely determined. \\
      \vskip1ex
      ${H}\models^{HOL} \textcolor{blue}{s}$ iff ${V}(\phi,\textcolor{blue}{s}) = T$ for all $\phi$.
      $\models \textcolor{blue}{s}$ iff ${H}\models^{HOL} \textcolor{blue}{s}$ for all
      ${H}$. \\
      \vskip2.7ex
        \chriscite{[Church'40],[Andrews'72a/b],[Benzm\"ullerEtAl'04]
          \vskip.5ex
        }
        \end{block}
      \end{column}
    \end{columns}
    \vskip2ex
    \begin{block}{\large Embedding QCLs in HOL --- In other words: QCLs are simple Fragments of HOL!} \footnotesize
      \begin{columns}[t]
        \begin{column}{.31\textwidth}
          The mapping $\textcolor{blue}{\lfloor \textcolor{red}{\cdot} \rfloor}$ identifies
          \qcl\ formulas $\textcolor{red}{\varphi}$ with HOL terms $\textcolor{blue}{\lfloor
          \textcolor{red}{\varphi} \rfloor}$ of type $\proptype:=\itype \typearrow o$. The mapping is recursively defined:
          \[
          \begin{array}{lcl}
            \textcolor{blue}{\lfloor \textcolor{red}{P} \rfloor} &=& \textcolor{blue}{P_\proptype} \\
            \textcolor{blue}{\lfloor \textcolor{red}{k(X^1,\ldots,X^n)} \rfloor} % &=& \lfloor k \rfloor \lfloor X^1 \rfloor \ldots \lfloor X^n \rfloor \\
            &=& \textcolor{blue}{k_{u^n \typearrow \proptype}\,X^1_\utype \ldots X^n_\utype}\\
            \textcolor{blue}{\lfloor \textcolor{red}{\neg \varphi} \rfloor} &=&  \textcolor{blue}{\neg_{\proptype\typearrow \proptype}}\,\lfloor \textcolor{red}{\varphi} \rfloor \\
            \textcolor{blue}{\lfloor \textcolor{red}{\varphi\vee \psi} \rfloor} &=&  \textcolor{blue}{\vee_{\proptype \typearrow \proptype \typearrow \proptype}}\,\lfloor\textcolor{red}{\varphi} \rfloor \lfloor \textcolor{red}{\psi} \rfloor\\
            \textcolor{blue}{\lfloor \textcolor{red}{\varphi\Rightarrow \psi} \rfloor} &=&  \textcolor{blue}{\Rightarrow_{\proptype \typearrow \proptype \typearrow \proptype}}\,\lfloor\textcolor{red}{\varphi} \rfloor \lfloor \textcolor{red}{\psi} \rfloor \\
            \textcolor{blue}{\lfloor \textcolor{red}{\forall^{co} X \varphi} \rfloor} &=&   \textcolor{blue}{\Pi^{co}_{(\utype\typearrow\proptype)\typearrow\proptype}\,\lambda X_\utype} \lfloor\textcolor{red}{\varphi} \rfloor \\
            \textcolor{blue}{\lfloor \textcolor{red}{\forall^{va} X \varphi} \rfloor} &=&  \textcolor{blue}{\Pi^{va}_{(\utype\typearrow\proptype)\typearrow\proptype}\,\lambda X_\utype} \lfloor\textcolor{red}{\varphi} \rfloor \\
            % \end{array}\]
            % \[
            % \begin{array}{lcl}
            \textcolor{blue}{\lfloor \textcolor{red}{\forall P \varphi} \rfloor} &=&  \textcolor{blue}{\Pi_{(\proptype\typearrow\proptype)\typearrow\proptype}\,\lambda P_\proptype} \lfloor\textcolor{red}{\varphi} \rfloor
          \end{array}
          \]
          $\textcolor{blue}{P_\proptype}$ and $\textcolor{blue}{X^1_\utype},\ldots,\textcolor{blue}{X^n_\utype}$ are variables and
          $\textcolor{blue}{k_{u^n \typearrow \proptype}}$ is a 
          constant symbol.  
        \end{column}
        \hfill
        \begin{column}{.31\textwidth}
          $\textcolor{blue}{\neg_{\proptype\typearrow \proptype}}$, $\textcolor{blue}{\vee_{\proptype \typearrow
            \proptype \typearrow \proptype}}$, $\textcolor{blue}{\Rightarrow_{\proptype \typearrow \proptype \typearrow \proptype}}$, $\textcolor{blue}{\Pi^{co,va}_{(\utype\typearrow\proptype)\typearrow\proptype}}$
          and $\textcolor{blue}{\Pi_{(\proptype\typearrow\proptype)\typearrow\proptype}}$ realize the \qcl\ connectives in
          HOL. They abbreviate the following HOL terms:% \footnote{Note the predicate argument $A$ of $f$ in the term for $\Rightarrow_{\proptype \typearrow \proptype \typearrow \proptype}$ and the second-order quantifier $\forall P_{\proptype}$ in the term for $\Pi_{(\proptype\typearrow\proptype)\typearrow\proptype}$. FOL encodings of both constructs, if feasible, will be less natural.}
          \[
          \begin{array}{ll}
            \textcolor{blue}{\neg_{\proptype \typearrow \proptype}} &=  \textcolor{blue}{\lambda A_\proptype \lambda X_\itype \neg(A\,X)}\\
             \textcolor{blue}{\vee_{\proptype \typearrow \proptype \typearrow \proptype}} &=  \textcolor{blue}{\lambda A_\proptype \lambda B_\proptype \lambda X_\itype (A\,X \vee B\,X)}\\
             \textcolor{blue}{\Rightarrow_{\proptype \typearrow \proptype \typearrow \proptype}} &=  \textcolor{blue}{\lambda A_\proptype \lambda B_\proptype \lambda X_\itype \forall V_\itype (f\, X\, A\, V \rightarrow B\, V)}\\
             \textcolor{blue}{\Pi^{co}_{(\utype\typearrow\proptype)\typearrow\proptype}} & =  \textcolor{blue}{\lambda {Q_{\utype\typearrow\proptype}}  \lambda {V_\itype}  \forall{X_\utype}  (Q\,X\,V)} \\
             \textcolor{blue}{\Pi^{va}_{(\utype\typearrow\proptype)\typearrow\proptype}} & =  \textcolor{blue}{\lambda {Q_{\utype\typearrow\proptype}}  \lambda {V_\itype}  \forall{X_\utype}  (\exInW\,V\,X \rightarrow Q\,X\,V)} \\
             \textcolor{blue}{\Pi_{(\proptype\typearrow\proptype)\typearrow\proptype}} & =  \textcolor{blue}{\lambda {R_{\proptype\typearrow\proptype}}  \lambda {V_\itype}  \forall {P_\proptype}  (R\,P\,V)}
          \end{array}
          \]
          % Constant symbol $f$ in the mapping of $\Rightarrow$ is of type ${i
%             \typearrow \proptype \typearrow \proptype}$. It realizes the
%           selection function. Constant symbol $\exInW$ (for 'exists in world'),
%           which is of type ${\itype\typearrow\utype\typearrow o}$,
%           is associated with the varying domains. 
          The interpretations of $\textcolor{blue}{f}$ and $\textcolor{blue}{\exInW}$ are chosen
          appropriately. %  below,
%           cf. Def.~\ref{def:hm}. Moreover, f
          For the varying domains
          non-emptiness is postulated:
          % \begin{align*}
             $\textcolor{blue}{\forall W_\itype \exists X_\utype (\exInW\,W\,X)}$ \\[1em]
          % \end{align*}
          Meta-level notion of validity defined as $
          \textcolor{blue}{\text{vld}_{\proptype\typearrow o}} = \textcolor{blue}{\lambda A_\proptype
          \forall S_\itype (A\, S)}$. \\
        \end{column}
        \hfill
        \begin{column}{.31\textwidth}
          \vskip-2ex
          \colorbox{GreenYellow}{
            \begin{minipage}{.96\textwidth}
              \vskip1ex
              \textbf{Theorem: Soundness and Completeness}
                $$\models^{\qcl}
                \textcolor{red}{\varphi} \,\, \text{ iff } \,\, \{\text{NE}\}\models^{HOL} \textcolor{blue}{\text{vld}_{\proptype\typearrow o}}\, \textcolor{blue}{\lfloor \textcolor{red}{\varphi}
                \rfloor} \text{\,\,(wrt Henkin semantics)}$$
                (Proof: By relating Kripke structures to Henkin models.)
                \vskip1ex
            \end{minipage}
          } 
          \vskip2ex
          \colorbox{GreenYellow}{
            \begin{minipage}{.96\textwidth}
              \vskip1ex
              \textbf{Corollary: Cut-elimination for QCL}
              $$\text{There are cut-free calculi for QCL.}$$ 
              (Proof: Take any cut-free calculus for HOL, e.g. the cut-free sequent calculus from [Benzm\"ullerEtAl'09]. Note, however, the potential impact of cut-simulation.) 
                \vskip2ex
            \end{minipage}        
          } 
          \vskip2ex
          \chriscite{Earlier work is reported in [Benzm.Genovese'11]}   
        \end{column}
      \end{columns}
      % \vskip2ex      \chriscite{[Benzm.Genovese'11]}
    \end{block}
    \vskip0ex
    \begin{columns}[t]
      \begin{column}{.27\textwidth}
        \begin{block}{\large The Encoding in THF0-Syntax} 
          \vskip1.2ex 
          \begin{minipage}{.92\textwidth}
               \VerbatimInput[frame=single,%
               commandchars=\\\{\},%
               fontfamily=courier,fontseries=b,%
               fontsize=\tiny,%
               rulecolor=\color{green},%
               fillcolor=\color{YellowGreen},%
               formatcom=\color{blue},%
               framerule=0pt,%
               framesep=0pt,numbers=right]%
               {Axioms_annotated.ax}
%            \textcolor{blue}{\bf\VerbatimInput[fontsize=\tiny]{Axioms.ax}}
          \end{minipage}
          \vskip2ex
          \chriscite{Introduction to THF0-Syntax [SutcliffeBenzm.'10]}
        \end{block}
      \end{column}
      \hfill
      \begin{column}{.70\textwidth}
        \begin{block}{\large Automating Prominent Examples from the Literature (in \qcl+ID+MP)}
          \begin{columns}[t]
            \begin{column}{.30\textwidth} \footnotesize
              \textbf{Example: Pegasus, the winged horse} \\[1ex]
              
              It can be consistently stated (in \qcl+ID+MP) that: \textcolor{green}{\textit{``Horses (h) contingently do not
              have wings (w) but Pegasus (p) is a winged horse.''}}
              \textcolor{red}{\begin{align*}
                & \textcolor{red}{\forall^{va} X (h(X) \rightarrow \neg w(X)),\quad h(p),\quad w(p)}
              \end{align*}}
              THF0 encoding of this example:
              \vskip1ex
              \begin{minipage}{.92\textwidth}
                \VerbatimInput[frame=single,%
                commandchars=\\\{\},%
                fontfamily=courier,fontseries=b,%
                fontsize=\tiny,%
                rulecolor=\color{green},%
                fillcolor=\color{YellowGreen},%
                formatcom=\color{blue},%
                framerule=0pt,%
                framesep=0pt,numbers=right]%
                {E1_annotated.thf}
                \end{minipage}
              %\textcolor{blue}{\bf \VerbatimInput[fontsize=\tiny]{E1.thf}}
              \vskip1ex
              \holp\ confirms the satisfiability of these formulas (with
              \holpNitpick=7.7). The finite model generated by Nitpick tells us that
              Pegasus is not 'actual', i.e., does not exist (cf. $\textcolor{blue}{eiw}$) in any world. As expected, when the example problem is formulated
              with $\textcolor{red}{\forall^{co}}$ instead of\, $\textcolor{red}{\forall^{va}}$ then \holp\ reports
              unsatisfiability (\holpLS=0.0, \holpIsabelle=5.8).\\
              %\chriscite{These examples have been discussed (but not automated!) in [Delgrande'98]}
            \end{column}
            \begin{column}{.61\textwidth} \footnotesize
              \vskip-3.6ex

              \, \hfill Notation: \textcolor{red}{$\phi \Rightarrow_X \psi$} := \textcolor{red}{$(\exists^{va} X \phi) \Rightarrow \forall^{va} X
              (\phi \rightarrow \psi)$} \,
              \vskip1ex

              \textbf{Example: Opus, the penguin} \\[1ex]

                 \textcolor{green}{\textit{``Birds (b) normally fly (f), but Opus (o) is a bird that normally does not fly.''}}
                \begin{align*}
                  &\textcolor{red}{b(X) \Rightarrow_X f(X), \quad b(o), \quad  b(o) \Rightarrow \neg f(o)}
                \end{align*}
                \holp\ reports a finite model (\holpNitpick=8.6). When $\textcolor{red}{\forall^{co}}$ is used:  \holp\  says unsatisfiable (\holpSatallax=0.0, \holpIsabelle=7.9).\\[.5em]
                
                 \textcolor{green}{\textit{``Birds normally fly and 
                necessarily Opus the bird does not fly.''}}
                \begin{align*}
                  &\textcolor{red}{b(X) \Rightarrow_X f(X), \quad \Box(b(o) \wedge \neg f(o))}
                \end{align*}
                \holp\ reports a finite model (\holpNitpick=8.7).  When $\textcolor{red}{\forall^{co}}$ is used:
                \holp\  says  unsatisfiable
                (\holpSatallax=0.0, \holpIsabelle=7.6).\\[.5em]
                
                \textcolor{green}{\textit{``Birds normally fly and necessarily there is a non-flying bird.''}}
                \begin{align*}
                  &\textcolor{red}{b(X) \Rightarrow_X f(X), \quad \Box\exists^{va}(b(X) \wedge \neg f(X))}
                \end{align*}
                \holp\ reports unsatisfiability  (\holpSatallax=0.0, \holpIsabelle=8.7), also when $\textcolor{red}{\forall^{co}}$ is used (\holpSatallax=0.0, \holpIsabelle=8.8).\\[.5em]
                
                \textcolor{green}{\textit{``Birds normally fly, penguins normally do not fly and that all penguins are necessarily birds.''}}
                \begin{align*}
                  &\textcolor{red}{b(X) \Rightarrow_X f(X), \quad p(X) \Rightarrow_X \neg f(X), \quad \forall^{va} \Box (p(X) \rightarrow b(X))} 
                \end{align*}
                \holp\ generates a finite model ($\textcolor{red}{\forall^{va}}$: \holpNitpick=8.8; $\textcolor{red}{\forall^{co}}$: \holpNitpick=7.9).\\[.5em]

 Moreover, \holp\ can conclude from the
                statements above  that \textcolor{green}{\textit{``Birds are normally
                not penguins.''}} ($\textcolor{red}{\forall^{va}}$: \holpSatallax=0.9, \holpLeo=10.2, \holpAgsyhol=9.4; $\textcolor{red}{\forall^{co}}$: \holpSatallax=0.8, \holpLeo=10.1, \holpAgsyhol=0.3):
                \begin{align*}
                  \textcolor{red}{b(X) \Rightarrow_X f(X), \quad p(X) \Rightarrow_X \neg f(X), \quad \forall^{va} \Box (p(X) \rightarrow b(X))  \quad \vdash \quad b(X) \Rightarrow_X \neg p(X)}
                \end{align*}
                In line with Delgrande, \holp\ reports a
                countermodel for the following statement (\holpNitpick=8.7):
                \begin{align*}
                  \textcolor{red}{b(X) \Rightarrow_X f(X), \quad p(X) \Rightarrow_X \neg f(X), \quad \forall^{va} \Box (p(X) \rightarrow b(X))  \quad \vdash \quad  b(o) \Rightarrow \neg p(o)}
                \end{align*}
                However, when $\textcolor{red}{\forall^{co}}$ is used, \holp\ reports a  theorem (\holpSatallax=0.8, \holpAgsyhol=0.4).
 \vskip2ex
        \chriscite{These examples have been discussed (but not automated) in [Delgrande'98]}
            \end{column}
          \end{columns}         
        \end{block}
      \end{column}
    \end{columns}
    \vskip2ex
    \begin{columns}[t]
      \begin{column}{.42\linewidth}
        \begin{block}{The HOL Metaprover $H$
          %  \holp=(\holpLeo,\holpSatallax,\holpIsabelle,\holpNitpick,\holpAgsyhol)
          }
          \footnotesize
          The \holp\ metaprover for HOL sequentially calls the following prover and model finders: \\[.5em]
          \quad\textbf{\holpLeo} LEO-II (Benzm\"uller/Sultana/Theiss): \url{http://www.leoprover.org} \\
          \quad\textbf{\holpSatallax} Satallax (Brown): \url{http://www.ps.uni-saarland.de/~cebrown/satallax/}\\
          \quad\textbf{\holpIsabelle} Isabelle (Blanchette/Paulson/Nipkow/\ldots): \url{http://isabelle.in.tum.de/} \\
          \quad\textbf{\holpNitpick} Nitpick (Blanchette): \url{http://www4.in.tum.de/~blanchet/nitpick.html} \\
          \quad\textbf{\holpAgsyhol} agsyHol (Lindblatt): \url{https://github.com/frelindb/agsyHOL}\\[.6em]
          These systems support THF0 syntax. These provers are remotely
          available via \\ SystemOnTPTP:
          \url{http://www.cs.miami.edu/~tptp/cgi-bin/SystemOnTPTP}
        \end{block}
      \end{column}
      \hfill
      \begin{column}{.56\linewidth}
        \begin{block}{References and Further Reading} \scriptsize 
         \vskip1ex
          \begin{tabular}{ll}
            [Andrews'72a] & P.B. Andrews.
            General models, descriptions, and choice in type theory.
            JSL, 37(2):385-394, 1972. \\
            
            [Andrews'72b] & P.B. Andrews.
            General models and extensionality.
            JSL, 37(2):395--397, 1972. \\
            
            [Benzm\"uller'13] & C. Benzm�ller. A top-down approach to 
            combining logics, Proc. of ICAART 2013, Barcelona, Spain, 2013. \\
            
            [Benzm\"ullerEtAl'04] & C. Benzm�ller, C. E. Brown, and M. 
            Kohlhase, Higher order semantics and extensionality. JSL, 69(4):1027-1088, 2004. \\ 

            [Benzm\"ullerEtAl'09] & C. Benzm{\"u}ller, C. E. Brown, and M. Kohlhase.
            Cut-simulation and impredicativity. LMCS, 5(1:6):1--21, 2009. \\ 

            [Benzm.Genovese'11] & C.~Benzm{\"u}ller and V.~Genovese.
            Quantified conditional logics are fragments of {HOL}. NCMPL 2011. arXiv:1204.5920v1\\

            [Church'40] & A.~Church. A formulation of the simple theory of types.  
            JSL, 5:56--68, 1940. \\       

            [Delgrande'98] & J.P. Delgrande. On first-order conditional logics.
 
            Artificial Intelligence, 105(1-2):105--137, 1998. \\
            
 %           [Henkin'50] & \\ 

            [Stalnaker'68] & R.C. Stalnaker.
            A theory of conditionals.
            In Studies in Logical Theory, pp. 98--112. Blackwell, 1968.\\ 

            [SutcliffeBenzm.'10] & G.~Sutcliffe and C.~Benzm. 
            Automated reasoning in HOL using the {TPTP THF}
            infrastructure.
            {\em JFR}, 3(1):1--27, 2010. \\[.9em]
           \end{tabular}
        \end{block}
      \end{column}
    \end{columns}
 
 \end{frame}
\end{document}


%%%%%%%%%%%%%%%%%%%%%%%%%%%%%%%%%%%%%%%%%%%%%%%%%%%%%%%%%%%%%%%%%%%%%%%%%%%%%%%%%%%%%%%%%%%%%%%%%%%%
%%% Local Variables: 
%%% mode: latex
%%% TeX-PDF-mode: t
%%% End:

\documentclass{ecai2014}
\usepackage{times}
\usepackage{graphicx}

\usepackage{verbatim}
\usepackage{wrapfig}
\usepackage{amsmath,latexsym}
\usepackage{xspace}
\usepackage{stmaryrd}
\usepackage{txgreeks}
\usepackage[hyphens]{url}


%%\ecaisubmission   % inserts page numbers. Use only for submission of paper.
                  % Do NOT use for camera-ready version of paper.

%%%%%%%%%%%% some personal definitions  %%%%%%%%%%%%%

% -----------------------------------------------
\newtheorem{definition}{Def.}
\newtheorem{example}{Ex.}
% -----------------------------------------------

\def\lambdot{\rule{0.6mm}{0.6mm}\hspace{0.4ex}} 
\def\all#1{\forall #1\lambdot}
\def\exi#1{\exists #1\lambdot}
\def\lam#1{\lambda #1\lambdot}


\def\modal#1{#1}
\def\mfalse{\modal\bot}
\def\mtrue{\modal\top}
\def\mnot{\modal\neg\,}
\def\mor{\,\modal\vee\,}
\def\mand{\,\modal\wedge\,}
\def\mimpl{\,\modal\supset\,}
\def\miff{\,\modal\Leftrightarrow\,}
\def\mball#1{\modal\Box_{#1}\,}
\def\mdexi#1{\modal\Diamond_{#1}\,}
\def\mall#1{\modal{\forall}{#1}\lambdot\,}
\def\mexi#1{\modal{\exists}{#1}\lambdot\,}
\def\mpi{\modal{\Pi}\,}
\def\mvalid{\modal{\texttt{valid}}} 

\def\Metaeq{=}
\def\bnormform#1{\left.#1\hspace*{-.4ex}\right\downarrow_\beta}
\def\benormform#1{\left.#1\hspace*{-.4ex}\right\downarrow_{\beta\eta}}
\def\Bnormform#1{\left.{#1}\hspace*{-.4ex}\right\downarrow_\beta}
\def\Benormform#1{\left.{#1}\hspace*{-.4ex}\right\downarrow_{\beta\eta}}
\def\ambnormform#1{{#1}\hspace*{-1.1ex}\downarrow_{\kern-.2em\scriptscriptstyle *}} % ``ambiguous'' normal form
\def\eqb{\Metaeq_{\beta}}
\def\eqe{\Metaeq_{\eta}}
\def\eqbe{\Metaeq_{\beta\eta}}
\def\convarrow{\rightarrow}


\newcommand\entity[1]{\text{\textrm{#1}}}
\def\QHL{\entity{QHL}}
\def\QML{\entity{QML}}
\def\NOM{\entity{NOM}}
\def\SVAR{\entity{SVAR}}
\def\CON{\entity{CON}}
\def\FVAR{\entity{FVAR}}
%\def\IC{\entity{IC}}
\def\FSYM{\entity{FSYM}}
\def\RSYM{\entity{RSYM}}
\def\QHLSTT{\entity{QHLSTT}}
\def\QMLSTT{\entity{QMLSTT}}
\def\IV{\entity{IV}}
\def\PV{\entity{PV}}
\def\SYM{\entity{SYM}}
\def\IVSTT{\entity{IVSTT}}
\def\PVSTT{\entity{PVSTT}}
\def\SYMSTT{\entity{SYMSTT}}
\def\SSTT{\entity{SSTT}}
\def\AR{\entity{AR}}
\def\STT{\entity{STT}\xspace}

\def\QKPIm{\entity{QK}\pi^-\xspace}
\def\QKPI{\entity{QK}\pi\xspace}
\def\QKPIp{\entity{QK}\pi^+\xspace}
\def\QSFPIm{\entity{QS5}\pi^-\xspace}
\def\QSFPI{\entity{QS5}\pi\xspace}
\def\QSFPIp{\entity{QS5}\pi^+\xspace}

\def\stt{\entity{STT}\xspace}
\def\tt{\entity{STT}}
\def\lm{\entity{MM}\xspace}
\def\ipl{\entity{IPL}\xspace}
\def\HOML{\entity{HOML}\xspace}
\def\HOL{\entity{HOL}\xspace}

\def\worldtype{\mu}
\def\indtype{\iota}
\def\mutype{\mu}
\def\boola{\omicron}
\def\boolb{\hat{\omicron}}

\def\ar{\shortrightarrow}

\newcommand\hol[1]{\boldsymbol{#1}}
\newcommand\lift[1]{\lceil #1 \rceil}
\newcommand\llift[1]{\dot{#1}}

\newcommand{\imp}{\supset}
\newcommand{\biimp}{\equiv}
\newcommand{\allq}{\forall}
\newcommand{\exq}{\exists}
\newcommand{\seq}{\vdash}
\newcommand{\nec}{\Box} % necessarily
\newcommand{\pos}{\Diamond} % possibly
\newcommand{\ess}[2]{#1 \ \mathit{ess.} \ #2}
\newcommand{\NE}{\mathit{NE}}

%%%%%%%%%%


\begin{document}

\title{Automating G\"odel's Ontological Proof of God's Existence with Higher-order Automated Theorem Provers}

\author{Christoph Benzm\"uller\institute{Freie Universit\"at Berlin,
    Germany, email: c.benzmueller@fu-berlin.de; this author has been
    supported by the German National Research Foundation (DFG) under
 grants BE 2501/9-1 and BE 2501/11-1. }
  \and Bruno Woltzenlogel Paleo\institute{Technical University
    Vienna, Austria,  email: bruno@logic.at}  
}

\maketitle


\begin{abstract}
  Kurt G{\"o}del's ontological argument for God's existence has been
  formalized and automated on a computer with higher-order automated
  theorem provers. From G{\"o}del's premises, the computer proved:
  necessarily, there exists God. On the other hand, the theorem
  provers have also confirmed prominent criticism on G{\"o}del's
  ontological argument, and they found some new results about it.  

  The background theory of the work presented here offers a novel
  perspective towards a \emph{computational theoretical philosophy}.
\end{abstract}

\section{INTRODUCTION}
Kurt G{\"o}del proposed an argumentation formalism to prove the
existence of God \cite{GoedelNotes,ScottNotes}. Attempts to prove the
existence (or non-existence) of God by means of abstract, ontological
arguments are an old tradition in western philosophy.  Before
G{\"o}del, several prominent philosophers, including St. Anselm of
Canterbury, Descartes and Leibniz, have presented similar
arguments. Moreover, there is an impressive body of recent and ongoing
work (cf.~\cite{sobel2004logic,Fitting,ContemporaryBibliography} and the references therein).
Ontological arguments, for or against the existence of God,
illustrate well an essential aspect of metaphysics: some (necessary) facts
for our existing world are deduced by purely a priori, analytical means from some
abstract definitions and axioms. % Contingent truths are to be
% distinguished from necessary truths.

What motivated G{\"o}del as a logician was the question, whether it is
possible to deduce the existence of God from a small number of
foundational (but debatable) axioms and definitions, with a mathematically precise,
formal argumentation chain in a well defined logic.



In theoretical philosophy, formal logical confrontations with such
ontological arguments had been so far (mainly) limited to paper
and pen.  Up to now, the use of computers was prevented, because the
logics of the available theorem proving systems were not expressive
enough to formalize the abstract concepts adequately. G{\"o}del's proof
uses, for example, a complex higher-order modal logic (\HOML) to handle
concepts such as \emph{possibility} and \emph{necessity} and to support
quantification over individuals and properties.

Current works \cite{J23,B9} of the first author and Paulson illustrate that many
expressive logics, including quantified (multi-)modal logics, can be
embedded into the classical higher-order logic (\HOL), which can thus be seen
as a universal logic \cite{C36}. For this universal logic, efficient automated
theorem provers have been developed in recent years, and these systems
were now employed in our work.

G\"{o}del defines God (see Fig.~\ref{fig1}) as a being who possesses all \emph{positive}
properties.  He does not extensively discuss what positive properties
are, but instead he states a few reasonable (but debatable) axioms
that they should satisfy.  Various slightly different versions of
axioms and definitions have been considered by G\"{o}del and by
several philosophers who commented on his proof
(cf.~\cite{sobel2004logic,anderson90:_some_emend_of_goedel_ontol_proof,AndersonGettings,Fitting,Adams,ContemporaryBibliography}).

\begin{figure}[t]
\noindent \framebox[\columnwidth][r]{
\begin{minipage}{.94\columnwidth}%\small
\begin{itemize}
\item[A1] Either a property or its negation is positive, but not
  both: \\ \phantom{b} \hfill 
  $\hol{\allq \phi [P(\neg \phi) \biimp \neg P(\phi)]}$ 
\item[A2] A property necessarily implied by a
  positive property is positive:  \hfill 
  $\hol{\allq \phi \allq \psi [(P(\phi) \wedge \nec \allq x [\phi(x)
  \imp \psi(x)]) \imp P(\psi)]}$ 
\item[T1] Positive properties are possibly exemplified: \\ \phantom{b} \hfill $\hol{\allq
  \phi [P(\phi) \imp \pos \exq x \phi(x)]}$ 
\item[D1] A \emph{God-like} being possesses all positive properties: \\ \phantom{b} \hfill
  $\hol{G(x) \biimp \forall \phi [P(\phi) \imp \phi(x)]}$ 
\item[A3]  The property of being God-like is positive: \hfill   $\hol{P(G)}$ 
\item[C\phantom{1}] Possibly, God exists: \hfill $\hol{\pos \exq x G(x)}$ 
\item[A4]  Positive properties are necessarily positive: \\  \phantom{b} \hfill 
  $\hol{\allq \phi [P(\phi) \imp \Box \; P(\phi)]}$ 
\item[D2] An \emph{essence} of an individual is a property possessed by it and necessarily implying any of its properties: \\
  \phantom{b} \hfill $\hol{\ess{\phi}{x} \biimp \phi(x) \wedge \allq
  \psi (\psi(x) \imp \nec \allq y (\phi(y) \imp \psi(y)))}$ 
\item[T2]  Being God-like is an essence of any
  God-like being:  \\ \phantom{b}  \hfill $\hol{\allq x [G(x) \imp \ess{G}{x}]}$ 
\item[D3] \emph{Necessary existence} of an individ.~is the necessary exemplification of all its essences: 
  \phantom{b} \hfill $\hol{\NE(x) \biimp \allq \phi [\ess{\phi}{x} \imp \nec
  \exq y \phi(y)]}$
\item[A5] Necessary existence is a positive property: \hfill $\hol{P(\NE)}$ 
\item[T3] Necessarily, God exists: \hfill $\hol{\nec \exq x G(x)}$ 
\end{itemize}
\end{minipage}
} \vskip-.5em
\caption{Scott's version of G\"odel's ontological argument \cite{ScottNotes}.\label{fig1}} 
\end{figure}

The overall idea of G{\"o}del's proof is in the tradition of Anselm's
argument, who defined God as some entity of which nothing greater can be
conceived. Anselm argued that existence in the actual world would
make such an assumed being even greater; hence, by definition God must
exist. G{\"o}del's ontological argument is clearly related to this
reasoning pattern. However, it also tries to fix some fundamental
weaknesses in Anselm's work. For example, G{\"o}del explicitly proves
that God's existence is possible, which has been a basic assumption of
Anselm. Because of this, Anselm's argument has been
criticized as incomplete by Leibniz. Leibniz instead claimed that the
assumption should be derivable from the definition of God as a perfect
being and from the notion of perfection.  G{\"o}del's proof 
addresses this critique, and it also addresses the critique of others,
including Kant's objection that existence should not be treated as a predicate. On the other hand, G{\"o}del's work
still leaves room for criticism, in particular, his axioms are so
strong that they imply \emph{modal collapse}, that is, a situation where
contingent truths and necessary
truths coincide. More information on the philosophical debate on
G{\"o}del's proof is provided in~\cite{sobel2004logic}.


We have analyzed Dana Scott's version of G\"{o}del's proof
\cite{ScottNotes} (cf.~Fig.~\ref{fig1}) for the first-time with an
unprecedented degree of detail and formality with the help of
higher-order automated theorem provers (HOL-ATPs).\footnote{All
  sources of our formalization are publicly available
  at~\url{https://github.com/FormalTheology/GoedelGod}. Our work has
  attracted major public interest, and leading media institutions
  worldwide have reported on it; some exemplary links to respective
  media reports and interviews are available at the above URL (see
  `Press' subfolder).}  The following has been done (and in this
order):
% \footnote{System URLs:
%  TPTP---\url{http://tptp.org};
%  Nitpick---\url{http://www4.in.tum.de/\~{}blanchet/nitpick.html};
%  LEO-II---\url{http://leoprover.org};
%  Satallax---\url{http://www.ps.uni-saarland.de/\~{}cebrown/satallax}; 
%  Coq---\url{http://coq.inria.fr};
%  Isabelle---\url{http://isabelle.in.tum.de}.
% }
(i) a detailed natural deduction proof;
%
(ii) a formalization in TPTP THF syntax \cite{J22};
%
(iii) an automatic verification of the consistency of the axioms and 
definitions with Nitpick \cite{Nitpick};
%
(iv) an automatic demonstration of the theorems with the provers LEO-II \cite{LEO-II} and Satallax \cite{Satallax};
%
(v) a step-by-step formalization using the Coq proof assistant \cite{Coq};
%
(vi) a formalization using the Isabelle proof assistant \cite{Isabelle}, where the
theorems (and some additional lemmata) have been automated with the
Isabelle tools Sledgehammer %~\cite{Sledgehammer} 
and Metis. %~\cite{Metis}.
Subsequently, we have studied additional consequences
of G\"odel's axioms, including modal collapse and monotheism, and we
have investigated variations of the proof, for example, by switching
from constant domain semantics to varying domain semantics.


In this paper we focus on the core aspect of our work related to AI: proof automation with HOL-ATPs (cf.~aspects (ii)--(iv)
above). % In addition, we present the detailed technical
% background, i.e., the embedding of \HOML in \HOL.  None of these
% aspects have been adressed in detail yet in any of our other existing 
% (short and partly non-reviewed) papers on the subject \cite{J30,W50,J29,J28}.
The particular contributions of this paper are as follows: 
% In Sec.~\ref{sec1} we present Scott's script \cite{ScottNotes} of
% G\"odel's proof, and we summarize the results of our
% recent automation of this script with HOL-ATPs. 
In Sec.~\ref{sec2} we present an elegant embedding of 
\HOML \cite{Gallin75,homl} in \HOL \cite{andrewsSEP,B5}. 
This background theory extends and
adapts the work as presented in \cite{B9,J23}.
In Sec.~\ref{sec3}, we present details on the
encoding of this embedding and of G{\"o}del's argument in the
concrete THF syntax \cite{J22} for \HOL, and we report on the
experiments we have conducted with HOL-ATPs.  The main findings of
these experiments are summarized in Sec.~\ref{sec4}.
% : in addition to
% automating G{\"o}del's proof script we have automated also some further
% results (e.g.~modal collapse), and we have studied
% the argument under modified logical settings (varying vs.~constant domain semantics).
% (by switching from constant
% domain semantics to varying domain semantics)
Related and future work is addressed in Sec.~\ref{sec5}, and the paper
is concluded in Sec.~\ref{sec6}.
%  where we also discuss potential
% implications towards a \emph{computational theoretical
%   philosophy}.
None of the above aspects have been addressed (at least
not in depth) in any of our other existing (short and partly
non-reviewed) publications on the subject \cite{J30,J28,W50,J29}.



\section{THEORY FRAMEWORK}\label{sec2}
An embedding of quantified modal logic (with first-order and
propositional quantifiers) in HOL has been presented in
\cite{J23}. The theory below extends this work: quantifiers for all
types are now supported, and nested uninterpreted predicate and
function symbols of arbitrary types
are allowed as opposed to allowing top-level uninterpreted
predicate symbols over individual variables only.

\subsection{Higher-order modal logic}
A notion of \HOML is introduced that extends \HOL with a modal
operator $\Box$. An appropriate notion of semantics for \HOML is
obtained by adapting Henkin semantics for \HOL (cf.~\cite{Henkin50}
and~\cite{Gallin75}). The presentation in this section is adapted
from~\cite{homl} and~\cite{andrewsSEP}.

% First-order quantification can be constant domain or varying domain.  Below we only consider the constant domain case: every possible world has the same domain. We adapt the presentation of syntax and
%  semantics of quantified modal logic from Fitting \cite{Fitting02} to
%  the higher-order case.

% Let $\IV$ be a set of first-order (individual) variables, $\PV$ a
% set of propositional variables, and $\SYM$ a set of predicate
% symbols of any arity. Like Fitting, we keep our definitions simple by not having function or constant symbols; our language has no terms other than variables.
%   While Fitting \cite{Fitting02} studies quantified monomodal logic, we
%   are interested in quantified multimodal logic. Hence, we introduce
%   multiple $\mball{r}$ operators for symbols $r$ from an index set
%   $S$. 

\begin{definition}\label{homltypes} The set $\entity{T}$ of \emph{simple
    types} is freely generated from the set of basic types $\{o,
  \mu\}$ ($o$ stands for Booleans and $\mu$ for individuals) using the
  function type constructor $\ar$. We may avoid parentheses, and
  $\alpha\ar\alpha\ar\alpha$ then stands for
  $(\alpha\ar(\alpha\ar\alpha))$, that is, function types associate to
  the right.
\end{definition} 

\begin{definition}\label{homlgrammar}
The \emph{grammar} for \HOML is:
\begin{align*} 
  s,t \quad ::= \quad & p_\alpha \mid X_\alpha \mid (\lam{X_\alpha}
  s_\beta)_{\alpha\ar\beta} \mid (s_{\alpha\ar\beta}\, t_\alpha)_\beta
  \mid (\neg_{o\ar o}\, s_o)_o \mid \\
  & ((\vee_{o\ar o\ar o} s_o)\, t_o)_o  
  \mid (\forall_{(\alpha\ar
    o)\ar o}(\lam{X_\alpha} s_o))_o  \mid (\Box_{o\ar o}\, s_o)_o
%P \mid k(X^1,\ldots,X^n) \mid \mnot s \mid s \mor t \mid \all{X} s  \mid \all{P} s  \mid \mball{r} s
\end{align*}
where $\alpha,\beta\in T$.  $p_\alpha$ denotes typed constants and
$X_\alpha$ typed variables (distinct from $p_\alpha$).  Complex typed
terms are constructed via abstraction and application. The type of
each term is given as a subscript.  Terms $s_o$ of type $o$ are called
formulas.  The logical connectives of choice are
$\neg_{o\ar o}$, $\vee_{o\ar
  o\ar o}$, $\forall_{(\alpha\ar
  o)\ar o}$ (for $\alpha\in T$), and
$\Box_{o\ar o}$.  Type subscripts may be dropped if
irrelevant or obvious. Similarly, parentheses may be avoided.  Binder
notation $\all{X_\alpha} s_o$ is used as shorthand for
$\forall_{(\alpha\ar o)\ar o}
(\lam{X_\alpha} s_o)$, and infix notation $s \vee t$ is employed
instead of $((\vee s)\, t)$. From the above connectives, other logical
connectives, such as $\top$, $\bot$, $\wedge$, $\supset$, $\equiv$, $\exists$,
and $\Diamond$, can be defined in the usual way.
\end{definition}

\begin{definition}\label{homlsubstitution}
  \emph{Substitution} of a term $A_\alpha$ for a variable $X_\alpha$
  in a term $B_\beta$ is denoted by $[A/X]B$.  Since we consider
  $\alpha$-conversion implicitly, we assume the bound variables of $B$
  avoid variable capture.
\end{definition}

\begin{definition}\label{homlbetaeta} %\footnote{\textbf{Chris:} Is Def.4 needed at all?}
  Two common relations on terms are given by $\beta$-reduction and
  $\eta$-reduction.  A $\beta$-redex has the form $(\lam{X}s)t$ and
  $\beta$-reduces to $[t/X]s$.  An $\eta$-redex has the form
  $(\lam{X}s X)$ where variable $X$ is not free in $s$; it
  $\eta$-reduces to $s$.  We write $s\eqb t$ to mean $s$ can be
  converted to $t$ by a series of $\beta$-reductions and expansions.
  Similarly, $s\eqbe t$ means $s$ can be converted to $t$ using both
  $\beta$ and $\eta$.  For each $s_\alpha\in \HOML$ there is a unique
  \emph{$\beta$-normal form} and a unique \emph{$\beta\eta$-normal
    form}.
% From this fact we know $s\eqb t$ ($s\eqbe t$) iff
% $\Bnormform s\Metaeq\Bnormform t$
% ($\Benormform s\Metaeq\Benormform t$).
% The semantics of $\stt$ is well understood and thoroughly documented
% in the literature \cite{Henkin50,Andrews72b,Andrews72a,BBK04}; our
% summary below is adapted from Andrews~\cite{sep-type-theory-church}.
\end{definition}


\begin{definition}\label{homlframe}
A \emph{frame} $D$ is a collection $\{D_\alpha\}_{\alpha\in\entity{T}}$
of nonempty sets $D_\alpha$, such that $D_o = \{T,F\}$
(for truth and falsehood).  The
$D_{\alpha\ar\beta}$ are collections of functions mapping
$D_\alpha$ into $D_\beta$. 
\end{definition}


\begin{definition}\label{homlassignment}
A \emph{variable assignment} $g$ maps
variables $X_\alpha$ to elements in $D_\alpha$. $g[d/W]$ denotes the
assignment that is identical to $g$, except for variable $W$, which is
now mapped to $d$.
\end{definition}

\begin{definition}\label{homlmodel}
  A \emph{model} for \HOML is a quadruple $M=\langle W, R, D,
  \{I_w\}_{w\in W} \rangle$, where $W$ is a set of worlds (or states),
  $R$ is an accessibility relation between the worlds in $W$, $D$ is a
  frame, and for each $w\in W$, $\{I_w\}_{w\in W}$ is a family of
  typed interpretation functions mapping constant symbols $p_\alpha$
  to appropriate elements of $D_\alpha$, called the \emph{denotation
    of $p_\alpha$ in world $w$} (the logical connectives $\neg$,
  $\vee$, $\forall$, and $\Box$ are always given the standard
  denotations, see below).  Moreover, it is assumed that the domains
  $D_{\alpha\ar\alpha\ar o}$ contain the respective identity relations
  on objects of type $\alpha$ (to overcome the extensionality
  issue discussed in \cite{Andrews:gmae72}).
\end{definition}


% \emph{Validity} of a formula $s$ for a model $M=\langle  W, R, \{D_\alpha\}_{\alpha\in\entity{T}}, \{I_\alpha\}_{\alpha\in\entity{T}} \rangle$,
% a world $w\in W$, and a variable assignment $g$ is denoted as $M,g,w\models s$ and defined as
% follows, where $[a/Z]g$ denotes the assignment identical to $g$ except that 
% $([a/Z]g)(Z) = a$:
% \begin{align*}
% M,g,w &\models k(X^1,\ldots,X^n) &\text{if and only if}& \quad \langle g^{iv}(X^1),\ldots,g^{iv}(X^n)\rangle\in I_w(k) \\
% M,g,w &\models P &\text{if and only if}& \quad  w\in g^{pv}(P) \\
% M,g,w &\models \mnot s &\text{if and only if}& \quad  M,g,w \not\models s \\
% M,g,w &\models s \mor t &\text{if and only if}& \quad  M,g,w \models s \text{ or } M,g,w \models t \\
% M,g,w &\models \mall{X}s &\text{if and only if}& \quad  M,([d/X]g^{iv},g^{pv}),w \models s \\ 
% & & & \quad \text{for all } d\in D \\
% M,g,w &\models \mall{Q}s &\text{if and only if}& \quad  M,(g^{iv},[p/Q]g^{pv}),w \models s \\ 
% & & & \quad \text{for all } p\in P \\
% M,g,w &\models \mball{r}s &\text{if and only if}& \quad  M,g,v \models s \text{ for all } v\in W \\
% & & & \quad \text{with } \langle w,v\rangle\in R_r
% \end{align*}


\begin{definition}\label{homlvalue}
The \emph{value} $\| s_\alpha\|^{M,g,w}$ of a \HOML term $s_\alpha$ on a model $M=\langle W, R, D, \{I_w\}_{w\in W}
\rangle$ in a world $w\in W$ under variable assignment $g$ is an element $d\in D_\alpha$ defined in the following way:
\begin{enumerate}
\item $\|p_\alpha\|^{M,g,w} = I_w(p_\alpha)$ and $\|X_\alpha\|^{M,g,w} = g(X_\alpha)$
\item $\|(s_{\alpha\ar\beta}\, t_\alpha)_\beta\|^{M,g,w} = \|s_{\alpha\ar\beta}\|^{M,g,w}(\|t_\alpha\|^{M,g,w})$
\item $\|(\lam{X_\alpha}s_\beta)_{\alpha\ar\beta}\|^{M,g,w} = $ the
  function $f$ from $D_\alpha$ to $D_\beta$ such that $f(d) = \|s_\beta\|^{M,g[d/X_\alpha],w}$ for all $d\in D_\alpha$
\item $\|(\neg_{o\ar o}\, s_o)_o\|^{M,g,w} = T$ iff $\|s_o\|^{M,g,w} = F$
\item $\|(({\vee_{o\ar o\ar o}}\,s_o)\,t_o)_o\|^{M,g,w} = T$ iff \,$\|s_o\|^{M,g,w} = T$ or $\|t_o\|^{M,g,w} = T$
\item $\|(\forall_{(\alpha\ar o)\ar o}(\lam{X_\alpha} s_o))_o\|^{M,g,w} = T$ iff for all $d\in D_\alpha$ we have  $\|s_o\|^{M,g[d/X_\alpha],w}  = T$
\item $\|(\Box_{o\ar o}\, s_o)_o\|^{M,g,w} = T$ iff for all $v \in W$ with $w R v$ we have  $\|s_o\|^{M,g,v}  = T$
\end{enumerate}
\end{definition}

\begin{definition}\label{homlhenkinmodel}
 A model $M=\langle W, R, D, \{I_w\}_{w\in W}\rangle$ is called a
 \emph{standard model} iff for all $\alpha,\beta\in T$ we have
 $D_{\alpha\ar\beta} = \{ f \mid f : D_\alpha \longrightarrow D_\beta
 \}$. In a \emph{Henkin model} function spaces are not necessarily
 full. Instead it is only required that $D_{\alpha\ar\beta}
 \subseteq \{ f \mid f : D_\alpha \longrightarrow D_\beta \}$ (for all
 $\alpha,\beta\in T$) and that the valuation function 
 $\|\cdot\|^{M,g,w}$ from above is total (i.e., every term denotes). Any standard model is obviously
 also a Henkin model. We consider Henkin models in the remainder.
\end{definition}

\begin{definition}\label{homlvalid}
A formula $s_o$ is \emph{true} in model $M$ for world $w$
under assignment $g$ iff $\|s_o\|^{M,g,w} = T$; this is also denoted
as $M,g,w \models s_o$.  A formula $s_o$ is called \emph{valid} in
$M$ iff $M,g,w \models s_o$ for all $w\in W$ and all
assignments $g$. Finally, a formula $s_o$ is called
\emph{valid}, which we denote by $\models s_o$, iff $s_o$ is valid for
all $M$. Moreover, we write $\Gamma\models \Delta$ (for sets of
 formulas $\Gamma$ and $\Delta$) iff there is a model $M=\langle W, R, D, \{I_w\}_{w\in W}
\rangle$, an assignment $g$, and a world $w\in W$, such that  
 $M,g,w \models s_o$ for all $s_o\in \Gamma$ and $M,g,w \models t_o$ for at least one $t_o\in \Delta$.
\end{definition}

The above definitions introduce higher-order modal logic K. In order
to obtain logics KB and S5 respective conditions on accessibility
relation $R$ are postulated: $R$ is a symmetric relation in logic KB,
and it is an equivalence relation in logic S5. If these restriction apply,
we use the notations $\models^{KB}$ and $\models^{S5}$.  G{\"o}del's
argument has been developed and studied in the context of logic
S5 (and logic S5 has subsequently been criticized). However, the
HOL-ATPs discovered (cf.~Sec.~\ref{sec4}) that logic KB
is sufficient.

An important issue for quantified modal logics is whether constant
domain or varying domain semantics is considered. The theory above
introduces constant domains. 
Terms (other than those of Boolean type) are modeled as rigid, that
is, their denotation is fixed for all worlds. 
%
An adaptation to varying or cumulative
domains is straightforward (cf.~\cite{fitting98}). Moreover, non-rigid
terms could be modeled; that is, terms whose denotation may switch
from world to world. 
%
The respective assumptions of 
G\"odel are not obvious to us.




\subsection{Classical higher-order logic}
\HOL is easily obtained from \HOML by removing the modal operator
$\Box$ from the grammar, and by dropping the set of possible worlds
$W$ and the accessibility relation $R$ from the definition of a model.
Nevertheless, we explicitly state the most relevant definitions for
the particular notion of \HOL as employed in this paper. One reason is
that we do want to carefully distinguish the \HOL and \HOML languages
in the remainder (we use boldface fonts for \HOL and standard fonts
for \HOML). There is also a subtle, but harmless, difference in the
\HOL language as employed here  in comparison to the standard
presentation: here three base types are employed, whereas
usually only two base types are considered. The third base type plays
a crucial role in our embedding of \HOML in \HOL.


\begin{definition}\label{holtypes}
  The set $\hol{T}$ of simple types freely generated from a set of
  basic types $\{\hol{o}, \hol{\mu}, \hol{\iota}\}$ using the function type
  constructor $\hol{\ar}$. $\hol{o}$ is the type of Booleans,
  $\hol{\mu}$ is the type of individuals, and type
  $\hol{\iota}$ is employed as the type of possible worlds below. As
  before we may avoid parentheses.
\end{definition}

\begin{definition}\label{holgrammar}
The grammar for higher-order logic $\HOL$ is:
\begin{align*} 
  \hol{s},\hol{t} \quad ::= \quad & \hol{p_\alpha} \mid \hol{X_\alpha} \mid \hol{(\lam{X_\alpha}
  s_\beta)_{\alpha\ar\beta}} \mid \hol{(s_{\alpha\ar\beta}\, t_\alpha)_\beta} \mid  \hol{\neg_{o\ar o}\, s_o} \mid \\
  & \hol{(({\vee_{o\ar o\ar o}}\,s_o)\,t_o)} \mid \hol{\forall_{(\alpha\ar
    o)\ar o}(\lam{X_\alpha} s_o)} 
\end{align*}
where $\hol{\alpha},\hol{\beta}\in \hol{T}$. The text from
Def.~\ref{homlgrammar} analogously applies, except that we do not
consider the modal  connectives $\hol{\Box}$ and $\hol{\Diamond}$.
\end{definition}

The definitions for substitution (Def.~\ref{homlsubstitution}), $\beta$- and $\eta$-reduction (Def.~\ref{homlbetaeta}), frame (Def.~\ref{homlframe}), and assignment (Def.~\ref{homlassignment}) remain unchanged.

\begin{definition}\label{holmodel}
  A \emph{model} for \HOL is a tuple $\hol{M}=\hol{\langle D, I \rangle}$, where
  $\hol{D}$ is a frame, and $\hol{I}$ is a family of typed interpretation
  functions mapping constant symbols $\hol{p_\alpha}$ to appropriate
  elements of $\hol{D_\alpha}$, called the \emph{denotation of $\hol{p_\alpha}$}
  (the logical connectives $\hol{\neg}$, $\hol{\vee}$, and $\hol{\forall}$ are always
  given the standard denotations, see below).  Moreover, we assume that the domains
  $\hol{D_{\alpha\ar\alpha\ar o}}$ contain the respective identity relations.
\end{definition}


\begin{definition}\label{holvalue}
  The \emph{value} $\hol{\| s_\alpha\|^{M,g}}$ of a \HOL term
  $\hol{s_\alpha}$ on a model $\hol{M}=\hol{\langle D, I \rangle}$ under assignment $\hol{g}$ is an element $\hol{d}\in \hol{D_\alpha}$
  defined in the following way:
\begin{enumerate}
\item $\hol{\|p_\alpha\|^{M,g}} = \hol{I(p_\alpha)}$ and $\hol{\|X_\alpha\|^{M,g}} = \hol{g(X_\alpha)}$
\item $\hol{\|(s_{\alpha\ar\beta}\, t_\alpha)_\beta\|^{M,g}} = \hol{\|s_{\alpha\ar\beta}\|^{M,g}(\|t_\alpha\|^{M,g})}$
\item $\hol{\|(\lam{X_\alpha}s_\beta)_{\alpha\ar\beta}\|^{M,g}} = $
  the function $\hol{f}$ from $\hol{D_\alpha}$ to $\hol{D_\beta}$ such
  that $\hol{f(d)} = \hol{\|s_\beta\|^{M,g[d/X_\alpha]}}$ for all
  $\hol{d}\in \hol{D_\alpha}$
\item $\hol{\|(\neg_{o\ar o}\, s_o)_o\|^{M,g}} = \hol{T}$ iff $\hol{\|s_o\|^{M,g}} = \hol{F}$
\item $\hol{\|(({\vee_{o\ar o\ar o}}\,s_o)\,t_o)_o\|^{M,g}} =
  \hol{T}$ iff\, $\hol{\|s_o\|^{M,g}} = \hol{T}$ or $\hol{\|t_o\|^{M,g}}
  = \hol{T}$
\item $\hol{\|(\forall_{(\alpha\ar o)\ar o}(\lam{X_\alpha}
    s_o))_o\|^{M,g}} = \hol{T}$ iff for all $\hol{d}\in
  \hol{D_\alpha}$ we have $\hol{\|s_o\|^{M,g[d/X_\alpha]}} = \hol{T}$
%\item $\|(\Box_{o\ar o}\, s_o)_o\|^{M,g,w} = T$ iff for all $v \in W$ with $w R v$ we have  $\|s_o\|^{M,g,v}  = T$
\end{enumerate}
\end{definition}


The definition for standard and Henkin models
(Def.~\ref{homlhenkinmodel}), and for truth in a model, validity,
etc. (Def.~\ref{homlvalid}) are adapted in the obvious way, and we use
the notation $\hol{M,g \models s_o}$, $\hol{\models s_o}$, and
$\hol{\Gamma \models \Delta}$. As for \HOML, we assume Henkin
semantics in the remainder.


\subsection{\HOML as a fragment of \HOL}
The encoding of \HOML in \HOL is simple: we identify \HOML formulas of
type $o$ with certain \HOL formulas of type $\hol{\iota \ar o}$. The \HOL
type $\hol{\iota \ar o}$ is abbreviated as $\hol{\sigma}$ in the
remainder. More generally, we define for each \HOML type $\alpha\in T$
the associated raised \HOL type $\lift{\alpha}$ as follows: $\lift{\mu}
= \hol{\mu}$, $\lift{o} = \hol{\sigma} = \hol{\iota \ar o}$, and
$\lift{\alpha\ar\beta} = \lift{\alpha} \hol{\ar} \lift{\beta}$.
Hence, all \HOML terms are rigid, except for those of type $o$.
% \footnote{This choice could be modified in an obvious way to
  % obtain non-rigid terms.}

\begin{definition}\label{operators}
\HOML terms $s_\alpha$ are associated with type-raised \HOL terms
$\lift{s_\alpha}$ in the following way:
\begin{align*} {\lift{p_\alpha}} & = \hol{p}_{\lift{\alpha}}
  \\
  {\lift{X_\alpha}} & = \hol{X}_{\lift{\alpha}}
  \\
  {\lift{(s_{\alpha\ar\beta}\,t_{\alpha})}} & =
  {(\lift{s_{\alpha\ar\beta}}\,\lift{ t_{\alpha}})}
  \\
  {\lift{(\lam{X_\alpha} s_\beta)}} & = {(\hol{\lambda}
    \lift{X_\alpha} \lambdot \lift{s_\beta})}
  \\
  {\lift{(\neg_{o\ar o}\,s_{o})}} & =
  (\hol{\llift{\neg}_{\sigma\ar\sigma}}\,\lift{ s_{\alpha}})
  \\
  {\lift{(({\vee}_{o\ar o\ar o}\,s_{o})\,t_{o})}} & =
  ((\hol{\llift{\vee}_{\sigma\ar\sigma\ar\sigma}}\,\lift{
    s_{\alpha}})\,\lift{ t_{\alpha}})
  \\
  {\lift{((\forall_{(\alpha\ar o)\ar o}\,(\lam{X_\alpha} s_\beta)}} & =
  (\hol{\llift{\forall}_{(\alpha\ar\sigma)\ar\sigma}}\,(\hol{\lambda}
  \lift{X_\alpha} \lambdot \lift{s_\beta})
  \\
  {\lift{(\Box_{o\ar o}\,s_{o})}} & =
  (\hol{\llift{\Box}_{\sigma\ar\sigma}}\,\lift{ s_{o}})
\end{align*}

$\hol{\llift{\neg}},\hol{\llift{\vee}},\hol{\llift{\forall}}$, and
$\hol{\llift{\Box}}$ are the \emph{type-raised modal \HOL connectives}
associated with the corresponding modal \HOML connectives. They are defined as follows (where $\hol{r_{\iota\ar
    \iota\ar o}}$ is a new constant symbol in \HOL associated with the
accessibility relation $R$ of \HOML):
\begin{align*}
\hol{\llift{\neg}_{\sigma\ar\sigma}} & = \hol{\lam{s_\sigma}\lam{W_\iota}\neg\,(s\,W)}
\\
\hol{\llift{\vee}_{\sigma\ar\sigma\ar\sigma}} & = \hol{\lam{s_\sigma} \lam{t_\sigma} \lam{W_\iota} s\,W \vee t\,W}
\\
\hol{\llift{\forall}_{(\alpha\ar \sigma)\ar
\sigma}} & = \hol{\lam{s_{\alpha\ar \sigma}} \lam{W_\iota}
\all{X_\alpha} s\,X\,W} 
\\
\hol{\llift{\Box}_{\sigma\ar\sigma}} & =  
\hol{\lam{s_\sigma} \lam{W_{\iota}} \all{V_{\iota}} \neg (r_{\iota\ar \iota\ar o}\,W\,V) \vee s\,V}
\end{align*}
As before, we write $\hol{\llift{\forall} {X_\alpha} \lambdot
  s_\sigma}$ as shorthand for $\hol{\llift{\forall}_{(\alpha\ar
    \sigma)\ar \sigma}(\lam{X_\alpha} s_\sigma)}$. Further operators,
such as $\hol{\llift{\top}}$, $\hol{\llift{\bot}}$,
$\hol{\llift{\wedge}}$, $\hol{\llift{\supset}}$, $\hol{\llift{\equiv}}$,
$\hol{\llift{\Diamond}}$, and $\hol{\llift{\exists}}$
($\hol{\llift{\exists} {X_\alpha} \lambdot s_\sigma}$ is used as
shorthand for $\hol{\llift{\exists}_{(\alpha\ar \sigma)\ar \sigma}
  (\lam{X_\alpha} s_\sigma)}$) can now be easily defined.\footnote{We
  could introduce further modal operators, such as the difference
  modality $\hol{D}$, the global modality $\hol{E}$, nominals with $\hol{!}$, and the $\hol{@}$
  operator (cf.~\cite{J23}).} The above equations can be treated as
abbreviations in \HOL theorem provers. Alternatively, they can be
 stated as axioms where $=$ is either Leibniz equality or
primitive equality (if additionally provided in the \HOL grammar, as
is the case for most modern \HOL provers).
\end{definition}
% \begin{align*}
% D_{(\iota\ar o)\ar(\iota\ar o)} & = \lam{s_{\iota\ar o}} \lam{W_\iota} \exi{V_\iota} W \not= V \wedge s V 
% \\
% E_{(\iota\ar o)\ar(\iota\ar o)} & = \lam{s_{\iota\ar o}} s \mor D\, s 
% \\
% !_{(\iota\ar o)\ar(\iota\ar o)} & = \lam{s_{\iota\ar o}} E\, (s \mand \mnotinpit{D\,s}) 
% \\
% @_{\iota\ar(\iota\ar o)\ar(\iota\ar o)} & = \lam{W_\iota} \lam{s_{\iota\ar o}} s\, W 
% \end{align*}
% % This illustrates the potential of our embedding for encoding quantified
% hybrid logic, an issue that we might explore in future work.

%  In
% particular, the \HOML connectives $\neg_{o\ar o}$, $\vee_{o\ar o\ar
%   o}$, $\Pi_{(\alpha\ar o)\ar o}$, and $\Box_{o\ar o}$ are associated
% with ``type-raised'' \HOL terms of
% $\hol{\lift{\neg}_{\sigma\ar\sigma}}$,
% $\hol{\lift{\vee}_{\sigma\ar\sigma\ar\sigma}}$,
% $\hol{\lift{\Pi}_{(\alpha\ar \sigma)\ar \sigma}}$, and
% $\hol{\lift{\Box}_{\sigma\ar\sigma}}$, respectively.

As a consequence of the above embedding we can express \HOML proof
problems elegantly in the type-raised syntax of \HOL. Using rewriting
or definition expanding, we can reduce these representations to
corresponding statements containing only the basic \HOL connectives
$\hol{\neg_{o\ar o}}$, $\hol{\vee_{o\ar o\ar o}}$, and
$\hol{\forall_{(\alpha\ar o)\ar o}}$.

\begin{example}\label{ex1} %\footnote{Maybe we should instead use an axiom or theorem from G{\"o}del's proof?}
  The \HOML formula $\Box\,\exi{P_{\mu \ar o}} P\,a_\mu$ is
  associated with the type raised \HOL formula
  $\hol{\llift{\Box}\,\llift{\exists} {P_{\mu\ar
        \sigma}} \lambdot P\ a_\mu}$, which rewrites into the following
  $\beta\eta$-normal \HOL term of type $\hol{\sigma}$
\[\hol{\lam{W_\iota} \all{V_\iota} \neg (r\,W\,V) \vee \neg \all{P_{\mu\ar \sigma}} \neg (P\, a_\mu\, V)}\]
\end{example}



Next, we define validity of type-raised modal \HOL propositions 
$\hol{s_\sigma}$ in the obvious way: 
$\hol{s_\sigma}$ is valid iff for all possible worlds
$\hol{w_\iota}$ we have 
$\hol{w_\iota\in s_\sigma}$,
that is, iff
$\hol{(s_\sigma\,w_\iota)}$ holds.

\begin{definition}\label{validity}
  \emph{Validity} is modeled as an abbreviation for the following
  $\lambda$-term: $\textbf{valid} = \hol{\lam{s_{\iota\ar o}} \all{W_\iota}
    s\,W}$ (alternatively, we could define validity simply as
  $\hol{\forall_{(\iota\ar o)\ar o}}$). Instead of 
  $\hol{\textbf{valid}\,s_\sigma}$ we also use the notation
  $\hol{[s_\sigma]}$.
\end{definition}

\begin{example} \label{ex2} \sloppy We analyze whether the type-raised
  modal \HOL formula $\hol{\llift{\Box}\,\llift{\exists} {P_{\mu\ar
        \sigma}} \lambdot (P\ a_\mu)}$ is valid or not. For this, we
  formalize the \HOL proof problem
  $\hol{[{\llift{\Box}\,\llift{\exists} {P_{\mu-\ar \sigma}} \lambdot (P\
      a_\mu)}]}$, which expands into $\hol{\all{W_\iota} \all{V_\iota} \neg
    (r\,W\,V) \vee \neg \all{P_{\mu\ar \sigma}} \neg (P\, a_\mu\, V)}$.
  It is easy to check that this term is valid in Henkin semantics: put
  $\hol{P} = \hol{\lam{X_\mu} \lam{Y_\iota} \top}$.
%  Proving the statement automatically is non-trivial though as we shall shortly see.
\end{example}


\begin{theorem}[Soundness and Completeness] \label{thm1} 
For all \HOML formulas $s_o$ we have: 
\[\models s_o \quad \text{iff} \quad \hol{\models} \hol{[}\lceil s_o \rceil\hol{]}\]
\textbf{Proof sketch:} The proof adapts the ideas presented
in \cite{J23}. By contraposition it is sufficient to show $\not\models
s_o \,\,\text{iff}\,\hol{\not\models} \hol{[}\lceil s_o \rceil\hol{]}$,
that is, $\| s_o \|^{M,g,w}$ (for some \HOML model $M$, assignment
$g$, and $w$) iff $\hol{\| \all{W_\iota} \lceil s_o \rceil\,W\|^{M,g}}$
(for some \HOL model $\hol{M}$ and assignment $\hol{g}$) iff $\hol{\|
  \lceil s_o \rceil\,W\|^{M,g[w/W]}}$ (for some $\hol{M}$, $\hol{g}$,
and $w$). We easily get the proof by choosing the obvious
correspondences between $D$ and $\hol{D}$, $W$ and $\hol{D_\iota}$, $I$
and $\hol{I}$, $g$ and $\hol{g}$, $R$ and $r_{\iota\ar \iota\ar o}$,  and $w$
and $\hol{w}$. \hfill $\Box$
\end{theorem}

From Theorem~\ref{thm1} we get the following corollaries:
\[\models^{KB} s_o  \quad \text{iff}  \quad \hol{(\textbf{symmetric}\,r_{\iota\ar \iota\ar o})} \hol{\models} \hol{[}\lceil s_o \rceil\hol{]}\]
\[\models^{S5} s_o  \quad \text{iff}  \quad
\hol{(\textbf{equiv-rel}\,r_{\iota\ar \iota\ar o})} \hol{\models}
\hol{[}\lceil s_o \rceil\hol{]}\]
where \textbf{symmetric} and \textbf{equiv-rel} are defined in an
obvious way.

Constant domain quantification is
addressed above. Techniques for handling varying domain and cumulative domain
quantification in the embedding of first-order modal logics in \HOL
have been outlined in \cite{C34}. These techniques, which have also been
adapted  for the theory above, cannot be presented here for space limitations.

Note that also non-rigid terms can easily be modeled by
type-raising. For example, a non-rigid \HOML constant symbol
$\mathrm{kingOfFrance}_\mu$ would be mapped to a type-raised
(and thus world-depended) \HOL constant symbol $\mathbf{kingOfFrance_{\iota\ar \mu}}$.



\section{EXPERIMENTS} \label{sec3} The above embedding 
%from Sec.~\ref{sec2} 
has been encoded in the concrete THF0 syntax \cite{J22} for \HOL; cf.~the
files \texttt{Quantified\_K/\_KB/\_S5.ax}\footnote{The formalization in
  these files slightly varies from the above theory w.r.t. technical
  details. For example, a generic $\Box$-operator is introduced that
  can be instantiated for different accessibility relations as
  e.g. required for multi-modal logic applications
  (cf.~\cite{J23}). Moreover, since THF0 does not support
  polymorphism, copies of the $\hol{\llift{\forall}_{(\alpha\ar
      \sigma)\ar \sigma}}$ and $\hol{\llift{\exists}_{(\alpha\ar
      \sigma)\ar \sigma}}$ connectives are provided only for the
  selected types ${(\mu \ar \sigma)\ar \sigma}$ and ${((\mu \ar
    \sigma)\ar \sigma)\ar \sigma}$ as precisely required in
  G{\"o}dels's proof. The Isabelle version~\cite{J28} and the Coq
  version of the encoding instead provide respective polymorphic
  definitions.} available at
\url{https://github.com/FormalTheology/GoedelGod/tree/master/Formalizations/THF}
(all files mentioned below are provided under this URL). The
definition for quantifier
$\hol{\llift{\forall}_{((\mu \ar \sigma)\ar \sigma)\ar \sigma}}$, for
example, is given as\footnote{\texttt{\$i}, \texttt{\$o}, and
  \texttt{mu} represent the HOL base types $\hol{i}$, $\hol{o}$, and
  $\hol{\mu}$. \texttt{\$i>\$o} encodes a function (predicate) type.
  Function application is represented by \texttt{@}, and for universal
  quantification, existential quantification and $\lambda$-abstraction
  the symbols \texttt{!}, \texttt{?}  and \texttt{\^{}} are
  employed. $\hol{\neg}$, $\hol{\vee}$, $\hol{\wedge}$, and
  $\hol{\supset}$ are written as \texttt{\url{~}}, \texttt{|},
  \texttt{\&}, and \texttt{=>}, respectively. The type-raised modal
  connectives are called \texttt{mforall\_*}, \texttt{mexists\_*},
  \texttt{mnot}, \texttt{mor}, \texttt{mand}, \texttt{mimplies}, etc.}

{\footnotesize
\begin{verbatim}
  thf(mforall_indset,definition,
      ( mforall_indset
      = ( ^ [S: ( mu > $i > $o ) > $i > $o,W: $i] :
          ! [X: mu > $i > $o] :
            ( S @ X @ W ) ) )).
\end{verbatim}
}


Subsequently the axioms, definitions, and theorems from
Fig.~\ref{fig1} and some further, related problems have been encoded
in THF0. Then the THF0 compliant HOL-ATPs LEO-II \cite{LEO-II},
Satallax \cite{Satallax}, and Nitpick \cite{Nitpick} have been
employed to automate the proof problems. LEO-II, which internally
cooperates with the first-order prover E~\cite{Schulz:AICOM-2002}, was
used exclusively in the initial round of experiments, that is, it was the
first prover to automate G{\"o}del's ontological argument.


%\subsection{Automation of Scott's version of G{\"o}del's proof}
Theorem T1 from Fig.~\ref{fig1}, for example, is formalized as 

{\footnotesize
\begin{verbatim}
  thf(thmT1,conjecture,
     ( v
      @ ( mforall_indset
        @ ^ [Phi: mu > $i > $o] :
            ( mimplies @ ( p @ Phi )
            @ ( mdia
              @ ( mexists_ind
                @ ^ [X: mu] :
                    ( Phi @ X ) ) ) ) ) )).
\end{verbatim}
} 

\noindent This encodes the \HOL formula
$$\hol{[\llift{\forall}\phi_{\mu\ar \sigma} \lambdot}\hol{p_{(\mu\ar
    \sigma)\ar \sigma}\phi\,\llift{\supset}\, \llift{\Diamond}
  \llift{\exists} X_\mu \lambdot \phi X]}$$ \texttt{v} in the THF0
encoding stands for \textbf{valid} and $p$ corresponds to the
uppercase $P$, for `positive', from Fig.~\ref{fig1}.  The
respective encodings and the results of a series of recent experiments with
LEO-II (version 1.6.2), Satallax (version 2.7), and Nitpick (version
2013) are provided in Fig.~\ref{fig2}.  The first row marked with T1,
for example, shows that theorem T1 follows from axioms A2 and A1
(where only the $\supset$-direction is needed); LEO-II and Satallax
confirm this in 0.1 second.  The experiments have been carried
out w.r.t.~the logics K and/or KB, and w.r.t.~constant (const) and
varying (vary) domain semantics for the domains of individuals. The
exact dependencies (available axioms and definitions) are displayed for each
single problem. The results of the prover calls are given in
seconds. `---' means timeout. `THM', `CSA', `SAT', and `UNS'
are the reported result statuses; they stand for `Theorem',
`CounterSatisfiable', `Satisfiable', and `Unsatisfiable',
respectively.  {The experiments can be easily reproduced: all relevant
  files have been provided at the above URL.  For example, the two
  THF0 problem files associated with the first table row for T1 are
  \texttt{T1\_K\_const\_min.p} and \texttt{T1\_K\_vary\_min.p}, and
  those associated with the second row for T1 are
  \texttt{T1\_K\_const\_max.p} and
  \texttt{T1\_K\_vary\_max.p}, respectively. Moreover, a simple shell script
  \texttt{call\_tptp.sh} is provided, which can be used to make remote
  calls to LEO-II, Satallax, and Nitpick installed at Sutcliffe's
  SystemOnTPTP infrastructure \cite{sutcliffe2009tptp} at the
  University of Miami. The experiments used standard 2.80GHz
  computers with 1GB memory remotely located in Miami. 
\begin{figure*}[t] 
\noindent \framebox[\textwidth][r]{
\begin{minipage}{.99\textwidth}
\small
$\begin{array}{llllllll} \\[-.5em]
   & \text{\HOL encoding} & \text{dependencies} & \text{logic} & \text{status} & \text{LEO-II} & \text{Satallax} & \text{Nitpick} \\  
   &                      &                     &              &               & \text{const/vary} & \text{const/vary} & \text{const/vary} 
\\
  \text{A1} &  {\hol{[\llift{\forall}\phi_{\mu\ar \sigma} \lambdot p_{(\mu\ar \sigma)\ar \sigma} (\lambda X_\mu \lambdot \llift{\neg} (\phi X)) \, \llift{\equiv}\,  \llift{\neg} (p \phi) ]}} \\
  % \text{A2} &  \allq \phi \allq \psi [(P(\phi) \wedge \nec \allq x [\phi(x) \imp \psi(x)]) \imp P(\psi)] \\
  \text{A2} &  \multicolumn{4}{l}{\hol{[\llift{\forall}\phi_{\mu\ar \sigma}\lambdot \llift{\forall}\psi_{\mu\ar \sigma}\lambdot (p_{(\mu\ar \sigma)\ar \sigma} \phi \,\llift{\wedge}\, \llift{\Box} \llift{\forall}X_\mu\lambdot (\phi X \, \llift{\supset}\, \psi X)) \, \llift{\supset}\, p \psi]}}\\
  \text{T1} & \hol{[\llift{\forall}\phi_{\mu\ar \sigma} \lambdot p_{(\mu\ar \sigma)\ar \sigma}\phi\, \llift{\supset}\, \llift{\Diamond} \llift{\exists} X_\mu \lambdot \phi X]} & \text{A1}(\supset), \text{A2} & \text{K} & \text{THM} 
  & 0.1/0.1 & 0.0/0.0 & \text{---/---}\\
  & & \text{A1}, \text{A2} & \text{K} & \text{THM} 
  & 0.1/0.1 & 0.0/5.2 & \text{---/---}\\
%\text{D1} & G(x) \biimp \forall \phi [P(\phi) \imp \phi(x)] \\
\text{D1} & {\hol{g_{\mu\ar \sigma} = \lambda X_\mu \lambdot  \llift{\forall}\phi_{\mu\ar \sigma} \lambdot p_{(\mu\ar \sigma)\ar \sigma}\phi\, \llift{\supset}\, \phi X}} \\
\text{A3} &  \hol{[p_{(\mu\ar \sigma)\ar \sigma} g_{\mu\ar \sigma}]} \\
\text{C} &  \hol{[\llift{\Diamond} \llift{\exists} X_\mu \lambdot g_{\mu\ar \sigma} X]} & \text{T1}, \text{D1}, \text{A3} & \text{K} & \text{THM} & 0.0/0.0 & 0.0/0.0 & \text{---/---}\\
 & & \text{A1}, \text{A2}, \text{D1}, \text{A3} & \text{K} & \text{THM} 
& 0.0/0.0 & 5.2/31.3 & \text{---/---}\\
%\text{A4} &  \allq \phi [P(\phi) \imp \Box \; P(\phi)] \\
\text{A4} &  \hol{[\llift{\forall}\phi_{\mu\ar \sigma} \lambdot p_{(\mu\ar \sigma)\ar \sigma}\phi\,\llift{\supset}\, \llift{\Box} p \phi]} \\
%\text{D2} & \ess{\phi}{x} \biimp \phi(x) \wedge \allq \psi (\psi(x) \imp \nec \allq y (\phi(y) \imp \psi(y)))   \\
\text{D2} & \multicolumn{4}{l}{\hol{\mathbf{ess}_{(\mu\ar \sigma)\ar \mu \ar \sigma} = \lambda \phi_{\mu\ar \sigma} \lambdot \lambda X_\mu \lambdot \phi X \,\llift{\wedge}\, \llift{\forall}\psi_{\mu\ar \sigma} \lambdot (\psi X \,\llift{\supset}\,\llift{\Box}   \llift{\forall} Y_\mu \lambdot (\phi Y \,\llift{\supset}\, \psi Y))}} \\
%\text{T2} &  \allq x [G(x) \imp \ess{G}{x}] \\
\text{T2} &  \hol{[\llift{\forall} X_\mu \lambdot g_{\mu\ar \sigma} X \,\llift{\supset}\, (\mathbf{ess}_{(\mu\ar \sigma)\ar \mu \ar \sigma} g X)]} & \text{A1}, \text{D1}, \text{A4}, \text{D2} & \text{K} & \text{THM} 
& 19.1/18.3 & 0.0/0.0 & \text{---/---}\\
  & & \text{A1}, \text{A2}, \text{D1}, \text{A3}, \text{A4}, \text{D2} & \text{K} & \text{THM} 
& 12.9/14.0 & 0.0/0.0 & \text{---/---}\\
%\text{D3} & NE(x) \biimp \allq \phi [\ess{\phi}{x} \imp \nec  \exq y \phi(y)] \\
\text{D3} & \multicolumn{4}{l}{\hol{\mathbf{NE}_{\mu\ar \sigma} = \lambda X_\mu \lambdot  \llift{\forall}\phi_{\mu\ar \sigma} \lambdot (\mathbf{ess}\, \phi X \,\llift{\supset}\, \llift{\Box}  \llift{\exists} Y_\mu \lambdot \phi Y ) }} \\
\text{A5} & \hol{[p_{(\mu\ar \sigma)\ar \sigma} \mathbf{NE}_{\mu\ar \sigma}]} \\
\text{T3} & \hol{[\llift{\Box}  \llift{\exists} X_\mu \lambdot g_{\mu\ar \sigma} X]} 
  &  \text{D1}, \text{C}, \text{T2}, \text{D3}, \text{A5}  & \text{K} & \text{CSA} 
& \text{---/---} & \text{---/---} & 3.8/6.2\\
  & & \text{A1}, \text{A2}, \text{D1}, \text{A3}, \text{A4}, \text{D2}, \text{D3}, \text{A5} & \text{K} & \text{CSA} 
& \text{---/---} & \text{---/---} & 8.2/7.5\\
 & &  \text{D1}, \text{C}, \text{T2}, \text{D3}, \text{A5}  & \text{KB} & \text{THM} & 0.0/0.1 & 0.1/5.3 & \text{---/---}\\
  & & \text{A1}, \text{A2}, \text{D1}, \text{A3}, \text{A4}, \text{D2}, \text{D3}, \text{A5} & \text{KB} & \text{THM} & \text{---/---} & \text{---/---} & \text{---/---}\\ \\
\text{MC} & \hol{[s_\sigma  \,\llift{\supset}\, \llift{\Box}  s_\sigma]} 
   & \text{D2}, \text{T2}, \text{T3} & \text{KB} & \text{THM} & 17.9/\text{---} & 3.3/3.2 & \text{---/---}\\
  & & \text{A1}, \text{A2}, \text{D1}, \text{A3}, \text{A4}, \text{D2}, \text{D3}, \text{A5} & \text{KB} & \text{THM} & \text{---/---} & \text{---/---} & \text{---/---}\\
\text{FG} & \multicolumn{2}{l}{\hol{[\llift{\forall}\phi_{\mu\ar \sigma} \lambdot  \llift{\forall}X_\mu\lambdot  (g_{\mu\ar \sigma} X \,\llift{\supset}\,  (\llift{\neg} (p_{(\mu\ar \sigma)\ar \sigma} \phi) \,\llift{\supset}\, \llift{\neg} (\phi X))) ]} \quad \text{A1}, \text{D1}} & \text{KB} & \text{THM} & 16.5/\text{---} & 0.0/0.0 & \text{---/---} \\
  & & \text{A1}, \text{A2}, \text{D1}, \text{A3}, \text{A4}, \text{D2}, \text{D3}, \text{A5}  
  & \text{KB} & \text{THM} & 12.8/15.1 & 0.0/5.4 & \text{---/---} \\
\text{MT} & \hol{[\llift{\forall}X_\mu\lambdot  \llift{\forall}Y_\mu\lambdot (g_{\mu\ar \sigma} X \,\llift{\supset}\,  (g_{\mu\ar \sigma} Y \,\llift{\supset}\,  X\,\llift{=}\,Y))]} & \text{D1}, \text{FG} & \text{KB} & \text{THM} & \text{---/---}  &  0.0/3.3 & \text{---/---} \\
  & & \text{A1}, \text{A2}, \text{D1}, \text{A3}, \text{A4}, \text{D2}, \text{D3}, \text{A5}  
  & \text{KB} & \text{THM} & \text{---/---}  &  \text{---/---} & \text{---/---} \\ \\
\text{CO} & \emptyset\, \text{(no goal, check for consistency)} & \text{A1}, \text{A2}, \text{D1}, \text{A3}, \text{A4}, \text{D2}, \text{D3}, \text{A5} & \text{KB} & \text{SAT} & \text{---/---} & \text{---/---} & 7.3/7.4 \\ 
% & & \text{A1}, \text{A2}, \text{D1}, \text{A3}, \text{A4}, \text{D2}, \text{D3}, \text{A5} & \text{KB} & \text{SAT} & \text{---/---} & \text{---/---} & \text{---/---} \\ 
\text{D2'} & \multicolumn{4}{l}{\hol{\mathbf{ess}_{(\mu\ar \sigma)\ar \mu \ar \sigma} = \lambda \phi_{\mu\ar \sigma} \lambdot \lambda X_\mu \lambdot \llift{\forall}\psi_{\mu\ar \sigma} \lambdot (\psi X \,\llift{\supset}\,\llift{\Box}   \llift{\forall} Y_\mu \lambdot (\phi Y \,\llift{\supset}\, \psi Y))}} \\
\text{CO'} & \emptyset\, \text{(no goal, check for consistency)} & \text{A1}(\supset), \text{A2}, \text{D2'}, \text{D3}, \text{A5} & \text{KB} & \text{UNS} & 7.5/7.8 & \text{---/---} & \text{---/---} \\ 
 & & \text{A1}, \text{A2}, \text{D1}, \text{A3}, \text{A4}, \text{D2'}, \text{D3}, \text{A5} & \text{KB} & \text{UNS} & \text{---/---} & \text{---/---} & \text{---/---} \\ \\[-.5em]
\end{array}$
\end{minipage}
}
\caption{\HOL encodings and experiment results for Scott's version of G{\"o}del's ontological argument from Fig.~\ref{fig1}. \label{fig2}}
\end{figure*}



%  Figure~\ref{fig2} shows the entire results
% of this experiment. The provers were given a timeout of 60 seconds. The mentioned dependencies for T1 are encoded in
% this table in the
% third column. We also present the results for maximal sets of axioms
% (and theorems) as being accumulated along the proof script. This
% demonstrates the provers robustness and effectiveness when respective
% (ir-)relevance information is still missing and has to be 
% determined first. It was in fact the latter kind of proof problems 
% we started with in our experiments. Then the former dependency information was
% extracted from the proof objects of LEO-II.

% We have additionally encoded some further
% theorems. Moreover, we have added some variations of the embedding in
% which we e.g. consider varying domain semantics as opposed to const
% domain semantics. These encodings are available at
% \url{https://github.com/FormalTheology/GoedelGod/tree/master/Formalizations/THF}.
% A subset of those has also been submitted to the public TPTP problem
% library at \url{www.tptp.org}, (they will be released in summer after
% the CASC-J7 competition).  Our contribution has in fact led to the set
% up of the new TPTP problem category \texttt{PHI} for philosophy
% problems.

\section{MAIN FINDINGS} \label{sec4}
Several interesting and partly novel findings have been contributed by the HOL-ATPs, including:
\begin{enumerate}
\item The axioms and definitions from Fig.~\ref{fig1} are consistent (cf.~CO in Fig.~\ref{fig2}). 
\item Logic K is sufficient for proving T1, C and T2.
\item For proving the final theorem T3, logic KB is sufficient (and
  for K a countermodel is reported). This is highly relevant since
  several philosophers have criticized G{\"o}del's argument for the
  use of logic S5. This criticism is thus provably pointless.
\item Only for T3 the HOL-ATPs still fail to produce a proof directly
  from the axioms; thus, T3 remains an interesting benchmark problem; T1, C, and T2 are rather trivial for HOL-ATPs.
\item G\"odel's original version of the proof \cite{GoedelNotes},
  which omits conjunct $\phi(x)$ in the definition of \emph{essence} (cf.~D2'),
  seems inconsistent (cf.~the failed consistency check for CO' in
  Fig.~\ref{fig2}). As far as we are aware of, this is a new result.
\item G{\"o}del's axioms imply what is called the modal collapse (cf.~MC
  in Fig.~\ref{fig2}) $\phi\supset\Box\phi$, that is, contingent truth
  implies necessary truth (which can even be interpreted as an
  argument against free will; cf.~\cite{sobel2004logic}). MC is probably the most fundamental
  criticism put forward against G{\"o}del's argument.
\item For proving T1, only the $\supset$-direction of A1 is
  needed. However, the $\subset$-direction of A1 is required for
  proving T2. Some philosophers (e.g.~\cite{anderson90:_some_emend_of_goedel_ontol_proof}) try to avoid MC by
  eluding/replacing the $\supset$-direction of A1.
\item G{\"o}del's axioms imply a `flawless God', that is, an entity
  that can only have `positive' properties (cf.~FG in
  Fig.~\ref{fig2}). However, a comment by G{\"o}del in \cite{GoedelNotes}
  explains that `positive' is to be interpreted in a moral aesthetic
  sense only.
% \footnote{In order to better understand G\"odel's notion of
%     `positive' properties, we have reformulated G{\"o}del's theory
%     and used `divine' instead of `positive'. Then we introduced
%     orthogonal predicates `positive' and `negative' and we showed that
%     God-like beings may well have positive and negative properties as
%     long as all these properties are classified as divine properties. A respective
%     formalization in Isabelle can be found at
%     \url{https://github.com/FormalTheology/GoedelGod/blob/master/Formalizations/Isabelle/DivineVersion/GoedelGodDivine.thy}.}
\item Another implication of G{\"o}del's axioms is monotheism (see MT
  in Fig.~\ref{fig2}). MT can easily be proved by Satallax from FG and
  D1. It remains non-trivial to prove it directly from
  G{\"o}del's axioms.
\item \label{vary} All of the above findings hold for both constant domain semantics and varying domain semantics (for the domain of individuals).
\end{enumerate}

The above findings, in particular \eqref{vary}, well illustrate that
the theory framework from Sec.~\ref{sec2} has a great potential
towards a flexible support system for \emph{computational theoretical
  philosophy}. In fact, G\"odel's ontological argument has been
verified and even automated not only for one particular setting of
logic parameters, but these logic parameters have been varied and the
validity of the argument has been reconfirmed (or falsified, cf.~D2'
and CO') for the modified setting.  Moreover, our framework is not
restricted to a particular theorem proving system, but has been
fruitfully employed with some of the most prominent automated and
interactive theorem provers available to date.



\section{RELATED AND FUTURE WORK}\label{sec5}
We are pioneering the computer-supported automation of modern versions
of the ontological argument. There are two related papers
\cite{oppenheimera11,rushby13}. Both focus on the comparably simpler
argument by Anselm. \cite{oppenheimera11} encodes (a variant) of
Anselm's argument in first-order logic and employs the theorem prover
PROVER9 in experiments; this work has been criticized in
\cite{garbacz12:_prover_simpl_expal_away}. The work in
\cite{rushby13}, which has evolved in parallel to ours, interactively
verifies Anselm's argument in the higher-order proof assistant
PVS. Note in particular, that both formalizations do not achieve the
close correspondence between the original formulations and the formal
encodings that can be found in our approach.

A particular strength of our universal logic framework is that it can
be easily adapted for logic variations and even supports flexible
combinations of logics (cf.~\cite{C36}). In ongoing and future work we
will therefore investigate further logic parameters for G{\"o}del's
argument, including varying domains at higher types and non-rigid
terms. We plan to make the entire landscape of results available to
the interested
%theoretical philosophy and theology 
communities. This is relevant, since philosophers are sometimes
imprecise about the very details of the logics they employ.


% the main activity of our project will be the use of state-of-the-art automated deduction tools and proof assistants to represent, verify and strengthen arguments about God’s existence.

\section{CONCLUSION} \label{sec6} While computers can now calculate,
play games, translate, plan, learn and classify data much better than
we humans do, tasks involving philosophical and theological inquiries
have remained mostly untouched by our technological progress up to
now. Due to the abstract and sophisticated types of reasoning they
require, they can be considered a challenging frontier for automated
reasoning.  

We accepted this challenge and decided to tackle, with automated
reasoning techniques, a philosophical problem that is almost 1000
years old: the ontological argument for God's existence, firstly
proposed by St. Anselm of Canterbury and greatly improved by
Descartes, Leibniz, G{\"o}del and many others throughout the
centuries.  % G{\"o}del's modern version of this argument can be
% formalized in a very expressive higher-order modal logic.
So far,
there was no AI system capable of dealing with
such complex problems. We created a prototypical infrastructure
extending widely used systems such as LEO-II, Satallax, and Nitpick
(and Isabelle and Coq) to allow them to cope with modalities; and
using the extended systems we were able to automatically reconstruct
and verify G{\"o}del's argument, as well as discover new facts and
confirm controversial claims about it.  This is a landmark result,
with media repercussion in a global scale, and yet it is only a
glimpse of what can be achieved by combining computer science,
philosophy and theology. 

Our work, in this sense, offers new perspectives for a
computational theoretical philosophy. The critical discussion of
the underlying concepts, definitions and axioms remains a human
responsibility, but the computer can assist in building and checking
rigorously correct logical arguments. In case of logico-philosophical
disputes, the computer can check the disputing arguments and partially
fulfill Leibniz' dictum: Calculemus --- Let us calculate!


%\paragraph{Acknowledgments}
\section*{ACKNOWLEDGEMENTS}
 We thank Alexander Steen,
 Max Wisniewski, and the anonymous reviewers for their comments and suggestions.

\bibliographystyle{ecai2014}
%\bibliography{Bibliography}
\begin{thebibliography}{10}

\bibitem{Adams}
R.M. Adams, `Introductory note to *1970', in {\em {Kurt G\"odel: Collected
  Works Vol. 3: Unpubl. Essays and Letters}}, Oxford Univ. Press, (1995).

\bibitem{AndersonGettings}
A.C. Anderson and M.~Gettings, `G\"odel ontological proof revisited', in {\em
  {G\"odel'96: Logical Foundations of Mathematics, Computer Science, and
  Physics: Lecture Notes in Logic 6}},  167--172, {Springer}, (1996).

\bibitem{anderson90:_some_emend_of_goedel_ontol_proof}
C.A. Anderson, `Some emendations of {G{\"o}del's} ontological proof', {\em
  Faith and Philosophy}, {\bf 7}(3), (1990).

\bibitem{Andrews:gmae72}
P.B. Andrews, `General models and extensionality', {\em Journal of Symbolic
  Logic}, {\bf 37}(2),  395--397, (1972).

\bibitem{andrewsSEP}
P.B. Andrews, `Church's type theory', in {\em The Stanford Encyclopedia of
  Philosophy}, ed., E.N. Zalta, spring 2014 edn., (2014).

\bibitem{C36}
C.~Benzm{\"u}ller, `{HOL} based universal reasoning', in {\em Handbook of the
  4th World Congress and School on Universal Logic}, ed., J.Y. Beziau~et al.,
  pp. 232--233, Rio de Janeiro, Brazil, (2013).

\bibitem{B5}
C.~Benzm{\"u}ller and D.~Miller, `Automation of higher-order logic', in {\em
  Handbook of the History of Logic, Volume 9 --- Logic and Computation},
  Elsevier, (2014).
\newblock Forthcoming; preliminary version available at
  {http://christoph-benzmueller.de/papers/B5.pdf}.

\bibitem{C34}
C.~Benzm{\"u}ller, J.~Otten, and Th. Raths, `Implementing and evaluating
  provers for first-order modal logics', in {\em Proc. of the 20th European
  Conference on Artificial Intelligence (ECAI)}, pp. 163--168, (2012).

\bibitem{B9}
C.~Benzm{\"u}ller and L.C. Paulson, `Exploring properties of normal multimodal
  logics in simple type theory with {LEO-II}', in {\em {Festschrift in Honor of
  {Peter B. Andrews} on His 70th Birthday}}, ed., C.~Benzm{\"u}ller~et al.,
  386--406, College Publications, (2008).

\bibitem{J23}
C.~Benzm{\"u}ller and L.C. Paulson, `Quantified multimodal logics in simple
  type theory', {\em Logica Universalis}, {\bf 7}(1),  7--20, (2013).

\bibitem{LEO-II}
C.~Benzm{\"u}ller, F.~Theiss, L.~Paulson, and A.~Fietzke, `{LEO-II} - a
  cooperative automatic theorem prover for higher-order logic', in {\em
  Proc.~of IJCAR 2008}, number 5195 in LNAI, pp. 162--170. Springer, (2008).

\bibitem{J30}
C.~Benzm{\"u}ller and B.~Woltzenlogel-Paleo, `Formalization, mechanization and
  automation of {G{\"o}del's} proof of {God's} existence', {\em
  arXiv:1308.4526}, (2013).

\bibitem{J28}
C.~Benzm\"uller and B.~Woltzenlogel-Paleo, `{G{\"o}del's God in Isabelle/HOL}',
  {\em Archive of Formal Proofs}, (2013).

\bibitem{W50}
C.~Benzm\"uller and B.~Woltzenlogel-Paleo, `G\"odel's {God} on the computer',
  in {\em Proceedings of the 10th International Workshop on the Implementation
  of Logics}, EPiC Series. EasyChair, (2013).
\newblock Invited abstract.

\bibitem{Coq}
Y.~Bertot and P.~Casteran, {\em {Interactive Theorem Proving and Program
  Development}}, Springer, 2004.

\bibitem{Nitpick}
J.C. Blanchette and T.~Nipkow, `Nitpick: A counterexample generator for
  higher-order logic based on a relational model finder', in {\em Proc. of ITP
  2010}, number 6172 in LNCS, pp. 131--146. Springer, (2010).

\bibitem{Satallax}
C.E. Brown, `Satallax: An automated higher-order prover', in {\em Proc. of
  IJCAR 2012}, number 7364 in LNAI, pp. 111 -- 117. Springer, (2012).

\bibitem{ContemporaryBibliography}
R.~Corazzon.
\newblock Contemporary~bibliography~on~ontological~arguments: {\scriptsize
  \url{http://www.ontology.co/biblio/ontological-proof-contemporary-biblio.htm}}.

\bibitem{Fitting}
M.~Fitting, {\em Types, Tableaux and G\"odel's God}, Kluwer, 2002.

\bibitem{fitting98}
M.~Fitting and R.L. Mendelsohn, {\em First-Order Modal Logic}, volume 277 of
  {\em Synthese Library}, Kluwer, 1998.

\bibitem{Gallin75}
D.~Gallin, {\em Intensional and Higher-Order Modal Logic}, North-Holland, 1975.

\bibitem{garbacz12:_prover_simpl_expal_away}
P.~Garbacz, `{PROVER9's} simplifications explained away', {\em Australasian
  Journal of Philosophy}, {\bf 90}(3),  585--592, (2012).

\bibitem{GoedelNotes}
K.~G\"odel, {\em Appx.A: Notes in Kurt G\"odel's Hand},  144--145.
\newblock In  \cite{sobel2004logic}, 2004.

\bibitem{Henkin50}
L.~Henkin, `Completeness in the theory of types', {\em Journal of Symbolic
  Logic}, {\bf 15}(2),  81--91, (1950).

\bibitem{homl}
R.~Muskens, `{Higher Order Modal Logic}', in {\em Handbook of Modal Logic},
  ed., P~Blackburn~et al.,  621--653, Elsevier, Dordrecht, (2006).

\bibitem{Isabelle}
T.~Nipkow, L.C. Paulson, and M.~Wenzel, {\em {Isabelle/HOL: A Proof Assistant
  for Higher-Order Logic}}, number 2283 in LNCS, Springer, 2002.

\bibitem{oppenheimera11}
P.E. Oppenheimera and E.N. Zalta, `A computationally-discovered simplification
  of the ontological argument', {\em Australasian Journal of Philosophy}, {\bf
  89}(2),  333--349, (2011).

\bibitem{rushby13}
J.~Rushby, `The ontological argument in {PVS}', in {\em Proc.~of CAV Workshop
  ``Fun With Formal Methods''}, St. Petersburg, Russia,, (2013).

\bibitem{Schulz:AICOM-2002}
S.~Schulz, `E -- a brainiac theorem prover', {\em {AI Communications}}, {\bf
  15}(2),  111--126, (2002).

\bibitem{ScottNotes}
D.~Scott, {\em Appx.B: Notes in Dana Scott's Hand},  145--146.
\newblock In  \cite{sobel2004logic}, 2004.

\bibitem{sobel2004logic}
J.H. Sobel, {\em Logic and Theism: Arguments for and Against Beliefs in God},
  Cambridge U. Press, 2004.

\bibitem{sutcliffe2009tptp}
G.~Sutcliffe, `The {TPTP} problem library and associated infrastructure', {\em
  Journal of Automated Reasoning}, {\bf 43}(4),  337--362, (2009).

\bibitem{J22}
G.~Sutcliffe and C.~Benzm{\"u}ller, `Automated reasoning in higher-order logic
  using the {TPTP THF} infrastructure.', {\em Journal of Formalized Reasoning},
  {\bf 3}(1),  1--27, (2010).

\bibitem{J29}
B.~Woltzenlogel-Paleo and C.~Benzm{\"u}ller, `Automated verification and
  reconstruction of {G\"odel's} proof of {God's} existence', {\em OCG J.},
  (2013).

\end{thebibliography}

\end{document}
%%%%%%%%%%%%%%%%%%%%%%%%%%%%%%%%%%%%%%%%%%%%%%%%%%%%%%%%%%%%%%%%%%%%%%

% This is LLNCS.DEM the demonstration file of
% the LaTeX macro package from Springer-Verlag
% for Lecture Notes in Computer Science,
% version 2.4 for LaTeX2e as of 16. April 2010
%
\documentclass{llncs}
%

\usepackage[utf8]{inputenc}
\usepackage{fancyvrb,amssymb,graphicx}
\usepackage[usenames,dvipsnames,svgnames,table]{xcolor}




\begin{document}

\title{On Logic Embeddings and G\"odel's God}
\author{Christoph Benzm\"uller\inst{1}\thanks{This work has been supported by
    the German Research Foundation DFG under grants BE2501/9-1,2 and
    BE2501/11-1.} \and Bruno Woltzenlogel Paleo\inst{2} 
}

\institute{%Department of Mathematics and Computer Science \\ 
 Freie Universit\"at Berlin, Germany \\ \url{c.benzmueller@fu-berlin.de}
\and
 Vienna Technical University, Austria \\\url{bruno@logic.at}}

\maketitle            



\begin{abstract}
  We have applied an elegant and flexible  logic embedding
  approach to verify and automate a prominent philosophical argument:
  the ontological argument for the existence of God. In our ongoing
  computer-assisted study, higher-order automated reasoning tools have
  made some interesting observations, some of which were previously unknown.
\end{abstract}

%   Our work attests the maturity of contemporary interactive and
%   automated deduction tools for classical higher-order logic and
%   demonstrates the elegance and practical relevance of the
%   embeddings-based approach. Most importantly, our work opens new
%   perspectives towards a computational metaphysics.

%   In addition to the above, I will also briefly discuss the relation
%   of the logic embeddings approach to axiomatic and algebraic
%   specifications

%   The study of G\"odel's ontological proof of God's existence is joint
%   work with Bruno Woltzenlogel Paleo.
% \end{abstract}
%
% \section{Introduction}
%

Logic embeddings provide an elegant means to formalize sophisticated
non-classical logics in classical higher-order logic (HOL, Church's
simple type theory \cite{Church40}). In previous work (cf.~\cite{C35}
and the references therein) the embeddings approach has been
successfully applied to automate object-level and meta-level reasoning
for a range of logics and logic combinations with off-the-shelf HOL
theorem provers. This also includes quantified modal logics (QML)
\cite{J23} and quantified conditional logics (QCL) \cite{C37}.  For
many of the embedded logics few or no automated theorem provers did
exist before. HOL is exploited in this approach to encode the
semantics of the logics to be embedded, for example, Kripke semantics
for QMLs \cite{fitting98} or selection function semantics for QCLs
\cite{Stalnaker68}.  

The embeddings approach is related to labelled deductive systems
\cite{gabbay96}, which employ meta-level (world-)labeling techniques
for the modeling and implementation of non-classical proof systems. In
our embeddings approach such labels are instead encoded in the HOL
logic. 

The embedding approach is flexible, because various modal logics (even
with multiple modalities or a mix of varying/cumulative domain
quantifiers) can be easily supported by stating their characteristic
axioms. Moreover, it is relatively simple to implement, because it
does not require any modification in the source code of the
higher-order prover. A minimal encoding of
second-order modal logic KB in TPTP THF syntax \cite{J22} --- this syntax is
accepted by a range of HOL automated theorem provers (ATPs) --- is
exemplarily provided in Fig.~\ref{fig1}.\footnote{Some Notes on THF, which is a concrete syntax for HOL: \texttt{\$i} and
                   \texttt{\$o} represent the HOL base types $i$ and
                   $o$ (Booleans). \texttt{\$i>\$o} encodes a function
                   (predicate) type.  Predicate
                   application as in $A(X,W)$ is encoded as
                   \texttt{((A@X)@W)} or simply as
                   \texttt{(A@X@W)}, i.e., function/predicate application
                   is represented by \texttt{@}; universal
                   quantification and $\lambda$-abstraction as in
                   $\lambda {A}_{i\rightarrow o} \forall {W_i} (A\,W)$
                   and are represented as in
                   \texttt{\^{}[X:\$i>\$o]:![W:\$i]:(A@W)}; comments begin with
                   \texttt{\%}.} The given set of axioms turns
any TPTP THF compliant HOL-ATP in a reasoning tool for second-order modal logic. A
Henkin-style semantics is thereby assumed for both logics: HOL and
second-order modal logic.


\begin{figure}[t]
\begin{flushright}
\begin{minipage}{.97\textwidth}
\VerbatimInput[frame=single,%
               commandchars=\\\{\},%
               fontfamily=courier,
               fontseries=b,%
               fontsize=\scriptsize,%
               %rulecolor=green,%
               %fillcolor=yellow,%
               %formatcom=blue,%
               framerule=1pt,%
               framesep=5pt,
               numbersep=5pt,
               numbers=left]%
               {Quantified_KB.ax}
\end{minipage}
\end{flushright}
                \vskip-.5em
               \caption{\label{fig1}HOL encoding of second-order modal
                 logic KB in THF syntax. 
                 Modal formulas are mapped to HOL 
                 predicates (with type \texttt{\$i>\$o}); 
                 type \texttt{\$i} now stands for possible worlds. 
                 The modal connectives $\neg$ (\texttt{mnot}), $\vee$ (\texttt{mor}) and
                 $\square$ (\texttt{mbox}), universal quantification for
                 individuals (\texttt{mall\_ind}) and for sets of 
                 individuals  (\texttt{mall\_indset}) are introduced in 
                 lines 7-18. 
                 Validity of lifted modal formulas is
                 defined in the standard way (lines 20-21). Symmetry
                 of accessibility relation $r$ is postulated in lines
                 23-26. Hence, second-order KB is realized here; 
                 for logic K the symmetry axiom can be dropped.
                 % \texttt{?}
                   % is the existential quantifier, and
                   % $\neg,\vee,\wedge, and \Rightarrow$ (mat. impl.)
                   % are written as \texttt{\url{~}}, \texttt{|},
                   % \texttt{\&}, and \texttt{=>}. 
                   % Better formatted and easier readable
                   % presentations of our THF code can easily be
                   % generated with the TPTP tools of
                   % Sutcliffe;
                   % here we did minimize space.
                 }
\end{figure}

In recent work \cite{C40,J28,CSR} we have applied the embedding approach
to verify and automate a philosophical argument that has fascinated
philosophers and theologians for about 1000 years: the ontological
argument for the existence of God~\cite{sobel2004logic}. We have
thereby concentrated on G\"odel's \cite{GoedelNotes} 
modern version of this argument and on
Scott's \cite{ScottNotes} modification, which
employ a second-order modal logic (S5) for which, until now, no
theorem provers were available.  In our computer-assisted study of the
argument, the HOL provers LEO-II \cite{C26}, Satallax \cite{Satallax}
and Nitpick \cite{Nitpick} have made some interesting observations,
some of which were unknown so far. This is a landmark result, with
media repercussion in a global scale, and yet it is only a glimpse of
what can be achieved by combining computer science, philosophy and
theology.


We briefly summarize some of these observations: Nitpick confirms that
Scott's axioms are consistent, while LEO-II and Satallax demonstrate
that Gödel's original, slightly different axioms are inconsistent. As
far as we are aware, this is a new result. As experiments with LEO-II
revealed, the problem lies in a subtle difference in the definitions
of the predicate \textit{essence} (characterizing the essential
properties of an entity) between Gödel and Scott. In recent papers on
the ontological argument (see e.g. below), some authors speak of an oversight/flaw by
Gödel, some silently replace Gödel's definition without commenting and
some simply stay with it.
Moreover, instead of using modal logic S5, LEO-II and Satallax can
prove the final theorem (that is, $\square \exists x . G(x)$,
necessarily there exists God) already for modal logic KB.  This is
highly relevant since some philosophers have criticized G{\"o}del's
argument for the use of logic S5.  Axiom B (symmetry), however, cannot be dropped,
which in turn is confirmed by Nitpick.  LEO-II and Satallax can also
show that G{\"o}del's and Scott's axioms imply what is called the
modal collapse: $\phi\supset\Box\phi$. This expresses that contingent
truth implies necessary truth (which can even be interpreted as an
argument against free will; cf.~\cite{sobel2004logic}) and is probably
the most fundamental criticism put forward against G{\"o}del's and
Scott's versions of the argument. Other theorems that can be shown by
LEO-II and Satallax include flawlessness of God and 
monotheism.




% \item The axioms and definitions from Fig.~\ref{fig1} are consistent (cf.~CO in Fig.~\ref{fig2}). 
% \item Logic K is sufficient for proving T1, C and T2.
% \item For proving the final theorem T3, logic KB is sufficient (and
%   for K a countermodel is reported). This is highly relevant since
%   several philosophers have criticized G{\"o}del's argument for the
%   use of logic S5. This criticism is thus provably pointless.
% \item Only for T3 the HOL-ATPs still fail to produce a proof directly
%   from the axioms; thus, T3 remains an interesting benchmark problem; T1, C, and T2 are rather trivial for HOL-ATPs.
% \item G\"odel's original version of the proof \cite{GoedelNotes},
%   which omits conjunct $\phi(x)$ in the definition of \emph{essence} (cf.~D2'),
%   seems inconsistent (cf.~the failed consistency check for CO' in
%   Fig.~\ref{fig2}). As far as we are aware of, this is a new result.
% \item G{\"o}del's axioms imply what is called the modal collapse (cf.~MC
%   in Fig.~\ref{fig2}) $\phi\supset\Box\phi$, that is, contingent truth
%   implies necessary truth (which can even be interpreted as an
%   argument against free will; cf.~\cite{sobel2004logic}). MC is probably the most fundamental
%   criticism put forward against G{\"o}del's argument.
% \item For proving T1, only the $\supset$-direction of A1 is
%   needed. However, the $\subset$-direction of A1 is required for
%   proving T2. Some philosophers (e.g.~\cite{anderson90:_some_emend_of_goedel_ontol_proof}) try to avoid MC by
%   eluding/replacing the $\supset$-direction of A1.
% \item G{\"o}del's axioms imply a `flawless God', that is, an entity
%   that can only have `positive' properties (cf.~FG in
%   Fig.~\ref{fig2}). However, a comment by G{\"o}del in \cite{GoedelNotes}
%   explains that `positive' is to be interpreted in a moral aesthetic
%   sense only.


% \footnote{In order to better understand G\"odel's notion of
%     `positive' properties, we have reformulated G{\"o}del's theory
%     and used `divine' instead of `positive'. Then we introduced
%     orthogonal predicates `positive' and `negative' and we showed that
%     God-like beings may well have positive and negative properties as
%     long as all these properties are classified as divine properties. A respective
%     formalization in Isabelle can be found at
%     \url{https://github.com/FormalTheology/GoedelGod/blob/master/Formalizations/Isabelle/DivineVersion/GoedelGodDivine.thy}.}
% \item Another implication of G{\"o}del's axioms is monotheism (see MT
%   in Fig.~\ref{fig2}). MT can easily be proved by Satallax from FG and
%   D1. It remains non-trivial to prove it directly from
%   G{\"o}del's axioms.
% \item \label{vary} All of the above findings hold for both constant domain semantics and varying domain semantics (for the domain of individuals).
% \end{enumerate}

Ongoing and future work concentrates on the systematic study of
G\"odel's and Scott's proofs. We have also begun to study more recent
variants of the argument
\cite{anderson90:_some_emend_of_goedel_ontol_proof,AndersonGettings,bjordal99,fuhrmann05:_exist_notwen,fitting02:_types_tableaus_god,Hajek2002,Hajek2008},
which claim to remedy some fundamental problems of G\"odel's and
Scott's proofs, especially the modal collapse \cite{C41}.  One interesting
and very encouraging observation from these studies is, that the
argumentation granularity typical of these philosophy
papers is already within reach of the capabilities 
of our higher-order automated theorem provers.
This provides good evidence for the potential 
relevance of the embedding approach (not only) w.r.t. 
other similar applications in metaphysics.


The long-term goal
is to methodically determine the range of logical parameters (e.g.,
constant vs. varying domains, rigid vs. non-rigid terms, logics KB vs.
S4 vs. S5, etc.) under which the proposed variants of the modern
ontological argument hold or fail.

There have been few related works~\cite{oppenheimera11,rushby13}, 
and they have
focused solely on the comparably simpler, original ontological
argument by Anselm of Canterbury. These works do not achieve the
close correspondence between the original formulations and the formal
encodings that can be found in our approach and they also do not reach
the same degree of proof automation.

Our work attests the maturity of contemporary interactive and
automated deduction tools for HOL and
demonstrates the elegance and practical relevance of the
embeddings-based approach. Most importantly, our work opens new
perspectives towards computational metaphysics.




% \paragraph{Acknowledgement:}
%   The study of G\"odel's ontological proof of God's existence is joint
%   work with Bruno Woltzenlogel Paleo.

\bibliographystyle{plain}
%\bibliography{chris,Bibliography}


\begin{thebibliography}{10}

\bibitem{anderson90:_some_emend_of_goedel_ontol_proof}
C.A.~Anderson.
\newblock Some emendations of {G{\"o}del's} ontological proof.
\newblock {\em Faith and Philosophy}, 7(3), 1990.


\bibitem{AndersonGettings}
C.A.~Anderson and M.~Gettings.
\newblock G\"odel ontological proof revisited.
\newblock In {\em {G\"odel'96: Logical Foundations of Mathematics, Computer
  Science, and Physics: Lecture Notes in Logic 6}}, pages 167--172. {Springer},
  1996.


\bibitem{C37}
C. Benzm{\"u}ller.
\newblock Automating quantified conditional logics in {HOL}.
\newblock In F.~Rossi, editor, Proc. of {\em IJCAI 2013}, pages 746--753, Beijing, China, 2013.

\bibitem{C35}
C. Benzm{\"u}ller.
\newblock A top-down approach to combining logics.
\newblock In {\em Proc. of ICAART 2013}, pages 346--351, Barcelona, Spain, 2013.
  SciTePress Digital Library.

\bibitem{J28}
C. Benzm\"uller and B.~Woltzenlogel Paleo.
\newblock {G{\"o}del's God in Isabelle/HOL}.
\newblock {\em Archive of Formal Proofs}, 2013.

\bibitem{C40}
C. Benzm{\"u}ller and B.~Woltzenlogel Paleo.
\newblock Automating {G\"{o}del's} ontological proof of god's existence with
  higher-order automated theorem provers.
\newblock In 
  {\em ECAI 2014}, volume 263 of {\em Frontiers in AI and
  Applications}, pages 163 -- 168. IOS Press, 2014.

\bibitem{C41}
C. Benzm{\"u}ller and L. Weber and B. Woltzenlogel
Paleo. Computer-Assisted Analysis of the {Anderson-H\'{a}jek}
Ontological Controversy. Handbook of the 1st World Congress on Logic
and Religion, Jo\~ao Pessoa, Brazil, 2015.

\bibitem{CSR}
C.~Benzm{\"u}ller and B. Woltzenlogel Paleo.
\newblock Interacting with Modal Logics in the Coq Proof Assistant,
{\em 10th Intl. Computer Science Symp. in Russia (CSR)}, 2015.

\bibitem{J23}
C. Benzm{\"u}ller and L. Paulson.
\newblock Quantified multimodal logics in simple type theory.
\newblock {\em Logica Universalis (Special Issue on Multimodal Logics)},
  7(1):7--20, 2013.

\bibitem{C26}
C. Benzm{\"u}ller, F. Theiss, L. Paulson, and A. Fietzke.
\newblock {LEO-II} - a cooperative automatic theorem prover for higher-order
  logic (system description).
\newblock In 
  {\em Proc. of IJCAR 2008}, volume 5195 of {\em
  LNCS}, pages 162--170. Springer, 2008.

\bibitem{bjordal99}
F. Bjørdal.
\newblock Understanding gödel’s ontological argument.
\newblock In T. Childers, editor, {\em The Logica Yearbook 1998}.
  Filosofia, 1999.

\bibitem{Nitpick}
J.C. Blanchette and T.~Nipkow.
\newblock Nitpick: A counterexample generator for
  higher-order logic based on a relational model finder. In {\em Proc. of ITP
  2010}, volume 6172 of LNCS, pp. 131--146. Springer, (2010).

\bibitem{Satallax}
C.E. Brown.
\newblock Satallax: An automated higher-order prover.
\newblock In {\em Proc. of IJCAR 2012}, number 7364 in LNAI, pages 111 -- 117.
  Springer, 2012.


\bibitem{Church40}
A.~Church.
\newblock A formulation of the simple theory of types.
\newblock {\em Journal of Symbolic Logic}, 5:56--68, 1940.

\bibitem{fitting98}
M.~Fitting and R.L. Mendelsohn.
\newblock {\em First-Order Modal Logic}, volume 277 of {\em Synthese Library}.
\newblock Kluwer, 1998.

\bibitem{fitting02:_types_tableaus_god}
M.~Fitting.
\newblock {\em Types, Tableaus, and {G}{\"o}del's God}.
\newblock Kluwer, 2002.

\bibitem{fuhrmann05:_exist_notwen}
A.~Fuhrmann.
\newblock {Existenz und Notwendigkeit --- Kurt G\"odels axiomatische
  Theologie}.
\newblock In W.~Spohn et~al., editor, {\em Logik in der Philosophie}.
  Heidelberg (Synchron), 2005.

\bibitem{gabbay96}
D.M. Gabbay.
\newblock {\em Labelled Deductive Systems}.
\newblock Clarendon Press, 1996.

\bibitem{GoedelNotes}
K.~G\"odel.
\newblock {\em Appx.A: Notes in Kurt G\"odel's Hand}, pages 144--145.
\newblock In  \cite{sobel2004logic}, 2004.

\bibitem{Hajek2002}
P.~Hajek.
\newblock A new small emendation of g\"odel's ontological proof.
\newblock {\em Studia Logica: An International Journal for Symbolic Logic},
  71(2):pp. 149-164, 2002.

\bibitem{Hajek2008}
P.~Hajek.
\newblock Ontological proofs of existence and non-existence.
\newblock {\em Studia Logica: An International Journal for Symbolic Logic},
  90(2):pp. 257--262, 2008.

\bibitem{oppenheimera11}
P.E. Oppenheimera and E.N. Zalta.
\newblock A computationally-discovered simplification of the ontological
  argument.
\newblock {\em Australasian J. of Philosophy}, 89(2):333--349, 2011.

\bibitem{rushby13}
J.~Rushby.
\newblock The ontological argument in {PVS}.
\newblock In {\em Proc.~of CAV Workshop ``Fun With Formal Methods''}, St.
  Petersburg, Russia,, 2013.

\bibitem{ScottNotes}
D.~Scott.
\newblock {\em Appx.B: Notes in Dana Scott's Hand}, pages 145--146.
\newblock In  \cite{sobel2004logic}, 2004.

\bibitem{sobel2004logic}
J.H. Sobel.
\newblock {\em Logic and Theism: Arguments for and Against Beliefs in God}.
\newblock Cambridge U. Press, 2004.

\bibitem{Stalnaker68}
R.C.~Stalnaker.
\newblock A theory of conditionals.
\newblock In {\em Studies in Logical Theory}, pages
98--112. Blackwell, 1968.

\bibitem{J22}
G.~Sutcliffe and C.~Benzm{\"u}ller.
\newblock Automated reasoning in higher-order logic
  using the {TPTP THF} infrastructure, {\em J. of Formalized Reasoning},
  3(1),  1--27, (2010).



\end{thebibliography}



\end{document}


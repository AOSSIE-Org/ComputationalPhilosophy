\documentclass [smallextended] {svjour3} [natbib]

%\usepackage{setspace}
%\doublespacing

\usepackage[utf8]{inputenc}
\usepackage{fancybox}
\usepackage{latexsym}
\usepackage{natbib}



\usepackage{proof}
\usepackage{bussproofs}
\EnableBpAbbreviations
\newcommand{\rl}[1]{\RightLabel{#1}}

\usepackage{amsmath}
\usepackage{ntheorem}

\usepackage[margin=3.5cm]{geometry}

\usepackage{calculi}
\usepackage{theorems}
\usepackage[numbers]{natbib}

\pagestyle{plain}


% Logical symbols
\newcommand{\imp}{\rightarrow}
\newcommand{\biimp}{\leftrightarrow}
\newcommand{\all}{\forall}
\newcommand{\ex}{\exists}
\newcommand{\seq}{\vdash}
\newcommand{\nec}{\Box} % necessarily
\newcommand{\pos}{\Diamond} % possibly


\newtheorem*{lemma*}{Lemma}

\title{A Variant of G\"{o}del's Ontological Proof \\
in a Natural Deduction Calculus}


\author{Annika Siders \and Bruno Woltzenlogel Paleo}

\authorrunning{A.\~Siders \and B.\~Woltzenlogel Paleo}

\institute{ 
 Annika Siders \at 
  Department of Philosophy \\
  University of Helsinki\\
  P.O. Box 24 (Unioninkatu 40 A) \\
  Finland \\ 
  \email{annika.siders@helsinki.fi} \\
  +358 2941 29241[2em]
  \and 
  Bruno Woltzenlogel Paleo \at 
  Theory and Logic Group \\ 
  Vienna University of Technology \\ 
  Austria \\
   \email{bruno@logic.at} \\
  \and
    Bruno Woltzenlogel Paleo \at 
  Logic and Computation Group \\
  College of Engineering and Computer Science \\
  Australian National University \\
  Australia
    \email{bruno.woltzenlogel.paleo@anu.edu.au} 
}





\begin{document}

%\maketitle

%\clearpage

%\title{A Variant of G\"{o}del's Ontological Proof \\ in a Natural Deduction Calculus}
\author{}
\authorrunning{}
\institute{}
\maketitle


\begin{abstract}
This paper presents two detailed formalizations of ontological arguments in a simple natural deduction calculus. The first formal proof closely follows the hints in Scott's manuscript about G\"odel's argument and fills in the gaps, thus verifying its correctness. The second formal proof improves the first one, by relying on the weaker modal logic {\bf KB} instead of the stronger modal logic {\bf S5}; and by avoiding the introduction of the equality predicate. The second proof is also technically shorter than the first one, because it eliminates unnecessary detours and uses Axiom 1 for the positivity of properties only once.
\end{abstract}

\noindent
\textbf{Keywords: } G\"odel's Ontological Proof, Higher-Order Modal Logics, Natural Deduction, 


\newcommand{\ess}[2]{#1 \ \mathit{ess} \ #2}
\newcommand{\NE}{E}


\noindent
\begin{footnotesize}
\begin{center}
``There is a scientific (exact) philosophy and theology,
which deals with concepts of the highest abstractness; and this is also most highly fruitful for science. [\ldots] \\Religions are, for the most part, bad; but religion is not.'' \\
- Kurt G\"{o}del \citep{Wang1996}[p. 316]
\end{center}
\end{footnotesize}



\section{Introduction}

Ontological arguments for the existence of God can be traced back at least to St. Anselm (1033-1109). His argument considers a greatest conceivable being, who must exist, because if it did not have the property of existence, then we could conceive of a greater being that, apart from the other properties, also had the property of existence. St. Anselm's argument was further elaborated by Descartes, Leibniz and Kant. 

Leibniz identified the possible existence of God as a critical missing step in St. Anselm's argument. To fill this gap, he argued that the properties of God, the perfections, are compatible. This means that it is possible to satisfy all perfections at once, which implies that the existence of a greatest conceivable being with all these properties is possible. 

G\"odel bulit on Leibniz's work \citep{adams} and brought the ontological argument to a modern form using a modal logic with higher-order quantification over properties. In this setting, he gave precise axioms describing the notion of \emph{positive} property and defined God as a being that has all positive properties. G\"odel's work was saved in his own notes \citep{Goedel} as well as in notes by Scott \citep{scott}, in whom he confided his proof. 

The increase in formality of the ontological argument has required a development of its basic notions. G\"odel's notion of positive property and Leibniz's notion of perfection differ. A formal distinction is that Leibniz's perfections are atomic whereas G\"odel's positive properties can consist of combinations of atomic properties \citep{fitting}[p.139]. In particular, one of G\"odel's axioms states that any conjunction of positive properties is itself positive. From this axiom, it is immediately deduced that the property of being God-like is positive. Intuitively, a (possibly infinite) conjunction of positive properties is deduced from the universal definition of God-likeness. This deductive inference is not formalizable in a finite first-order calculus. The interplay between universal quantification (in the definition of a God-like being) and infinite conjunctions (in G\"odel's axiom for positive properties) could explain why, starting with Scott \citep{scott}, this axiom of G\"odel has been replaced by another that simply assumes the positivity of the property of being god-like. 

The aim of this paper is to present two detailed formalizations of ontological arguments in a natural deduction calculus. For a comprehensive introduction to natural deduction, the reader can consult \citep{prawitz}. The higher-order modal natural deduction calculus proposed here has special introduction and elimination rules for modalities, as defined in Section \ref{sec:calculus}, and is sound and complete relative to an axiomatic modal calculus, as proven in Section \ref{sec:compl-sound}.

The natural deduction style was chosen for three reasons. Firstly, presentations of G\"odel's proof are typically either informal or formalized in other styles of proof calculi (e.g. Fitting's tableaux \citep{fitting} or Sobel's sentential modal calculus \citep{sobel2}). Therefore, a formalization in natural deduction is a valuable complement to the existing presentations. Secondly, it makes the ontological proof accessible to people who are familiar with a natural deduction style. Thirdly, as natural deduction is the style used by proof assistants such as Coq \citep{coq} and Isabelle \citep{isabelle}, natural deduction formalizations can be verified step-by-step in such proof assistants, and we have in fact done this.

The first contribution of this paper is a detailed formalization of Scott's version \citep{scott} of G\"odel's ontological argument \citep{Goedel} (as shown in Section \ref{sec:scott}) in the proposed natural deduction calculus. The second contribution of the paper is a new proof (also in natural deduction style). In contrast to Scott's proof \citep{scott}, which requires the modal logic \textbf{S5}, the new proof requires only the weaker modal logic \textbf{KB}. The new proof also does not rely on the equality predicate and is much shorter.

A major criticism against G\"odel's formal argument is an undesirable consequence of the stipulated axioms, called \emph{modal collapse}. This is discussed in greater detail in Section \ref{sec:collapse}, where a natural deduction derivation of the collapse is presented. Although many recent works on the ontological argument have proposed modifications of the argument that do not entail a modal collapse, these solutions are beyond the scope of this paper.


\section{Natural Deduction}
\label{sec:calculus}

The language of higher-order modal logic used here is inspired by that of Church's simple type theory \citep{church}.

\begin{definition} \emph{Simple types} are given by the following grammar:
$$
  \theta,\tau \quad ::= \quad \mu \ \mid \ o \ \mid \ \theta \imp \tau
$$
where $\mu$ is the atomic type for individuals, $o$ is the atomic type for propositions and $\theta \imp \tau$ is the type for functions taking an argument of type $\theta$ and returning something of type $\tau$. `$\imp$' is assumed to be right associative.
\end{definition}

\begin{definition} \emph{Terms and formulas} are given by the following grammar:
\begin{align*}
 s,t \quad ::= \quad & 
  p_\tau \ \mid \ 
  X_\tau \ \mid \
  (\lambda X_\theta.s_\tau)_{\theta\imp\tau} \ \mid \ 
  (s_{\theta\imp\tau}\, t_\theta)_\tau \ \mid \\
& \bot_o \ \mid \
  \imp_{o\imp o\imp o} \ \mid \ 
  \wedge_{o\imp o\imp o} \ \mid \
  \vee_{o\imp o\imp o} \ \mid \\
& \all_{(\tau\imp o)\imp o} \ \mid \ 
  \ex_{(\tau\imp o)\imp o} \ \mid \
  \nec_{o\imp o} \ \mid \
  \pos_{o\imp o}
\end{align*}
where $p_\tau$ and $X_\tau$ range over, respectively, constants and variables of type $\tau$. Parenthesis conventions, infix notation for propositional connectives and binding notation for quantifiers are assumed. Furthermore, subscript types are omitted when they are clear from the context. Negation ($\neg_{o\imp o}$) and equivalence ($\biimp_{o\imp o\imp o}$) are defined by $\neg A\equiv A\imp \bot$ and $ (A\biimp B)\equiv (A\imp B)\wedge (B\imp A)$.
\end{definition}

The natural deduction calculus used here has standard rules for propositional connectives and quantifiers, as shown in Figures \ref{fig:PropositionalRules} and \ref{fig:QuantifierRules}. The extension to classical logic is achieved by adding a rule for double negation elimination, shown in Figure \ref{fig:Classical}. Finally, modal operators are handled by special rules that insert or remove formulas from boxes, as shown in Figure \ref{fig:NDK}.
%Apart from the use of labels and the dual rules for `$\pos$', these rules are essentially the modal rules from \cite{todo}. 
Beta-reduction is implicit; all rules are assumed to operate modulo beta-reduction. A \emph{derivation} is a directed acyclic graph whose nodes are formulas and whose edges correspond to applications of the inference rules. A \emph{proof} of a formula $F$ is a derivation without open assumptions and having $F$ as root not inside any box. 

\newcommand{\subproof}{\star}

Double lines are used to abbreviate tedious propositional reasoning steps in the derivations. Dashed lines are used to refer to an axiom or theorem with the proof shown elsewhere. When proof trees are too large to fit on the page, some branches may be displayed separately. In such cases, the conclusion of the branch and the location in the main proof tree where the branch belongs are annotated with the same symbol (a subscripted $\subproof$). Dotted lines are used to indicate folding and unfolding of definitions. Furthermore, as it is inconvenient to draw boxes around large derivations in \LaTeX, formulas inside boxes are labeled with the names of the boxes surrounding them. Therefore, the boxes can be omitted without loss of information. 

The calculus having only the rules shown in Figures \ref{fig:PropositionalRules}, \ref{fig:Classical} and \ref{fig:QuantifierRules} is named \ND, while the calculus with the additional rules shown in Figure \ref{fig:NDK} is named \NDK.

\newcommand{\s}{\qquad}




\begin{calculus}
{Propositional rules}
{fig:PropositionalRules}

\vspace{1em}

\s\s
\infer[\bot_E]{A}{ \bot }
\s
\infer[\imp_I]{A \imp B}{ B }
\s
\infer[\imp_I^n]{A \imp B}{ \infer*{B}{\infer[n]{A}{}} }
\s
\infer[\imp_E]{B}{A & A \imp B}

\vspace{2em}

\s\s
\infer[\wedge_I]{A \wedge B}{A & B}
\s\s
\infer[\wedge_{E_1}]{A}{A \wedge B}
\s\s
\infer[\wedge_{E_2}]{B}{A \wedge B}

% \vspace{2em}

% \s\s
% \infer[\vee_E]{C}{A \vee B & \infer*{C}{\infer{A}{}} & \infer*{C}{\infer{B}{}}}
% \s\s
% \infer[\vee_{I_1}]{A \vee B}{A}
% \s\s
% \infer[\vee_{I_2}]{A \vee B}{B}

\vspace{1em}
\end{calculus}

\begin{calculus}
{Double negation elimination}
{fig:Classical}
\infer[\neg\neg_E]{A}{ \neg\neg A }
\end{calculus}


\begin{calculus}
{Quantifier rules}
{fig:QuantifierRules}

\vspace{1em}

\s
\infer[\all_I]{\all x_{\tau}. A[x]}{ A[\alpha] }
\s
\infer[\all_E]{A[t]}{ \all x_{\tau}. A[x] }
\s\s
\infer[\ex_I]{\ex x_{\tau}. A[x]}{ A[t] }
\s
\infer[\ex_E]{A[\beta]}{ \ex x_{\tau}. A[x] }

\vspace{1em}

\begin{center}
\textbf{eigen-variable conditions:} \\
if $\rho$ is a $\all_I$ inference eliminating a variable $\alpha$, then any occurrence of $\alpha$ in the proof should be an ancestor of the occurrence of $\alpha$ eliminated by $\rho$; \\
if $\rho$ is a $\ex_E$ inference introducing a variable $\beta$, then any occurrence of $\beta$ in the proof should be a descendant of the occurrence of $\beta$ introduced by $\rho$.
\end{center}

\vspace{1em}

\end{calculus}



\begin{calculus}
{Rules for modal operators}
{fig:NDK}

\vspace{1em}

\s\s\s\s
\infer[\nec_I]{\nec A}{\omega: \fbox{\infer*{A}{}} }
\s\s\s\s
\infer[\nec_E]{w: \fbox{ \infer*{}{A} } }{\nec A}

\vspace{2em}

\s\s\s\s
\infer[\pos_I]{\pos A}{w: \fbox{\infer*{A}{}} }
\s\s\s\s
\infer[\pos_E]{\omega: \fbox{ \infer*{}{A} } }{\pos A}

\vspace{1em}


\begin{center}
\textbf{eigen-box condition:}\\ 
$\nec_I$ and $\pos_E$ are \emph{strong} modal rules: \\
$\omega$ must be a fresh name for the box they access \\ 
(in analogy to the eigen-variable condition for strong quantifier rules). \\
Every box must be accessed by \emph{exactly one} strong modal inference. \\
\vspace{0.5em}
\textbf{boxed assumption condition:} \\
assumptions should be discharged within the box where they are created.
\end{center}

\vspace{1em}

\end{calculus}


\subsection{Suitability for Rigid Higher-Order Modal Logic K}\label{sec:compl-sound}

Adding the modal rules results in a calculus that is suitable for the basic modal logic \textbf{K}.
In other words, {\NDK} is sound and complete relative to {\ND} extended with axiom K ($\nec(A\imp B)\imp (\nec A\imp \nec B)$) and the necessitation rule (which establishes that $\nec A$ is a theorem if $A$ is a theorem).


\begin{theorem}
\label{theorem:completeness}
{\NDK} is complete, relative to {\ND} extended with axiom K and the necessitation rule.
\end{theorem}
\begin{proof}
The necessitation rule can be immediately simulated with the $\nec_I$ rule. Axiom K can be derived in {\NDK} as shown below:

\begin{small}
\begin{prooftree}
\AXC{$ $}\RightLabel{2}
\UIC{$\nec(A\imp B)$}\RightLabel{$\nec_E$}
\UIC{$\omega: A\imp B$}
      \AXC{$ $}\RightLabel{1}
      \UIC{$\nec A $}\RightLabel{$\nec_E$}
      \UIC{$\omega: A$} \RightLabel{$\imp_E$}
   \BIC{$ \omega: B$} \RightLabel{$\nec_I$}
   \UIC{$ \nec B$} \RightLabel{$\imp_I^1$}
   \UIC{$\nec A\imp \nec B$} \RightLabel{$\imp_I^2$}
   \UIC{$\nec(A\imp B)\imp (\nec A\imp \nec B)$}
\end{prooftree}
\end{small}


\end{proof}


\begin{theorem}
\label{theorem:soundness}
{\NDK} is sound, relative to {\ND} extended with axiom K and the necessitation rule.
\end{theorem}
\begin{proof}
It is necessary to show that {\NDK} proofs of the following form can be translated to proofs in {\ND} extended with the axiom K and the necessitation rule.

\begin{small}
$$
\infer[\nec_I]{\nec B}{
\infer{\omega: B}{
  \infer{\vdots}{\infer[\nec_E]{\omega: A_1}{\nec A_1}} &
  \ldots & 
  \infer{\vdots}{\infer[\nec_E]{\omega: A_n}{\nec A_n}}}
}
$$
\end{small}

\noindent
A translation to {\ND} extended with axiom K and necessitation is shown below for the case when $n=1$:

 % Assuming the axiom K and the necessitation rule $\Box_I$, the open formula $\nec A $ and the existence of a derivation of $ B$ from the open assumption $ A$, then we can derive $\nec B$ without the rules for boxed parts of derivations.

\begin{small}
\begin{prooftree}
\AXC{$ $}\RightLabel{1}
\UIC{$A_1$}\noLine
\UIC{$\vdots$}\noLine
\UIC{$ B$} \RightLabel{$\imp_I^1$}
\UIC{$A_1\imp B$} \RightLabel{necessitation}
\UIC{$\nec(A_1\imp B)$}
      \AXC{Axiom K}\dashedLine
      \UIC{$\nec(A_1\imp B)\imp (\nec A_1\imp \nec B)$} \RightLabel{$\imp_E$}
  \BIC{$\nec A_1\imp \nec B$}
        \AXC{$\nec A_1 $}\RightLabel{$\imp_E$}
    \BIC{$\nec B$}
\end{prooftree}
\end{small}

\noindent
For $n > 1$, the translation is a straightforward generalization:

\begin{scriptsize}
\begin{prooftree}
\AXC{$ $}\RightLabel{1}
\UIC{$A_1$}\noLine
\UIC{$\vdots$}
%
\AXC{$\ldots$}
%
\AXC{$ $}\RightLabel{n}
\UIC{$A_n$}\noLine
\UIC{$\vdots$} 
%
\TIC{$ B$} \doubleLine \RightLabel{$\imp_I^*$}
\UIC{$A_1\imp \ldots \imp A_n\imp B$} \RightLabel{nec.}
\UIC{$\nec(A_1\imp \ldots \imp A_n\imp B)$}
      \AXC{Axiom K, iterated}\doubleLine\dashedLine
      \UIC{$\nec(A_1\imp \ldots \imp A_n\imp B)\imp (\nec A_1\imp \ldots \imp \nec A_n\imp \nec B)$} \RightLabel{$\imp_E$}
  \BIC{$\nec A_1\imp \ldots \imp \nec A_n\imp \nec B$}
\end{prooftree}
\end{scriptsize}

\begin{scriptsize}
\begin{prooftree}
  \AXC{$ $} \dashedLine
  \UIC{$\nec A_1\imp \ldots \imp \nec A_n\imp \nec B$}
        \AXC{$\nec A_1 \quad \ldots \quad \nec A_n$} \doubleLine \RightLabel{$\imp_E$}
    \BIC{$\nec B$}
\end{prooftree}
\end{scriptsize}

\end{proof}

Without the restriction that every box must be accessed by exactly one strong modal inference, the calculus would be unsound for the modal logic \textbf{K}. For example, the formula $\all \psi. (\nec \psi \imp \pos \psi)$ is not valid in \textbf{K} but would be provable without this restriction:

\begin{prooftree}
  \AXC{$ $} \RightLabel{1}
  \UIC{$\nec \psi$} \RightLabel{$\nec_E$}
        \UIC{$\omega: \psi$} \RightLabel{$\pos_I$}
        \UIC{$\pos \psi$} \RightLabel{$\imp_I^1$}
    \UIC{$\nec \psi \imp \pos \psi$} \RightLabel{$\forall_I$}
    \UIC{$\all \psi. (\nec \psi \imp \pos \psi)$}
\end{prooftree}

This example proof is unsound according to the {\NDK} calculus, because the eigen-box condition is violated: the box labelled by $\omega$ is not accessed by any strong inference. 

The straightforward combinations of the quantifier rules of {\ND} either with the modal rules of {\NDK} or with axiom K and the necessitation rule are suitable for a higher-order modal logic where constants and variables are \emph{rigid}. From the point of view of a \emph{possible world semantics}, rigidity means that their interpretation is independent of the world in which they are being interpreted. Rigidity is silently assumed by most works investigating the ontological argument, and is explicitly assumed here. Nevertheless, it should be noted that its adequacy for the ontological argument has already been contested \citep{fitting}. Another assumption made here is that the quantification domains are constant (i.e. independent of the possible worlds). Neither G\"odel's manuscript nor Scott's manuscript reveal whether they use constant or varying domains, and this is also the case for many variants (e.g. \citep{hajek}). Nevertheless, some authors of variants \citep[footnotes 11 and 14]{and90} of G\"odel's ontological argument do explicitly state a preference for varying domains. Our choice of assuming constant domains is motivated by simplicity.



\section{Some Useful Derivable Modal Principles}
\label{sec:ModalLogics}

From an axiomatic point of view, modal logics differ with respect to which additional axiom schemas they admit. Some common axiom schemas of relevance to the ontological argument are T ($\nec A \imp A$), B ($A \imp \nec \pos A$), 4 ($\nec A \imp \nec \nec A $) and 5 ($\pos A \imp \nec \pos A$). From a semantic point of view, these axioms correspond to geometric frame conditions that must be satisfied by the accesibility relation in the Kripke models \citep{negri}: T corresponds to reflexivity, B corresponds to symmetry, 4 corresponds to transitivity and 5 corresponds to euclidianity. In the models of logic {\bf S5}, all these axioms are satisfied; in the models of logic {\bf KB}, K and B are satisfied; and in the models of logic {\bf B}, K, T and B are satisfied.

In this section, natural deduction proofs of three convenient derived modal principles are presented. These principles are used in the ontological proofs formalized in the next sections. Although they are well-known among modal logicians, we include their proofs here in order to have a self-contained and fully formal presentation of the ontological proofs, without any gaps.

\bigskip


\noindent
The \emph{distribution principle} can be seen as a form of modus ponens within the scope of modalities: if $A \imp B$ holds in all accessible worlds and $A$ holds in an accessible world, then $B$ holds in an accessible world. This principle is provable in the modal logic {\bf K}.

\begin{lemma*}[Distribution Principle]
\label{DP} 
$$\nec (A\imp B)\imp(\pos A\imp \pos B)$$
\end{lemma*}

\begin{proof}\hfill

\begin{small}
\begin{prooftree}
\AXC{$ $} \RightLabel{2}
\UIC{$\nec (A\imp B)$}\RightLabel{$\nec_E$}
\UIC{$\omega: A\imp B$}
\AXC{$ $} \RightLabel{1}
\UIC{$\pos A$}\RightLabel{$\pos_E$}
\UIC{$\omega: A$}\RightLabel{$\imp_E$}
\BIC{$\omega: B$} \RightLabel{$\pos_I$}
\UIC{$\pos B$}\RightLabel{$\imp_I^1$}
\UIC{$\pos A\imp \pos B$}\RightLabel{$\imp_I^2$}
\UIC{$\nec (A\imp B)\imp(\pos A\imp \pos B)$}
\end{prooftree}
\end{small}

\end{proof}

\bigskip

\noindent
\emph{Brouwer's reduction principle} is derivable in modal logic {\bf KB}, using axiom B. The proof of G\"odel and Scott (Section \ref{sec:scott}) do not make direct use of this principle, but the new proof presented in Section \ref{sec:newproof} does.


\begin{lemma*}[Brouwer's Reduction Principle]
\label{BRP}
$$
\pos\nec A\imp A
$$ 
\end{lemma*}


\begin{proof}\hfill

\begin{small}
\begin{prooftree}
\AXC{$ $}\RightLabel{$2$}
\UIC{$\pos \nec A $} \doubleLine
\UIC{$\neg\nec \neg \nec A$}
\AXC{$ $}\RightLabel{$ 1$}
\UIC{$\neg A$}
\AXC{Axiom B}\dashedLine
\UIC{$\neg A\imp \nec\pos\neg A$} \doubleLine
\UIC{$\neg A\imp \nec\neg\nec\neg\neg A$}\RightLabel{$\imp_E$}
\BIC{$\nec\neg\nec\neg\neg A$}\doubleLine
\UIC{$\nec\neg\nec A$} \RightLabel{$\neg_E$}
\BIC{$\bot$}\RightLabel{$\neg_I^1$}
\UIC{$\neg\neg A$} \RightLabel{$\neg\neg_E$}
\UIC{$A$} \RightLabel{$\imp_I^2$}
\UIC{$\pos \nec A\imp A$}
\end{prooftree}
\end{small}

\end{proof}

\bigskip

\noindent
In the modal logic {\bf S5}, a sequential iteration of modalities can be collapsed to the last modality in the sequence. In other words, the following principles hold in {\bf S5}:
\begin{itemize}
\item $\pos^n \nec A \imp \nec A$
\item $\nec^n \pos A \imp \pos A$
\item $\pos^n \pos A \imp \pos A$
\item $\nec^n \nec A \imp \nec A$
\end{itemize}

\noindent
Below we exhibit a natural deduction proof for the first principle listed above  when $n = 1$. Informally, this particular case can be read as the claim that anything that is possibly necessary is in fact necessary, and it is sufficient for the ontological proofs of G\"odel and Scott. The proof of this principle depends on the previously proven Brouwer's reduction principle, on the Modal Axiom 5, and on K, which was shown to be derivable in the proof of Theorem \ref{theorem:completeness}.


\begin{lemma*}[Iteration Principle]
$$
\pos \nec A \imp \nec A
$$
\end{lemma*}

\begin{proof}\hfill

\begin{footnotesize}
\begin{prooftree}

\AXC{Brouwer's Reduction}\dashedLine
\UIC{$\pos \nec A\imp A$}\RightLabel{$\nec_I$}
\UIC{$\nec(\pos \nec A\imp A)$}
\AXC{K}\dashedLine
\UIC{$\nec(\pos \nec A\imp A)\imp (\nec\pos \nec A\imp \nec A)$}\RightLabel{$\imp_E$}
\BIC{$\nec\pos \nec A\imp \nec A$}

\AXC{Axiom 5 for $\nec A$}\dashedLine
\UIC{$\pos \nec A\imp \nec \pos \nec A$}

\AXC{$ $}\RightLabel{$1$}
\UIC{$\pos \nec A$}\RightLabel{$\imp_E$}
\BIC{$\nec \pos \nec A$}\RightLabel{$\imp_E$}
\BIC{$\nec A$}\RightLabel{$\imp_I^1$}
\UIC{$\pos \nec A \imp  \nec A$}
\end{prooftree}
\end{footnotesize}


\end{proof}



\section{Scott's Proof in Natural Deduction}\label{sec:scott}

\setcounter{axiom}{0}
\setcounter{lemma}{0}
\setcounter{theorem}{0}
\setcounter{corollary}{0}
\setcounter{definition}{0}


\newcommand{\scott}{^{\dagger}}


In this section we present a detailed formalization of Scott's proof in the natural deduction calculus defined in Section \ref{sec:calculus}. The reasoning in Scott's manuscript has been reproduced step-by-step and all reasoning gaps are filled in by using the deduction considered most natural. All derived formulas that do appear in Scott's manuscript are marked with a $\dagger$ here. Unmarked formulas were derived in the process of filling the gaps between the marked formulas mentioned in Scott's manuscript. 

% ToDo: uncomment this post-acceptance: the journal uses double blind reviewing.
%The formal natural deduction proof presented here was verified in the Coq proof assistant\footnote{\url{https://github.com/FormalTheology/GoedelGod/blob/master/Formalizations/Coq/GoedelGod_Scott.v}}. 

Scott's version of G\"odel's proof depends on 5 axioms that circumscribe the notion of \emph{positive} property, with positivity being denoted by the undefined second-order predicate symbol $P$. Additionally, 3 definitions are used for defining the notions of \emph{God-like}, \emph{essence} and \emph{necessary existence}. Technically, these definitions only abbreviate certain complex formulas. The argument would still go through if all defined symbols were replaced by the complex formulas they define. This observation is particularly relevant in the case of \emph{necessary existence}. Since this notion of ``existence'' is just an abbreviation, G\"odel's argument is not susceptible to Kant's criticism against Anselm's argument (that existence should not be treated as a predicate). In G\"odel's proof, existence is properly denoted by the existential quantifier. It is, therefore, unfortunately misleading to refer to the defined predicate symbol $E$ as ``necessary existence'', when in fact it is just a convenient abbreviation.

Beside the 5 axioms and 3 definitions that pertain specifically to the argument, the proof also uses {\bf S5}'s iteration principle (in the proof of Lemma 2) and the equality axiom of reflexivity (in the proof of Theorem 1). The use of reflexivity could be avoided, but allows a shorter proof.


\begin{axiom}
\label{A1}
Either a property or its negation is positive, but not both:
$$
\all \varphi. [P(\neg \varphi) \biimp \neg P(\varphi)]
$$
\end{axiom}

\begin{axiom}
\label{A2}
A property necessarily implied by a positive property is positive:
$$
\all \varphi. \all \psi.[(P(\varphi) \wedge \nec \all x.[\varphi(x) \imp \psi(x)]) \imp P(\psi)]
$$
\end{axiom}


\begin{theorem}
\label{T1}
Positive properties are possibly exemplified:
$$
\all \varphi. [P(\varphi) \imp \pos \ex x.\varphi(x)]
$$
\end{theorem}
\begin{proof} \hfill


\begin{prooftree}
\AXC{$ $} \RightLabel{5}
\UIC{$P(\rho)$}
		\AXC{Reflexivity} \dashedLine
		\UIC{$\omega: \gamma = \gamma$} \RightLabel{$\imp_I$}
		\UIC{$\omega: \rho(\gamma) \imp \gamma = \gamma$} \RightLabel{$\all_I$}
		\UIC{$\omega: \all x. \rho(x) \imp x = x$} \RightLabel{$\nec_I$}
		\UIC{$\nec (\all x. \rho(x) \imp x = x)\scott $} \RightLabel{$\wedge_I$}
	\BIC{$P(\rho) \wedge \nec (\all x. \rho(x) \imp x = x) $} 
			\AXC{Axiom 2 for $\rho$ and $\lambda x. x = x$} \dashedLine
	    	\UIC{$P(\rho) \wedge \nec (\all x. \rho(x) \imp x = x) \imp P(\lambda x. x = x)$} 
		\BIC{$[\subproof_1] \qquad P(\lambda x. x = x)\scott$} 
\end{prooftree}


\begin{prooftree}
\AXC{$ $} \RightLabel{5}
\UIC{$P(\rho)$}
	\AXC{$ $} \RightLabel{1}
	\UIC{$\omega: \rho(\beta)$}
			\AXC{$ $} \RightLabel{3}
			\UIC{$\nec \all x. \neg \rho(x)$} \RightLabel{$\nec_E$}
			\UIC{$\omega: \all x. \neg \rho(x) $} \RightLabel{$\all_E$}
			\UIC{$\omega: \neg \rho(\beta)$} \RightLabel{$\neg_E$}
		\BIC{$\omega: \bot$} \RightLabel{$\bot_E$}
		\UIC{$\omega: \neg (\beta = \beta)$} \dottedLine
		\UIC{$\omega: (\beta \neq \beta)$} \RightLabel{$\imp_I^1$}
		\UIC{$\omega: (\rho(\beta) \imp (\beta \neq \beta))$} \RightLabel{$\all_I$}
		\UIC{$\omega: (\all x. \rho(x) \imp (x \neq x))$} \RightLabel{$\nec_I$}
		\UIC{$\nec (\all x. \rho(x) \imp (x \neq x))\scott$}  \RightLabel{$\wedge_I$}
	\BIC{$P(\rho) \wedge \nec (\all x. \rho(x) \imp x \neq x) $} 
			\AXC{Axiom 2 for $\rho$ and $\lambda x. x \neq x$} \dashedLine
	    	\UIC{$P(\rho) \wedge \nec (\all x. \rho(x) \imp x \neq x) \imp P(\lambda x. x \neq x)$} 
		\BIC{$[\subproof_2] \qquad P(\lambda x. x \neq x)\scott$} 
\end{prooftree}


\begin{prooftree}
		\AXC{$\subproof_2$} \dashedLine
		\UIC{$P(\lambda x. x \neq x)\scott$}    
    		\AXC{$\subproof_1$} \dashedLine
			  \UIC{$P(\lambda x. x = x)\scott$} 
					\AXC{Half of Axiom 1} \dashedLine
					\UIC{$P(\lambda x. x = x) \imp \neg P(\lambda x. x \neq x)$} \RightLabel{$\imp_E$}
				\BIC{$\neg P(\lambda x. x \neq x)$} \RightLabel{$\neg_E$}
			\BIC{$\bot$}
\end{prooftree}


\begin{prooftree}
\AXC{$ $} \RightLabel{4}
\UIC{$\nec \neg \ex x. \rho(x)$} \RightLabel{$\nec_E$}
\UIC{$\omega: \neg \ex x. \rho(x)$}
        \AXC{$ $} \RightLabel{2}
        \UIC{$\rho(\alpha)$} \RightLabel{$\ex_I$}
		\UIC{$\omega: \ex x. \rho(x)$} \RightLabel{$\neg_E$}
    \BIC{$\omega: \bot$} \RightLabel{$\neg_I^2$}
	\UIC{$\omega: \neg \rho(\alpha)$} \RightLabel{$\all_I$}
	\UIC{$\omega: \all x. \neg \rho(x)$} \RightLabel{$\nec_I$}
	\UIC{$\nec \all x. \neg \rho(x) \scott$}
	        \AXC{$ $} \dashedLine
	        \UIC{$\bot$} \RightLabel{$\neg_I^3$}
    		\UIC{$\neg \nec \all x. \neg \rho(x)$} \RightLabel{$\neg_E$}
  		\BIC{$\bot$} \RightLabel{$\neg_I^4$}
  		\UIC{$ \neg \nec \neg \ex x.\rho(x)$} \doubleLine %\RightLabel{definition of $\pos$}
  		\UIC{$ \pos \ex x.\rho(x)$}  \RightLabel{$\imp_I^5$}
  		\UIC{$ P(\rho) \imp \pos \ex x.\rho(x)$} \RightLabel{$\all_I$}
      \UIC{$ \all \varphi. [P(\varphi) \imp \pos \ex x.\varphi(x)] $}
\end{prooftree}

\end{proof}


\begin{definition}
\label{D1}
A \emph{God-like} being possesses all positive properties:
$$
G(x) \biimp \forall \varphi. [P(\varphi) \to \varphi(x)]
$$
\end{definition}



\begin{axiom}
\label{A3}
The property of being god-like is positive:
$$
P(G)
$$
\end{axiom}
\begin{corollary}
\label{C1}
Possibly, God exists:
$$
\pos \ex x. G(x)
$$
\end{corollary}
\begin{proof} \hfill
\begin{prooftree}
\AXC{Axiom \ref{A3}} \dashedLine
\UIC{$P(G)$}
\AXC{Theorem \ref{T1} for G} \dashedLine
\UIC{$ P(G) \imp \pos \ex x.G(x) $} \RightLabel{$\imp_E$}
\BIC{$\pos \ex x. G(x)$}
\end{prooftree}
\end{proof}


\begin{axiom}
\label{A4}
Positive properties are necessarily positive:
$$
\all \varphi.[P(\varphi) \to \Box \; P(\varphi)]
$$
\end{axiom}

\begin{definition}
\label{D2}
An \emph{essence} of an individual is a property possessed by it and necessarily implying any of its properties:
$$
\ess{\varphi}{x} \biimp \varphi(x) \wedge \all \psi. (\psi(x) \imp \nec \all x. (\varphi(x) \imp \psi(x)))
$$
\end{definition}


\begin{theorem}
\label{T2}
Being god-like is an essence of any God-like being:
$$
\all x. G(x) \imp \ess{G}{x}
$$
\end{theorem}
\begin{proof}\hfill

\begin{small}
\begin{prooftree}


	\AXC{$ $} \RightLabel{5}
	\UIC{$\rho(\gamma)$}

		\AXC{$ $} \RightLabel{2}
		\UIC{$\neg P(\rho)$} 
				\AXC{Axiom 1} \dashedLine
        \UIC{$\all \varphi. [P(\neg \varphi) \biimp \neg P(\varphi)]$} \RightLabel{$\all_E$}
        \UIC{$P(\lambda x. \neg \rho(x)) \biimp \neg P(\rho)$} \RightLabel{$\wedge_E$}
        \UIC{$\neg P(\rho) \imp P(\lambda x. \neg \rho(x))$} \RightLabel{$\imp_E$}
			\BIC{$P(\lambda x. \neg \rho(x))$}  

					\AXC{$ $} \RightLabel{1}
					\UIC{$G(\gamma)$} \dottedLine
					\UIC{$\all \varphi. P(\varphi) \imp \varphi(\gamma)$} \RightLabel{$\all_E$}
					\UIC{$P(\lambda x. \neg \rho(x)) \imp \neg \rho(\gamma)$} \RightLabel{$\imp_E$}
				\BIC{$\neg \rho(\gamma)$} \RightLabel{$\neg_E$}      
	   \BIC{$\bot$} \RightLabel{$\neg_I^2$}
	   \UIC{$\neg\neg P(\rho)$} \RightLabel{$\neg\neg_E$}
	   \UIC{$[\subproof_3] \qquad P(\rho)$}
	   \end{prooftree}
\end{small}


\begin{small}
\begin{prooftree}
\AXC{$ $} \RightLabel{1}
\UIC{$G(\gamma)$}

 \AXC{$\subproof_3$}\dashedLine
 \UIC{$P(\rho)$}
	      \AXC{Axiom 4} \dashedLine
        \UIC{$\all \varphi.[P(\varphi) \to \Box \; P(\varphi)]$} \RightLabel{$\all_E$}
	   	  \UIC{$P(\rho) \imp \nec P(\rho)$} \RightLabel{$\imp_E$}
		 \BIC{$\nec P(\rho)\scott$}
			\AXC{$ $} \RightLabel{4}
			\UIC{$\omega: P(\rho)$}
					\AXC{$ $} \RightLabel{3}
					\UIC{$\omega: G(\delta) $} \dottedLine
					\UIC{$\omega: \all \varphi. P(\varphi) \imp \varphi(\delta)$} \RightLabel{$\all_E$} 
					\UIC{$\omega: P(\rho) \imp \rho(\delta)$} \RightLabel{$\imp_E$}
				\BIC{$\omega: \rho(\delta) $} \RightLabel{$\imp_I^3$}
				\UIC{$\omega: G(\delta) \imp \rho(\delta) $} \RightLabel{$\all_I$}
				\UIC{$\omega: \all y. G(y) \imp \rho(y) $} \RightLabel{$\imp_I^4$}
				\UIC{$\omega: P(\rho) \imp (\all y. G(y) \imp \rho(y) )$} \RightLabel{$\nec_I$}
				\UIC{$\nec (P(\rho) \imp (\all y. G(y) \imp \rho(y) ))$} \RightLabel{$\imp_E$}
						\AXC{ K} \RightLabel{$\imp_E$}
					\BIC{$\nec P(\rho) \imp \nec (\all y. G(y) \imp \rho(y) )$}
			\BIC{$\nec (\all y. G(y) \imp \rho(y) )$} \RightLabel{$\imp_I^5$}
			\UIC{$\rho(\gamma) \imp \nec (\all y. G(y) \imp \rho(y) )$} \RightLabel{$\all_I$}
			\UIC{$\all \varphi. \varphi(\gamma) \imp \nec (\all y. G(y) \imp \varphi(y) )$} \RightLabel{$\wedge_I$}
		\BIC{$ G(\gamma) \wedge (\all \varphi. \varphi(\gamma) \imp \nec (\all y. G(y) \imp \varphi(y) ))$} \dottedLine
		\UIC{$\ess{G}{\gamma}$} \RightLabel{$ \imp_I^1$}
		\UIC{$G(\gamma) \imp \ess{G}{\gamma}$} \RightLabel{$\all_I$}
		\UIC{$\all x. G(x) \imp \ess{G}{x}$}
\end{prooftree}
\end{small}

\end{proof}


\begin{definition}
\label{D3}
\emph{Necessary existence} of an individual is the necessary exemplification of all its essences:
$$
E(x) \biimp \all \varphi.[\ess{\varphi}{x} \imp \nec \ex y.\varphi(y)]
$$
\end{definition}


\begin{axiom}
\label{A5}
Necessary existence is a positive property:
$$
P(E)
$$
\end{axiom}



\begin{lemma}
\label{L1}
If there is a God-like being, then there is a God-like being necessarily:
$$
\ex z. G(z) \imp \nec \ex x. G(x)
$$
\end{lemma}

\begin{proof}\hfill

\begin{prooftree}
\AXC{$ $} \RightLabel{1}
\UIC{$\ex z. G(z)$} \RightLabel{$\ex_E$}
\UIC{$[\subproof_4] \qquad G(\gamma)$}
\end{prooftree}


\begin{prooftree}
\AXC{$\subproof_4$} \dashedLine
\UIC{$G(\gamma)$}
		\AXC{Theorem 2} \dashedLine 
		\UIC{$\all x. G(x) \imp \ess{G}{x}$} \RightLabel{$\all_E$}
		\UIC{$G(\gamma) \imp \ess{G}{\gamma}$}
	\BIC{$\ess{G}{\gamma}\scott$}
				\AXC{Axiom 5} \dashedLine
				\UIC{$P(E)$}
						\AXC{$\subproof_4$} \dashedLine
						\UIC{$G(\gamma)$} \dottedLine
						\UIC{$\all \varphi. P(\varphi) \imp \varphi(\gamma)$} \RightLabel{$\all_E$}
						\UIC{$P(E) \imp E(\gamma)$} 
					\BIC{$E(\gamma)\scott$} \dottedLine
					\UIC{$\all \varphi. \ess{\varphi}{\gamma} \imp \nec \ex x. \varphi(x)$} \RightLabel{$\all_E$}
					\UIC{$\ess{G}{\gamma} \imp \nec \ex x. G(x)$}
		\BIC{$\nec \ex x. G(x)$} \RightLabel{$\imp_I^1$}
		\UIC{$\ex z. G(z) \imp \nec \ex x. G(x)$}

\end{prooftree}

\end{proof}


\begin{lemma}
\label{L2}
If the existence of a God-like being is possible, then it is necessary:
$$
\pos \ex z. G(z) \imp \nec \ex x. G(x)
$$
\end{lemma}

\begin{proof}\hfill

\begin{scriptsize}
\begin{prooftree}

\AXC{$ $} \RightLabel{1}
\UIC{$\pos \ex x. G(x)$} 
		\AXC{Lemma 1} \dashedLine
		\UIC{$\omega: \ex x. G(x) \imp \pos \nec \ex x. G(x)$} \RightLabel{$\nec_I$}
		\UIC{$\nec (\ex x. G(x) \imp \nec \ex x. G(x)) $}
				\AXC{Distribution Principle}
			\BIC{$\pos \ex x. G(x) \imp \pos \nec \ex x. G(x)$} \RightLabel{$\imp_E$}
	\BIC{$\pos \nec \ex x. G(x)\scott$}
			\AXC{S5 Iteration Principle} \dashedLine
			\UIC{$\pos \nec \ex x. G(x) \imp \nec \ex x. G(x)$}
		\BIC{$\nec \ex x. G(x)$} \RightLabel{$\imp_I^1$}
		\UIC{$\pos \ex z. G(z) \imp \nec \ex x. G(x)$}
\end{prooftree}
\end{scriptsize}

\end{proof}

\begin{theorem}
\label{T3}
Necessarily, there exists a God-like being:
$$
\nec \ex x. G(x)
$$
\end{theorem}

\begin{proof}\hfill

\begin{prooftree}
	\AXC{Corollary 1} \dashedLine
	\UIC{$\pos \ex x. G(x)$}
			\AXC{Lemma 2} \dashedLine
			\UIC{$\pos \ex x. G(x) \imp \nec \ex x. G(x)$} \RightLabel{$\imp_E$}
		\BIC{$\nec \ex x. G(x)$}
\end{prooftree}

\end{proof}



\section{New Proof}\label{sec:newproof}

In this section we present a new proof in {\bf KB}. This proof uses exactly the same axioms and definitions of Scott's proof shown in the previous sections, but it uses neither the equality axiom of reflexivity nor {\bf S5}'s modal iteration principle. Instead, it relies only on Brouwer's reduction principle (in the proof of Theorem 3). The new proof is also shorter than Scott's proof. 



\setcounter{axiom}{0}
\setcounter{lemma}{0}
\setcounter{theorem}{0}
\setcounter{corollary}{0}
\setcounter{definition}{0}


\begin{axiom}
\label{A1}
Either a property or its negation is positive, but not both:
$$
\all \varphi. [P(\neg \varphi) \biimp \neg P(\varphi)]
$$
\end{axiom}

\begin{axiom}
\label{A2}
A property necessarily implied by a positive property is positive:
$$
\all \varphi. \all \psi.[(P(\varphi) \wedge \nec \all x.[\varphi(x) \imp \psi(x)]) \imp P(\psi)]
$$
\end{axiom}
\begin{theorem}
\label{T1}
Positive properties are possibly exemplified:
$$
\all \varphi. [P(\varphi) \imp \pos \ex x.\varphi(x)]
$$
\end{theorem}
\begin{proof} \hfill

\begin{prooftree}
\AXC{$ $} \RightLabel{3}
\UIC{$P(\rho)$} 
\AXC{$ $} \RightLabel{2}
\UIC{$\nec \neg \ex x.\rho(x) $} \RightLabel{$\nec_E$}
\UIC{$\omega: \neg \ex x.\rho(x) $} 
\AXC{$ $} \RightLabel{$1$}
\UIC{$\omega: \rho(x)$} \RightLabel{$\ex_I$}
\UIC{$\omega: \ex x.\rho(x) $} \RightLabel{$\imp_E$}
\BIC{$\omega:  \bot $}\RightLabel{$\imp_I^1$}
\UIC{$\omega:  \neg \rho(x) $} \RightLabel{$\imp_I$}
\UIC{$\omega:  \rho(x) \imp \neg \rho(x) $} \RightLabel{$\all_I$}
\UIC{$\omega:  \all x.(\rho(x) \imp \neg \rho(x)) $}  \RightLabel{$\nec_I$}
\UIC{$ \nec \all x.(\rho(x) \imp \neg \rho(x)) $}\RightLabel{$\wedge_I$}
\BIC{$ P(\rho) \wedge \nec \all x.[\rho(x) \imp \neg \rho(x)]$}
\AXC{Axiom \ref{A2} for $\rho$ and $\lambda x. \neg \rho(x)$} \dashedLine
\UIC{$(P(\rho) \wedge \nec \all x.[\rho(x) \imp \neg \rho(x)]) \imp P(\lambda x. \neg \rho(x))$} \RightLabel{$\imp_E$}
\BIC{$[\subproof_5]\qquad P(\lambda x. \neg \rho(x))$} \RightLabel{$\imp_E$}
\end{prooftree}



\begin{prooftree}
\AXC{$\subproof_5$} \dashedLine
\UIC{$P(\lambda x. \neg \rho(x))$} \RightLabel{$\imp_E$}
\AXC{Axiom 1} \dashedLine
\UIC{$\all \varphi. [P(\neg \varphi) \biimp \neg P(\varphi)]$} \RightLabel{$\all_E$}
\UIC{$P(\lambda x. \neg \rho(x)) \biimp \neg P(\rho)$} \RightLabel{$\wedge_E$}
\UIC{$ P(\lambda x. \neg \rho(x)) \imp \neg P(\rho) $}  \RightLabel{$\imp_E$}
\BIC{$\neg P(\rho) $} 
\AXC{$ $} \RightLabel{3} 
\UIC{$P(\rho)$} 
\BIC{$\bot$} \RightLabel{$\imp_I^2$}
\UIC{$ \neg \nec \neg \ex x.\rho(x)$} \doubleLine %\RightLabel{definition of $\pos$}
\UIC{$ \pos \ex x.\rho(x)$}  \RightLabel{$\imp_I^3$}
\UIC{$ P(\rho) \imp \pos \ex x.\rho(x) $} \RightLabel{$\all_I$}
\UIC{$ \all \varphi. [P(\varphi) \imp \pos \ex x.\varphi(x)] $} 

\end{prooftree}


\end{proof}

Comparing this new proof of Theorem 1 with the proof of Theorem 1 indicated by G\"odel and Scott, it is noticeable that new proof is significantly shorter. The key idea to obtain a shorter proof is to instantiate the second universally quantified property of Axiom 2 by $\lambda x. \neg \rho(x)$ instead of using the predicate symbols $=$ or $\neq$ as G\"odel and Scott did. This eliminates the reliance of the ontological argument on the presence of $=$ and $\neq$ in the logical language. Furthermore, Axiom 2 is then needed only once.

Although the simpler proof of Theorem 1 presented here was developed independently, it seems conceptually related to a possible simplification briefly and informally described by Anderson \citep[footnote 2]{and90} and attributed by him to an anonymous referee. The simpler proof of Theorem 1 is not particularly special. In fact, it is probably the most likely proof that any well-trained logician would construct, if asked to derive Theorem 1 from Axioms 1 and 2 using methodical formal proof search techniques. Modern techniques usually rely on unification to compute or guess good instances for weakly quantified variables, and therefore they would never result in instantiations containing $=$ and $\neq$, which are symbols that do not appear anywhere in the axioms or in the theorems to be proved. We may only wonder why G\"odel preferred to use those unusual instantiations. It might be an interesting question from a historical point of view, if this suggests that there might have been a change (perhaps evolution) in the way logicians construct formal proofs; or even from a philosophical point of view, if G\"odel's preference was philosophically motivated.



\begin{definition}
\label{D1}
A \emph{God-like} being possesses all positive properties:
$$
G(x) \biimp \forall \varphi. [P(\varphi) \to \varphi(x)]
$$
\end{definition}

\begin{axiom}
\label{A3}
The property of being god-like is positive:
$$
P(G)
$$
\end{axiom}

\begin{corollary}
\label{C1}
Possibly, a God-like being exists:
$$
\pos \ex x. G(x)
$$
\end{corollary}
\begin{proof} \hfill
\begin{prooftree}
\AXC{Axiom \ref{A3}} \dashedLine
\UIC{$P(G)$}
\AXC{Theorem \ref{T1} for G} \dashedLine
\UIC{$ P(G) \imp \pos \ex x.G(x) $} \RightLabel{$\imp_E$}
\BIC{$\pos \ex x. G(x)$}
\end{prooftree}
\end{proof}


\begin{axiom}
\label{A4}
Positive properties are necessarily positive:
$$
\all \varphi.[P(\varphi) \to \Box \; P(\varphi)]
$$
\end{axiom}

\begin{definition}
\label{D2}
An \emph{essence} of an individual is a property possessed by it and necessarily implying any of its properties:
$$
\ess{\varphi}{x} \biimp \varphi(x) \wedge \all \psi. (\psi(x) \imp \nec \all x. (\varphi(x) \imp \psi(x)))
$$
\end{definition}


At this point, G\"odel and Scott proceed to prove Theorem 2, which is then used to prove Lemma 1. However, a close inspection of Scott's proof of Lemma 1 reveals that Theorem 2 is an unnecessary detour. Therefore, we do not include it in the new proof.


\begin{definition}
\label{D3}
\emph{Necessary existence} of an individual is the necessary exemplification of all its essences:
$$
E(x) \biimp \all \varphi.[\ess{\varphi}{x} \imp \nec \ex y.\varphi(y)]
$$
\end{definition}


\begin{axiom}
\label{A5}
Necessary existence is a positive property:
$$
P(E)
$$
\end{axiom}


The new proof of Lemma 1 does not rely on Theorem 2. Overall, it is slightly shorter than Scott's proofs of Theorem 2 and Lemma 1 combined, but a large part of the new proof of Lemma 1 is structurally very similar to Scott's proof of Theorem 2, and their underlying intuitive ideas are still essentially the same. However, despite being an unnecessary detour from a technical point of view, Theorem 2 is very interesting from a philosophical perspective. It breaks an otherwise long proof of Lemma 1 in a point that facilitates comprehension by humans at an intuitive and informal level.

This phenomenon is very intriguing: the new proof of Lemma 1 is technically simpler (because it has fewer inferences), but Scott's proof of Lemma 1 with Theorem 2 can be considered intuitively simpler (because it is arguably easier to understand). This constitutes an interesting case for Hilbert's 24th Problem \citep{hilbert}, which asks for criteria to properly compare the simplicity of proofs. While Hilbert had mathematical proofs in mind, insights into the 24th problem could be gained by analyzing philosophical proofs as well. As G\"odel suggested, ``a scientific (exact) philosophy and theology [\ldots] is also most highly fruitful for science''.


\begin{lemma}
\label{L1}
If there is a God-like being, then there is a God-like being necessarily:
$$
\ex z. G(z) \imp \nec \ex x. G(x)
$$
\end{lemma}

\begin{proof} \hfill
\begin{prooftree}
\AXC{$ $} \RightLabel{1}
\UIC{$\ex z. G(z)$}\RightLabel{$\ex_E$}
\UIC{$[\subproof_6] \qquad G(\gamma)$}
\end{prooftree}

\begin{scriptsize}
\begin{prooftree}
\AXC{$ $} \RightLabel{2}
\UIC{$\neg P(\rho)$} 
\AXC{Axiom 1} \dashedLine
\UIC{$\all \varphi. [P(\neg \varphi) \biimp \neg P(\varphi)]$} \RightLabel{$\all_E$}
\UIC{$P(\lambda x. \neg \rho(x)) \biimp \neg P(\rho)$} \RightLabel{$\wedge_E$}
\UIC{$\neg P(\rho) \imp P(\lambda x. \neg \rho(x))$}\RightLabel{$\imp_E$}
\BIC{$P(\lambda x. \neg \rho(x))$}
\AXC{$\subproof_6$} \dashedLine
\UIC{$G(\gamma)$}  \dottedLine \RightLabel{D\ref{D1}}
\UIC{$\forall \varphi.(P(\varphi) \imp \varphi(\gamma))$}\RightLabel{$\forall_E$}
\UIC{$P(\lambda x. \neg \rho(x)) \imp \neg \rho(\gamma)$}\RightLabel{$\imp_E$}
\BIC{$\neg \rho(\gamma)$}
\AXC{$ $} \RightLabel{3}
\UIC{$\rho(\gamma)$} \RightLabel{$\imp_E$}
\BIC{$\bot$}\RightLabel{$\imp_I^2$}
\UIC{$\neg\neg P(\rho)$}\RightLabel{$\neg\neg_E$}
\UIC{$P(\rho)$}
\AXC{Axiom \ref{A4}} \dashedLine
\UIC{$\all \varphi.[P(\varphi) \imp \nec\; P(\varphi)]$}\RightLabel{$\forall_E$}
\UIC{$P(\rho) \imp \nec\; P(\rho)$}\RightLabel{$\imp_E$}
\BIC{$\nec P(\rho)$}\RightLabel{$\imp_I^3$}
\UIC{$[\subproof_7] \qquad \rho(\gamma) \imp \nec\; P(\rho)$}
\end{prooftree}
\end{scriptsize}


\begin{small}
\begin{prooftree}
\AXC{$ $} \RightLabel{5}
\UIC{$\nec P(\rho)$} \RightLabel{$\nec_E$}
\UIC{$\omega: P(\rho)$}
\AXC{$ $} \RightLabel{4}
\UIC{$\omega: G(y)$}  \dottedLine \RightLabel{D\ref{D1}}
\UIC{$\omega: \forall \varphi.(P(\varphi) \imp \varphi(y))$}\RightLabel{$\forall_E$}
\UIC{$\omega: P(\rho) \imp \rho(y)$}\RightLabel{$\imp_E$}
\BIC{$\omega: \rho(y)$} \RightLabel{$\imp_I^4$}
\UIC{$\omega: G(y) \imp \rho(y)$}\RightLabel{$\forall_I$}
\UIC{$\omega: \forall y.(G(y) \imp \rho(y))$}\RightLabel{$\nec_I$}
\UIC{$\nec \forall y.(G(y) \imp \rho(y))$} \RightLabel{$\imp_I^5$}
\UIC{$\nec P(\rho)\imp \nec \forall y .(G(y)\imp \rho(y))$} \RightLabel{$\imp_E$}
\AXC{$ $} \RightLabel{6}
\UIC{$\rho(\gamma)$} \RightLabel{}
\AXC{$\subproof_7$} \dashedLine
\UIC{$\rho(\gamma)\imp\Box P(\rho)$}\RightLabel{$\imp_E$}
\BIC{$\nec P(\rho)$} \RightLabel{$\imp_E$}
\BIC{$\nec \forall y .(G(y)\imp \rho(y))$}\RightLabel{$\imp_I^6$}
\UIC{$\rho(\gamma) \imp \nec \forall y.(G(y) \imp \rho(y))$}\RightLabel{$\forall_I$}
\UIC{$\forall \psi.(\psi(\gamma) \imp \nec \forall y.(G(y) \imp \psi(y)))$}
\AXC{$\subproof_6$} \dashedLine
\UIC{$G(\gamma)$}  
\BIC{$G(\gamma) \wedge \forall \psi.(\psi(\gamma) \imp \nec \forall y.(G(y) \imp \psi(y)))$}\dottedLine \RightLabel{D\ref{D2}}
\UIC{$[\subproof_8] \qquad \ess{G}{\gamma}$}
\end{prooftree}
\end{small}

\begin{small}
\begin{prooftree}
\AXC{$\subproof_8$} \dashedLine
\UIC{$\ess{G}{\gamma}$}
\AXC{Axiom \ref{A5}} \dashedLine
\UIC{$P(E)$}\dottedLine \RightLabel{D\ref{D3}}
\AXC{$\subproof_6$} \dashedLine
\UIC{$G(\gamma)$}\dottedLine \RightLabel{D\ref{D1}}
\UIC{$\forall \varphi.(P(\varphi) \imp \varphi(\gamma))$}\RightLabel{$\forall_E$}
\UIC{$P(E) \imp E(\gamma)$}\RightLabel{$\imp_E$}
\BIC{$E(\gamma)$}\dottedLine \RightLabel{D\ref{D3}}
\UIC{$ \all \varphi.[\ess{\varphi}{\gamma} \imp \nec \ex x.\varphi(x)] $}\RightLabel{$\all_E$}
\UIC{$ \ess{G}{\gamma} \imp \nec \ex x. G(x) $}\RightLabel{$\imp_E$}
\BIC{$\nec \ex x. G(x)$} \RightLabel{$\imp_I^1$}
\UIC{$\ex z. G(z) \imp \nec \ex x. G(x)$}
\end{prooftree}
\end{small}
\end{proof}


Lemma 2 also turns out to be a superfluous detour and therefore the new proof does not include it. Inspecting Scott's proof of Lemma 2, we see that an important step is the derivation of $\nec \ex x. G(x)$ from $\pos \nec \ex x. G(x)$ and {\bf S5}'s iteration principle. Instead, we can derive $\ex x. G(x)$ from $\pos \nec \ex x. G(x)$ and Brouwer's reduction theorem and then derive $\nec \ex x. G(x)$ from $\ex x. G(x)$ by the $\nec_I$ (necessitation) inference rule. This is precisely what the new proof of Theorem 3 does.


\setcounter{theorem}{2}
\begin{theorem}
\label{T3}
Necessarily, there exists a God-like being:
$$
\nec \ex x. G(x)
$$
\end{theorem}

\begin{proof}

\begin{small}
\begin{prooftree}
\AXC{ Distribution Principle} \dashedLine
\UIC{$\nec [\ex x. G(x)\imp \nec \ex x. G(x)]\imp [\pos \ex x. G(x)\imp \pos\nec \ex x. G(x)]$}
\AXC{ Lemma \ref{L1} } \dashedLine
\UIC{$\ex x. G(x)\imp \nec \ex x. G(x)$}\RightLabel{$\nec_I$}
\UIC{$\nec[\ex x. G(x)\imp \nec \ex x. G(x)]$}\RightLabel{$\imp_E$}
\BIC{$\pos \ex x. G(x)\imp \pos\nec \ex x. G(x)$}
\end{prooftree}
\end{small}
\begin{small}
\begin{prooftree}
\AXC{$[\pos \ex x. G(x)\imp \pos\nec \ex x. G(x)]$}
\AXC{ Corollary \ref{C1} } \dashedLine
\UIC{$\pos \ex x. G(x)$}\RightLabel{$\imp_E$}
\BIC{$\pos\nec \ex x. G(x)$}
\AXC{ Brouwer's Reduction Principle } \dashedLine
\UIC{$\pos\nec \ex x. G(x) \imp \ex x. G(x)$}\RightLabel{$\imp_E$}
\BIC{$ \ex x. G(x)$} \RightLabel{$\nec_I$}
\UIC{$\nec \ex x. G(x)$} 
\end{prooftree}
\end{small}

\end{proof}


Since the iteration principle requires the strong modal logic {\bf S5} while Brouwer's reduction principle relies only the much weaker modal logic {\bf KB},  the possibility of proving Theorem 3 using Brouwer's reduction principle instead of {\bf S5}'s iteration principle is philosophically profound. There have been many discussions of what modal axioms are needed for the ontological argument. In their manuscripts, G\"odel and Scott do not write explicitly which logic they use. However, a proof step in their notes clearly relies on {\bf S5}'s \emph{iteration principle}.
Anderson \citep[footnote 5]{and90} cites several articles questioning the adequacy of {\bf S5} for the ontological argument and conjectures that the ontological argument could be restricted to the weaker modal logic {\bf B}. Sobel \citep{sobel2}[p. 152] acknowledges Anderson's conjecture, claims that it is correct and tries to informally explain how his formal proofs could be modified to rely on {\bf B} instead of {\bf S5}. Nevertheless, neither Anderson nor Sobel presented any formal proof relying only on {\bf B}. The new formal proof presented in Section \ref{sec:newproof} relies on {\bf KB}, which is weaker than {\bf B}. 

Curiously, in the new proof of Theorem 3, the actual existence of a God-like being ($\ex x. G(x)$) is proven \emph{before} its necessity ($\nec \ex x. G(x)$). This contrasts with G\"odel's proof, where the actual existence is never actually proven, although it can be derived as a trivial corollary of its necessity by applying the modal axiom T.




\section{Modal Collapse} \label{sec:collapse}

A major criticism against G\"odel's proof is that its axioms lead to the so-called \emph{modal collapse} \citep{sobel}: it is possible to prove that everything that is the case is so necessarily, and hence actuality, possibility and necessity coincide \citep{sobel2}[Ch. 4, section 6, theorems 9 and 10]. That is: for all propositions $A$, 
$$
A \biimp \pos A \biimp \nec A
$$

Below we show natural deduction derivations of the modal collapse, thereby confirming that it holds\footnote{It is well-known and uncontroversial that the modal collapse holds. We show a natural deduction proof here merely for the sake of self-containment and so that the proof becomes more accessible to readers who might be more familiarized with natural deduction than with Sobel's proof system.} for the axioms used in the previous sections. We show only the strongest implication above. The others are easy corollaries in modal logics where T holds. % ToDo: do we really need T?

\begin{theorem}
\label{Collapse}
For all propositions $A$, the following \emph{modal collapse} proposition is provable:
$$
A \imp \nec A
$$ 
\end{theorem}



\begin{small}
\begin{prooftree}
\AXC{Theorem 2}\dashedLine
\UIC{$\all y.[G(y) \imp \ess{G}{y}]$} \RightLabel{D2 }\doubleLine
\UIC{$\all y.[G(y) \imp G(y) \wedge \all \psi. (\psi(y) \imp \nec \all x. (G(x) \imp \psi(x)))]$}
\UIC{$\all y.[G(y) \imp \all \psi. (\psi(y) \imp \nec \all x. (G(x) \imp \psi(x)))]$} \doubleLine
\UIC{$\all y.[G(y) \imp (A(y) \imp \nec \all x. (G(x) \imp A(x)))]$} \RightLabel{$A$ is constant}
\UIC{$\all y.[G(y) \imp (A \imp \nec \all x. (G(x) \imp A))]$} \doubleLine
\UIC{$[\subproof_9] \qquad \ex y.G(y) \imp (A \imp \nec \all x. (G(x) \imp A))$}
\end{prooftree}
\end{small}



\begin{small}
\begin{prooftree}
\AXC{$ \subproof_9 $}\dashedLine
\UIC{$\ex y.G(y) \imp (A \imp \nec \all x. (G(x) \imp A))$}
\AXC{Theorem 3}\dashedLine
\UIC{$\nec \ex y.G(y)$} \RightLabel{$\nec_E$}
\UIC{$\ex y.G(y)$} \RightLabel{$\imp_E$}
\BIC{$ A \imp \nec \all x. (G(x) \imp A)$} \doubleLine
\UIC{$ A \imp \nec (\ex x. G(x) \imp A)$}
\AXC{$ $} \RightLabel{1}
\UIC{$A$} \RightLabel{$\imp_E$}
\BIC{$ \nec (\ex x. G(x) \imp A)$} \RightLabel{$\nec_E$}
\UIC{$ \omega:  \ex x. G(x) \imp A$} \RightLabel{$\imp_E$}
\AXC{Theorem 3}\dashedLine
\UIC{$\nec \ex y.G(y)$} \RightLabel{$\nec_E$}
\UIC{$\omega: \ex x.G(x)$} \RightLabel{$\imp_E$}
\BIC{$\omega: A$} \RightLabel{$\nec_I$}
\UIC{$ \nec A$} \RightLabel{$\imp_I^1$}
   \UIC{$A\imp \nec A$}
\end{prooftree}
\end{small}



In G\"odel's ontological proof, we are proving a restricted modal collapse, which applies to one specific formula, the existence of a God-like being. The interest in the proof naturally decreases if a consequence of the axiomatization is a modal collapse for \emph{all} formulas. Therefore, an improvement would be obtained if the modal collapse was limited only to the property of being god-like or at least to a restricted collection of properties. Several solutions to the problem of modal collapse have been proposed. 

Anderson's solution \citep{and90} modifies the definitions of God-like being and essence, and eliminates half of an axiom. This not only avoids the modal collapse, but also makes two of G\"odel's five axioms derivable from the others \citep{hajek} under some implicit additional assumptions \citep{fuhrmann}. Another solution involving more substantial modifications is that of Bj{\o}rdal \citep{bjordal,fuhrmann}. 

On another track, Fitting has argued that greater care has to be taken with the semantics of higher-order modal logics. Quantified variables may be rigid or flexible; and properties may be treated as intensional or extensional. Making the right choices may prevent the modal collapse \citep{fitting}[Sections 11.9 and 11.10].

Anderson \citep{and90}[p. 292] and Sobel \citep{sobel2}[p. 133] also discuss the idea that the notion of property over which quantification is allowed might be too general and restrictions might be appropriate.  

A reduction from {\bf S5} to {\bf KB} is possible for these variants as well, since the technique we used to replace {\bf S5}'s iteration principle by Brouwer's reduction principle depends only on the structure of the final steps of the proof of Theorem 3, but not on details of the axioms.


\section{A natural deduction proof based on Anderson's axiomatization}


\setcounter{axiom}{0}
\setcounter{lemma}{0}
\setcounter{theorem}{0}
\setcounter{corollary}{0}
\setcounter{definition}{0}

%Anderson modification of the axioms in A1 equivalence removed and definition of essence altered. 
%Anderson's implication instead of equivalence. 
\begin{axiom}
\label{A1}
Either a property or its negation is positive, but not both:
$$
\all \varphi. [P(\neg \varphi) \imp \neg P(\varphi)]
$$
\end{axiom}

\begin{axiom}
\label{A2}
A property necessarily implied by a positive property is positive:
$$
\all \varphi. \all \psi.[(P(\varphi) \wedge \nec \all x.[\varphi(x) \imp \psi(x)]) \imp P(\psi)]
$$
\end{axiom}


\begin{theorem}
\label{T1}
Positive properties are possibly exemplified:
$$
\all \varphi. [P(\varphi) \imp \pos \ex x.\varphi(x)]
$$
\end{theorem}

The proof is otherwise identical to the proof of theorem \ref{T1} in section \ref{sec:newproof} but the equivalence of axiom \ref{A1} does not need to be reduced to an implication by implication elimination. 




\begin{definition}[Anderson's alternative Godlike]
\label{D1}
%A \emph{God-like} being possesses all positive properties:
$$
G(x) \biimp \forall \varphi. [P(\varphi) \biimp \nec \varphi(x)]
$$
\end{definition}

\begin{axiom}
\label{A3}
The property of being god-like is positive:
$$
P(G)
$$
\end{axiom}

\begin{corollary}
\label{C1}
Possibly, a God-like being exists:
$$
\pos \ex x. G(x)
$$
\end{corollary}



\begin{axiom}
\label{A4}
Positive properties are necessarily positive:
$$
\all \varphi.[P(\varphi) \to \Box \; P(\varphi)]
$$
\end{axiom}

\begin{definition}[Anderson's alternative essence with Scott's additional conjunct]
\label{D2}
%sAn \emph{essence} of an individual is a property possessed by it and necessarily implying any of its properties:
$$
\ess{\varphi}{x} \biimp \varphi(x) \wedge \all \psi. (\nec \psi(x) \biimp \nec \all x. (\varphi(x) \imp \psi(x)))
$$
\end{definition}


%At this point, G\"odel and Scott proceed to prove Theorem 2, which is then used to prove Lemma 1. However, a close inspection of Scott's proof of Lemma 1 reveals that Theorem 2 is an unnecessary detour. Therefore, we do not include it in the new proof.


\begin{definition}[Anderson's alternative where G\"odel's definition of essence is replaced by def. \ref{D2}]
\label{D3}
\emph{Necessary existence} of an individual is the necessary exemplification of all its essences:
$$
E(x) \biimp \all \varphi.[\ess{\varphi}{x} \imp \nec \ex y.\varphi(y)]
$$
\end{definition}


\begin{axiom}
\label{A5}
Necessary existence is a positive property:
$$
P(E)
$$
\end{axiom}



\begin{lemma}
\label{L1}
If there is a God-like being, then there is a God-like being necessarily:
$$
\ex z. G(z) \imp \nec \ex x. G(x)
$$
\end{lemma}

\begin{proof} \hfill
\begin{prooftree}
\AXC{$ $} \RightLabel{1}
\UIC{$\ex z. G(z)$}\RightLabel{$\ex_E$}
\UIC{$[\subproof_6] \quad G(\gamma)$}
\end{prooftree}

\begin{scriptsize}
\begin{prooftree}
\AXC{$\subproof_6$} \dashedLine
\UIC{$G(\gamma)$}  \dottedLine \RightLabel{D\ref{D1}}
\UIC{$\forall \varphi. [P(\lambda x.\varphi(x)) \biimp \nec \varphi(\gamma)]$}\RightLabel{$\forall_E$}
\UIC{$P(\lambda x.\rho(x)) \biimp \nec \rho(\gamma)$}
\AXC{$ $} \RightLabel{5}
\UIC{$\nec \rho(\gamma)$} \RightLabel{$\imp_E$}
\BIC{$P(\lambda x.\rho(x))$} 
\AXC{$ $} \RightLabel{6}
\UIC{$G(y) $}\dottedLine \RightLabel{D\ref{D1}}
\UIC{$\forall \varphi. [P(\lambda x.\varphi(x)) \biimp \nec \varphi(y)]$}
\UIC{$[P(\lambda x.\rho(x)) \biimp \nec \rho(y)]$}
\BIC{$\nec \rho(y)$}\RightLabel{$\nec_E$}
\UIC{$\omega':  \rho(y)$}\RightLabel{$\imp_I^6$}
\UIC{$\omega':  G(y)\imp \rho(y)$}\RightLabel{$\forall_I$}
\UIC{$\omega':  \forall y. (G(y)\imp \rho(y))$}\RightLabel{$\nec_I$}
\UIC{$\nec \forall y. (G(y)\imp \rho(y))$}\RightLabel{$\imp_I^5$}
\UIC{$\nec \rho(\gamma)\imp \nec \forall y. (G(y)\imp \rho(y))$}
\end{prooftree}
\end{scriptsize}



\begin{scriptsize}
\begin{prooftree}
\AXC{$\nec P(\rho)\imp \nec \forall y .(G(y)\imp \rho(y))$} \RightLabel{$\imp_E$}
\AXC{$ $} \RightLabel{6}
\UIC{$\nec \rho(\gamma)$} \RightLabel{}
\UIC{$\rho(\gamma)$} \RightLabel{$\nec_E$}
\AXC{$\subproof_7$} \dashedLine
\UIC{$\omega': \rho(\gamma)\imp\Box P(\rho)$}\RightLabel{$\imp_E$}
\BIC{$\nec P(\rho)$} \RightLabel{$\imp_E$}
\BIC{$\nec \forall y .(G(y)\imp \rho(y))$}\RightLabel{$\imp_I^6$}
\UIC{$\rho(\gamma) \imp \nec \forall y.(G(y) \imp \rho(y))$}\RightLabel{$\forall_I$}
\UIC{$\forall \psi.(\psi(\gamma) \imp \nec \forall y.(G(y) \imp \psi(y)))$}
\AXC{$\subproof_6$} \dashedLine
\UIC{$G(\gamma)$}  
\BIC{$G(\gamma) \wedge \forall \psi.(\psi(\gamma) \imp \nec \forall y.(G(y) \imp \psi(y)))$}\dottedLine \RightLabel{D\ref{D2}}
\UIC{$[\subproof_8] \qquad \ess{G}{\gamma}$}
\end{prooftree}
\end{scriptsize}

%implication for equivalence in essence: 
\begin{small}
\begin{prooftree}
\AXC{$ $} \RightLabel{7}
\UIC{$\nec  \forall y. [G(y)\imp \varphi(y)]$}\RightLabel{$\nec_E$}
\UIC{$\omega'':   \forall y. [G(y)\imp \varphi(y)]$}\RightLabel{$\forall_E$}
\UIC{$\omega'':   G(\gamma)\imp \varphi(\gamma)$}
\AXC{$\subproof_6$} \dashedLine
\UIC{$\omega'':  G(\gamma)$}  \RightLabel{$\imp_E$}
\BIC{$\omega'':   \varphi(\gamma)$}\RightLabel{$\nec_I$}
\UIC{$\nec \varphi(\gamma)$}\RightLabel{$\imp_I^7$}
\UIC{$\nec  \forall y. [G(y)\imp \varphi(y)]\imp \nec \varphi(\gamma)$}
\end{prooftree}
\end{small}

\begin{small}
\begin{prooftree}
\AXC{$\subproof_8$} \dashedLine
\UIC{$\nec \varphi(\gamma)\biimp \nec  \forall y. [G(y)\imp \varphi(y)] $}
\AXC{$ $} \RightLabel{6}
\UIC{$G(x)$}\RightLabel{$\wedge_I$}
\BIC{$\ess{G}{\gamma}$}
\AXC{Axiom \ref{A5}} \dashedLine
\UIC{$P(E)$}\dottedLine 
\AXC{$\subproof_6$} \dashedLine
\UIC{$G(\gamma)$}\dottedLine \RightLabel{D\ref{D1}}
\UIC{$\forall \varphi.(P(\varphi) \biimp \nec \varphi(\gamma))$}\RightLabel{$\forall_E$}
\UIC{$P(E) \biimp \nec E(\gamma)$}\RightLabel{$\imp_E$}
\BIC{$\nec E(\gamma)$}\RightLabel{$\nec_E$ (reflexive frame)}
\UIC{$ E(\gamma)$} \dottedLine \RightLabel{D\ref{D3}}
\UIC{$ \all \varphi.[\ess{\varphi}{\gamma} \imp \nec \ex x.\varphi(x)] $}\RightLabel{$\all_E$}
\UIC{$ \ess{G}{\gamma} \imp \nec \ex x. G(x) $}\RightLabel{$\imp_E$}
\BIC{$\nec \ex x. G(x)$} \RightLabel{$\imp_I^1$}
\UIC{$\ex z. G(z) \imp \nec \ex x. G(x)$}
\end{prooftree}
\end{small}
\end{proof}



\setcounter{theorem}{2}
\begin{theorem}
\label{T3}
Necessarily, there exists a God-like being:
$$
\nec \ex x. G(x)
$$
\end{theorem}




\subsection{Why is the proof of modal collapse prevented?}

The alteration of the definition of essence in Anderson's proof prevents the standard proof of modal collapse \ref{Collapse} 
$$\vdash A \imp \nec A
$$ 
by the additional necessitation, which converts the proof of modal collapse into a proof of a tautology.
$$
\vdash \nec A \imp \nec A
$$ 
   
%For all propositions $A$, the following \emph{modal collapse} proposition is not provable with Anderson's alternative definitions with the same proof as G\"odel or Scott's:
 




\begin{small}
\begin{prooftree}
\AXC{Theorem 2}\dashedLine
\UIC{$\all y.[G(y) \imp \ess{G}{y}]$} \RightLabel{D2 }\doubleLine
\UIC{$\all y.[G(y) \imp G(y) \wedge \all \psi. (\nec \psi(y) \biimp \nec \all x. (G(x) \imp \psi(x)))]$}
\UIC{$\all y.[G(y) \imp \all \psi. (\nec \psi(y) \biimp \nec \all x. (G(x) \imp \psi(x)))]$} \doubleLine
\UIC{$\all y.[G(y) \imp (\nec A(y) \biimp \nec \all x. (G(x) \imp A(x)))]$} \RightLabel{$A$ is constant}
\UIC{$\all y.[G(y) \imp (\nec A \biimp \nec \all x. (G(x) \imp A))]$} \doubleLine
\UIC{$[\subproof_9] \qquad \ex y.G(y) \imp (\nec A \biimp \nec \all x. (G(x) \imp A))$}
\end{prooftree}
\end{small}



\begin{small}
\begin{prooftree}
\AXC{$ \subproof_9 $}\dashedLine
\UIC{$\ex y.G(y) \imp (\nec A \biimp \nec \all x. (G(x) \imp A))$}
\AXC{Theorem 3}\dashedLine
\UIC{$\nec \ex y.G(y)$} \RightLabel{$\nec_E$}
\UIC{$\ex y.G(y)$} \RightLabel{$\imp_E$}
\BIC{$ \nec A \biimp \nec \all x. (G(x) \imp A)$} \doubleLine
\UIC{$ \nec A \biimp \nec (\ex x. G(x) \imp A)$}
\AXC{$ $} \RightLabel{1}
\UIC{$\nec A$} \RightLabel{$\imp_E$}
\BIC{$ \nec (\ex x. G(x) \imp A)$} \RightLabel{$\nec_E$}
\UIC{$ \omega:  \ex x. G(x) \imp A$} \RightLabel{$\imp_E$}
\AXC{Theorem 3}\dashedLine
\UIC{$\nec \ex y.G(y)$} \RightLabel{$\nec_E$}
\UIC{$\omega: \ex x.G(x)$} \RightLabel{$\imp_E$}
\BIC{$\omega: A$} \RightLabel{$\nec_I$}
\UIC{$ \nec A$} \RightLabel{$\imp_I^1$}
   \UIC{$\nec A\imp \nec A$}
\end{prooftree}
\end{small}

The exchange of the implication in the G\"odel-Scott definition of essence for an equivalence and the additional  modal necessitation of the property $A$ prevents the implication of a necessary fact from its contingent counterpart as is seen in the second part of the proof above because the hypothetical assumption of the contingent property $A$ has to be exchanged for its necessitation for the proof structure to be preserved. 



\section{A natural deduction proof based on Bj\o rdal's axiomatization}


In Bj\o rdal's axiomatization godlikeness is taken as primitive, $G_B$. 

\begin{definition}
\label{B:D1}
A property is positive iff it is necessarily possessed by every God-like being.
$$
P_B(\varphi) \equiv \nec \all x. [G_B(x) \imp \varphi(x)]
$$
\end{definition}

\begin{definition}
\label{B:D2}
A maximal composite of an individual's positive properties is a positive property possessed by the individual and necessarily implying every positive property possessed by the individual.
$$MCP(\varphi, x) \equiv (\varphi(x) \wedge P_B (\varphi)) \wedge \all\psi.((\psi(x) \wedge P_B (\psi)) \imp \nec\all y.(\varphi(y) \imp \psi(y)))
$$
\end{definition}

\begin{definition}
\label{B:D3}
Necessary existence of an individual is the necessary exemplification of all its maximal composites.
$$
NE_B (x) \equiv \all\varphi. (MCP(\varphi, x) \imp \nec \ex y.\varphi(y))
$$
\end{definition}

\begin{axiom}
\label{A:A1'} If a property is positive, its negation is not positive: 
$$
\all \varphi. [P_B (\varphi) \imp \neg P_B (\neg\varphi)]
$$
\end{axiom}

\begin{axiom}
\label{A:A5'} Necessary existence is a positive property. 
$$
P_B (NE_B )
$$
\end{axiom}

The proof of the possible instantiation of $G_B$ is adjusted by proving Theorem 1 of section \ref{sec:newproof} directly for $G_B$ and not for properties in general. Corollary 1 of section \ref{sec:newproof} follows by proving the positivity of godlikeness as in lemma \ref{B:L1} below. 

\begin{theorem}
\label{T1}
Positive properties are possibly exemplified:
$$
\all \varphi. [P(G_B) \imp \pos \ex x.G_B(x)]
$$
\end{theorem}
\begin{proof} \hfill

\begin{prooftree}
\AXC{$ $} \RightLabel{2}
\UIC{$\nec \neg \ex x.G_B(x) $} \RightLabel{$\nec_E$}
\UIC{$\omega: \neg \ex x.G_B(x) $} 
\AXC{$ $} \RightLabel{$1$}
\UIC{$\omega: G_B(x)$} \RightLabel{$\ex_I$}
\UIC{$\omega: \ex x.G_B(x) $} \RightLabel{$\imp_E$}
\BIC{$\omega:  \bot $}\RightLabel{$\imp_I^1$}
\UIC{$\omega:  \neg G_B(x) $} \RightLabel{$\imp_I$}
\UIC{$\omega:  G_B(x) \imp \neg G_B(x) $} \RightLabel{$\all_I$}
\UIC{$\omega:  \all x.(G_B(x) \imp \neg G_B(x)) $}  \RightLabel{$\nec_I$}
\UIC{$ \nec \all x.(G_B(x) \imp \neg G_B(x)) $}\RightLabel{$\wedge_I$}
\UIC{$P(\neg G_B)$}
\AXC{Axiom 6***} \dashedLine
\UIC{$\all \varphi. [P(\neg G_B) \imp \neg P(G_B)]$} \RightLabel{$\all_E$}
\UIC{$ P( \neg G_B(x)) \imp \neg P(G_B) $}  \RightLabel{$\imp_E$}
\BIC{$\neg P(G_B) $} 
\AXC{$ $} \RightLabel{3} 
\UIC{$P(G_B)$} 
\BIC{$\bot$} \RightLabel{$\imp_I^2$}
\UIC{$ \neg \nec \neg \ex x.G_B(x)$} \doubleLine %\RightLabel{definition of $\pos$}
\UIC{$ \pos \ex x.G_B(x)$}  \RightLabel{$\imp_I^3$}
\UIC{$ P(\rho) \imp \pos \ex x.G_B(x) $} \RightLabel{$\all_I$}
\UIC{$ \all \varphi. [P(G_B) \imp \pos \ex x.G_B(x)] $} 

\end{prooftree}


\end{proof}

\begin{lemma}
\label{B:L1}
If there is a God-like being, then there is a God-like being necessarily:
$$
\ex z. G_B(z) \imp \nec \ex x. G_B(x)
$$
\end{lemma}

\begin{small}
\begin{prooftree}
\AXC{$�$}
\UIC{$G_B(x) $} 
\AXC{$ $}\doubleLine
\UIC{$\nec \all x. [G_B(x) \imp G_B(x)] $}\RightLabel{ D\ref{B:D1}}\dottedLine
\UIC{$P_B(G_B) $}\RightLabel{ $\wedge_I$}
\BIC{$G_B(x) \wedge P_B(G_B) $}
\end{prooftree}
\end{small}

\begin{small}
\begin{prooftree}
\AXC{$�$}\dashedLine
\UIC{$G_B(x) \wedge P_B(G_B) $}
\AXC{$�$}\RightLabel{ $1$}
\UIC{$(\psi(x) \wedge P_B (\psi))$} \RightLabel{ $\wedge_E$}
\UIC{$P_B (\psi)$}\RightLabel{ D\ref{B:D1}}\dottedLine
\UIC{$\nec\all y.(G_B(y) \imp \psi(y))$} \RightLabel{ $\imp_I^1$}
\UIC{$(\psi(x) \wedge P_B (\psi))\imp \nec\all y.(G_B(y) \imp \psi(y))$}\RightLabel{ $\forall_I$}
\UIC{$\all\psi.((\psi(x) \wedge P_B (\psi)) \imp \nec\all y.(G_B(y) \imp \psi(y)))$} \RightLabel{ $\wedge_I$}
\BIC{$MCP(G_B, x) $} 

\end{prooftree}
\end{small}


\begin{small}
\begin{prooftree}
\AXC{Axiom \ref{A:A5'}}\dashedLine
\UIC{$P_B (NE_B)$} \RightLabel{ D\ref{B:D1}}\dottedLine
\UIC{$\nec \all x. [G_B(x) \imp NE_B(x)]$}\RightLabel{ D\ref{B:D3}}\dottedLine
\UIC{$\nec \all x. [G_B(x) \imp \all\varphi. (MCP(\varphi, x) \imp \nec \ex y.\varphi(y))]$}  \RightLabel{ $\nec_E$}
\UIC{$\all x. [G_B(x) \imp \all\varphi. (MCP(\varphi, x) \imp \nec \ex y.\varphi(y))]$}  \RightLabel{ $\forall_E$}
\UIC{$ [G_B(x) \imp \all\varphi. (MCP(\varphi, x) \imp \nec \ex y.\varphi(y))]$}
\AXC{$�$} \RightLabel{ $2$}
\UIC{$G_B(x) $}  \RightLabel{ $\imp_E$}
\BIC{$\all\varphi. (MCP(\varphi, x) \imp \nec \ex y.\varphi(y))]$} \RightLabel{ $\forall_E$}
\UIC{$(MCP(G_B, x) \imp \nec \ex y.G_B(y))]$}
\AXC{$�$}\dashedLine
\UIC{$MCP(G_B, x) $} \RightLabel{ $\imp_E$}
\BIC{$\nec \ex y.G_B(y)$}\RightLabel{ $\imp_I^2$}
\UIC{$G_B(x) \imp \nec \ex y.G_B(y)$}
\end{prooftree}
\end{small}




%ToDo: acknowledgements


\begin{thebibliography}{9}

\bibitem[{\itshape Adams(1995)}]{adams}
Adams, R. M. 1995. Introductory note to [G\"odel 1970]. 

\bibitem[{\itshape Anderson(1990)}]{and90}
Anderson, C. A. 1990. {\itshape Some Emendations of G\"odel's Ontological Proof}. Faith and Philosophy, Vol. 7, Issue 3, pp. 291-303. 

\bibitem[{\itshape Anderson \& Gettings (1996)}]{and}
Anderson, C. A.\& Gettings, M. 1996.  {\itshape G\"odel's ontological proof revisited}. In: edited by Hajek P. {\itshape G\"odel '96},  Springer. 

\bibitem[{\itshape Bj{\o}rdal(1999)}]{bjordal}
Bj{\o}rdal, F. 1999. Understanding G\"odel's Ontological Argument, in Timothy Childers (ed.) The Logica Yearbook 1998, pp. 214-217, Filosofia.

\bibitem[{\itshape Church(1940)}]{church}
Church, A. 1940. {\itshape A Formulation of the Simple Theory of Types}, Journal of Symbolic Logic, 5: 56--68. 

\bibitem[{\itshape Fitting(2002)}]{fitting}
Fitting, M. 2002.  {\itshape Types, Tableaus, and G\"odel's God}, Kluwer Academic Publishers.  

\bibitem[{\itshape Fuhrmann(2005)}]{fuhrmann}
Fuhrmann, A. 2005.
Existenz und Notwendigkeit -- Kurt G\"odel's Axiomatische Theologie, in Olsson e.J., Schr\"oder-Heister, P. and Spohn, W. (eds.) Logik in der Philosophie, pp. 349-374, Synchr.-Wissenschafts-Verlag.

\bibitem[{\itshape Goedel(1970)}]{Goedel} 
G\"odel, K. Ontological Proof. In Kurt G\"odel Collected Works vol. III. Ed. Feferman et al., pp. 403--404, 139, 145 or {\itshape Appendix A. Notes in Kurt G\"odel's hand} in [Sobel 2001]. 

\bibitem[{\itshape Hajek(1996)}]{hajek}
Hajek, P. 1996. Magari and others on G\"odel's Ontological Proof, in Ursini et alii
(eds.) Logic and Logical Algebra, pp. 125--136, Marcel Dekker.

\bibitem[{\itshape Kant(1781)}]{kant}
Kant, I.  original 1781. {\itshape Critique of Pure Reason}, J. M. Dent \& Sons LTD, edition from 1959.

\bibitem[{\itshape Negri(2005)}]{negri}
Negri, S. 2005. {\itshape Proof analysis in modal logic}. J. Philosophical Logic, vol. 34, pp. 507--544. 

\bibitem[{\itshape Nipkow \& Paulson \& Wenzel(2002)}]{isabelle} Nipkow, T. \& Paulson, L.C. \& Wenzel, M. 2002. {\itshape Isabelle/HOL: A Proof Assistant for
Higher-Order Logic}. Number 2283 in LNCS, Springer.

\bibitem[{\itshape Paulin-Mohring(2015)}]{coq} Paulin-Mohring, C. 2015. {\itshape Introduction to the calculus of inductive constructions}. In %D. Delahaye and B. Woltzenlogel Paleo, editors, 
All about Proofs, Proofs for All,
Mathematical Logic and Foundations. College Publications, London.

\bibitem[{\itshape Prawitz(2006)}]{prawitz}
Prawitz, D. 2006 (1st ed. 1965). {\itshape Natural deduction: a proof-theoretical study}. Mineola, New York: Dover publications. 

\bibitem[{\itshape Scott(2001)}]{scott}
Scott, D. {\itshape Appendix B. Notes in Dana Scott's hand} in Logic and Theism: Arguments for and against Beliefs in God by Sobel, J. H. 

\bibitem[{\itshape Sobel(1987)}]{sobel}
Sobel, J. H. 1987. {\itshape G\"odel's Ontological Proof}. In: J. J. Thompson (ed.). {\itshape On being and saying: essays for Richard Cartwright},  MIT Press. 

\bibitem[{\itshape Sobel(2001)}]{sobel2}
Sobel, J. H. 2001. {\itshape Logic and Theism: Arguments for and against Beliefs in God}, Cambridge University Press. 

\bibitem[{\itshape Thiele(2003)}]{hilbert} Thiele, R. 2003. {\itshape Hilberts twenty-fourth problem}, American Mathematical Monthly, January. 

\bibitem[{\itshape Wang(1996)}]{Wang1996}
Wang, H. 1996. {\itshape A Logical Journey: From G\"odel to Philosophy}, The MIT Press. 








\end{thebibliography}


\end{document}


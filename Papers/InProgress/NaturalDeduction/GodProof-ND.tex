\documentclass{llncs}

\usepackage{fancybox}
\usepackage{latexsym}
\usepackage{proof}
\usepackage{bussproofs}
\usepackage{amsmath}
\EnableBpAbbreviations
\newcommand{\rl}[1]{\RightLabel{#1}}


\usepackage{calculi}
\usepackage{theorems}
\usepackage[numbers]{natbib}


% Logical symbols
\newcommand{\imp}{\rightarrow}
\newcommand{\biimp}{\leftrightarrow}
\newcommand{\all}{\forall}
\newcommand{\ex}{\exists}
\newcommand{\seq}{\vdash}
\newcommand{\nec}{\Box} % necessarily
\newcommand{\pos}{\Diamond} % possibly

\title{A Variant of G\"{o}del's Ontological Proof \\
in a Natural Deduction Calculus \\
(Draft)}


\author{Annika Siders\inst{1} \and Bruno Woltzenlogel Paleo\inst{2}}


\authorrunning{A.\~Siders \and B.\~Woltzenlogel Paleo}


\institute{
  Helsinki, Finland\\
  \email{annika.siders@helsinki.fi}
  \and 
  Vienna, Austria \\
  \email{bruno@logic.at}
}



\begin{document}

\maketitle

\newcommand{\ess}[2]{#1 \ \mathit{ess} \ #2}
\newcommand{\NE}{E}


\noindent
``There is a scientific (exact) philosophy and theology,
which deals with concepts of the highest abstractness; and this is also most highly fruitful for science. [\ldots] Religions are, for the most part, bad; but religion is not.'' - Kurt G\"{o}del

\section{Introduction}

Ontological arguments for the existence of God can be traced back at least to St. Anselm (1033-1109). His argument considered a greatest conceivable being, who must exist, because if it did not have the property of existence, then we could conceive of a greater being that, apart from the other properties, also has the property of existence. 

A major critique of this argument is that we do not know whether the concept greatest conceivable being in fact designates anything or if it is inconsistent, like a round square \citep{fitting}[p.134].  
%The question is if the argument is sound. 
As Bertrand Russell pointed out, the definition of greatest allows us to define properties, like having boots, which the greatest being then also must have \citep{citation needed}. 

Kant argued against the ontological argument on the basis that existence is not an analytic property \citep{kant}. This means that existence cannot be contained in the definition of a concept, because it is generally synthetic. All that we can say is that if God exists, then God necessarily exists \citep{citation needed}.

St. Anselm's argument was elaborated further by Descartes and Leibniz. Leibniz identified that establishing the possible existence of God is a critical missing step in St. Anselm's argument. To fill this gap, he argued that the properties of God, the perfections, are compatible. This implies that it is possible to have all perfections at once and therefore the existence of a greatest being with all these properties is possible. 

G\"odel studied Leibniz's work \citep{Adams} and brought his ontological argument to a modern form using a modal logic with higher-order quantification over properties. In this setting, he gave precise axioms describing the notion of \emph{positive} property and defined God as a being having all positive properties. 

G\"odel's notion of positive property and Leibniz's notion of perfection are not exactly the same \citep{Adams}. On a technical level, the main distinction seems to be that G\"odel's positive properties are not just atomic properties, like Leibniz's perfections, but can also consist of complex combinations of atomic properties \citep{fitting}[p.139.] (TODO: Check this claim). In particular, one of G\"odel's axioms states that the conjunction of any set of positive properties is positive. And from this axiom, it follows immediately that the property of being God-like is positive. While this step is intuitively and informally clear, it is not easily formalizable in a typical logical calculus, because it requires inferring that being god-like is a (possibly infinite) conjuntion of positive properties, while it has only been defined as the property holding for individuals who have all positive properties. This interplay between universal quantification (in the definition of a God-like being) and conjunction (in G\"odel's mentioned axiom) is a technical difficulty that probably explains why, starting with Scott \cite{scott}, to whom G\"odel confided his manuscript, this axiom of G\"odel's has been replaced by another one that simply assumes the positivity of the property of being God-like. 

The main criticism against G\"odel's ontological argument is the \emph{modal collapse}, an undesirable consequence of the stipulated axioms. Most recent works \cite{todo} on the ontological argument have focused on proposing modifications of the argument that would not entail a modal collapse. This is discussed in more detail in Section \ref{sec:collapse}.

%This axiom is unfortunately equivalent to the possibility of God's existence given the other axioms, leading to criticisms that it might be too strong \citep{sobel2}\citep{fitting}[Section 11.4. Objections.]. 

%The possibility  God's existence in turn is equivalent to  God's necessary existence as well as  God's existence itself. Thus, we have an axiom which seems to be equivalent to the conclusion. 


 % This axiom of the positiveness of conjunctions can be given a third order formulation which is the reason for the claim that third-order quantification may be required \citep{and}. %\citep{fitting}[p. 148.]   

% Another, possibly undesirable, consequence of the formalism is that there can only be one God, which is defined by the axioms. Namely, if one assumes that the theory contains the equality relation then a God has the property of being identical to itself. Since any other god-like being would have at least the same properties it shares the property of being identical to the one God. Thus, we have proved monotheism \citep{sobel2}[Ch. 4, section 3.3.2]. %\citep{fitting}[The proof is Exercise 7.1, p.163]. 

The first contribution of this paper is a detailed formalization of Scott's version \cite{Scott} of G\"odel's ontological argument \cite{Goedel} (as shown in Section \ref{sec:scott}) in a natural deduction calculus (as defined in Section \ref{sec:calculus}). A natural deduction style \citep{gentzen,prawitz} was chosen mainly for three reasons. Firstly, presentations of G\"odel's proof are typically either informal or formalized in other styles of calculi (e.g. Fitting's tableaux \cite{fitting} or Sobel's sentential modal calculus \cite{sobel2}). Therefore, a formalization in natural deduction is a valuable complement to the existing presentations. Secondly, it makes the ontological proof accessible to people who are familiarized with a natural deduction style. Finally, as natural deduction is the style used by proof assistants such as Coq \cite{coq} and Isabelle \cite{isabelle}, the natural deduction formalizations have been easily checked step-by-step in Coq \cite{coqformalizations}.

The main contribution of the paper, however, is the presentation of new proofs (also in natural deduction style) of the lemmas, theorems and corollaries in G\"odel's manuscript. In contrast to Scott's proofs \cite{Scott}, the proofs presented here are simpler and shorter, as discussed in Section \ref{sec:conclusion}.



\section{Natural Deduction}
\label{sec:calculus}

The language of higher-order modal logic used here is inspired by that of Church's simple type theory \citep{church}.

\begin{definition} \emph{Simple types} are given by the following grammar:
$$
  \theta,\tau \quad ::= \quad \mu \ \mid \ o \ \mid \ \theta \imp \tau
$$
where $\mu$ is the atomic type for individuals, $o$ is the atomic type for propositions and $\theta \imp \tau$ is the type for functions taking an argument of type $\theta$ and returning something of type $\tau$. `$\imp$' is assumed to be right associative.
\end{definition}

ToDo: decide which connectives and corresponding inference rules will be primitive.

\begin{definition} \emph{Terms and formulas} are given by the following grammar:
\begin{align*}
 s,t \quad ::= \quad & 
  p_\tau \ \mid \ 
  X_\tau \ \mid \
  (\lambda X_\theta.s_\tau)_{\theta\imp\tau} \ \mid \ 
  (s_{\theta\imp\tau}\, t_\theta)_\tau \ \mid \\
& \bot_o \ \mid \
  \imp_{o\imp o\imp o} \ \mid \ 
  \wedge_{o\imp o\imp o} \ \mid \
  \vee_{o\imp o\imp o} \ \mid \\
& \all_{(\tau\imp o)\imp o} \ \mid \ 
  \ex_{(\tau\imp o)\imp o} \ \mid \
  \nec_{o\imp o} \ \mid \
  \pos_{o\imp o}
\end{align*}
where $p_\tau$ and $X_\tau$ range over, respectively, constants and variables of type $\tau$. Parenthesis conventions, infix notation for propositional connectives and binding notation for quantifiers are assumed. Furthermore, subscript types are omitted when they are clear from the context. Negation ($\neg_{o\imp o}$) and equivalence ($\biimp_{o\imp o\imp o}$) are defined by $\neg A\equiv A\imp \bot$ and $ (A\biimp B)\equiv (A\imp B)\wedge (B\imp A)$.
\end{definition}

ToDo: decide if rules for diamond should be added.


The natural deduction calculus used here has standard rules for propositional connectives and quantifiers, as shown in Figures \ref{fig:PropositionalRules} and \ref{fig:QuantifierRules}. The extension to classical logic is achieved by adding a rule for double negation elimination, shown in Figure \ref{fig:Classical}. Finally, modal operators are handled by special rules that insert or remove formulas from boxes, as shown in Figure \ref{fig:NDK}. Apart from the use of labels and the dual rules for `$\pos$', these rules are essentially the modal rules from \cite{todo}. Beta-reduction is implicit; all rules are assumed modulo beta-reduction. A \emph{derivation} is a directed acyclic graph whose nodes are formulas and whose edges correspond to applications of the inference rules. A \emph{proof} of a formula $F$ is a derivation without open assumptions and having $F$ as root not inside any box. 

Double lines are used to abbreviate tedious propositional reasoning steps in the derivations. Dashed lines are used to refer to an axiom or theorem with proof shown elsewhere. Dotted lines are used to indicate folding and unfolding of definitions. Furthermore, as it is inconvenient to draw boxes around large derivations in \LaTeX, formulas inside boxes are labeled with the names of the boxes surrounding them. Therefore, the boxes can be omitted without loss of information. 

\newcommand{\s}{\qquad}

% ToDo: check if disjunction rules are used in Goedel's proof. If not, we could define disjunction as we already do for negation and equivalence.


\begin{calculus}
{propositional rules}
{fig:PropositionalRules}

\vspace{1em}

\s\s
\infer[\bot_E]{A}{ \bot }
\s
\infer[\imp_I]{A \imp B}{ B }
\s
\infer[\imp_I^n]{A \imp B}{ \infer*{B}{\infer[n]{A}{}} }
\s
\infer[\imp_E]{B}{A & A \imp B}

\vspace{2em}

\s\s
\infer[\wedge_I]{A \wedge B}{A & B}
\s\s
\infer[\wedge_{E_1}]{A}{A \wedge B}
\s\s
\infer[\wedge_{E_2}]{B}{A \wedge B}

% \vspace{2em}

% \s\s
% \infer[\vee_E]{C}{A \vee B & \infer*{C}{\infer{A}{}} & \infer*{C}{\infer{B}{}}}
% \s\s
% \infer[\vee_{I_1}]{A \vee B}{A}
% \s\s
% \infer[\vee_{I_2}]{A \vee B}{B}

\vspace{1em}
\end{calculus}

\begin{calculus}
{double negation elimination}
{fig:Classical}
\infer[\neg\neg_E]{A}{ \neg\neg A }
\end{calculus}


\begin{calculus}
{quantifier rules}
{fig:QuantifierRules}

\vspace{1em}

\s
\infer[\all_I]{\all x_{\tau}. A[x]}{ A[\alpha] }
\s
\infer[\all_E]{A[t]}{ \all x_{\tau}. A[x] }
\s\s
\infer[\ex_I]{\ex x_{\tau}. A[x]}{ A[t] }
\s
\infer[\ex_E]{A[\beta]}{ \ex x_{\tau}. A[x] }

\vspace{1em}

\begin{center}
\textbf{eigen-variable conditions:} \\
if $\rho$ is a $\all_I$ inference eliminating a variable $\alpha$, then any occurrence of $\alpha$ in the proof should be an ancestor of the occurrence of $\alpha$ eliminated by $\rho$; \\
if $\rho$ is a $\ex_E$ inference introducing a variable $\beta$, then any occurrence of $\beta$ in the proof should be a descendant of the occurrence of $\beta$ introduced by $\rho$.
\end{center}

\vspace{1em}

\end{calculus}



\begin{calculus}
{Rules for Modal Operators}
{fig:NDK}

\vspace{1em}

\s\s\s\s
\infer[\nec_I]{\nec A}{\omega: \fbox{\infer*{A}{}} }
\s\s\s\s
\infer[\nec_E]{w: \fbox{ \infer*{}{A} } }{\nec A}

% \vspace{2em}

% \s\s\s\s
% \infer[\pos_I]{\pos A}{w: \fbox{\infer*{A}{}} }
% \s\s\s\s\s
% \infer[\pos_E]{\omega: \fbox{ \infer*{}{A} } }{\pos A}

\vspace{1em}

\begin{center}
\textbf{eigen-box condition:}\\ 
$\omega$ must be a fresh name for a box. \\
\vspace{0.5em}
\textbf{boxed assumption condition:} \\
assumptions should be discharged within the box where they are made.
\end{center}

\vspace{1em}

\end{calculus}

%\clearpage

\noindent
The calculus having only the rules shown in Figures \ref{fig:PropositionalRules}, \ref{fig:Classical} and \ref{fig:QuantifierRules} is named \ND. The calculus with the additional rules shown in Figure \ref{fig:NDK} is called \NDK.

%\clearpage

\subsection{Suitability for Rigid Higher-Order Modal Logic K}

Adding the modal rules results in a calculus that is suitable for the basic normal modal logic K.
In other words, {\NDK} is sound and complete relative to {\ND} extended with axiom K ($\nec(A\imp B)\imp (\nec A\imp \nec B)$) and the necessitation rule (which establishes that $\nec A$ is a theorem if $A$ is a theorem).

The straightforward combinations of the quantifier rules of {\ND} either with the modal rules of {\NDK} or with axiom K and the necessitation rule are suitable for a higher-order modal logic where constants and variables are \emph{rigid}. From the point of view of a \emph{possible worlds} semantics, rigidity means that their interpretation is independent of the world in which they are being interpreted. Rigidity is silently assumed by most works investigating the ontological argument, and is explicitly assumed here. Nevertheless, it should be noted that its adequacy has already been contested \cite{fitting}.

ToDo: Talk about constant domains and varying domains, mention Conchiarella's semantics and Anderson's footnote 14.


\begin{theorem}
\label{theorem:completeness}
{\NDK} is complete, relative to {\ND} extended with axiom K and the necessitation rule.
\end{theorem}
\begin{proof}
The necessitation rule can be immediately simulated with the $\nec_I$ rule. Axiom K can be derived in {\NDK} as shown below:

\begin{small}
\begin{prooftree}
\AXC{$\nec(A\imp B)^2 $}\RightLabel{$\nec_E$}
\UIC{$\omega: A\imp B$}
      \AXC{$\nec A ^1 $}\RightLabel{$\nec_E$}
      \UIC{$\omega: A$} \RightLabel{$\imp_E$}
   \BIC{$ \omega: B$} \RightLabel{$\nec_I$}
   \UIC{$ \nec B$} \RightLabel{$\imp_I^1$}
   \UIC{$\nec A\imp \nec B$} \RightLabel{$\imp_I^2$}
   \UIC{$\nec(A\imp B)\imp (\nec A\imp \nec B)$}
\end{prooftree}
\end{small}


\end{proof}


\begin{theorem}
\label{theorem:soundness}
{\NDK} is sound, relative to {\ND} extended with axiom K and the necessitation rule.
\end{theorem}
\begin{proof}
It is necessary to show that {\NDK} proofs of the following form can be translated to proofs in {\ND} extended with the axiom K and the necessitation rule.

\begin{small}
$$
\infer[\nec_I]{\nec B}{
\infer{\omega: B}{
  \infer{\vdots}{\infer[\nec_E]{\omega: A_1}{\nec A_1}} &
  \ldots & 
  \infer{\vdots}{\infer[\nec_E]{\omega: A_n}{\nec A_n}}}
}
$$
\end{small}

\noindent
A translation to {\ND} extended with axiom K and necessitation is shown below for the case when $n=1$:

 % Assuming the axiom K and the necessitation rule $\Box_I$, the open formula $\nec A $ and the existence of a derivation of $ B$ from the open assumption $ A$, then we can derive $\nec B$ without the rules for boxed parts of derivations.

\begin{small}
\begin{prooftree}
\AXC{$ $}\RightLabel{1}
\UIC{$A_1$}\noLine
\UIC{$\vdots$}\noLine
\UIC{$ B$} \RightLabel{$\imp_I^1$}
\UIC{$A_1\imp B$} \RightLabel{necessitation}
\UIC{$\nec(A_1\imp B)$}
      \AXC{Axiom K}\dashedLine
      \UIC{$\nec(A_1\imp B)\imp (\nec A_1\imp \nec B)$} \RightLabel{$\imp_E$}
  \BIC{$\nec A_1\imp \nec B$}
        \AXC{$\nec A_1 $}\RightLabel{$\imp_E$}
    \BIC{$\nec B$}
\end{prooftree}
\end{small}

\noindent
For $n > 1$, the translation is a straightforward generalization:

\begin{scriptsize}
\begin{prooftree}
\AXC{$ $}\RightLabel{1}
\UIC{$A_1$}\noLine
\UIC{$\vdots$}
%
\AXC{$\ldots$}
%
\AXC{$ $}\RightLabel{n}
\UIC{$A_n$}\noLine
\UIC{$\vdots$} 
%
\TIC{$ B$} \doubleLine \RightLabel{$\imp_I^*$}
\UIC{$A_1\imp \ldots \imp A_n\imp B$} \RightLabel{nec.}
\UIC{$\nec(A_1\imp \ldots \imp A_n\imp B)$}
      \AXC{Axiom K, iterated}\doubleLine\dashedLine
      \UIC{$\nec(A_1\imp \ldots \imp A_n\imp B)\imp (\nec A_1\imp \ldots \imp \nec A_n\imp \nec B)$} \RightLabel{$\imp_E$}
  \BIC{$\nec A_1\imp \ldots \imp \nec A_n\imp \nec B$}
\end{prooftree}
\end{scriptsize}

\begin{scriptsize}
\begin{prooftree}
  \AXC{$ $} \dashedLine
  \UIC{$\nec A_1\imp \ldots \imp \nec A_n\imp \nec B$}
        \AXC{$\nec A_1 \quad \ldots \quad \nec A_n$} \doubleLine \RightLabel{$\imp_E$}
    \BIC{$\nec B$}
\end{prooftree}
\end{scriptsize}

\end{proof}





\section{Scott's Proof in Natural Deduction}

\setcounter{axiom}{0}
\setcounter{lemma}{0}
\setcounter{theorem}{0}
\setcounter{corollary}{0}

ToDo: natural language explanation of the proof in Anderson.

The acceptance of the correctness of the ontological argument by G\"odel's work boils down to the intuitive correctness of the axioms and definitions and the belief in the soundness of the deductive system. The formal argument of G\"odel is based on Leibniz proof, which in turn is based on Descartes proof. These proofs have two parts; a proof that if God's existence is possible, then it is necessary and a proof that God's existence is in fact possible. 


\begin{axiom}
\label{A1}
Either a property or its negation is positive, but not both:
$$
\all \varphi. [P(\neg \varphi) \biimp \neg P(\varphi)]
$$
\end{axiom}

\begin{axiom}
\label{A2}
A property necessarily implied by a positive property is positive:
$$
\all \varphi. \all \psi.[(P(\varphi) \wedge \nec \all x.[\varphi(x) \imp \psi(x)]) \imp P(\psi)]
$$
\end{axiom}


\section{New Proof}

\setcounter{axiom}{0}
\setcounter{lemma}{0}
\setcounter{theorem}{0}
\setcounter{corollary}{0}


\subsection{Possibly, God Exists}



\begin{axiom}
\label{A1}
Either a property or its negation is positive, but not both:
$$
\all \varphi. [P(\neg \varphi) \biimp \neg P(\varphi)]
$$
\end{axiom}

\begin{axiom}
\label{A2}
A property necessarily implied by a positive property is positive:
$$
\all \varphi. \all \psi.[(P(\varphi) \wedge \nec \all x.[\varphi(x) \imp \psi(x)]) \imp P(\psi)]
$$
\end{axiom}


\begin{theorem}
\label{T1}
Positive properties are possibly exemplified:
$$
\all \varphi. [P(\varphi) \imp \pos \ex x.\varphi(x)]
$$
\end{theorem}
\begin{proof} \hfill
\begin{prooftree}
        \AXC{Axiom \ref{A2}} \dashedLine
        \UIC{$ \all \varphi. \all \psi.[(P(\varphi) \wedge \nec \all x.[\varphi(x) \imp \psi(x)]) \imp P(\psi)]$} \RightLabel{$\all_E$}
        \UIC{$ \all \psi.[(P(\rho) \wedge \nec \all x.[\rho(x) \imp \psi(x)]) \imp P(\psi)]$} \RightLabel{$\all_E$}
        \UIC{$(P(\rho) \wedge \nec \all x.[\rho(x) \imp \neg \rho(x)]) \imp P(\neg \rho)$} \doubleLine
        \UIC{$(P(\rho) \wedge \nec \all x.[\neg \rho(x)]) \imp P(\neg \rho)$}
                        \AXC{Axiom \ref{A1}} \dashedLine
                        \UIC{$\all \varphi.[ P(\neg \varphi) \biimp \neg P(\varphi) ]$} \RightLabel{$\all_E$}
                        \UIC{$ P(\neg \rho) \biimp \neg P(\rho) $} \doubleLine
                 \BIC{$ (P(\rho) \wedge \nec \all x.[\neg \rho(x)]) \imp \neg P(\rho) $} \doubleLine
                 \UIC{$ P(\rho) \imp \pos \ex x.\rho(x) $} \RightLabel{$\all_I$}
                 \UIC{$\all \varphi.[ P(\varphi) \imp \pos \ex x.\varphi(x) ] $}
\end{prooftree}
\end{proof}

\begin{definition}
\label{D1}
A \emph{God-like} being possesses all positive properties:
$$
G(x) \biimp \forall \varphi. [P(\varphi) \to \varphi(x)]
$$
\end{definition}

\begin{axiom}
\label{A3}
The property of being God-like is positive:
$$
P(G)
$$
\end{axiom}

\begin{corollary}
\label{C1}
Possibly, God exists:
$$
\pos \ex x. G(x)
$$
\end{corollary}
\begin{proof} \hfill
\begin{prooftree}
\AXC{Axiom \ref{A3}} \dashedLine
\UIC{$P(G)$}
                 \AXC{Theorem \ref{T1}} \dashedLine
                 \UIC{$\all \varphi.[ P(\varphi) \imp \pos \ex x.\varphi(x) ]$} \RightLabel{$\all_E $}
                 \UIC{$ P(G) \imp \pos \ex x.G(x) $} \RightLabel{$\imp_E$}
    \BIC{$\pos \ex x. G(x)$}
\end{prooftree}
\end{proof}


\subsection{Being God is an essence of any God}

\begin{axiom}
\label{A4}
Positive properties are necessarily positive:
$$
\all \varphi.[P(\varphi) \to \Box \; P(\varphi)]
$$
\end{axiom}

\begin{definition}
\label{D2}
An \emph{essence} of an individual is a property possessed by it and necessarily implying any of its properties:
$$
\ess{\varphi}{x} \biimp \varphi(x) \wedge \all \psi. (\psi(x) \imp \nec \all x. (\varphi(x) \imp \psi(x)))
$$
\end{definition}


\begin{theorem}
\label{T2}
Being God-like is an essence of any God-like being:
$$
\all y.[G(y) \imp \ess{G}{y}]
$$
\end{theorem}
\begin{proof}
Let the following derivation with the open assumption $G(x)$ be $\Pi_1[G(x)]$:

\begin{prooftree}
\AXC{$ \neg P(\psi)^1$}
       \AXC{Axiom \ref{A1}} \dashedLine
       \UIC{$\forall \varphi.(\neg P(\varphi)\imp P(\neg\varphi))$}\RightLabel{$\all_E$}
       \UIC{$\neg P(\psi)\imp P(\neg\psi)$}\RightLabel{$\imp_E$}
 \BIC{$P(\neg\psi)$}\RightLabel{}
                   \AXC{$G(x)$} \dottedLine\RightLabel{D\ref{D1}}
                 \UIC{$\forall \varphi .(P(\varphi)\imp \varphi(x))$}\RightLabel{$\forall_E$}
                      \UIC{$P(\neg \psi)\imp \neg \psi(x)$}\RightLabel{$\imp_E$}
           \BIC{$ \neg \psi(x)$}\RightLabel{$\imp_E$}
            \AXC{$ \psi(x)^2$} \RightLabel{$\imp_E$}
             \BIC{$\bot $}\RightLabel{$\imp_I^1$}
             \UIC{$\neg \neg P(\psi)$}\RightLabel{$\neg\neg_E$}
              \UIC{$ P(\psi)$}\RightLabel{$\imp_I^2$}
              \UIC{$ \psi(x)\imp P(\psi)$}
\end{prooftree}


\noindent
Let the following derivation with the open assumption $G(x)$ be $\Pi_2[G(x)]$:
\begin{prooftree}
       \AXC{$\psi(x)^3$}
                     \AXC{$\Pi_1[G(x)] $} \dashedLine
                \UIC{$\psi(x)\imp P(\psi)$}\RightLabel{$\imp_E$}
                       \BIC{$P(\psi)$}
                 \AXC{Axiom \ref{A4}} \dashedLine
                \UIC{$\all \varphi .(P(\varphi)\imp \Box P(\varphi))$}\RightLabel{$\all_E$}
                 \UIC{$P(\psi)\imp \Box P(\psi)$}\RightLabel{$\imp_E$}
           \BIC{$\Box P(\psi)$}\RightLabel{$\imp_I^3$}
           \UIC{$\psi(x)\imp\Box P(\psi)$}
\end{prooftree}

\noindent
Let the following derivation without open assumptions be $\Pi_3$:

\begin{prooftree}
       \AXC{$P(\psi)^4$}
                     \AXC{$G(x)^5$} \dottedLine\RightLabel{D\ref{D1}}
                \UIC{$\all\varphi .(P(\varphi)\imp \varphi(x))$}\RightLabel{$\all_E$}
                \UIC{$P(\psi)\imp \psi(x)$}\RightLabel{$\imp_E$}
                       \BIC{$\psi(x)$}\RightLabel{$\imp_I^5$}
                       \UIC{$G(x)\imp \psi(x)$}\RightLabel{$\all_I$}
                \UIC{$\all x .(G(x)\imp \psi(x))$} \RightLabel{$\imp_I^4$}
                \UIC{$P(\psi)\imp \forall x .(G(x)\imp \psi(x))$}
\end{prooftree}

\noindent
Let the following derivation with the open assumption $G(x)$ be $\Pi_4[G(x)]$:
\begin{prooftree}
       \AXC{$\psi(x)^6 $} \RightLabel{}
                     \AXC{$\Pi_2$} \dashedLine
                \UIC{$\psi(x)\imp\Box P(\psi)$}\RightLabel{$\imp_E$}
                \BIC{$\Box P(\psi)$}
                         \AXC{$\Box P(\psi)^7 $}\RightLabel{$\Box_E$}
                               \UIC{$P(\psi)$}
                                \AXC{$\Pi_3$} \dashedLine\RightLabel{}
                                       \UIC{$P(\psi)\imp \all x .(G(x)\imp \psi(x))$}\RightLabel{$\imp_E$}
                               \BIC{$\all x .(G(x)\imp \psi(x))$}\RightLabel{$\Box_I$}
                        \UIC{$\Box \all x .(G(x)\imp \psi(x))$} \RightLabel{$\imp_I^7$}
                        \UIC{$\Box P(\psi)\imp \Box \all x .(G(x)\imp \psi(x))$} \RightLabel{$\imp_E$}
                  \BIC{$\Box \all x .(G(x)\imp \psi(x))$}\RightLabel{$\imp_I^6$}
           \UIC{$\psi(x)\imp\Box \all x .(G(x)\imp \psi(x))$}
\end{prooftree}


\noindent
The use of the necessitation rule above is correct, because the only open assumption $\Box P(\psi)$ is boxed. In the derivation of Theorem \ref{T2} below, the assumption $G(x)$ in the subderivation $\Pi_4[G(x)^8]$ is discharged by the rule labeled $8$.

\begin{prooftree}
       \AXC{$G(x)^8 $} \RightLabel{}
                     \AXC{$\Pi_4[G(x)^8]$} \dashedLine
                \UIC{$\psi(x)\imp\Box \all x .(G(x)\imp \psi(x))$}\RightLabel{$\all_I$}
                \UIC{$\all \psi .(\psi(x)\imp\Box \all x .(G(x)\imp \psi(x)))$}\RightLabel{$\wedge_I$}
                \BIC{$G(x)\wedge \forall \psi .(\psi(x)\imp\Box \all x .(G(x)\imp \psi(x)))$}\dottedLine\RightLabel{D\ref{D2}}
                               \UIC{$\ess{G}{x}$} \RightLabel{$\imp_I^8$}
                        \UIC{$G(x)\imp \ess{G}{x}$} \RightLabel{$\all_I$}
                         \UIC{$\all y.[G(y) \imp \ess{G}{y}]$}
\end{prooftree}
\end{proof}

\subsection{If God's existence is possible, it is necessary}

\begin{definition}
\label{D3}
\emph{Necessary existence} of an individual is the necessary exemplification of all its essences:
$$
E(x) \biimp \all \varphi.[\ess{\varphi}{x} \imp \nec \ex y.\varphi(y)]
$$
\end{definition}

\begin{axiom}
\label{A5}
Necessary existence is a positive property:
$$
P(E)
$$
\end{axiom}

\begin{lemma}
\label{L1}
If there is a God, then necessarily there exists a God:
$$
\ex z. G(z) \imp \nec \ex x. G(x)
$$
\end{lemma}
\begin{proof} \hfill
\begin{prooftree}
\AXC{$ $} \RightLabel{1}
\UIC{$\ex z. G(z)$}\RightLabel{$\ex_E$}
\UIC{$G(g)$}
\end{prooftree}

\begin{prooftree}
\AXC{$ $} \dashedLine
\UIC{$G(g)$}
        \AXC{Theorem \ref{T2}} \dashedLine
        \UIC{$\all y.[G(y) \imp \ess{G}{y}]$}\RightLabel{$\all_E$}
        \UIC{$G(g) \imp \ess{G}{g}$}\RightLabel{$\imp_E$}
    \BIC{$\ess{G}{g}$}
                \AXC{Axiom \ref{A5}} \dashedLine
                \UIC{$P(E)$}
                        \AXC{$ $} \dashedLine
                        \UIC{$G(g)$} \dottedLine \RightLabel{D\ref{D1}}
                        \UIC{$\all \varphi.[P(\varphi) \imp \varphi(g)]$}\RightLabel{$\all_E$}
                        \UIC{$P(E) \imp E(g)$}\RightLabel{$\imp_E$}
                     \BIC{$E(g)$} \dottedLine\RightLabel{D\ref{D3}}
                     \UIC{$ \all \varphi.[\ess{\varphi}{g} \imp \nec \ex x.\varphi(x)] $}\RightLabel{$\all_E$}
                     \UIC{$ \ess{G}{g} \imp \nec \ex x. G(x) $}\RightLabel{$\imp_E$}
        \BIC{$\nec \ex x. G(x)$} \RightLabel{$\imp_I^1$}
        \UIC{$\ex z. G(z) \imp \nec \ex x. G(x)$}
\end{prooftree}
\end{proof}


\subsection{God exists}

ToDo: this is proven in a way that is slightly different from G\"odel's 1970.


\begin{theorem}
\label{T4}
God exists:
$$
\ex x. G(x)
$$
\end{theorem}


\noindent
\textbf{Formal proof:}

\begin{small}
\begin{prooftree}
\AXC{$ $} \RightLabel{1}
\UIC{$\nec\neg\nec \ex x. G(x)$}\RightLabel{$\nec_E$, axiom T}
\UIC{$\neg\nec \ex x. G(x)$}
\AXC{ Lemma \ref{L1} } \dashedLine
\UIC{$\neg\nec \ex x. G(x) \imp \neg\ex x. G(x)$}\RightLabel{$\imp_E$}
 \BIC{$ \neg\ex x. G(x)$} \RightLabel{$\nec_I$}
\UIC{$\nec \neg\ex x. G(x)$} \dottedLine
\UIC{$\neg \pos\ex x. G(x)$}
\end{prooftree}
\end{small}

\begin{small}
\begin{prooftree}
\AXC{$ $}\dashedLine
\UIC{$\neg \pos\ex x. G(x)$}
\AXC{ Corollary \ref{C1}} \dashedLine
\UIC{$ \pos\ex x. G(x)$} \RightLabel{$\imp_E$}
 \BIC{$ \bot$} \RightLabel{$\imp_I^1$}
 \UIC{$\neg\nec\neg\nec \ex x. G(x)$} \dottedLine
 \UIC{$\neg\nec\pos \neg \ex x. G(x)$}
\AXC{ Axiom B} \dashedLine
\UIC{$ \neg \ex x. G(x)\imp\nec\pos \neg \ex x. G(x)$} \doubleLine
\UIC{$ \neg\nec\pos \neg \ex x. G(x)\imp \neg \neg \ex x. G(x)$} \RightLabel{$\imp_E$}
 \BIC{$ \neg \neg \ex x. G(x)$}\RightLabel{$ \neg \neg$ E}
 \UIC{$ \ex x. G(x)$}
\end{prooftree}
\end{small}

Note that the last step is classical and we do not prove the existential statement by providing an object for which the statement holds.
This proof makes section \ref{GodsE} superflous and the use of axiom "M" unnecessary.

The system used contains the $\nec_E$-rule with the restriction that we have a $\nec_I$ below it. This is equivalent to modal sytem K that contains axiom K and necessitation rule N. We aslo use axiom B ($ A\imp\nec\pos A$). No other modal axioms are needed.

\subsection{Necessarily, God exists}

We can also prove that god exists necessarily in our system by simply introducing box on a theorem by rule N in modal sytem K. So we get:


\begin{corollary}
\label{T3}
Necessarily, God exists:
$$
\nec \ex x. G(x)
$$
\end{corollary}




\subsection{God exists}\label{GodsE}
This section is superfluous.

\begin{axiom}[M]
\label{M} What is necessary is the case:
$$
\all \varphi. [\nec \varphi \imp \varphi]
$$
\end{axiom}

\begin{corollary}
\label{C2}
There exists a God:
$$
\ex x. G(x)
$$
\end{corollary}
\begin{proof}\hfill
\begin{prooftree}
\AXC{Theorem \ref{T3}} \dashedLine
\UIC{$ \nec \ex x. G(x)$}
        \AXC{$ \all \varphi. [\nec \varphi \imp \varphi]$}
        \UIC{$\nec \ex x. G(x) \imp \ex x. G(x)$}
    \BIC{$\ex x. G(x)$}
\end{prooftree}
\end{proof}





\section{Modal collapse}

A major criticism against G\"odel's proof is that its axioms lead to the so-called \emph{modal collapse} \citep{sobel}: it is possible to prove that everything that is the case is so necessarily, and hence actuality, possibility and necessity coincide \citep{sobel2}[Ch. 4,  section 6, theorems 9 and 10]. As this is an undesirable consequence, many solutions to the problem of the modal collapse have been proposed.

Anderson's solution \citep{citation needed} modifies the definitions of God-like being and essence, and eliminates half of an axiom. This not only avoids the modal collapse, but also makes two of G\"odel's five axioms derivable from the others \citep{hajek} under some implicit additional assumptions \citep{fuhrmann}. Another solution involving more substantial modifications is that of Bj{\o}rdal \citep{bjordal,fuhrmann}. 

On another track, Fitting has argued that greater care has to be taken with the semantics of higher-order modal logics. Quantified variables may be rigid or flexible; and properties may be treated as intensional or extensional. Making the right choices may prevent the modal collapse \citep{fitting}[Sections 11.9 and 11.10].

Anderson \citep{anderson1990}[p. 292] and Sobel \citep{sobel2}[p. 133] also discuss the idea that the notion of property over which quantification is allowed might be too general and restrictions might be appropriate.  

%In his manuscript, G\"odel warned that ``positive means positive in the moral aesthetic sense [\ldots]. Only then [are] the axioms true''. 

It is beyond the scope of this paper to analyze these solutions in detail or propose new solutions. The purpose of this section is simply to show natural deduction derivations of the modal collapse, thus confirming that it holds for the axioms used in the previous sections.


\begin{theorem}
\label{T5}
For all constant fomulas (without free variables), $A$, we have:
$$
A \imp \nec A
$$
\end{theorem}

Note that in intuitionistic predicate logic we have $\all y.[B \imp C] \biimp [\ex y.B \imp C]$ if $y$ is not free in $C$.

\begin{small}
\begin{prooftree}
\AXC{Theorem \ref{T2}}\dashedLine
\UIC{$\all y.[G(y) \imp \ess{G}{y}]$} \RightLabel{D2 }
\UIC{$\all y.[G(y) \imp G(y) \wedge \all \psi. (\psi(y) \imp \nec \all x. (G(x) \imp \psi(x)))]$} \RightLabel{Prop. logic }
\UIC{$\all y.[G(y) \imp \all \psi. (\psi(y) \imp \nec \all x. (G(x) \imp \psi(x)))]$} \RightLabel{Second order quantifier elimination and logical steps }
\UIC{$\all y.[G(y) \imp (A(y) \imp \nec \all x. (G(x) \imp A(x)))]$} \RightLabel{$A$ is constant}
\UIC{$\all y.[G(y) \imp (A \imp \nec \all x. (G(x) \imp A))]$} \RightLabel{intuitionistic predicate logic}
\UIC{$\ex y.G(y) \imp (A \imp \nec \all x. (G(x) \imp A))$}
\end{prooftree}
\end{small}
\begin{small}
\begin{prooftree}
\AXC{}\dashedLine
\UIC{$\ex y.G(y) \imp (A \imp \nec \all x. (G(x) \imp A))$}
\AXC{Theorem \ref{T4}}\dashedLine
\UIC{$\ex y.G(y)$} \RightLabel{$\imp_E$}
\BIC{$ A \imp \nec \all x. (G(x) \imp A)$} \RightLabel{intuitionistic predicate logic}
\UIC{$ A \imp \nec (\ex x. G(x) \imp A)$}
\AXC{$ $} \RightLabel{1}
\UIC{$A$} \RightLabel{$\imp_E$}
\BIC{$ \nec (\ex x. G(x) \imp A)$} \RightLabel{$\nec_E$}
\UIC{$ \ex x. G(x) \imp A$} \RightLabel{$\imp_E$}
\AXC{Theorem \ref{T4}}\dashedLine
\UIC{$\ex x.G(x)$} \RightLabel{$\imp_E$}
\BIC{$A$} \RightLabel{$\nec_I$}
\UIC{$ \nec A$} \RightLabel{$\imp_I^1$}
   \UIC{$A\imp \nec A$}
\end{prooftree}
\end{small}



% This collapse of the system was first proved in \citep{sobel} and the argument is also given in \citep{fitting}[ch. 11, sections 8-11] who provides summaries of two solutions by the modifications of the axioms. The argument may require that one doesn't instantiate the second order quantifier with $A$ directly but with $\lambda x.A$ to make $A$ a property. But still we have $(\lambda x.A)(y)\equiv A$ because $A$ is constant.


% The question arises what kind of second order quantification is allowed. The instantiations that are used in our proof are instantiated with predicate variables that later work as eigenvariables for the second order universal introduction. In the proof of corollary \ref{C1} we instantiate with $G$ and in the proof of lemma \ref{L1}. we instantiate with $G$ and $E$. %In Sobel on G\"{o}del's ontological proof by Robert E. Koons this question is raised.

% One can also discuss if a second order modal logic is sound and complete. With Henkin models, which restrict the second order quantification with comprehension, we have that normal second order logic is sound and complete. There is a completeness proof for a second order modal logic found in A completeness theorem in second order modal logic by Cocchiarella.


\section{Conclusions}

The proofs of theorem T1 and corollary C1 ($\pos \ex x. G(x)$) do not rely on equality and use axiom A2 ($\all \varphi. \all \psi.[(P(\varphi) \wedge \nec \all x.[\varphi(x) \imp \psi(x)]) \imp P(\psi)]$) only once. 
Corollary C2 ($\ex x. G(x)$), which is usually regarded as a trivial corollary of main theorem T3 ($\nec \ex x. G(x)$) using the modal logic axiom T, is here derived directly from lemma L1 and corollary C1, not relying on T. The main theorem T3 becomes derivable from C2 by a single application of the necessitation rule. Furthermore, all proofs are done in the modal logic K, except for the proof of corollary C2, which requires one use of the axiom B of modal logic KB.

ToDo: Sobel Anderson on KB: Sobel page 152

Anderson's footnote 5 on KB

ToDo: talk about anderson's footnote 2, where he argues that an anonymous referee knew the simpler proof of T1.

Hajek (magari and others) on non-provability of ex x. G(x) in KD45.


\begin{thebibliography}{9}


\bibitem[{\itshape Anderson \& Gettings(1996)}]{and}
Anderson, C. A.\& Gettings, M. 1996.  {\itshape G\"odel's ontological proof revisited}. In: edited by Hajek P. {\itshape G\"odel '96},  Springer. 


\bibitem[{\itshape Fitting(2002)}]{fitting}
Fitting, M. 2002.  {\itshape Types, Tableaus, and G\"odel's God}, Kluwer Academic Publishers.  

\bibitem[{\itshape Kant(1781)}]{kant}
Kant, I.  original 1781.   {\itshape Critique of Pure Reason}, J. M. Dent \& Sons LTD, edition from 1959.

\bibitem[{\itshape Sobel(1987)}]{sobel}
Sobel, J. H. 1987. {\itshape G\"odel's Ontological Proof}. In: edited by J. J. Thompson. {\itshape On being and saying : essays for Richard Cartwright},  MIT Press. 

\bibitem[{\itshape Sobel(2001)}]{sobel2}
Sobel, J. H. 2001. {\itshape Logic and Theism: Arguments for and against Beliefs in God}, Cambridge University Press. 



\end{thebibliography}




\end{document}


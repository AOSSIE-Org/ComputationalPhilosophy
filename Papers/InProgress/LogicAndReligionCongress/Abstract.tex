% ------------------------------------------------------------------------
% bjourdoc.tex for birkjour.cls*******************************************
% ------------------------------------------------------------------------
%%%%%%%%%%%%%%%%%%%%%%%%%%%%%%%%%%%%%%%%%%%%%%%%%%%%%%%%%%%%%%%%%%%%%%%%%%

\documentclass{birkjour}

\usepackage[utf8]{inputenc}

\usepackage{amsmath}
\usepackage{xspace}
\usepackage{url}
\usepackage{commands}
\usepackage{xcolor}
\usepackage{comment}
\usepackage{modallogics}
\usepackage[backend=bibtex,sorting=ydnt,maxnames=999,maxcitenames=999,maxbibnames=999,natbib]{biblatex}
\addbibresource[label=bibs]{Bibliography.bib}


\newcommand{\AOE}{$\mathcal{AOE}$}
\newcommand{\AOEH}{$\mathcal{AOE}'$}
\newcommand{\AOEHH}{$\mathcal{AOE}'_0$}
\newcommand{\AOEHHH}{$\mathcal{AOE}''$}



%
%
% THEOREM Environments (Examples)-----------------------------------------
%
 \newtheorem{thm}{Theorem}[section]
 \newtheorem{cor}[thm]{Corollary}
 \newtheorem{lem}[thm]{Lemma}
 \newtheorem{prop}[thm]{Proposition}
 \theoremstyle{definition}
 \newtheorem{defn}[thm]{Definition}
 \theoremstyle{remark}
 \newtheorem{rem}[thm]{Remark}
 \newtheorem*{ex}{Example}
 \numberwithin{equation}{section}

\def\HOML{\entity{HOML}\xspace}
\def\HOL{\entity{HOL}\xspace}


\begin{document}

%-------------------------------------------------------------------------
% editorial commands: to be inserted by the editorial office
%
%\firstpage{1} \volume{228} \Copyrightyear{2004} \DOI{003-0001}
%
%
%\seriesextra{Just an add-on}
%\seriesextraline{This is the Concrete Title of this Book\br H.E. R and S.T.C. W, Eds.}
%
% for journals:
%
%\firstpage{1}
%\issuenumber{1}
%\Volumeandyear{1 (2004)}
%\Copyrightyear{2004}
%\DOI{003-xxxx-y}
%\Signet
%\commby{inhouse}
%\submitted{March 14, 2003}
%\received{March 16, 2000}
%\revised{June 1, 2000}
%\accepted{July 22, 2000}
%
%
%
%---------------------------------------------------------------------------
%Insert here the title, affiliations and abstract:
%


\title[The Anderson-Hájek Ontological Controversy]
{Computer-Assisted Analysis of the \\ 
Anderson-Hájek Ontological Controversy}



%----------AUTHOR 1
\author[Benzm\"uller]{C. Benzm\"uller}

\address{%
Dep.~of Mathematics and Computer Science, Freie Universit\"at Berlin, Germany
}
\email{c.benzmueller@fu-berlin.com}

\thanks{This work was supported by German National Research Foundation (DFG) under
 grants BE 2501/9-1 and BE 2501/11-1.}


%----------Author 2
\author[Weber]{L. Weber}
\address{ 
Dep.~of Mathematics and Computer Science, Freie Universit\"at Berlin, Germany
}
\email{leon.weber@fu-berlin.de}


%----------Author 3
\author[Woltzenlogel-Paleo]{B. Woltzenlogel Paleo}
\address{ 
Room HA0402, Favoritenstra{\ss}e 9, 1040 Wien, Austria
}
\email{bruno.wp@gmail.com}




%----------classification, keywords, date
% \subjclass{
% % Prim. 03A02;  % Philosophical aspects of logic and foundations
% % Sec. 68T02 % Artificial Intelligence 
% }

%\keywords{Ontological Argument, Modal Collapse}

\date{\today}
%----------additions
\dedicatory{ }
%%% ----------------------------------------------------------------------

% \begin{abstract}
% \end{abstract}

%%% ----------------------------------------------------------------------
\maketitle
%%% ----------------------------------------------------------------------
%\tableofcontents


\noindent The axioms in G\"odel's ontological proof
\citep{GoedelNotes,ScottNotes} (cf. Appendix \ref{apx:Goedel}) entail
what is called \emph{modal collapse}
\citep{Sobel1987,SobelBook2004}: the formula $\varphi \rightarrow \Box
\varphi$, abbreviated as MC, holds for any formula $\varphi$ and not
just for $\exists x. \mathit{God}(x)$ as intended. This fact, which
has recently been confirmed with higher-order automated theorem
provers \citep{C40,J30}, has led to strong criticism of the argument
and stimulated attempts to remedy the problem.
\citet{Hajek_Magari_and_others_1996,Hajek_der_Mathematiker_2001}
proposed the use of cautious instead of full comprehension principles,
and \citet{fitting02:_types_tableaus_god} took
greater care of the semantics of higher-order quantifiers
in the presence of modalities. Others, such as
\citet{anderson90:_some_emend_of_goedel_ontol_proof}, \citet{Hajek2002}
and \citet{bjordal99}, proposed emendations of G\"odel's
axioms and definitions. They require neither comprehension
restrictions nor more complex semantics. Therefore, they are
technically simpler to analyze with computer support. We have
formalized them using the proof assistant Isabelle/HOL \citep{Isabelle}
together with the automated higher-order reasoners Leo-II \citep{C26},
Satallax \citep{brown2012satallax}, Metis 
\citep{Hurd03first-orderproof}, and Nitpick \citep{Nitpick}.   Our
formalizations\footnote{The   formalizations are available in the
subdirectories \url{Anderson}, \url{Hajek} and \url{Bjordal} at \url{gi
thub.com/FormalTheology/GoedelGod/blob/master/Formalizations/Isabelle/
}.} employ the embedding of higher-order modal logic (HOML) in
classical higher-order logic (HOL) as introduced in previous work
\citep{C40,J30,J23}. We explored the effect of different domain
conditions on the provability of lemmas, theorems and even axioms.
This was motivated by a controversy between Hájek and Anderson
regarding the redundancy of some axioms in Anderson's emendation. In
\emph{constant domain semantics}, the individual domains are the same
in all possible worlds. In \emph{varying domain semantics}, the
domains may vary from world to world. This variation is technically
encoded with the help of an existence relation expressing which
individuals actually exist in each world. Quantifiers are then
uniformly formalized as \emph{actualistic quantifiers} (i.e. guarded by
the existence relation). Our main results are summarized here.

For all emendations and variants discussed here, the axioms and
definitions have been shown to be consistent and not to entail modal
collapse.

For both \textbf{constant domain semantics} and \textbf{varying domain
semantics}, the following results hold for \emph{Anderson's
Emendation} (cf. Appendix \ref{apx:Anderson}): T1, C
and T3' can be quickly automated (in logics \K, \K and \KB,
respectively); the axioms A4 and A5 are proven redundant\footnote{
An axiom $A$ is redundant w.r.t. a set of axioms $S$ iff $A$ is
entailed by $S$.  } (the former in logic \KFourB and the latter
already in \K); a trivial countermodel (with two worlds and  two
individuals) for MC is generated by Nitpick (for all mentioned
logics); all axioms and definitions are shown to be mutually
consistent.

The redundancy of A4 and A5 is particularly controversial. Magari
\cite{Magari1988} claimed that A4 and A5 are superfluous\footnote{
An axiom $A$ is superfluous w.r.t. a set of axioms $S$ iff T3 is
entailed by $S \setminus \{ A \}$. }, arguing that T3 is true in all
models of the other axioms and definitions by Gödel.
\citet[p.~5-6]{Hajek_Magari_and_others_1996} investigated this
further, and claimed that Magari's claim is not valid, but is
nevertheless true  under additional silent assumptions by Magari.
Moreover, \citet[p.~2]{Hajek_Magari_and_others_1996} cites his earlier
work\footnote{Although \citep{Hajek_der_Mathematiker_2001} precedes
\citep{Hajek_Magari_and_others_1996} in writing, it was published only
5 years later, in German.} \citep{Hajek_der_Mathematiker_2001}, where
he claims (in Theorem 5.3) that for Anderson's emended theory
\citep{anderson90:_some_emend_of_goedel_ontol_proof}, A4 and A5 are
not only superfluous, but also redundant. \citet[footnote 1 in
p.~1]{AndersonGettings}, in a footnote, rebutted Hájek's claim,
arguing that the redundancy of A4 and A5 holds only under constant
domain semantics, while Anderson's emended theory ought to be taken
under Cocchiarella's semantics \citep{Cocchiarella} (a varying domain
semantics). Our results show that Hájek was originally right, under
both constant and varying domain semantics.

Nevertheless, \citet[p.~7]{Hajek2002} acknowledges Anderson's rebuttal, 
and apparently accepts it, as evidenced by his use\footnote{
  A4 and A5 are used by \citet[p.~11]{Hajek2002} in, 
  respectively, Lemma 4 and Theorem 4.
} 
of A4 and A5', as well as varying domain semantics, in his new
emendation (named \AOEH \citep[sec.~4]{Hajek2002}, cf. Appendix
\ref{apx:Hajek1}), which replaces Anderson's A:A1 and A2 by a simpler
axiom H:A12. Surprisingly, the computer-assisted formalization of
\AOEH\ shows that A4 and A5' are still superfluous. Moreover, 
A4 and A5' are independent\footnote{
  An axiom $A$ is \emph{independent} of a set of axioms $S$ iff 
  there are models of $S$ where $A$ is true and other models of 
  $S$ where $A$ if false. 
} 
of the other axioms and definitions. 
Therefore, A4 and A5' are not redundant, despite their superfluousness. 

Although Hájek did not notice the superfluousness of A4 and A5' in his
\AOEH, he did describe yet another emendation (his \AOEHH, cf.
Appendix \ref{apx:Hajek2}) where A4 and A5' are superfluous (though no
claim is made w.r.t. to their redundancy), if A3' is replaced by a
stronger axiom (H:A3) additionally stating that the property of actual
existence is positive when it comes to God-like beings
\cite[sec.~5]{Hajek2002}. Formalization of \AOEHH\ shows that A4 is
not only superflous, but also redundant. For A5', no conclusive results
were achieved; neither a proof nor a countermodel could be
automatically generated. Surprisingly, a countermodel for the weaker
A3' was succesfully generated. This is somewhat unsatisfactory (for
theistic goals), because it shows that \AOEHH\ does not entail the
positiveness of being God-like.

Nevertheless, \AOEHH\ is explicitly regarded by
\citet[p.~12]{Hajek2002} as just an intermediary step towards a more
natural theory, based on a more sophisticated notion of positiveness.
That is his final emendation (\AOEHHH, cf. \ref{apx:Hajek3}), which
restores A3' and does use A4 and A5', albeit in a modified form (i.e.
H:A4 and H:A5). The formalization of \AOEHHH\ shows that H:A4 is
independent. The old A5' is independent as well, and both H:A4 and H:A5
are superfluous, but no conclusive results were achieved regarding
independence or redundancy of H:A5.

Additionally, Anderson \citep[footnote
14]{anderson90:_some_emend_of_goedel_ontol_proof} remarks that only
the quantifiers in T3' and in A:D2 need to be interpreted as
actualistic quantifiers, while others may be taken as possibilistic
quantifiers. Our computer-assisted study of this mixed variant shows
that A4 is still redundant, but A5' becomes independent (hence not
redundant). Unfortunately, a countermodel for T3 can then be found.

The controversy over the superfluousness of A4 and A5 indicates a
trend to reduce the ontological argument to its bare essentials. In
this regard,  already
\citet[p.~7]{anderson90:_some_emend_of_goedel_ontol_proof} indicates
that, by taking a notion of \emph{defective} as primitive and defining
the notion of \emph{positive} upon it, axioms A:A1, A2 and A4 become
derivable. These claims have been confirmed by the automated theorem
provers (in logic \KFourB). Within the same trend, the alternative
proposed by \citet{bjordal99} (cf. Appendix \ref{apx:Bjordal})
achieves a high level of minimality. He takes the property of being
God-like as a primitive and defines (B:D1) the positive properties as
those properties necessarily possessed by every God-like being. He
then briefly indicates (B:L1) that B:D1 is logically equivalent, under
modal logic \SFour, to the conjunction of D1', A2', A3' and A4'. This
has been confirmed in the computer-assisted formalization: A2' and A3'
can be quickly automatically derived in logic \K. A4' can be proved in
logic \KT (i.e. assuming reflexivity of the accessibility relation).
For constant domain semantics, proving D1 is possible in logic \KFour,
whereas for varying domain semantics, a counter model can be found
even in logic \SFive. Conversely, the proof that B:D1 is entailed by
D1', A2', A3' and A4' is possible already in logic \K. The provers
also show that theorem T3' follows from B:D1, B:D2, B:D3, A:A1' and
A5' already in logic \KB. Bjordal's last paragraph briefly mentions
Hájek's ideas about the superfluousness of A5' and claims that it is
possible, with (unclear) additional modifications of the definitions,
to eliminate A5' from his theory as well. Without any additional
modification, the automated reasoners show that A5' is actually not
superfluous. All these results, with the exception of the
aforementioned countermodel for D1', hold for both constant and
varying domain semantics.

Using our approach, the formalization and (partly) automated analysis
of several variants of G\"odel's ontological argument has been
surprisingly straightforward. The provers not only confirmed many
claimed results, but also exposed a few mistakes and novel insights
in a long standing controversy. 
We believe the technology employed in
this work is ready to be  fruitfully adopted in larger scale by
philosophers.

% ToDo: \\

% - Investigate Magari's proof

% - Hájek actually uses multi-sorted first-order formalizations. We should check that our results still hold in a multi-sorted first-order setting. They should.

% - Add apendix for Anderson's simplification

\sloppy 
\printbibliography


\newpage
\begin{appendix}
\noindent 


\small

\section{Scott's version of G\"odel's ontological argument} \label{apx:Goedel} 

\framebox[\columnwidth][r]{
\begin{minipage}{.94\columnwidth}\small
\begin{itemize}
\item[A1] Either a property or its negation is positive, but not
  both:
  $$\hol{\allq \varphi [P(\neg \varphi) \biimp \neg P(\varphi)]}$$ 
\item[A2] A property necessarily implied by a
  positive property is positive:
  $$\hol{\allq \varphi \allq \psi [(P(\varphi) \wedge \nec \allq x [\varphi(x)
  \imp \psi(x)]) \imp P(\psi)]}$$
\item[T1] Positive properties are possibly exemplified: 
  $$\hol{\allq \varphi [P(\varphi) \imp \pos \exq x \varphi(x)]}$$ 
\item[D1] A \emph{God-like} being possesses all positive properties: 
  $$\hol{G(x) \equiv \forall \varphi [P(\varphi) \imp \varphi(x)]}$$ 
\item[A3]  The property of being God-like is positive: 
  $$\hol{P(G)}$$
\item[C\phantom{1}] Possibly, a God-like being exists: $$\hol{\pos \exq x G(x)}$$
\item[A4]  Positive properties are necessarily positive: 
  $$\hol{\allq \varphi [P(\varphi) \imp \Box \; P(\varphi)]}$$ 
\item[D2] An \emph{essence} of an individual is a property possessed by it and necessarily implying any of its properties: $$\hol{\ess{\varphi}{x} \equiv \varphi(x) \wedge \allq
  \psi (\psi(x) \imp \nec \allq y (\varphi(y) \imp \psi(y)))}$$ 
\item[T2]  Being God-like is an essence of any
  God-like being: $$\hol{\allq x [G(x) \imp \ess{G}{x}]}$$
\item[D3] \emph{Necessary existence} of an individual is the necessary exemplification of all its essences: 
  $$\hol{\NE(x) \equiv \allq \varphi [\ess{\varphi}{x} \imp \nec
  \exq y \varphi(y)]}$$
\item[A5] Necessary existence is a positive property: $$\hol{P(\NE)}$$ 
\item[L1] If a god-like being exists, then necessarily a god-like being exists: 
  $$\hol{\exq x G(x) \imp \nec \exq y G(y)}$$
\item[L2] If possibly a god-like being exists, then necessarily a god-like being exists: 
  $$\hol{\pos \exq x G(x) \imp \nec \exq y G(y)} $$
%
\item[T3] Necessarily, a God-like being exists: $$\hol{\nec \exq x G(x)}$$ 
\end{itemize}
\end{minipage}
} %\vskip-.5em

\clearpage


\section{Anderson's Emendation} \label{apx:Anderson}

\framebox[\columnwidth][r]{
\begin{minipage}{.94\columnwidth}\small
\begin{itemize}
\item[A:A1] If a property is positive, its negation is not positive:
  $$\hol{\allq \varphi [P(\varphi) \imp \neg P(\neg \varphi)]}$$ 
\item[A2] A property necessarily implied by a
  positive property is positive:
  $$\hol{\allq \varphi \allq \psi [(P(\varphi) \wedge \nec \allq x [\varphi(x)
  \imp \psi(x)]) \imp P(\psi)]}$$
\item[T1] Positive properties are possibly exemplified: 
  $$\hol{\allq \varphi [P(\varphi) \imp \pos \exq x \varphi(x)]}$$ 
\item[A:D1] A \emph{God-like} being necessarily possesses those and only those properties that are positive: 
  $$\hol{G_A(x) \equiv \forall \varphi [P(\varphi) \biimp \nec \varphi(x)]}$$ 
\item[A3']  The property of being God-like is positive: 
  $$\hol{P(G_A)}$$
\item[C\phantom{1}] Possibly, a God-like being exists: $$\hol{\pos \exq x G(x)}$$
\item[A4]  Positive properties are necessarily positive: 
  $$\hol{\allq \varphi [P(\varphi) \imp \Box \; P(\varphi)]}$$ 
\item[A:D2] An \emph{essence} of an individual is a property that necessarily implies those and only those properties that the individual has necessarily: $$\hol{\essA{\varphi}{x} \equiv \allq
  \psi [\nec \psi(x) \biimp \nec \allq y (\varphi(y) \imp \psi(y))]}$$ 
\item[T2']  Being God-like is an essence of any
  God-like being: $$\hol{\allq x [G_A(x) \imp \essA{G_A}{x}]}$$
\item[D3'] \emph{Necessary existence} of an individual is the necessary exemplification of all its essences: 
  $$\hol{\NE_A(x) \equiv \allq \varphi [\essA{\varphi}{x} \imp \nec
  \exq y \varphi(y)]}$$
\item[A5'] Necessary existence is a positive property: $$\hol{P(\NE_A)}$$ 
\item[L1'] If a god-like being exists, then necessarily a god-like being exists: 
  $$\hol{\exq x G_A(x) \imp \nec \exq y G_A(y)}$$
\item[L2'] If possibly a god-like being exists, then necessarily a god-like being exists: 
  $$\hol{\pos \exq x G_A(x) \imp \nec \exq y G_A(y)} $$
%
\item[T3'] Necessarily, a God-like being exists: $$\hol{\nec \exq x G_A(x)}$$ 
\end{itemize}
\end{minipage}
} %\vskip-.5em

\clearpage



\section{Hájek's First Emendation \AOEH} \label{apx:Hajek1}

\framebox[\columnwidth][r]{
\begin{minipage}{.94\columnwidth}\small
\begin{itemize}
\item[H:A12] The negation of a property necessarily implied by a
  positive property is not positive:
  $$\hol{\allq \varphi \allq \psi [(P(\varphi) \wedge \nec \allq x [\varphi(x)
  \imp \psi(x)]) \imp \neg P(\neg \psi)]}$$
\item[H:D1] A \emph{God-like} being necessarily possesses those and only those properties that are necessarily implied by a positive property: 
  $$\hol{G_H(x) \equiv \forall \varphi [\nec \varphi(x) \biimp \exq \psi[P(\psi) \wedge \nec \allq x [\psi(x)  \imp \varphi(x))]]}$$ 
\item[A3']  The property of being God-like is positive: 
  $$\hol{P(G_H)}$$
\item[A4]  Positive properties are necessarily positive: 
  $$\hol{\allq \varphi [P(\varphi) \imp \Box \; P(\varphi)]}$$ 
\item[A:D2] An \emph{essence} of an individual is a property that necessarily implies those and only those properties that the individual has necessarily: $$\hol{\essA{\varphi}{x} \equiv \allq
  \psi [\nec \psi(x) \biimp \nec \allq y (\varphi(y) \imp \psi(y))]}$$ 
\item[D3'] \emph{Necessary existence} of an individual is the necessary exemplification of all its essences: 
  $$\hol{\NE_A(x) \equiv \allq \varphi [\essA{\varphi}{x} \imp \nec
  \exq y [\varphi(y)]]}$$
\item[A5'] Necessary existence is a positive property: $$\hol{P(\NE_A)}$$ 

\item[L3] 
  \begin{itemize}
  \item[(1)] The negation of a positive property is not positive:
$$\hol{\allq \varphi[P(\varphi) \imp \neg P(\neg \varphi)] }$$
  \item[(2)] Positive properties are possibly exemplified:
$$\hol{\allq \varphi[P(\varphi) \imp \pos \exq x \varphi(x)] }$$
  \item[(3)] If a god-like being exists, then necessarily a god-like being exists:
$$\hol{\allq x[G_H(x) \imp \nec G_H(x)))] }$$  
  \item[(4)] All positive properties are necessarily implied by the property of being god-like:
$$\hol{\allq \varphi[P(\varphi) \imp \nec \allq x[G_H(x) \imp \varphi(x)]] }$$
  \end{itemize}
\item[L4]  Being God-like is an essence of any
  God-like being: $$\hol{\allq x [G_H(x) \imp \ess{G_H}{x}]}$$

\item[T3'] Necessarily, a God-like being exists: $$\hol{\nec \exq x G_H(x)}$$
% 
\end{itemize}
\end{minipage}
} %\vskip-.5em


\clearpage

\section{Hájek's Second Emendation \AOEHH} \label{apx:Hajek2}

\framebox[\columnwidth][r]{
\begin{minipage}{.94\columnwidth}\small
\begin{itemize}
\item[H:A12] The negation of a property necessarily implied by a
  positive property is not positive:
  $$\hol{\allq \varphi \allq \psi [(P(\varphi) \wedge \nec \allq x [\varphi(x)
  \imp \psi(x)]) \imp \neg P(\neg \psi)]}$$
\item[A:D1] A \emph{God-like} being necessarily possesses those and only those properties that are positive: 
  $$\hol{G_A(x) \equiv \forall \varphi [P(\varphi) \biimp \nec \varphi(x)]}$$ 
\item[H:A3]  The property of being God-like and existing actually is positive: 
  $$\hol{P(G_A \wedge E)}$$
\item[T3'] Necessarily, a God-like being exists: $$\hol{\nec \exq x G_A(x)}$$
% 
\end{itemize}
\end{minipage}
} %\vskip-.5em



\section{Hájek's Third Emendation \AOEHHH} \label{apx:Hajek3}

\framebox[\columnwidth][r]{
\begin{minipage}{.94\columnwidth}\small
\begin{itemize}
\item[H:A12] The negation of a property necessarily implied by a
  positive property is not positive:
  $$\hol{\allq \varphi \allq \psi [(P(\varphi) \wedge \nec \allq x [\varphi(x)
  \imp \psi(x)]) \imp \neg P(\neg \psi)]}$$
\item[D4]  A property is positive\textsuperscript{\#} iff it is necessarily implied by a positive property: 
  $$\hol{P^\#(\varphi) \equiv \exq \psi[P(\psi) \wedge \nec \allq x[\psi(x) \imp \varphi(x)]]}$$
\item[H:D1] A \emph{God-like} being necessarily possesses those and only those properties that are positive\textsuperscript{\#}: 
  $$\hol{G_H(x) \equiv \forall \varphi [P^\#(\varphi) \biimp \nec \varphi(x)]}$$ 
\item[A3']  The property of being God-like is positive: 
  $$\hol{P(G_H)}$$
\item[H:A4]  Positive\textsuperscript{\#} properties are necessarily positive\textsuperscript{\#}: 
  $$\hol{\allq \varphi [P^\#(\varphi) \imp \Box \; P^\#(\varphi)]}$$ 
\item[A:D2] An \emph{essence} of an individual is a property that necessarily implies those and only those properties that the individual has necessarily: $$\hol{\essA{\varphi}{x} \equiv \allq
  \psi [\nec \psi(x) \biimp \nec \allq y (\varphi(y) \imp \psi(y))]}$$ 
\item[D3'] \emph{Necessary existence} of an individual is the necessary exemplification of all its essences: 
  $$\hol{\NE_A(x) \equiv \allq \varphi [\essA{\varphi}{x} \imp \nec
  \exq y [\varphi(y)]]}$$
\item[H:A5] Necessary existence is a positive\textsuperscript{\#} property: $$\hol{P^\#(\NE_A)}$$ 
% 
\end{itemize}
\end{minipage}
} %\vskip-.5em

\clearpage

\section{Bjordal's Alternative} \label{apx:Bjordal}




\framebox[\columnwidth][r]{
\begin{minipage}{.94\columnwidth}\small
$G_B$ (God-like) is taken as primitive and $P_B$ (Positive) is defined.

\bigskip

\begin{itemize}
\item[B:D1] A property is positive iff it is necessarily possessed by every God-like being.
  $$\hol{P_B(\phi) \equiv \nec \allq x (G_B(x) \imp \phi(x))}$$ 
\item[B:L1] B:D1 is logically equivalent in \SFour with the union of D1' and axioms A2', A3' and A4'.
  $$\hol{\textrm{B:D1} \biimp \textrm{D1'} \wedge \textrm{A2'} \wedge \textrm{A3'} \wedge \textrm{A4'} }$$
%  The proof splits into the two implication directions B:L1$^\rightarrow$ and B:L1$^\leftarrow$. B:L1$^\rightarrow$ can be further split into four single steps.
\item[B:D2] a \emph{maximal composite} of an individual's positive properties is a positive property possessed by the individual and necessarily implying every positive property possessed by the individual.
  $$\hol{\MCP(\phi,x) \equiv (\phi(x) \wedge P_B(\phi)) \wedge \allq \psi ((\psi(x) \wedge P_B(\psi)) \imp \nec \allq y (\phi(y) \imp \psi(y)))}$$
\item[B:D3] \emph{Necessary existence} of an individual is the necessary exemplification of all its maximal composites.
  $$\hol{\NE_B(x) \equiv \allq \phi (\MCP(\phi,x) \imp \nec \exq y \phi(y))}$$
\item[A:A1'] If a property is positive, its negation is not positive:
  $$\hol{\allq \varphi [P_B(\varphi) \imp \neg P_B(\neg \varphi)]}$$ 
\item[A5'] Necessary existence is a positive property.
 $$\hol{P_B(\NE_B)}$$
\item[T3'] Necessarily, a God-like being exists: $$\hol{\nec \exq x G_B(x)}$$ 
\end{itemize}
\end{minipage}
} %\vskip-.5em

\vskip 8em


\end{appendix}


% ------------------------------------------------------------------------
\end{document}
% ------------------------------------------------------------------------

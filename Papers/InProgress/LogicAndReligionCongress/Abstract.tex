% ------------------------------------------------------------------------
% bjourdoc.tex for birkjour.cls*******************************************
% ------------------------------------------------------------------------
%%%%%%%%%%%%%%%%%%%%%%%%%%%%%%%%%%%%%%%%%%%%%%%%%%%%%%%%%%%%%%%%%%%%%%%%%%

\documentclass{birkjour}

\usepackage{xspace}
\usepackage{url}
\usepackage{commands}
\usepackage{xcolor}
\usepackage{comment}
\usepackage{modallogics}


\usepackage[utf8]{inputenc}

%
%
% THEOREM Environments (Examples)-----------------------------------------
%
 \newtheorem{thm}{Theorem}[section]
 \newtheorem{cor}[thm]{Corollary}
 \newtheorem{lem}[thm]{Lemma}
 \newtheorem{prop}[thm]{Proposition}
 \theoremstyle{definition}
 \newtheorem{defn}[thm]{Definition}
 \theoremstyle{remark}
 \newtheorem{rem}[thm]{Remark}
 \newtheorem*{ex}{Example}
 \numberwithin{equation}{section}

\def\HOML{\entity{HOML}\xspace}
\def\HOL{\entity{HOL}\xspace}


\begin{document}

%-------------------------------------------------------------------------
% editorial commands: to be inserted by the editorial office
%
%\firstpage{1} \volume{228} \Copyrightyear{2004} \DOI{003-0001}
%
%
%\seriesextra{Just an add-on}
%\seriesextraline{This is the Concrete Title of this Book\br H.E. R and S.T.C. W, Eds.}
%
% for journals:
%
%\firstpage{1}
%\issuenumber{1}
%\Volumeandyear{1 (2004)}
%\Copyrightyear{2004}
%\DOI{003-xxxx-y}
%\Signet
%\commby{inhouse}
%\submitted{March 14, 2003}
%\received{March 16, 2000}
%\revised{June 1, 2000}
%\accepted{July 22, 2000}
%
%
%
%---------------------------------------------------------------------------
%Insert here the title, affiliations and abstract:
%


\title[Analysis of Emendations of G\"odel's Ontological Argument]
 {Computer-Assisted Analysis of 
 Emendations of G\"odel's Ontological Proof}



%----------AUTHOR 1
\author[Benzm\"uller]{C. Benzm\"uller}

\address{%
Dep.~of Mathematics and Computer Science, Freie Universit\"at Berlin, Germany
}
\email{c.benzmueller@fu-berlin.com}

\thanks{This work was supported by German National Research Foundation (DFG) under
 grants BE 2501/9-1 and BE 2501/11-1.}


%----------Author 2
\author[Weber]{L. Weber}
\address{ 
Dep.~of Mathematics and Computer Science, Freie Universit\"at Berlin, Germany
}
\email{leon.weber@fu-berlin.de}


%----------Author 3
\author[Woltzenlogel-Paleo]{B. Woltzenlogel Paleo}
\address{ 
Room HA0402, Favoritenstra{\ss}e 9, 1040 Wien, Austria
}
\email{bruno.wp@gmail.com}




%----------classification, keywords, date
% \subjclass{
% % Prim. 03A02;  % Philosophical aspects of logic and foundations
% % Sec. 68T02 % Artificial Intelligence 
% }

%\keywords{Ontological Argument, Modal Collapse}

\date{\today}
%----------additions
\dedicatory{ }
%%% ----------------------------------------------------------------------

% \begin{abstract}
% \end{abstract}

%%% ----------------------------------------------------------------------
\maketitle
%%% ----------------------------------------------------------------------
%\tableofcontents


\noindent The axioms in G\"odel's ontological argument
\cite{GoedelNotes,ScottNotes} (cf. Appendix \ref{apx:Goedel}) entail
what is called \emph{modal collapse}
\cite{Sobel1987,SobelBook2004}: the formula $\varphi \rightarrow \Box
\varphi$, abbreviated as MC, holds for any formula $\varphi$ and not
just for $\exists x. \mathit{God}(x)$ as intended. This fact, which
has recently been confirmed with higher-order automated theorem
provers \cite{C40,J30}, has led to strong criticism of the argument
and stimulated attempts to remedy the problem. Hájek
\cite{Hajek_der_Mathematiker_2002,Hajek_Magari_and_others_1996}
proposed the use of cautious instead of full comprehension principles,
and Fitting \cite{fitting02:_types_tableaus_god} suggested that
greater care is necessary in the semantics of higher-order quantifiers
in the presence of modalities. Others, such as Anderson
\cite{anderson90:_some_emend_of_goedel_ontol_proof,AndersonGettings}
and Bjordal \cite{bjordal99}, proposed slight emendations of G\"odel's
axioms and definitions. They require neither comprehension
restrictions nor more complex semantics. Therefore, they are
technically simpler to analyze with computer support. We have
formalized them using the proof assistant Isabelle/HOL \cite{Isabelle}
together with the automated higher-order reasoners Leo-II \cite{C26},
Satallax \cite{brown2012satallax}, Metis 
\cite{Hurd03first-orderproof}, and Nitpick \cite{Nitpick}.   Our
formalizations\footnote{The   formalizations are available in the
subdirectories \url{Anderson}   and \url{Bjordal} at   \url{https://gi
thub.com/FormalTheology/GoedelGod/blob/master/Formalizations/Isabelle/
}.} employ the embedding of higher-order modal logic (HOML) in
classical higher-order logic (HOL) as introduced in previous work
\cite{C40,J30,J23}. We explored the effect of different domain
conditions  on the provability of lemmas, theorems and even axioms.
This was motivated by a controversy between Hájek and Anderson
regarding the redundancy of some axioms in Anderson's emendation. In
\emph{constant domain semantics}, the individual domains are the same
in all possible worlds. In \emph{varying domain semantics}, the
domains may vary from world to world. This variation is technically
encoded with the help of an existence relation expressing which
individuals actually exist in each world. Quantifiers are then
uniformly formalized as \emph{actualistic quantifiers} (i.e. guarded by
the existence relation). Our main results are summarized in the
following sections.


\section{Anderson's Emendation (cf. Appendix \ref{apx:Anderson})}

%\BWP{Bruno: I do not yet address T2', L1' and L2' below; we could eventually include those as well?}

For both \textbf{constant domain semantics} and \textbf{varying domain semantics}, the following results hold: 
%
the theorems T1, C and T3' can be quickly automated (in logics \K, \K and \KB, respectively);
%
the axioms A4 and A5 are proven redundant (the former in
logic \KFourB and the latter already in \K);
%
a trivial countermodel (consisting of two worlds and 
two individuals) to MC generated by Nitpick (for all mentioned logics), which also shows the consistency of the axioms. 

The redundancy of A4 and A5 is particularly controversial. Magari \cite{TODO} claimed that the redundancy occurs already with Gödel's original axioms and definitions. Hájek \cite{Hajek 1996 TODO} investigated this further and claimed that Magari's claim is invalid, but true under the assumption of an additional axiom (PEP). Moreover, Hájek \cite{Hajek 1996 TODO} claimed that the redundancy occurs for Anderson's emended axioms and definitions \cite{Anderson 1990}. Anderson and Gettings \cite{AG1996, footnote 1, page 1} rebutted Hájek's claim, arguing that it holds only under constant domain semantics, while the emended argument by Anderson ought to be taken under Cocchiarella's semantics \cite{Cocchiarella} (a varying domain semantics). Hájek \cite{Hajek 2002, page 7} acknowledges this rebuttal, and apparently accepts it, as evidenced by his use of A4 and A5 in his new small emendation (replacing A:A1 and A2 by a new axiom H:A12) of Anderson's variant with varying domains \cite{Hajek 2002, section 4}. Nevertheless, he does show yet another emended version where A4 and A5 are redundant, if A3 is replaced by a stronger axiom additionally stating that the property of actual existence is positive \cite{Hajek 2002, section 5}. Our results show that Hájek was originally right, under both constant and varying domain semantics with no need to strengthen A3 (though this need might exist when H:A12 is used instead of A:A1 and A2). It should be noted, however, that \cite{Anderson, footnote 14} vaguely remarks that only the quantifiers in T3' and in A:D2  need to be interpreted as actualistic quantifiers, while other may be taken as possibilistic quantifiers. We have inv  mixed variant.

ToDo: tell the story of the controversy here, and then link with the mixed variant. 



\paragraph{Mixed variant}
(varying domain
  quantifiers are used only in the definitions of essence and NE;
  cf.~Fitting's comments to Anderson in
  \cite{fitting02:_types_tableaus_god}): Also in this setting we
  obtain the same results as above. However, if a varying domain
  quantifier is used only in the definition of NE, then the
  situation changes slightly. Now axiom A5 is no longer provable and
  a countermodel is reported by Nitpick. The remaining results are as before.

%   G\"odel's axioms A4 and A5
%   remain redundant (the former one in logic K4B, the latter already in
%   K). The situation changes when only NE is modeled with a varying
%   domain quantifier. Now, Nitpick reports a countermodel.

 
\section{Bjordal's Emendation (cf. Appendix \ref{apx:Bjordal})}

For both \textbf{constant domain semantics} and \textbf{varying domain semantics}, the following results hold:

    G\"odel's axiom
  A2, A3 can be quickly automatically derived in logic \K from
  Bjordal's definition B:D. A4 can be proved in logic \KT
  (reflexivity). Proving G\"odel's D1 from B:D is possible in logic
  \KFour. Conversely, the proof that B:D follows from D1, A2, A3 and
  A4 is possible already in logic \K. Hence, Bjordal's lemma
  B:L1 holds in logic \SFour. The provers also show that theorem T3
  follows from B:D, B:A1 and B:A2 already in logic \KB. Modal collapse
  does not follow in Bjordal's setting as Nitpick demonstrates with a
  countermodel (consisting of two worlds and one individual). 

  \BWP{@Bruno: we could eventually also
  also say that for MC1 we need a bigger countermodel?}

\BWP{ToDo: discuss the fact that Bjordal's proof is more resistant to MC than Anderson's}


\section{Conclusions}

Anderson's emendation (cf. Appendix \ref{apx:Anderson}; ,
which we have analysed for different domain conditions. These
variations were motivated by various comments on Anderson's work in
the literature. 

In this approach full comprehension is
naturally ``built-in'' since the underlying HOL supports
$\lambda$-abstraction.

Summary (what else can we say here, feel free to add): Using our approach, the formalization and (partly) automated
analysis of different variants of Anderson's and Bjordal's emendations
of G\"odel's ontological argument has been surprisingly
straightforward. The provers confirmed the claimed results and in a
few cases they have even contributed some novel insights. The
weakening of the comprehension principles would clearly constitute
another interesting parameter for further experiments. However, this
seems hard to achieve in our approach, since full comprehension is
naturally built-in.


\bibliographystyle{plain}
\bibliography{Bibliography}


\newpage
\begin{appendix}
\noindent 

\BWP{I think we need more compact presentation of the proofs.}


\small

\section{Scott's version of G\"odel's ontological argment} \label{apx:Goedel} 
\begin{itemize}
\item[A1] Either a property or its negation is positive, but not
  both:
  $$\hol{\allq \varphi [P(\neg \varphi) \biimp \neg P(\varphi)]}$$ 
\item[A2] A property necessarily implied by a
  positive property is positive:
  $$\hol{\allq \varphi \allq \psi [(P(\varphi) \wedge \nec \allq x [\varphi(x)
  \imp \psi(x)]) \imp P(\psi)]}$$
\item[T1] Positive properties are possibly exemplified: 
  $$\hol{\allq \varphi [P(\varphi) \imp \pos \exq x \varphi(x)]}$$ 
\item[D1] A \emph{God-like} being possesses all positive properties: 
  $$\hol{G(x) \equiv \forall \varphi [P(\varphi) \imp \varphi(x)]}$$ 
\item[A3]  The property of being God-like is positive: 
  $$\hol{P(G)}$$
\item[C\phantom{1}] Possibly, a God-like being exists: $$\hol{\pos \exq x G(x)}$$
\item[A4]  Positive properties are necessarily positive: 
  $$\hol{\allq \varphi [P(\varphi) \imp \Box \; P(\varphi)]}$$ 
\item[D2] An \emph{essence} of an individual is a property possessed by it and necessarily implying any of its properties: $$\hol{\ess{\varphi}{x} \equiv \varphi(x) \wedge \allq
  \psi (\psi(x) \imp \nec \allq y (\varphi(y) \imp \psi(y)))}$$ 
\item[T2]  Being God-like is an essence of any
  God-like being: $$\hol{\allq x [G(x) \imp \ess{G}{x}]}$$
\item[D3] \emph{Necessary existence} of an individual is the necessary exemplification of all its essences: 
  $$\hol{\NE(x) \equiv \allq \varphi [\ess{\varphi}{x} \imp \nec
  \exq y \varphi(y)]}$$
\item[A5] Necessary existence is a positive property: $$\hol{P(\NE)}$$ 
\item[L1] If a god-like being exists, then necessarily a god-like being exists: 
  $$\hol{\exq x G(x) \imp \nec \exq y G(y)}$$
\item[L2] If possibly a god-like being exists, then necessarily a god-like being exists: 
  $$\hol{\pos \exq x G(x) \imp \nec \exq y G(y)} $$
%
\item[T3] Necessarily, a God-like being exists: $$\hol{\nec \exq x G(x)}$$ 
\end{itemize}



\section{Anderson's Emendation} \label{apx:Anderson}

% \begin{figure}[t]
% \noindent \framebox[\columnwidth][r]{
% \begin{minipage}{.94\columnwidth}\small
\begin{itemize}
\item[A:A1] If a property is positive, its negation is not positive:
  $$\hol{\allq \varphi [P(\varphi) \imp \neg P(\neg \varphi)]}$$ 
\item[A2] A property necessarily implied by a
  positive property is positive:
  $$\hol{\allq \varphi \allq \psi [(P(\varphi) \wedge \nec \allq x [\varphi(x)
  \imp \psi(x)]) \imp P(\psi)]}$$
\item[T1] Positive properties are possibly exemplified: 
  $$\hol{\allq \varphi [P(\varphi) \imp \pos \exq x \varphi(x)]}$$ 
\item[A:D1] A \emph{God-like} being necessarily possesses those and only those properties that are positive: 
  $$\hol{G_A(x) \equiv \forall \varphi [P(\varphi) \biimp \nec \varphi(x)]}$$ 
\item[A3']  The property of being God-like is positive: 
  $$\hol{P(G_A)}$$
\item[C\phantom{1}] Possibly, a God-like being exists: $$\hol{\pos \exq x G(x)}$$
\item[A4]  Positive properties are necessarily positive: 
  $$\hol{\allq \varphi [P(\varphi) \imp \Box \; P(\varphi)]}$$ 
\item[A:D2] An \emph{essence} of an individual is a property that necessarily implies those and only those properties that the individual has necessarily: $$\hol{\essA{\varphi}{x} \equiv \allq
  \psi [\nec \psi(x) \biimp \nec \allq y (\varphi(y) \imp \psi(y))]}$$ 
\item[T2']  Being God-like is an essence of any
  God-like being: $$\hol{\allq x [G_A(x) \imp \essA{G_A}{x}]}$$
\item[D3'] \emph{Necessary existence} of an individual is the necessary exemplification of all its essences: 
  $$\hol{\NE_A(x) \equiv \allq \varphi [\essA{\varphi}{x} \imp \nec
  \exq y \varphi(y)]}$$
\item[A5'] Necessary existence is a positive property: $$\hol{P(\NE_A)}$$ 
\item[L1'] If a god-like being exists, then necessarily a god-like being exists: 
  $$\hol{\exq x G_A(x) \imp \nec \exq y G_A(y)}$$
\item[L2'] If possibly a god-like being exists, then necessarily a god-like being exists: 
  $$\hol{\pos \exq x G_A(x) \imp \nec \exq y G_A(y)} $$
%
\item[T3'] Necessarily, a God-like being exists: $$\hol{\nec \exq x G_A(x)}$$ 
\end{itemize}
% \end{minipage}
% } \vskip-.5em
% \caption{Anderson's Emendation \cite{ToDo:Anderson}.\label{fig:anderson}} 
% \end{figure}


% I have meanwhile added another file ``Anderson\_var\_partial.thy''. In this version the actualist e-Quantifier is used exclusively in the definition of NE and everywhere else ithe possibilist Quantifiers are used. It is still unclear to me, what is actually meant. Reading Anderson 1990, last sentence on p. 302, it could mean that ``Anderson\_var\_partial.thy'' is the correct version; but when reading AndersonGettings, then one could think that ``Anderson\_var.thy'' is meant. But ok we have now both versions. Interestingly in ``Anderson\_var\_partial.thy'' A5 is not inferable anymore, and this also holds for ``anderson\_implies\_fuhrmann". Nitpick finds countermodels.


%\clearpage

\section{Bjordal's Alternative} \label{apx:Bjordal}

% \begin{figure}[t]
% \noindent \framebox[\columnwidth][r]{
% \begin{minipage}{.94\columnwidth}\small
In Bjordal's emendation $G$ (God-like) is taken as primitive and $P$ (Positive) is defined (cf. definition D).
\begin{itemize}
\item[B:D] A formulas $\phi$ is positive iff it is necessarily the case
  that anything which is God-like has the property $\phi$.
  $$\hol{P(\phi) \equiv \nec \allq x (G(x) \imp \phi(x))}$$ 
\item[B:L1] D is logically equivalent in S4 with the union of G\"odel's definition D1 and axioms A2, A3 and A4.
  $$\hol{D \biimp D1 \wedge A2 \wedge A3 \wedge A4}$$
  The proof splits into the two implication directions B:L1$^\rightarrow$ and B:L1$^\leftarrow$. B:L1$^\rightarrow$ can be further split into four single steps.
\item[B:D2] $\phi$ is a maximal composite of object $x$'s positive properties iff $x$ has $\phi$ and $\phi$ is positive and all positive properties $\psi$ which $x$ has are such that is necessarily the case that all objects which have $\phi$ also have $\psi$.
  $$\hol{MCP(\phi,x) \equiv (\phi(x) \wedge P(\phi)) \wedge \allq \psi ((\psi(x) \wedge P(\psi)) \imp \nec \allq y (\phi(y) \imp \psi(y)))}$$
\item[B:D3] $x$ has the $N$-property iff x is such that if $\phi$ is a maximal composite of $x$'s positive properties then it is necessary that some object $y$ has the property $\phi$.
  $$\hol{N(x) \equiv \allq \phi (MCP(\phi,x) \imp \nec \allq y \phi(y))}$$
\item[B:A1] If a property is positive, its negation is not positive:
  $$\hol{\allq \varphi [P(\varphi) \imp \neg P(\neg \varphi)]}$$ 
\item[B:A2] The $N$-property is positive.
 $$\hol{P(N)}$$
\item[T3] Necessarily, a God-like being exists: $$\hol{\nec \exq x G(x)}$$ 
\end{itemize}
% \end{minipage}
% } \vskip-.5em
% \caption{Frode's Alternative \cite{ToDo:Frode}.\label{fig:frode}} 
% \end{figure}

%\clearpage
\end{appendix}


% ------------------------------------------------------------------------
\end{document}
% ------------------------------------------------------------------------

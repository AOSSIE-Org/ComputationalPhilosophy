% ------------------------------------------------------------------------
% bjourdoc.tex for birkjour.cls*******************************************
% ------------------------------------------------------------------------
%%%%%%%%%%%%%%%%%%%%%%%%%%%%%%%%%%%%%%%%%%%%%%%%%%%%%%%%%%%%%%%%%%%%%%%%%%

\documentclass{birkjour}

\usepackage{url}
\usepackage{commands}

\usepackage[utf8]{inputenc}

%
%
% THEOREM Environments (Examples)-----------------------------------------
%
 \newtheorem{thm}{Theorem}[section]
 \newtheorem{cor}[thm]{Corollary}
 \newtheorem{lem}[thm]{Lemma}
 \newtheorem{prop}[thm]{Proposition}
 \theoremstyle{definition}
 \newtheorem{defn}[thm]{Definition}
 \theoremstyle{remark}
 \newtheorem{rem}[thm]{Remark}
 \newtheorem*{ex}{Example}
 \numberwithin{equation}{section}

\def\HOML{\entity{HOML}\xspace}
\def\HOL{\entity{HOL}\xspace}


\begin{document}

%-------------------------------------------------------------------------
% editorial commands: to be inserted by the editorial office
%
%\firstpage{1} \volume{228} \Copyrightyear{2004} \DOI{003-0001}
%
%
%\seriesextra{Just an add-on}
%\seriesextraline{This is the Concrete Title of this Book\br H.E. R and S.T.C. W, Eds.}
%
% for journals:
%
%\firstpage{1}
%\issuenumber{1}
%\Volumeandyear{1 (2004)}
%\Copyrightyear{2004}
%\DOI{003-xxxx-y}
%\Signet
%\commby{inhouse}
%\submitted{March 14, 2003}
%\received{March 16, 2000}
%\revised{June 1, 2000}
%\accepted{July 22, 2000}
%
%
%
%---------------------------------------------------------------------------
%Insert here the title, affiliations and abstract:
%


\title[Analysis of Modern Variants of G\"odel's Ontological Argument]
 {Computer-supported Analysis of Modern Variants of G\"odel's Ontological Argument}



%----------AUTHOR 1
\author[Benzm\"uller]{C. Benzm\"uller}

\address{%
Dep.~of Mathematics and Computer Science, Freie Universit\"at Berlin, Germany
}
\email{c.benzmueller@fu-berlin.com}

\thanks{This work was supported by German National Research Foundation (DFG) under
 grants BE 2501/9-1 and BE 2501/11-1 and by ????}


%----------Author 2
\author[Weber]{L. Weber}
\address{ 
Dep.~of Mathematics and Computer Science, Freie Universit\"at Berlin, Germany
}
\email{leon.weber@fu-berlin.de}


%----------Author 3
\author[Woltzenlogel-Paleo]{B. Woltzenlogel Paleo}
\address{ 
Favoritenstra{\ss}e 9 \\
Room HA0402 \\
1040 Wien \\
Austria
}
\email{bruno.wp@gmail.com}




%----------classification, keywords, date
% \subjclass{
% % Prim. 03A02;  % Philosophical aspects of logic and foundations
% % Sec. 68T02 % Artificial Intelligence 
% }

%\keywords{Ontological Argument, Modal Collapse}

\date{\today}
%----------additions
\dedicatory{ }
%%% ----------------------------------------------------------------------

% \begin{abstract}
% The \emph{modal collapse} that afflicts G\"odel's modal ontological 
% argument for God's existence is discussed from the perspective of the 
% modal square of opposition. Furthermore, a computer-assisted verification 
% of the claims that the emendations by Anderson and by Frode are immune to 
% the modal collapse is presented.
% \end{abstract}

%%% ----------------------------------------------------------------------
\maketitle
%%% ----------------------------------------------------------------------
%\tableofcontents
%\section{Introduction}

\vspace*{-5em}

%\section{The Modal Collapse}
The axioms in G\"odel's ontological argument 
\cite{GoedelNotes,ScottNotes} (cf. Appendix \ref{apx:Goedel}) imply what is called the modal collapse
\cite{Sobel1987,SobelBook2004} (formula $\varphi \rightarrow \Box
\varphi$, abbreviated as MC, holds for any formula $\varphi$ and not
just for $\exists x. God(x)$ as intended). This fact, which has
recently been confirmed with higher-order automated theorem provers
\cite{C40,J30}, has led to strong criticism of the argument, and
it has therefore stimulated attempts to remedy the problem. Some
authors, including Hajek
\cite{Hajek_der_Mathematiker_2002,Hajek_Magari_and_others_1996}, suggest the use
of cautious instead of full comprehension principles in order to
escape the modal collapse. Others, including Anderson
\cite{anderson90:_some_emend_of_goedel_ontol_proof,AndersonGettings}
and Bjordal \cite{bjordal99}, propose to slightly modify
G\"odel's axioms and definitions.

We have studied --- with computer-support --- the emendations proposed
by Anderson and Bjordal (thereby we have retained full comprehension).
More precisely, we have formalized their emendations within the proof
assistant Isabelle/HOL \cite{Isabelle}, and we have then automated
(different variations of) their emendations with the higher-order
reasoners Leo-II \cite{C26}, Satallax \cite{brown2012satallax}, Metis
\cite{Hurd03first-orderproof} and Nitpick \cite{Nitpick}; these systems
are available in Isabelle/HOL.  Our formalizations\footnote{The
  formalizations are available in the subdirectories \url{Anderson}
  and \url{Bjordal} at
  \url{https://github.com/FormalTheology/GoedelGod/blob/master/Formalizations/Isabelle/}.}
employ the embedding of higher-order modal logic (HOML) in classical
higher-order logic (HOL) as introduced and employed in previous
work \cite{C40,J30,J23}. In this approach full comprehension is
naturally ``built-in'' since the underlying HOL supports
$\lambda$-abstraction.

We summarize the main results of our experiments, and we start with
Anderson's emendation (cf. Appendix \ref{apx:Anderson}; @Bruno: I do
not yet address T2', L1' and L2' below; we could eventually include those as well?),
which we have analysed for different domain conditions. These
variations were motivated by various comments on Anderson's work in
the literature (@Bruno: maybe we should be more explicit about these
comments).
\begin{itemize}
\item Constant domain semantics\footnote{Cf. Isabelle/HOL source file
    \url{Anderson/Anderson_constant_domain.thy}} (the individual
  domains remain unchanged for all worlds): G\"odel's theorems T1, C
  and T3' can be quickly automated (in logics K and KB, respectively),
  G\"odel's axioms A4 and A5 are proven redundant (the former one in
  logic K4B and the latter one already in K). Moreover, a trivial
  countermodel (consisting of two worlds and two individuals) to MC
  generated by Nitpick (for all mentioned logics); this model also
  proves consistency. 
\item Varying domain semantics\footnote{Cf. Isabelle/HOL source file
    \url{Anderson/Anderson_varying_domain.thy}} (the domains for
  indivduals may vary from world to world, all other domains remain
  constant; various authors have pointed out that varying domain
  quantifiers should be used in Anderson's work; here we consistently
  used varying domain for all individual quantifiers; the modeling of
  the varying domain quantifiers employs an explicit ``Existence''
  relation between indiviudals and worlds as guard): In this setting we obtain the
  same results as above.
\item Mixed variant\footnote{Cf. Isabelle/HOL source files
    \url{Anderson/Anderson_mixed_domain_1/2.thy}} (varying domain
  quantifiers are used only in the definitions of essence and NE;
  cf.~Fitting comments to Anderson in
  \cite{fitting02:_types_tableaus_god}): Also in this setting we
  obtain the same results as above. However, if a varying domain
  quantifier is used only in the definition of NE, then the
  situation slightly changes. Now axiom A5 is no longer provable and
  a countermodel is reported by Nitpick. The remaining results are as before.

%   G\"odel's axioms A4 and A5
%   remain redundant (the former one in logic K4B, the latter already in
%   K). The situation changes when only NE is modeled with a varying
%   domain quantifier. Now, Nitpick reports a countermodel.
\end{itemize}
 

Our analysis of Bjordal emendation of G\"odel's argument (cf. Appendix \ref{apx:Bjordal}) has produced the following results:
\begin{itemize}
\item Constant domain semantics\footnote{Cf. Isabelle/HOL source files
    \url{Bjordal/Bjordal_A/B/C_constant_domain.thy}}: G\"odel's axiom
  A2, A3 can be quickly automatically derived in logic K from
  Bjordal's definition B:D. A4 can be proved in logic T
  (reflexivity). Proving G\"odel's D1 from B:D is possible in logic
  K4. Vice versa, for the proof that B:D follows from D1, A2, A3 and
  A4 the povers succeed already in logic K. Hence, Bjordal's lemma
  B:L1 holds in logic S4. The provers also show that theorem T3
  follows from B:D, B:A1 and B:A2 already in logic KB. Modal collapse
  does not follow in Bjordal's setting as Nitpick demonstrates with a
  countermodel (consisting of two worlds and one individual). @Bruno: we could eventually also
  also say that for MC1 we need a bigger countermodel?
\item Varying domain semantics\footnote{Cf. Isabelle/HOL source files
    \url{Bjordal/Bjordal_A/B/C_varying_domain.thy}}: The results are
  the same as above. 
\item @Bruno: we could also check the mixed case her; but there doesn't seem to be strong reason for doing so.
\end{itemize}

Summary (what else can we say here, feel free to add): Using our approach, the formalization and (partly) automated
analysis of different variants of Anderson's and Bjordal's emendations
of G\"odel's ontological argument has been surprisingly
straightforward. The provers confirmed the claimed results and in a
few cases they have even contributed some novel insights. The
weakening of the comprehension principles would clearly constitute
another interesting parameter for further experiments. However, this
seems hard to achieve in our approach, since full comprehension is
naturally built-in.

% \section{Other Solutions}

% ToDo: Fitting


% \section{Conclusions}


% Various slightly different versions of
% axioms and definitions have been considered by G\"{o}del and by
% several philosophers who commented on his proof
% (cf{sobel2004logic,anderson90:_some_emend_of_goedel_ontol_proof,AndersonGettings,Fitting,Adams,ContemporaryBibliography}).



% In theoretical philosophy, formal logical confrontations with such
% ontological arguments had been so far (mainly) limited to paper
% and pen.  Up to now, the use of computers was prevented, because the
% logics of the available theorem proving systems were not expressive
% enough to formalize the abstract concepts adequately. G{\"o}del's proof
% uses, for example, a complex higher-order modal logic (\HOML) to handle
% concepts such as \emph{possibility} and \emph{necessity} and to support
% quantification over individuals and properties.

% controversies, care with parameters

% ToDo: Leibniz calculemus, Rushby, Zalta \cite{oppenheimera11,rushby13}


% \begin{defn}
% This serves as environment for definitions. Note that the text
% appears not in italics.
% \end{defn}

% \begin{equation}\label{testequation}
% \text{This is a sample equation: } c^2=a^2+b^2
% \end{equation}

% \begin{thm}[Main Theorem]
% In contrast to definitions, theorems appear typeset in italics as
% it has become more or less standard in most textbooks and
% monographs. Equations can be cited using the \verb+\eqref+ command which
% automatically adds brackets: \verb+\eqref{testequation}+ results in \eqref{testequation}.
% \end{thm}

% \begin{proof}
% A special environment is predefined: the \textit{proof} environment. Please use
% \begin{verbatim}\begin{proof}\end{verbatim}
% proof of the statement
% \begin{verbatim}\end{proof}\end{verbatim}
% for typesetting your proofs. The end-of-proof symbol $\Box$ will be added automatically.
% \end{proof}

% There are two known problems with the placement of the end-of-proof sign:

% \begin{enumerate}
%   \item if your proof ends with a\ \ s i n g l e\ \ displayed line, the end-of-proof sign would
% be placed in the line below; if you want to avoid this, write your line in the form
% \begin{verbatim}$$displayed math line \eqno\qedhere$$\end{verbatim}
% which results in

% \begin{proof}
% $$displayed math line \eqno\qedhere$$
% \end{proof}
% \item if your proof ends with an aligned displayed environment, the command
% \verb+\tag*{\qed}+ can be used to place the end-of-proof sign properly:
% \begin{verbatim}
% \begin{align*}
% \alpha&=\beta+\gamma\\
% &=\delta+\epsilon\tag*{\qed}
% \end{align*}
% \end{verbatim}
% results in
% \begin{align*}
% \alpha&=\beta+\gamma\\
% &=\delta+\epsilon\tag*{\qed}
% \end{align*}
% \end{enumerate}
% Please try to avoid using the obsolete \verb+\eqnarray+ environment. This environment has several bugs
% and has been replaced by the more flexible \AmS\ environments \verb+align, split, multline+.


% \begin{rem}
% Additional comments are being typeset without boldfaced entrance
% word as they may be minor important.
% \end{rem}

% \begin{ex}
% For some constructs, even no number is required.
% \end{ex}

% Displayed equations may be numbered like the following one:
% \begin{equation}
% \sqrt{1-\sin^2(x)}=|\cos(x)|.
% \end{equation}



% ------------------------------------------------------------------------

% \subsection*{Acknowledgment}
% Many thanks to ...

\bibliographystyle{plain}
\bibliography{Bibliography}

% \begin{thebibliography}{1}

% \bibitem{Adams}
% R.M. Adams, `Introductory note to *1970', in {\em {Kurt G\"odel: Collected
%   Works Vol. 3: Unpubl. Essays and Letters}}, Oxford Univ. Press, (1995).

% \bibitem{AndersonGettings}
% A.C. Anderson and M.~Gettings, `G\"odel ontological proof revisited', in {\em
%   {G\"odel'96: Logical Foundations of Mathematics, Computer Science, and
%   Physics: Lecture Notes in Logic 6}},  167--172, {Springer}, (1996).

% \bibitem{anderson90:_some_emend_of_goedel_ontol_proof}
% C.A. Anderson, `Some emendations of {G{\"o}del's} ontological proof', {\em
%   Faith and Philosophy}, {\bf 7}(3), (1990).

% \bibitem{Andrews:gmae72}
% P.B. Andrews, `General models and extensionality', {\em Journal of Symbolic
%   Logic}, {\bf 37}(2),  395--397, (1972).

% \bibitem{andrewsSEP}
% P.B. Andrews, `Church's type theory', in {\em The Stanford Encyclopedia of
%   Philosophy}, ed., E.N. Zalta, spring 2014 edn., (2014).

% \bibitem{C36}
% C.~Benzm{\"u}ller, `{HOL} based universal reasoning', in {\em Handbook of the
%   4th World Congress and School on Universal Logic}, ed., J.Y. Beziau~et al.,
%   pp. 232--233, Rio de Janeiro, Brazil, (2013).

% \bibitem{B5}
% C.~Benzm{\"u}ller and D.~Miller, `Automation of higher-order logic', in {\em
%   Handbook of the History of Logic, Volume 9 --- Logic and Computation},
%   Elsevier, (2014).
% \newblock Forthcoming; preliminary version available at
%   {http://christoph-benzmueller.de/papers/B5.pdf}.

% \bibitem{C34}
% C.~Benzm{\"u}ller, J.~Otten, and Th. Raths, `Implementing and evaluating
%   provers for first-order modal logics', in {\em Proc. of the 20th European
%   Conference on Artificial Intelligence (ECAI)}, pp. 163--168, (2012).

% \bibitem{B9}
% C.~Benzm{\"u}ller and L.C. Paulson, `Exploring properties of normal multimodal
%   logics in simple type theory with {LEO-II}', in {\em {Festschrift in Honor of
%   {Peter B. Andrews} on His 70th Birthday}}, ed., C.~Benzm{\"u}ller~et al.,
%   386--406, College Publications, (2008).

% \bibitem{J23}
% C.~Benzm{\"u}ller and L.C. Paulson, `Quantified multimodal logics in simple
%   type theory', {\em Logica Universalis}, {\bf 7}(1),  7--20, (2013).

% \bibitem{LEO-II}
% C.~Benzm{\"u}ller, F.~Theiss, L.~Paulson, and A.~Fietzke, `{LEO-II} - a
%   cooperative automatic theorem prover for higher-order logic', in {\em
%   Proc.~of IJCAR 2008}, number 5195 in LNAI, pp. 162--170. Springer, (2008).

% \bibitem{J30}
% C.~Benzm{\"u}ller and B.~Woltzenlogel-Paleo, `Formalization, mechanization and
%   automation of {G{\"o}del's} proof of {God's} existence', {\em
%   arXiv:1308.4526}, (2013).

% \bibitem{J28}
% C.~Benzm\"uller and B.~Woltzenlogel-Paleo, `{G{\"o}del's God in Isabelle/HOL}',
%   {\em Archive of Formal Proofs}, (2013).

% \bibitem{W50}
% C.~Benzm\"uller and B.~Woltzenlogel-Paleo, `G\"odel's {God} on the computer',
%   in {\em Proceedings of the 10th International Workshop on the Implementation
%   of Logics}, EPiC Series. EasyChair, (2013).
% \newblock Invited abstract.

% \bibitem{Coq}
% Y.~Bertot and P.~Casteran, {\em {Interactive Theorem Proving and Program
%   Development}}, Springer, 2004.

% \bibitem{Nitpick}
% J.C. Blanchette and T.~Nipkow, `Nitpick: A counterexample generator for
%   higher-order logic based on a relational model finder', in {\em Proc. of ITP
%   2010}, number 6172 in LNCS, pp. 131--146. Springer, (2010).

% \bibitem{Satallax}
% C.E. Brown, `Satallax: An automated higher-order prover', in {\em Proc. of
%   IJCAR 2012}, number 7364 in LNAI, pp. 111 -- 117. Springer, (2012).

% \bibitem{ContemporaryBibliography}
% R.~Corazzon.
% \newblock Contemporary~bibliography~on~ontological~arguments: {\scriptsize
%   \url{http://www.ontology.co/biblio/ontological-proof-contemporary-biblio.htm}}.

% \bibitem{Fitting}
% M.~Fitting, {\em Types, Tableaux and G\"odel's God}, Kluwer, 2002.

% \bibitem{fitting98}
% M.~Fitting and R.L. Mendelsohn, {\em First-Order Modal Logic}, volume 277 of
%   {\em Synthese Library}, Kluwer, 1998.

% \bibitem{Gallin75}
% D.~Gallin, {\em Intensional and Higher-Order Modal Logic}, North-Holland, 1975.

% \bibitem{garbacz12:_prover_simpl_expal_away}
% P.~Garbacz, `{PROVER9's} simplifications explained away', {\em Australasian
%   Journal of Philosophy}, {\bf 90}(3),  585--592, (2012).

% \bibitem{GoedelNotes}
% K.~G\"odel, {\em Appx.A: Notes in Kurt G\"odel's Hand},  144--145.
% \newblock In  \cite{sobel2004logic}, 2004.

% \bibitem{Henkin50}
% L.~Henkin, `Completeness in the theory of types', {\em Journal of Symbolic
%   Logic}, {\bf 15}(2),  81--91, (1950).

% \bibitem{homl}
% R.~Muskens, `{Higher Order Modal Logic}', in {\em Handbook of Modal Logic},
%   ed., P~Blackburn~et al.,  621--653, Elsevier, Dordrecht, (2006).

% \bibitem{Isabelle}
% T.~Nipkow, L.C. Paulson, and M.~Wenzel, {\em {Isabelle/HOL: A Proof Assistant
%   for Higher-Order Logic}}, number 2283 in LNCS, Springer, 2002.

% \bibitem{oppenheimera11}
% P.E. Oppenheimer and E.N. Zalta, `A computationally-discovered simplification
%   of the ontological argument', {\em Australasian Journal of Philosophy}, {\bf
%   89}(2),  333--349, (2011).

% \bibitem{rushby13}
% J.~Rushby, `The ontological argument in {PVS}', in {\em Proc.~of CAV Workshop
%   ``Fun With Formal Methods''}, St. Petersburg, Russia,, (2013).

% \bibitem{Schulz:AICOM-2002}
% S.~Schulz, `E -- a brainiac theorem prover', {\em {AI Communications}}, {\bf
%   15}(2),  111--126, (2002).

% \bibitem{ScottNotes}
% D.~Scott, {\em Appx.B: Notes in Dana Scott's Hand},  145--146.
% \newblock In  \cite{sobel2004logic}, 2004.

% \bibitem{sobel2004logic}
% J.H. Sobel, {\em Logic and Theism: Arguments for and Against Beliefs in God},
%   Cambridge U. Press, 2004.

% \bibitem{sutcliffe2009tptp}
% G.~Sutcliffe, `The {TPTP} problem library and associated infrastructure', {\em
%   Journal of Automated Reasoning}, {\bf 43}(4),  337--362, (2009).

% \bibitem{J22}
% G.~Sutcliffe and C.~Benzm{\"u}ller, `Automated reasoning in higher-order logic
%   using the {TPTP THF} infrastructure.', {\em Journal of Formalized Reasoning},
%   {\bf 3}(1),  1--27, (2010).

% \bibitem{J29}
% B.~Woltzenlogel-Paleo and C.~Benzm{\"u}ller, `Automated verification and
%   reconstruction of {G\"odel's} proof of {God's} existence', {\em OCG J.},
%   (2013).

%\end{thebibliography}

\newpage
\begin{appendix}
\noindent @Bruno: I think we need more compact presentation of the proofs.
% \begin{figure}[t]
% \noindent \framebox[\columnwidth][r]{
% \begin{minipage}{.94\columnwidth}\small
% \begin{itemize}
% \item[MC] Everything that is the case is so necessarily:
%   $$\hol{\allq \phi [\phi \imp \nec \phi]}$$ 
% \item[MC'] Everything that is possible is necessary:
%   $$\hol{\allq \phi [\pos \phi \imp \nec \phi]}$$ 
% \item[MC''] Modalities collapse:
%   $$\hol{\allq \phi [\phi \biimp \pos \phi \biimp \nec \phi]}$$
% \end{itemize}
% \end{minipage}
% } \vskip-.5em
% \caption{Modal Collapse \cite{todo:Sobel1987,SobelBook}.\label{fig:collapse}} 
% \end{figure}

%\clearpage

\small

\section{Scott's version of G\"odel's ontological argment} \label{apx:Goedel} 
\begin{itemize}
\item[A1] Either a property or its negation is positive, but not
  both:
  $$\hol{\allq \varphi [P(\neg \varphi) \biimp \neg P(\varphi)]}$$ 
\item[A2] A property necessarily implied by a
  positive property is positive:
  $$\hol{\allq \varphi \allq \psi [(P(\varphi) \wedge \nec \allq x [\varphi(x)
  \imp \psi(x)]) \imp P(\psi)]}$$
\item[T1] Positive properties are possibly exemplified: 
  $$\hol{\allq \varphi [P(\varphi) \imp \pos \exq x \varphi(x)]}$$ 
\item[D1] A \emph{God-like} being possesses all positive properties: 
  $$\hol{G(x) \equiv \forall \varphi [P(\varphi) \imp \varphi(x)]}$$ 
\item[A3]  The property of being God-like is positive: 
  $$\hol{P(G)}$$
\item[C\phantom{1}] Possibly, a God-like being exists: $$\hol{\pos \exq x G(x)}$$
\item[A4]  Positive properties are necessarily positive: 
  $$\hol{\allq \varphi [P(\varphi) \imp \Box \; P(\varphi)]}$$ 
\item[D2] An \emph{essence} of an individual is a property possessed by it and necessarily implying any of its properties: $$\hol{\ess{\varphi}{x} \equiv \varphi(x) \wedge \allq
  \psi (\psi(x) \imp \nec \allq y (\varphi(y) \imp \psi(y)))}$$ 
\item[T2]  Being God-like is an essence of any
  God-like being: $$\hol{\allq x [G(x) \imp \ess{G}{x}]}$$
\item[D3] \emph{Necessary existence} of an individual is the necessary exemplification of all its essences: 
  $$\hol{\NE(x) \equiv \allq \varphi [\ess{\varphi}{x} \imp \nec
  \exq y \varphi(y)]}$$
\item[A5] Necessary existence is a positive property: $$\hol{P(\NE)}$$ 
\item[L1] If a god-like being exists, then necessarily a god-like being exists: 
  $$\hol{\exq x G(x) \imp \nec \exq y G(y)}$$
\item[L2] If possibly a god-like being exists, then necessarily a god-like being exists: 
  $$\hol{\pos \exq x G(x) \imp \nec \exq y G(y)} $$
%
\item[T3] Necessarily, a God-like being exists: $$\hol{\nec \exq x G(x)}$$ 
\end{itemize}



\section{Anderson's Emendation} \label{apx:Anderson}

% \begin{figure}[t]
% \noindent \framebox[\columnwidth][r]{
% \begin{minipage}{.94\columnwidth}\small
\begin{itemize}
\item[A:A1] If a property is positive, its negation is not positive:
  $$\hol{\allq \varphi [P(\varphi) \imp \neg P(\neg \varphi)]}$$ 
\item[A2] A property necessarily implied by a
  positive property is positive:
  $$\hol{\allq \varphi \allq \psi [(P(\varphi) \wedge \nec \allq x [\varphi(x)
  \imp \psi(x)]) \imp P(\psi)]}$$
\item[T1] Positive properties are possibly exemplified: 
  $$\hol{\allq \varphi [P(\varphi) \imp \pos \exq x \varphi(x)]}$$ 
\item[A:D1] A \emph{God-like} being necessarily possesses those and only those properties that are positive: 
  $$\hol{G_A(x) \equiv \forall \varphi [P(\varphi) \biimp \nec \varphi(x)]}$$ 
\item[A3']  The property of being God-like is positive: 
  $$\hol{P(G_A)}$$
\item[C\phantom{1}] Possibly, a God-like being exists: $$\hol{\pos \exq x G(x)}$$
\item[A4]  Positive properties are necessarily positive: 
  $$\hol{\allq \varphi [P(\varphi) \imp \Box \; P(\varphi)]}$$ 
\item[A:D2] An \emph{essence} of an individual is a property that necessarily implies those and only those properties that the individual has necessarily: $$\hol{\essA{\varphi}{x} \equiv \allq
  \psi [\nec \psi(x) \biimp \nec \allq y (\varphi(y) \imp \psi(y))]}$$ 
\item[T2']  Being God-like is an essence of any
  God-like being: $$\hol{\allq x [G_A(x) \imp \essA{G_A}{x}]}$$
\item[D3'] \emph{Necessary existence} of an individual is the necessary exemplification of all its essences: 
  $$\hol{\NE_A(x) \equiv \allq \varphi [\essA{\varphi}{x} \imp \nec
  \exq y \varphi(y)]}$$
\item[A5'] Necessary existence is a positive property: $$\hol{P(\NE_A)}$$ 
\item[L1'] If a god-like being exists, then necessarily a god-like being exists: 
  $$\hol{\exq x G_A(x) \imp \nec \exq y G_A(y)}$$
\item[L2'] If possibly a god-like being exists, then necessarily a god-like being exists: 
  $$\hol{\pos \exq x G_A(x) \imp \nec \exq y G_A(y)} $$
%
\item[T3'] Necessarily, a God-like being exists: $$\hol{\nec \exq x G_A(x)}$$ 
\end{itemize}
% \end{minipage}
% } \vskip-.5em
% \caption{Anderson's Emendation \cite{ToDo:Anderson}.\label{fig:anderson}} 
% \end{figure}


% I have meanwhile added another file ``Anderson\_var\_partial.thy''. In this version the actualist e-Quantifier is used exclusively in the definition of NE and everywhere else ithe possibilist Quantifiers are used. It is still unclear to me, what is actually meant. Reading Anderson 1990, last sentence on p. 302, it could mean that ``Anderson\_var\_partial.thy'' is the correct version; but when reading AndersonGettings, then one could think that ``Anderson\_var.thy'' is meant. But ok we have now both versions. Interestingly in ``Anderson\_var\_partial.thy'' A5 is not inferable anymore, and this also holds for ``anderson\_implies\_fuhrmann". Nitpick finds countermodels.


%\clearpage

\section{Bjordal's Alternative} \label{apx:Bjordal}

% \begin{figure}[t]
% \noindent \framebox[\columnwidth][r]{
% \begin{minipage}{.94\columnwidth}\small
In Bjordal's emendation $G$ (God-like) is taken as primitive and $P$ (Positive) is defined (cf. definition D).
\begin{itemize}
\item[B:D] A formulas $\phi$ is positive iff it is necessarily the case
  that anything which is God-like has the property $\phi$.
  $$\hol{P(\phi) \equiv \nec \allq x (G(x) \imp \phi(x))}$$ 
\item[B:L1] D is logically equivalent in S4 with the union of G\"odel's definition D1 and axioms A2, A3 and A4.
  $$\hol{D \biimp D1 \wedge A2 \wedge A3 \wedge A4}$$
  The proof splits into the two implication directions B:L1$^\rightarrow$ and B:L1$^\leftarrow$. B:L1$^\rightarrow$ can be further split into four single steps.
\item[B:D2] $\phi$ is a maximal composite of object $x$'s positive properties iff $x$ has $\phi$ and $\phi$ is positive and all positive properties $\psi$ which $x$ has are such that is necessarily the case that all objects which have $\phi$ also have $\psi$.
  $$\hol{MCP(\phi,x) \equiv (\phi(x) \wedge P(\phi)) \wedge \allq \psi ((\psi(x) \wedge P(\psi)) \imp \nec \allq y (\phi(y) \imp \psi(y)))}$$
\item[B:D3] $x$ has the $N$-property iff x is such that if $\phi$ is a maximal composite of $x$'s positive properties then it is necessary that some object $y$ has the property $\phi$.
  $$\hol{N(x) \equiv \allq \phi (MCP(\phi,x) \imp \nec \allq y \phi(y))}$$
\item[B:A1] If a property is positive, its negation is not positive:
  $$\hol{\allq \varphi [P(\varphi) \imp \neg P(\neg \varphi)]}$$ 
\item[B:A2] The $N$-property is positive.
 $$\hol{P(N)}$$
\item[T3] Necessarily, a God-like being exists: $$\hol{\nec \exq x G(x)}$$ 
\end{itemize}
% \end{minipage}
% } \vskip-.5em
% \caption{Frode's Alternative \cite{ToDo:Frode}.\label{fig:frode}} 
% \end{figure}

%\clearpage
\end{appendix}


% ------------------------------------------------------------------------
\end{document}
% ------------------------------------------------------------------------

\documentclass{llncs}

\usepackage{url,amsmath,amssymb}

% Logical symbols
\newcommand{\imp}{\rightarrow}
\newcommand{\biimp}{\leftrightarrow}
\newcommand{\all}{\forall}
\newcommand{\ex}{\exists}
\newcommand{\seq}{\vdash}
\newcommand{\nec}{\Box} % necessarily
\newcommand{\pos}{\Diamond} % possibly


\title{ Formalization, Mechanization and Automation % of Variants
  of G\"{o}del's Proof of God's Existence\thanks{This work has been
    supported by the German Research Foundation under grant
    BE2501/9-1. Bruno ????}  }

\author{
  Christoph Benzm\"{u}ller\inst{1} 
  \and 
  Bruno Woltzenlogel Paleo\inst{2}
}

\authorrunning{C.\~Benzm\"{u}ller \and B.\~Woltzenlogel Paleo}


\institute{
  Dahlem Center for Intelligent Systems, Freie Universit\"{a}t Berlin, Germany\\
  \email{c.benzmueller@gmail.com}
  \and 
  Theory and Logic Group, Vienna University of Technology, Austria \\
  \email{bruno@logic.at}
}

\begin{document}

\maketitle

Attempts to prove the existence (or non-existence) of God by means of
abstract ontological arguments are an old tradition in philosophy and
theology.  G\"{o}del's proof \cite{Goedel1970} is a modern culmination of
this tradition, following particularly the footsteps of Leibniz.
%
G\"{o}del defines God as a being who possesses all \emph{positive} properties.
He does not extensively discuss what positive properties are, 
but instead he states a few reasonable but debatable axioms that they should satisfy.
Various slightly different versions of axioms and definitions have been considered by G\"{o}del and by several philosophers who commented on his proof (e.g. \cite{Scott,Sobel,AndersonGettings,Fitting,Adams,ContemporaryBibliography}). Our formalization employs the following axioms (A*) and definitions (D*):\footnote{
(A1), (A2), (A5), (D1), (D3) are logically equivalent to, respectively, axioms 5, 2 and 4 and definitions 1 and 3 in G\"odel's manuscript \cite{Goedel1970}. (A3) was introduced by Scott \cite{Scott} and could be derived from G\"odel's axiom 1 and (D1) in a logic with infinitary conjunction. (A4) is a weaker form of G\"odel's axiom 3. (D2) has an extra conjunct lacking in G\"odel's definition 2; this lack is believed to have been an oversight by G\"odel \cite{Hazen}.
}
\marginpar{\scriptsize Chris: The reference to Scott's version is kind of confusing, we need to verify this sources.}
\allowdisplaybreaks[1] 
\begin{align}
& \text{Any property necessarily implied by a positive property is positive.} \notag \\
& \quad \forall P \forall Q (pos\,P \wedge (\Box \forall x (P\,x \Rightarrow Q\,x)) \Rightarrow pos\,Q) \tag{A1} \\
& \text{A property is positive if and only if its negation is not positive.} \notag \\
& \quad \forall P (pos\,P \Leftrightarrow \neg (pos\,\neg P)) \tag{A2} \\
& \text{The property of being God-like is positive.} \notag \\
& \quad pos\,god \tag{A3} \\
& \text{Positive properties are necessarily positive.} \notag \\
& \quad \forall P (pos\,P \Rightarrow \Box (pos\,P))\tag{A4} \\
<<<<<<< HEAD
& \text{Necessary existence is a positive property.} \notag \\
& \quad pos\,nec\_exists \tag{A5} 
\end{align}
\begin{align}
& \text{x is God-like if and only if x has every positive property.} \notag \\
& \quad god\,x := \forall P (pos\,P \Rightarrow P\,x) \tag{D1} \\ 
& \parbox[c][2em][c]{0.8\textwidth}{A property P is an essence of x if and only if P is a property of x and every property Q that x has is necessarily implied by P.}\notag \\
& \quad ess\,P\,x := P\,x \wedge \forall Q (Q\,x \Rightarrow \Box \forall y (P\,y \Rightarrow Q\,y)) \tag{D2} \\
& \parbox[c][2em][c]{0.8\textwidth}{x necessarily exists if and only if every essence of x is necessarily exemplified.} \notag \\
& \quad nec\_exists\,x := \forall P (ess\,P\,x \Rightarrow \Box \exists y (P\,y)) \tag{D3} 
=======
& \text{Necessary existence is a positive property:} \notag \\
& \quad pos\,nec\_exists \tag{A5} \\
& \parbox[c][2em][c]{0.8\textwidth}{A property P is an essence of x if and only if P is a property of x and every property Q that x has is necessarily implied by P:}\notag \\
& \quad ess\,P\,x := P\,x \wedge \forall Q (Q\,x \Rightarrow \Box (\forall y (P\,y \Rightarrow Q\,y))) \tag{D1} \\
& \text{x is God-like if and only if x has every positive property:} \notag \\
& \quad god\,x := \forall P (pos\,P \Rightarrow P x) \tag{D2}
>>>>>>> 5a379f5268dc8d4392b2f6671030a686821135aa
\end{align}

% \begin{description}
% \item[(A1)] Any property necessarily implied by a positive property is positive: \\
%        $\forall P \forall Q ((positive P \wedge (\Box \forall x (P x \Rightarrow Q x))) \Rightarrow positive Q)$
% \item[(A2)] A property is positive if and only if its negation is not positive:\\
%        $\forall P ((positive P) \Leftrightarrow \neg (positive \neg P))$ 
% \item[(A3)] The property of being God-like is positive:
      
<<<<<<< HEAD
% \item[(A4)] positive properties are necessarily positive; 
% \item[(A5)] necessary existence is a positive property.
% \item[(D1)] x is God-like if and only if x has every positive property, 
% \item[(D2)] a property P is an essence of x if and only if P is a property of x and every property Q that x has is necessarily implied by P, 
% \item[(D3)] x necessarily exists if and only if every essence of x is necessarily exemplified.
% \end{description}
=======
\item[(A4)] positive properties are necessarily positive; 
\item[(A5)] necessary existence is a positive property.
\item[(D1)] a property P is an essence of x if and only if P is a property of x and every property Q that x has is necessarily implied by P, 
\item[(D2)] x is God-like if and only if x has every positive property, 
\item[(D3)] x necessarily exists if and only if every essence of x is necessarily exemplified.
\end{description}
>>>>>>> 5a379f5268dc8d4392b2f6671030a686821135aa

\noindent
From these axioms and definitions we then infer:
\begin{align}
& \text{Positive properties are possibly exemplified.} \notag \\
& \quad \forall P (pos\,P \Rightarrow \Diamond \exists x (P\,x)) \tag{L1} \\
& \text{Possibly God exists.} \notag \\
& \quad \Diamond \exists x (god\,x) \tag{L2} \\
& \parbox[c][2em][c]{0.8\textwidth}{If x is God-like, then the property of being God-like is an essence of x.} \notag \\
& \quad \forall x (god\,x \Rightarrow ess\,god\,x) \tag{L3} \\
& \text{Necessarily God exists.} \notag \\
& \quad \Box \exists x (god\,x) \tag{T}
\end{align}


%
%ToDo: Include formulas too??
% \begin{description}
% \item[(L1)] Positive properties are possibly exemplified.
% \item[(L2)] Possibly God exists.
% \item[(L3)] If x is God-like, then the property of being 
% God-like is an essence of x.
% \item[(T)] \ Necessarily God exists.
% \end{description}

% From these axioms, he claims (without detailed proof) that:
% \begin{enumerate}
% \item If it is possible that God exists, then it is necessary that God exists.
% \item It is possible that God exists.
% \end{enumerate}

The above variant of G\"{o}del's proof has now been 
constructed for the first-time
with an unprecedent degree of detail and formality; cf.~\cite{FormalTheologyRepository}. The following has been done (and in this order):
\begin{itemize}
\item A detailed natural deduction proof.\marginpar{\scriptsize Chris: We need to make sure that this ND proof is based on a sound calculus.}
%
\item A formalization of the axioms, definitions and theorems in the TPTP THF syntax \cite{J22}.
%
\item Automatic verification of the consistency of the axioms and 
definitions with Nitpick \cite{Nitpick}.
%
\item Automatic demonstration of the theorems with the provers LEO-II \cite{LEO-II} and Satallax \cite{Satallax}.

\item A step-by-step formalization using the Coq proof assistant \cite{Coq}.

\item A formalization using the Isabelle proof assistant \cite{Isabelle} where the theorems (and some additional lemmata) have been automated with Sledgehammer \cite{Sledgehammer} and Metis \cite{Hurd03first-orderproof}.
\end{itemize}

G\"{o}del's proof is challenging to formalize and verify because it
requires an expressive logical language with modal operators
(\emph{possibily} and \emph{necessarily}) and with
quantififiers for individuals and sets of individuals (properties).  Our computer-assisted formalizations rely on an
embedding of the modal logic S5 into classical higher-order logic with
Henkin semantics \cite{J23,B9}. The formalization is thus essentially
done in classical higher-order logic where quantified S5 is emulated.

This work attests the maturity of contemporary interactive and
automated deduction tools for classical higher-order logic and it
demonstrates the elegance and practical relevance of the embeddings
based approach.  Most importantly, our work opens new perspectives for
a computer-assisted theoretical philosophy.  The critical discussion
of the underlying concepts, definitions and axioms remains a human
responsibility, but the computer can assist in building and checking
rigorously correct logical arguments. In case of logico-philosophical
disputes, the computer can check the disputing arguments and partially
fulfill Leibniz' dictum: Calculemus --- Let us calculate!

Future work includes an extensive study of other formalizations of
ontological arguments with our machinery. Variations of the
distinctive features of the base logic S5 (e.g. non-rigid symbols, 
varying domains, etc.) are enabled in these studies due
to the flexibility of the embeddings based approach.

\bibliographystyle{plain}
\bibliography{Bibliography}

\end{document}
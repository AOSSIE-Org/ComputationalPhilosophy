\documentclass{llncs}

\usepackage{latexsym}
\usepackage{bussproofs}
\EnableBpAbbreviations
\newcommand{\rl}[1]{\RightLabel{#1}}

% Logical symbols
\newcommand{\imp}{\rightarrow}
\newcommand{\biimp}{\leftrightarrow}
\newcommand{\all}{\forall}
\newcommand{\ex}{\exists}
\newcommand{\seq}{\vdash}
\newcommand{\nec}{\Box} % necessarily
\newcommand{\pos}{\Diamond} % possibly


\title{
  Formalization and Automation of Variants of G\"{o}del's Ontological Proof of 
  God's Existence
  %\thanks{Supported by ... }
}

\author{
  Christoph Benzm\"{u}ller\inst{1} 
  \and 
  Bruno Woltzenlogel Paleo\inst{2}
}

\authorrunning{C.\~Benzm\"{u}ller \and B.\~Woltzenlogel Paleo}


\institute{
  Dahlem Center for Intelligent Systems, Freie Universit\"{a}t Berlin, Germany\\
  \email{c.benzmueller@gmail.com}
  \and 
  Theory and Logic Group, Vienna University of Technology, Austria \\
  \email{bruno@logic.at}
}

\begin{document}

\maketitle

Attempts to prove the existence (or non-existence) of God by means of
abstract ontological arguments are an old tradition in philosophy and
theology.  G\"{o}del's proof \cite{Goedel} is a modern culmination of
this tradition, following particularly the footsteps of Leibniz.
%
G\"{o}del defines God as a being who possesses all \emph{positive} properties.
He does not extensively discuss what positive properties are, 
but instead he states a few reasonable but debatable axioms that they should satisfy.
Various slightly different versions of axioms and definitions have been considered by G\"{o}del and by several philosophers who commented on his proof. Our formalization employs the following axioms\footnote{Say where they come from and how they relate to G\"odel.}: (1) Any property necessarily implied by a positive property is positive;
(2) A property is positive if and only if its negation is not positive; (3) the property of being God-like is positive; (4) positive properties are necessarily positive; and (5) necessary existence is a positive property.
Moreover, the following definitions are employed:
(D1) x is God-like if and only if x has every positive property, (D2) a property P is an essence of x if and only if P is a property of x and every property Q that x has is necessarily implied by P, (D3) x necessarily exists if and only if every essence of x is necessarily exemplified.

From these axioms and definitions we then infer:
\begin{enumerate}
\item (L1) Positive properties are possibly exemplified.
\item (L2) Possibly God exists.
\item (L3) If x is God-like, then the property of being 
God-like is an essence of x.
\item (T) Necessarily God exists.
\end{enumerate}

% From these axioms, he claims (without detailed proof) that:
% \begin{enumerate}
% \item If it is possible that God exists, then it is necessary that God exists.
% \item It is possible that God exists.
% \end{enumerate}

The above variant of G\"{o}del's proof has now been 
constructed for the first-time
with the utmost degree of detail and formality; cf.~\cite{ToDo:GitHubRepository}. The following has been done:
\begin{itemize}
\item A detailed natural deduction proof.
%
\item A formalization of the axioms and theorems in the TPTP THF format \cite{J22}.
%
\item Automatic verification of the consistency of the axioms and 
definitions with Nitpick \cite{Nitpick}.
%
\item Automatic demonstration of the theorems with the provers LEO-II \cite{LEO-II} and Satallax \cite{Satallax}.

\item A step-by-step formalization using the Coq proof assistant \cite{ToDo}.

\item A formalization using the Isabelle proof assistant \cite{Isabelle} partially automated with Sledgehammer \cite{ToDo} and Metis \cite{ToDo}.
\end{itemize}

G\"{o}del's proof is challenging to formalize and verify 
because it requires very expressive logical languages with 
modal operators (\emph{possibily} and \emph{necessarily}) and higher-order quantifiers. 
Our computer-assisted formalizations rely on an embedding of the modal logic S5 
into classical higher-order logic with Henkin 
semantics \cite{J23} and 
employed recently developed interactive and automated deduction tools designed for this logic.

This work attests the maturity of contemporary interactive and
automated deduction tools and opens new perspectives for a
computer-assisted theoretical philosophy.  
The critical discussion of the underlying concepts, definitions and
axioms remains a human responsibility, but the computer can assist in
building and checking rigorously correct logical arguments. In case of
logico-philosophical disputes, the computer can check the disputing
arguments and partially fulfill Leibniz' dictum: Calculemus --- Let us
calculate!

Future work includes an
extensive study of other formalizations of ontological arguments with
our machinery.

\bibliographystyle{plain}
\bibliography{Bibliography}

\end{document}
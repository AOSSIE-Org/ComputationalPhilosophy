\documentclass{llncs}

\usepackage{latexsym}
\usepackage{bussproofs}
\EnableBpAbbreviations
\newcommand{\rl}[1]{\RightLabel{#1}}

% Logical symbols
\newcommand{\imp}{\rightarrow}
\newcommand{\biimp}{\leftrightarrow}
\newcommand{\all}{\forall}
\newcommand{\ex}{\exists}
\newcommand{\seq}{\vdash}
\newcommand{\nec}{\Box} % necessarily
\newcommand{\pos}{\Diamond} % possibly


\title{
  Formalizations of G\"{o}del's Ontological Proof of \\
  God's Existence
  %\thanks{Supported by ... }
}

\author{
  Christoph Benzm\"{u}ller\inst{1} 
  \and 
  Bruno Woltzenlogel Paleo\inst{2}
}

\authorrunning{C.\~Benzm\"{u}ller \and B.\~Woltzenlogel Paleo}


\institute{
  Theory and Logic Group, Vienna University of Technology, Austria \\
  \email{bruno@logic.at}
  \and
  Dahlem Center for Intelligent Systems, Freie Universit\"{a}t Berlin, Germany\\
  \email{c.benzmueller@gmail.com}
}

\begin{document}

\maketitle

Attempts to prove the existence (or non-existence) of God by means of 
abstract ontological arguments are an old tradition in philosophy and theology,
G\"{o}del's proof is a modern culmination of this tradition, following particularly 
the footsteps of Leibniz.
%
G\"{o}del defines God as a being who possesses all \emph{positive} properties.
He does not extensively discuss what positive properties are, 
but instead he states a few reasonable but debatable axioms that they should satisfy.
From these axioms, he claims (without detailed proof) that:
\begin{enumerate}
\item If it is possible that God exists, then it is necessary that God exists.
\item It is possible that God exists.
\end{enumerate}

In the literature \cite{ToDo} slightly different versions of the axioms exist 
and derivations of the above mentioned claims are presented with various degrees
of detail and formality. Here \cite{ToDo: GitHub Repository} (a version of) G\"{o}del's proof is 
constructed for the first-time
with the utmost degree of detail and formality. The following has been done:
\begin{itemize}
\item A detailed natural deduction proof.
%
\item A formalization of the axioms and theorems in the TPTP THF format \cite{ToDo}.
%
\item Automatic verification of the consistency of the axioms and 
definitions with ToDo \cite{ToDo}.
%
\item Automatic demonstration of the theorems with the provers LEO-II \cite{ToDo} and Satallax \cite{ToDo}.

\item A step-by-step formalization using the Coq proof assistant \cite{ToDo}.

\item A formalization using the Isabelle proof assistant \cite{ToDo} partially automated with Sledgehammer \cite{ToDo} and Metis \cite{ToDo}.
\end{itemize}

G\"{o}del's proof is challenging to formalize and automatically verify 
because it requires very expressive logical languages with 
modal operators (\emph{possible} and \emph{necessary}) and higher-order quantifiers. 
The computer-assisted formalizations rely on an embedding of the modal logic S5 
into classical higher-order logic with Henkin 
semantics \cite{ToDo: Current works of Christoph Benzmüller and Larry Paulson} and 
employed recently developed interactive and automated deduction tools designed for this logic.

This work attests the maturity of such automated deduction tools and 
opens new perspectives for a computer-assisted theoretical philosophy, 
where the critical discussion of the underlying 
concepts, definitions and axioms remains a human responsibility, but the computer can assist 
in building and checking rigorously correct logical arguments. And in case of 
logico-philosophical disputes, the computer can check the disputing arguments
and partially fulfill Leibniz' dictum:
Calculemus --- Let us calculate!
\end{document}
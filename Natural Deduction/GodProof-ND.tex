\documentclass{article}

\usepackage{latexsym}
\usepackage{bussproofs}
\EnableBpAbbreviations
\newcommand{\rl}[1]{\RightLabel{#1}}

% Logical symbols
\newcommand{\imp}{\rightarrow}
\newcommand{\biimp}{\leftrightarrow}
\newcommand{\all}{\forall}
\newcommand{\ex}{\exists}
\newcommand{\seq}{\vdash}
\newcommand{\nec}{\Box} % necessarily
\newcommand{\pos}{\Diamond} % possibly

\author{Bruno Woltzenlogel Paleo, Annika Siders}

\title{G\"{o}del's Ontological Proof of God's Existence (Draft)}

\begin{document}

\maketitle

\newcommand{\ess}[2]{#1 \ \mathit{ess} \ #2}


\noindent
``There is a scientific (exact) philosophy and theology,
which deals with concepts of the highest abstractness; and this is also most highly fruitful for science. [\ldots] Religions are, for the most part, bad; but religion is not.'' - Kurt G\"{o}del


\section{Possible witnessing of positive properties}

\textbf{Axioms:}
\begin{itemize}
\item \textbf{(A1)} Properties necessarily entailed by \emph{positive} properties are also positive:
$$
\all \varphi. \all \psi.[(P(\varphi) \wedge \nec \all x.[\varphi(x) \imp \psi(x)]) \imp P(\psi)]
$$
%
\item \textbf{(A2)} A property's negation is positive iff the property is not positive:
$$
\all \varphi. [P(\neg \varphi) \biimp \neg P(\varphi)]
$$
\end{itemize}

\noindent
\textbf{Lemma 1:} Positive properties possibly have a witness:
$$
\all \varphi. [P(\varphi) \imp \pos \ex x.\varphi(x)]
$$


\noindent
\textbf{Formal proof:}

\begin{prooftree}
        \AXC{$ \all \varphi. \all \psi.[(P(\varphi) \wedge \nec \all x.[\varphi(x) \imp \psi(x)]) \imp P(\psi)]$}
        \UIC{$ \all \psi.[(P(\varphi') \wedge \nec \all x.[\varphi'(x) \imp \psi(x)]) \imp P(\psi)]$}
        \UIC{$(P(\varphi') \wedge \nec \all x.[\varphi'(x) \imp \neg \varphi'(x)]) \imp P(\neg \varphi')$} \doubleLine
        \UIC{$(P(\varphi') \wedge \nec \all x.[\neg \varphi'(x)]) \imp P(\neg \varphi')$}
                        \AXC{$\all \varphi.[ P(\neg \varphi) \biimp \neg P(\varphi) ]$}
                        \UIC{$ P(\neg \varphi') \biimp \neg P(\varphi') $} \doubleLine
                 \BIC{$ (P(\varphi') \wedge \nec \all x.[\neg \varphi'(x)]) \imp \neg P(\varphi') $} \doubleLine
                 \UIC{$ P(\varphi') \imp \pos \ex x.\varphi'(x) $}
                 \UIC{$\all \varphi.[ P(\varphi) \imp \pos \ex x.\varphi(x) ] $}
\end{prooftree}



\section{Possible existence of a God}

\textbf{Axioms:}
\begin{itemize}
\item \textbf{(A3)} Being God is a positive property:
$$
P(G)
$$
\end{itemize}

\noindent
\textbf{Lemma 2:} It is possible that a God exists:
$$
\pos \ex x. G(x)
$$

\noindent
\textbf{Formal proof:}

\begin{prooftree}
\AXC{$P(G)$}
                 \AXC{$ $} \dashedLine \RightLabel{Th. 1}
                 \UIC{$\all \varphi.[ P(\varphi) \imp \pos \ex x.\varphi(x) ]$}
                 \UIC{$ P(G) \imp \pos \ex x.G(x) $}
    \BIC{$\pos \ex x. G(x)$}
\end{prooftree}


\section{Essentiality of being God}

\textbf{Definitions:}
\begin{itemize}
\item \textbf{(D1)} A property is \emph{essential} for an individual if and only if it holds for that inidividual and necessarily entails every other property that holds for that individual: $\ess{\varphi}{x} \biimp \varphi(x) \wedge \all \psi. (\psi(x) \imp \nec \all x. (\varphi(x) \imp \psi(x)))$
\end{itemize}

\noindent
\textbf{Axioms:}
\begin{itemize}
\item \textbf{(A4)} Positive properties are necessarily positive:
$$
\all \varphi.[P(\varphi) \to \Box \; P(\varphi)]
$$
\end{itemize}

\noindent
\textbf{Lemma 3}: If an individual is a God, then being God is an essential property for that individual:
$$
\all y.[G(y) \imp \ess{G}{y}]
$$

\noindent
\textbf{Formal proof:}

Let the following derivation with the open assumption $G(x)$ be $\Pi_1$: 

\begin{prooftree}
\AXC{$ \neg P(\psi)^1$} 
       \AXC{$ $} \RightLabel{Ax. 2}
       \UIC{$\forall \varphi.(\neg P(\varphi)\imp P(\neg\varphi))$}\RightLabel{$\forall$E}  
       \UIC{$\neg P(\psi)\imp P(\neg\psi)$}\RightLabel{$\imp$E}   
 \BIC{$P(\neg\psi)$}\RightLabel{} 
           	\AXC{$G(x)$} \dottedLine\RightLabel{Definition of G}
		 \UIC{$\forall \varphi .(P(\varphi)\imp \varphi(x))$}\RightLabel{$\forall$E}
              	\UIC{$P(\varphi)\imp \varphi(x)$}\RightLabel{$\imp$ E}
           \BIC{$ \neg \psi(x)$}\RightLabel{$\imp$E}  
            \AXC{$ \psi(x)^2$} \RightLabel{$\imp$E}  
             \BIC{$\bot $}\RightLabel{RAA, 1}  
              \UIC{$ P(\psi)$}\RightLabel{$\imp$ I, 2}
              \UIC{$ \psi(x)\imp P(\psi)$}
\end{prooftree}



Let the following derivation with the open assumption $G(x)$ be $\Pi_2$: 
\begin{prooftree}
       \AXC{$\psi(x)^1$} \RightLabel{}
             	\AXC{$\Pi_1 $}  \dashedLine
		\UIC{$\psi(x)\imp P(\psi)$}\RightLabel{$\imp$E}
       		\BIC{$P(\psi)$}\RightLabel{$\imp$E}  
		\AXC{$$} \RightLabel{Ax. 4}
       		\UIC{$\forall \psi .(P(\psi)\imp \Box P(\psi))$}\RightLabel{$\forall$E}  
		\UIC{$P(\psi)\imp \Box P(\psi)$}\RightLabel{$\imp$E}  
          \BIC{$\Box P(\psi)$}\RightLabel{$\imp$I, 1} 
           \UIC{$\psi(x)\imp\Box P(\psi)$} 
\end{prooftree}


Let the following derivation without open assumptions be $\Pi_3$: 

\begin{prooftree}
       \AXC{$P(\psi)^1$} \RightLabel{}
             	\AXC{$G(x)^2$} \dottedLine\RightLabel{Definition of G}
		\UIC{$\forall\varphi .(P(\varphi)\imp \varphi(x))$}\RightLabel{$\forall$E}
		\UIC{$P(\psi)\imp \psi(x)$}\RightLabel{$\imp$E}  
       		\BIC{$\psi(x)$}\RightLabel{$\imp$I, 2}  
       		\UIC{$G(x)\imp \psi(x)$}\RightLabel{$\forall$I} 
		\UIC{$\forall x .(G(x)\imp \psi(x))$} \RightLabel{$\imp$I, 1}  
		\UIC{$P(\psi)\imp \forall x .(G(x)\imp \psi(x))$}
\end{prooftree}

Let the following derivation with the open assumption $G(x)$ be $\Pi_4$: 
\begin{prooftree}
       \AXC{$\psi(x)^1 $} \RightLabel{}
             	\AXC{$\Pi_2$}  \dashedLine
		\UIC{$\psi(x)\imp\Box P(\psi)$}\RightLabel{$\imp$E}
		\BIC{$\Box P(\psi)$}
		  	\AXC{$\Box P(\psi)^2 $}\RightLabel{$\Box$E}  
       			\UIC{$P(\psi)$}
				\AXC{$\Pi_3$}  \dashedLine\RightLabel{}
       				\UIC{$P(\psi)\imp \forall x .(G(x)\imp \psi(x))$}\RightLabel{$\imp$E}  
       			\BIC{$\forall x .(G(x)\imp \psi(x))$}\RightLabel{Necessitation}
			\UIC{$\Box \forall x .(G(x)\imp \psi(x))$}  \RightLabel{$\imp$I, 2} 
			\UIC{$\Box P(\psi)\imp \Box \forall x .(G(x)\imp \psi(x))$} \RightLabel{$\imp$E} 
         	 \BIC{$\Box \forall x .(G(x)\imp \psi(x)$}\RightLabel{$\imp$I, 1}
           \UIC{$\psi(x)\imp\Box \forall x .(G(x)\imp \psi(x))$}  
\end{prooftree}


The use of the necessitation rule is valid, because the only open assumption $\Box P(\psi)$ is boxed. 


We construct a derivation of theorem 3 with a subderivation $\Pi_4[G(x)^1]$, which means that the open assumption $G(x)$ in $\Pi_4$ is discharged with the rule labeled 1. 

\begin{prooftree}
       \AXC{$G(x)^1 $} \RightLabel{}
             	\AXC{$\Pi_4[G(x)^1]$}  \dashedLine
		\UIC{$\psi(x)\imp\Box \forall x .(G(x)\imp \psi(x))$}\RightLabel{$\forall $I}
		\UIC{$\forall \psi .(\psi(x)\imp\Box \forall x .(G(x)\imp \psi(x)))$}\RightLabel{$\&$I}
		\BIC{$G(x)\& \forall \psi .(\psi(x)\imp\Box \forall x .(G(x)\imp \psi(x)))$}\dottedLine\RightLabel{Definition of ess} 
       			\UIC{$\ess{G}{x}$} \RightLabel{$\imp$I, 1}
			\UIC{$G(x)\imp \ess{G}{x}$}
			  \UIC{$\all y.[G(y) \imp \ess{G}{y}]$}
\end{prooftree}


\section{Necessity of God's existence}

\textbf{Definitions:}
\begin{itemize}
\item \textbf{(D2)} An individual is a \emph{God} if and only if he possesses all positive properties:
$$
G(x) \biimp \forall \varphi. [P(\varphi) \to \varphi(x)]
$$
\item \textbf{(D3)} An individual \emph{necessarily exists} if and only if all its essential properties are necessarily witnessed:
$$
E(x) \biimp \all \varphi.[\ess{\varphi}{x} \imp \nec \ex x.\varphi(x)]
$$
\end{itemize}

\noindent
\textbf{Axioms:}
\begin{itemize}
\item \textbf{(A5)} Necessary existence is a positive property:
$$
P(E)
$$
\end{itemize}

\noindent
\textbf{Auxiliary Lemma:} If there is a God, then there necessarily exists a God:
$$
\ex z. G(z) \imp \nec \ex x. G(x)
$$


\noindent
\textbf{Formal proof:}

\begin{prooftree}
\AXC{$ $} \RightLabel{1}
\UIC{$\ex z. G(z)$}
\UIC{$G(g)$}
\end{prooftree}

\begin{prooftree}
\AXC{$ $} \dashedLine
\UIC{$G(g)$}
        \AXC{$ $} \dashedLine \RightLabel{Th. 3}
        \UIC{$\all y.[G(y) \imp \ess{G}{y}]$}
        \UIC{$G(g) \imp \ess{G}{g}$}
    \BIC{$\ess{G}{g}$}
                \AXC{$P(E)$}
                        \AXC{$ $} \dashedLine
                        \UIC{$G(g)$} \dottedLine
                        \UIC{$\all \varphi.[P(\varphi) \imp \varphi(g)]$}
                        \UIC{$P(E) \imp E(g)$}
                     \BIC{$E(g)$} \dottedLine
                     \UIC{$ \all \varphi.[\ess{\varphi}{g} \imp \nec \ex x.\varphi(x)] $}
                     \UIC{$ \ess{G}{g} \imp \nec \ex x. G(x) $}
        \BIC{$\nec \ex x. G(x)$} \RightLabel{1}
        \UIC{$\ex z. G(z) \imp \nec \ex x. G(x)$}
\end{prooftree}


\section{Necessary existence of a God}


\noindent
\textbf{Theorem:} The existence of a God is necessary:
$$
\nec \ex x. G(x)
$$


\noindent
\textbf{Formal proof:}

\begin{small}
\begin{prooftree}
\AXC{$ $} \dashedLine \RightLabel{\textbf{S5}}
\UIC{$ \all \varphi. [\pos \ldots \pos \nec \varphi \biimp \nec \varphi] $}
\UIC{$ \pos \nec \ex x. G(x) \biimp \nec \ex x. G(x) $}
        \AXC{$ $} \dashedLine \RightLabel{Th. 2}
        \UIC{$\pos \ex x. G(x)$}
                \AXC{$ $} \dashedLine \RightLabel{Th. A}
                \UIC{$\ex z. G(z) \imp \nec \ex x. G(x)$}
            \BIC{$\pos \nec \ex x. G(x)$}
    \BIC{$\nec \ex x. G(x)$}
\end{prooftree}
\end{small}


\section{God's existence}


\noindent
\textbf{Axioms:}
\begin{itemize}
\item \textbf{(M)} What is necessary is the case:
$$
\all \varphi. [\nec \varphi \imp \varphi]
$$
\end{itemize}

\noindent
\textbf{Corollary:} There exists a God:
$$
\ex x. G(x)
$$

\noindent
\textbf{Formal proof:}


\begin{prooftree}
\AXC{$ $} \dashedLine \RightLabel{Th. 4}
\UIC{$ \nec \ex x. G(x)$}
        \AXC{$ \all \varphi. [\nec \varphi \imp \varphi]$}
        \UIC{$\nec \ex x. G(x) \imp \ex x. G(x)$}
    \BIC{$\ex x. G(x)$}
\end{prooftree}



\end{document}
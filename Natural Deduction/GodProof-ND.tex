\documentclass{article}

\usepackage{fancybox}
\usepackage{latexsym}
\usepackage{proof}
\usepackage{bussproofs}
\EnableBpAbbreviations
\newcommand{\rl}[1]{\RightLabel{#1}}


\usepackage{calculi}
\usepackage{theorems}


% Logical symbols
\newcommand{\imp}{\rightarrow}
\newcommand{\biimp}{\leftrightarrow}
\newcommand{\all}{\forall}
\newcommand{\ex}{\exists}
\newcommand{\seq}{\vdash}
\newcommand{\nec}{\Box} % necessarily
\newcommand{\pos}{\Diamond} % possibly

\author{Bruno Woltzenlogel Paleo, Annika Siders}

\title{G\"{o}del's Ontological Proof of God's Existence (Draft)}

\begin{document}

\maketitle

\newcommand{\ess}[2]{#1 \ \mathit{ess} \ #2}
\newcommand{\NE}{E}


\noindent
``There is a scientific (exact) philosophy and theology,
which deals with concepts of the highest abstractness; and this is also most highly fruitful for science. [\ldots] Religions are, for the most part, bad; but religion is not.'' - Kurt G\"{o}del

\section{Introduction}

ToDo: Do also Scott's and Gödel's proofs.


\section{Natural Deduction}

\newcommand{\s}{\qquad}

\begin{calculus}
{The intuitionistic natural deduction calculus \ND}
{fig:ND}

\vspace{1em}

\s\s
\infer[\imp_I]{A \imp B}{ B }
\s\s
\infer[\imp_I^n]{A \imp B}{ \infer*{B}{\infer[n]{A}{}} }
\s\s
\infer[\imp_E]{B}{A & A \imp B}

\vspace{2em}

\s\s
\infer[\wedge_I]{A \wedge B}{A & B}
\s\s
\infer[\wedge_{E_1}]{A}{A \wedge B}
\s\s
\infer[\wedge_{E_2}]{B}{A \wedge B}

\vspace{2em}

\s\s
\infer[\vee_E]{C}{A \vee B & \infer*{C}{\infer{A}{}} & \infer*{C}{\infer{B}{}}}
\s\s
\infer[\vee_{I_1}]{A \vee B}{A}
\s\s
\infer[\vee_{I_2}]{A \vee B}{B}

\vspace{2em}

\s
\infer[\all_I]{\all x. A[x]}{ A[\alpha] }
\s
\infer[\all_E]{A[t]}{ \all x. A[x] }
\s\s
\infer[\ex_I]{\ex x. A[x]}{ A[t] }
\s
\infer[\ex_E]{A[\beta]}{ \ex x. A[x] }

\vspace{1em}

ToDo: add intuitionistic rule for negation/bottom
\end{calculus}

\begin{calculus}
{Classical Rules/Axioms}
{fig:Classical}

ToDo: add classical rules/axioms that we need
\end{calculus}




\begin{calculus}
{Rules for Modal Operators}
{fig:NDK}

\vspace{1em}

\s\s\s\s
\infer[\nec_I]{\nec A}{\fbox{\infer*{A}{} }}
\s\s\s
\infer[\nec_E]{\fbox{ \infer*{}{A} }  }{\nec A}

\vspace{1em}
\end{calculus}

\noindent
A \emph{derivation} is a directed acyclich graph whose nodes are formulas and whose edges correspond to applications of the inference rules shown in Figures \ref{fig:ND} and \ref{fig:NDK}. Parts of a derivation may be surrounded by boxes. A \emph{proof} is a derivation that additionally satisfies the following conditions:

\begin{itemize}
\item \textbf{eigen-variable conditions:}
if $\rho$ is a $\all_I$ inference eliminating a variable $\alpha$, then any occurrence of $\alpha$ in the proof should be an ancestor of the occurrence of $\alpha$ eliminated by $\rho$;
if $\rho$ is a $\ex_E$ inference introducing a variable $\beta$, then any occurrence of $\beta$ in the proof should be a descendant of the occurrence of $\beta$ introduced by $\rho$.
%
\item \textbf{boxed assumption condition:} any assumption should be discharged within the box where it is made.
%
\item \textbf{unboxed root condition:} the proof's root should not be inside any box.
\end{itemize}

\noindent
Double lines are used to abbreviate tedious propositional reasoning steps in the derivations. Dashed lines are used to refer to a proof shown elsewhere. Dotted lines are used to indicate folding and unfolding of definitions.

 



\section{Possibly, God Exists}

\begin{axiom}
\label{A1}
Either a property or its negation is positive, but not both:
$$
\all \varphi. [P(\neg \varphi) \biimp \neg P(\varphi)]
$$
\end{axiom}

\begin{axiom}
\label{A2}
A property necessarily implied by a positive property is positive:
$$
\all \varphi. \all \psi.[(P(\varphi) \wedge \nec \all x.[\varphi(x) \imp \psi(x)]) \imp P(\psi)]
$$
\end{axiom}


\begin{theorem}
\label{T1}
Positive properties are possibly exemplified:
$$
\all \varphi. [P(\varphi) \imp \pos \ex x.\varphi(x)]
$$
\end{theorem}
\begin{proof} \hfill
\begin{prooftree}
        \AXC{Axiom \ref{A2}} \dashedLine
        \UIC{$ \all \varphi. \all \psi.[(P(\varphi) \wedge \nec \all x.[\varphi(x) \imp \psi(x)]) \imp P(\psi)]$} \RightLabel{$\all_E$}
        \UIC{$ \all \psi.[(P(\rho) \wedge \nec \all x.[\rho(x) \imp \psi(x)]) \imp P(\psi)]$} \RightLabel{$\all_E$}
        \UIC{$(P(\rho) \wedge \nec \all x.[\rho(x) \imp \neg \rho(x)]) \imp P(\neg \rho)$} \doubleLine
        \UIC{$(P(\rho) \wedge \nec \all x.[\neg \rho(x)]) \imp P(\neg \rho)$}
                        \AXC{Axiom \ref{A1}} \dashedLine
                        \UIC{$\all \varphi.[ P(\neg \varphi) \biimp \neg P(\varphi) ]$} \RightLabel{$\all_E$}
                        \UIC{$ P(\neg \rho) \biimp \neg P(\rho) $} \doubleLine
                 \BIC{$ (P(\rho) \wedge \nec \all x.[\neg \rho(x)]) \imp \neg P(\rho) $} \doubleLine
                 \UIC{$ P(\rho) \imp \pos \ex x.\rho(x) $} \RightLabel{$\all_I$}
                 \UIC{$\all \varphi.[ P(\varphi) \imp \pos \ex x.\varphi(x) ] $}
\end{prooftree}
\end{proof}

\begin{definition}
\label{D1}
A \emph{God-like} being possesses all positive properties:
$$
G(x) \biimp \forall \varphi. [P(\varphi) \to \varphi(x)]
$$
\end{definition}

\begin{axiom}
\label{A3}
The property of being God-like is positive:
$$
P(G)
$$
\end{axiom}

\begin{corollary}
\label{C1}
Possibly, God exists:
$$
\pos \ex x. G(x)
$$
\end{corollary}
\begin{proof} \hfill
\begin{prooftree}
\AXC{Axiom \ref{A3}} \dashedLine
\UIC{$P(G)$}
                 \AXC{Theorem \ref{T1}} \dashedLine
                 \UIC{$\all \varphi.[ P(\varphi) \imp \pos \ex x.\varphi(x) ]$} \RightLabel{$\all_E $}
                 \UIC{$ P(G) \imp \pos \ex x.G(x) $} \RightLabel{$\imp_E$}
    \BIC{$\pos \ex x. G(x)$}
\end{prooftree}
\end{proof}


\section{Being God is an essence of any God}

\begin{axiom}
\label{A4}
Positive properties are necessarily positive:
$$
\all \varphi.[P(\varphi) \to \Box \; P(\varphi)]
$$
\end{axiom}

\begin{definition}
\label{D2}
An \emph{essence} of an individual is a property possessed by it and necessarily implying any of its properties: 
$$
\ess{\varphi}{x} \biimp \varphi(x) \wedge \all \psi. (\psi(x) \imp \nec \all x. (\varphi(x) \imp \psi(x)))
$$
\end{definition}


\begin{theorem}
\label{T2}
Being God-like is an essence of any God-like being:
$$
\all y.[G(y) \imp \ess{G}{y}]
$$
\end{theorem}
\begin{proof}
Let the following derivation with the open assumption $G(x)$ be $\Pi_1[G(x)]$: 

\begin{prooftree}
\AXC{$ \neg P(\psi)^1$} 
       \AXC{Axiom \ref{A1}} \dashedLine
       \UIC{$\forall \varphi.(\neg P(\varphi)\imp P(\neg\varphi))$}\RightLabel{$\all_E$}  
       \UIC{$\neg P(\psi)\imp P(\neg\psi)$}\RightLabel{$\imp_E$}   
 \BIC{$P(\neg\psi)$}\RightLabel{} 
           	\AXC{$G(x)$} \dottedLine\RightLabel{D\ref{D1}}
		 \UIC{$\forall \varphi .(P(\varphi)\imp \varphi(x))$}\RightLabel{$\forall_E$}
              	\UIC{$P(\neg \psi)\imp \neg \psi(x)$}\RightLabel{$\imp_E$}
           \BIC{$ \neg \psi(x)$}\RightLabel{$\imp_E$}  
            \AXC{$ \psi(x)^2$} \RightLabel{$\imp_E$}  
             \BIC{$\bot $}\RightLabel{$\mathit{RAA}^1$}  
              \UIC{$ P(\psi)$}\RightLabel{$\imp_I^2$}
              \UIC{$ \psi(x)\imp P(\psi)$}
\end{prooftree}


\noindent
Let the following derivation with the open assumption $G(x)$ be $\Pi_2[G(x)]$: 
\begin{prooftree}
       \AXC{$\psi(x)^3$}
             	\AXC{$\Pi_1[G(x)] $}  \dashedLine
		\UIC{$\psi(x)\imp P(\psi)$}\RightLabel{$\imp_E$}
       		\BIC{$P(\psi)$} 
		              \AXC{Axiom \ref{A4}} \dashedLine
       	          \UIC{$\all \varphi .(P(\varphi)\imp \Box P(\varphi))$}\RightLabel{$\all_E$}  
		              \UIC{$P(\psi)\imp \Box P(\psi)$}\RightLabel{$\imp_E$}  
           \BIC{$\Box P(\psi)$}\RightLabel{$\imp_I^3$} 
           \UIC{$\psi(x)\imp\Box P(\psi)$} 
\end{prooftree}

\noindent
Let the following derivation without open assumptions be $\Pi_3$: 

\begin{prooftree}
       \AXC{$P(\psi)^4$}
             	\AXC{$G(x)^5$} \dottedLine\RightLabel{D\ref{D1}}
		\UIC{$\all\varphi .(P(\varphi)\imp \varphi(x))$}\RightLabel{$\all_E$}
		\UIC{$P(\psi)\imp \psi(x)$}\RightLabel{$\imp_E$}  
       		\BIC{$\psi(x)$}\RightLabel{$\imp_I^5$}  
       		\UIC{$G(x)\imp \psi(x)$}\RightLabel{$\all_I$} 
		\UIC{$\all x .(G(x)\imp \psi(x))$} \RightLabel{$\imp_I^4$}  
		\UIC{$P(\psi)\imp \forall x .(G(x)\imp \psi(x))$}
\end{prooftree}

\noindent
Let the following derivation with the open assumption $G(x)$ be $\Pi_4[G(x)]$: 
\begin{prooftree}
       \AXC{$\psi(x)^6 $} \RightLabel{}
             	\AXC{$\Pi_2$}  \dashedLine
		\UIC{$\psi(x)\imp\Box P(\psi)$}\RightLabel{$\imp_E$}
		\BIC{$\Box P(\psi)$}
		  	\AXC{$\Box P(\psi)^7 $}\RightLabel{$\Box_E$}  
       			\UIC{$P(\psi)$}
				\AXC{$\Pi_3$}  \dashedLine\RightLabel{}
       				\UIC{$P(\psi)\imp \all x .(G(x)\imp \psi(x))$}\RightLabel{$\imp_E$}  
       			\BIC{$\all x .(G(x)\imp \psi(x))$}\RightLabel{Necessitation}
			\UIC{$\Box \all x .(G(x)\imp \psi(x))$}  \RightLabel{$\imp_I^7$} 
			\UIC{$\Box P(\psi)\imp \Box \all x .(G(x)\imp \psi(x))$} \RightLabel{$\imp_E$} 
         	 \BIC{$\Box \all x .(G(x)\imp \psi(x))$}\RightLabel{$\imp_I^6$}
           \UIC{$\psi(x)\imp\Box \all x .(G(x)\imp \psi(x))$}  
\end{prooftree}

\noindent
The use of the necessitation rule above is correct, because the only open assumption $\Box P(\psi)$ is boxed. In the derivation of Theorem \ref{T2} below, the assumption $G(x)$ in the subderivation $\Pi_4[G(x)^8]$ is discharged by the rule labeled $8$.

\begin{prooftree}
       \AXC{$G(x)^8 $} \RightLabel{}
             	\AXC{$\Pi_4[G(x)^8]$}  \dashedLine
		\UIC{$\psi(x)\imp\Box \all x .(G(x)\imp \psi(x))$}\RightLabel{$\all_I$}
		\UIC{$\all \psi .(\psi(x)\imp\Box \all x .(G(x)\imp \psi(x)))$}\RightLabel{$\wedge_I$}
		\BIC{$G(x)\wedge \forall \psi .(\psi(x)\imp\Box \all x .(G(x)\imp \psi(x)))$}\dottedLine\RightLabel{D\ref{D2}} 
       			\UIC{$\ess{G}{x}$} \RightLabel{$\imp_I^8$}
			\UIC{$G(x)\imp \ess{G}{x}$} \RightLabel{$\all_I$}
			  \UIC{$\all y.[G(y) \imp \ess{G}{y}]$}
\end{prooftree}
\end{proof}

\section{If God's existence is possible, it is necessary}

\begin{definition}
\label{D3}
\emph{Necessary existence} of an individual is the necessary exemplification of all its essences:
$$
E(x) \biimp \all \varphi.[\ess{\varphi}{x} \imp \nec \ex y.\varphi(y)]
$$
\end{definition}

\begin{axiom}
\label{A5}
Necessary existence is a positive property:
$$
P(E)
$$
\end{axiom}

\begin{lemma}
\label{L1}
If there is a God, then necessarily there exists a God:
$$
\ex z. G(z) \imp \nec \ex x. G(x)
$$
\end{lemma}
\begin{proof} \hfill
\begin{prooftree}
\AXC{$ $} \RightLabel{1}
\UIC{$\ex z. G(z)$}
\UIC{$G(g)$}
\end{prooftree}

\begin{prooftree}
\AXC{$ $} \dashedLine
\UIC{$G(g)$}
        \AXC{Theorem \ref{T2}} \dashedLine
        \UIC{$\all y.[G(y) \imp \ess{G}{y}]$}
        \UIC{$G(g) \imp \ess{G}{g}$}
    \BIC{$\ess{G}{g}$}
                \AXC{Axiom \ref{A5}} \dashedLine
                \UIC{$P(E)$}
                        \AXC{$ $} \dashedLine
                        \UIC{$G(g)$} \dottedLine
                        \UIC{$\all \varphi.[P(\varphi) \imp \varphi(g)]$}
                        \UIC{$P(E) \imp E(g)$}
                     \BIC{$E(g)$} \dottedLine
                     \UIC{$ \all \varphi.[\ess{\varphi}{g} \imp \nec \ex x.\varphi(x)] $}
                     \UIC{$ \ess{G}{g} \imp \nec \ex x. G(x) $}
        \BIC{$\nec \ex x. G(x)$} \RightLabel{1}
        \UIC{$\ex z. G(z) \imp \nec \ex x. G(x)$}
\end{prooftree}
\end{proof}

\section{Necessarily, God exists}

ToDo: this is proven in a way that is slightly different from G\"odel's 1970.


\begin{theorem}
\label{T3}
Necessarily, God exists:
$$
\nec \ex x. G(x)
$$
\end{theorem}


\noindent
\textbf{Formal proof:}
Let the following derivation be $ \Pi$: 
 
\begin{small}
\begin{prooftree}
\AXC{$ $} \RightLabel{1}
\UIC{$\nec \ex x. G(x)$} \RightLabel{$\nec_I$}
\UIC{$\nec\nec \ex x. G(x)$}\RightLabel{$\imp_I^1$}
\UIC{$\nec \ex x. G(x)\imp\nec\nec \ex x. G(x)$} \RightLabel{propositional logic}
\UIC{$\neg\nec\nec \ex x. G(x)\imp\neg  \nec \ex x. G(x)$}\RightLabel{$\nec_I$}
\UIC{$\nec(\neg\nec\nec \ex x. G(x)\imp\neg  \nec \ex x. G(x))$}
\AXC{Axiom K} \dashedLine 
\BIC{$\nec\neg\nec\nec \ex x. G(x)\imp\nec\neg  \nec \ex x. G(x)$} \dottedLine \RightLabel{$\pos$}
 \UIC{$\nec\pos\neg\nec \ex x. G(x) \imp\nec\neg\nec \ex x. G(x) $}

\end{prooftree}
\end{small}

\begin{small}
\begin{prooftree}
\AXC{$ $} \RightLabel{1}
\UIC{$\nec\pos\neg\nec \ex x. G(x)$}
\AXC{$ \Pi$} \dashedLine 
\UIC{$\nec\pos\neg\nec \ex x. G(x) \imp\nec\neg\nec \ex x. G(x) $}
 \BIC{$\nec\neg\nec \ex x. G(x)$}\RightLabel{$\nec_E$, axiom T}
\UIC{$\neg\nec \ex x. G(x)$} 
\AXC{$ $} \dashedLine  \RightLabel{Lemma \ref{L1}}
\UIC{$\neg\nec \ex x. G(x) \imp \neg\ex x. G(x)$}\RightLabel{$\imp_E$}
 \BIC{$ \neg\ex x. G(x)$} \RightLabel{$\nec_I$}
\UIC{$\nec \neg\ex x. G(x)$}  \dottedLine
\UIC{$\neg \pos\ex x. G(x)$} 

\end{prooftree}
\end{small}

\begin{small}
\begin{prooftree}
\AXC{$ $}\dashedLine
\UIC{$\neg \pos\ex x. G(x)$} 
\AXC{ Corollary \ref{C1}} \dashedLine 
\UIC{$ \pos\ex x. G(x)$} \RightLabel{$\imp_E$}
 \BIC{$ \bot$} \RightLabel{$\imp_I^1$}
 \UIC{$\neg\nec\pos\neg\nec \ex x. G(x)$}
\AXC{ Axiom B} \dashedLine 
\UIC{$ \neg\nec \ex x. G(x)\imp \nec\pos\neg\nec \ex x. G(x)$} \doubleLine
\UIC{$ \neg \nec\pos\neg\nec \ex x. G(x)\imp  \nec \ex x. G(x)$} \RightLabel{$\imp_E$}
 \BIC{$ \nec \ex x. G(x)$}
\end{prooftree}
\end{small}

\textbf{Better proof, that directly proves $\ex x. G(x)$:}
\begin{small}
\begin{prooftree}
\AXC{$ $} \RightLabel{1}
\UIC{$\nec\neg\nec \ex x. G(x)$}\RightLabel{$\nec_E$, axiom T}
\UIC{$\neg\nec \ex x. G(x)$} 
\AXC{ Lemma \ref{L1} } \dashedLine
\UIC{$\neg\nec \ex x. G(x) \imp \neg\ex x. G(x)$}\RightLabel{$\imp_E$}
 \BIC{$ \neg\ex x. G(x)$} \RightLabel{$\nec_I$}
\UIC{$\nec \neg\ex x. G(x)$}  \dottedLine
\UIC{$\neg \pos\ex x. G(x)$} 
\end{prooftree}
\end{small}

\begin{small}
\begin{prooftree}
\AXC{$ $}\dashedLine
\UIC{$\neg \pos\ex x. G(x)$} 
\AXC{ Corollary \ref{C1}} \dashedLine 
\UIC{$ \pos\ex x. G(x)$} \RightLabel{$\imp_E$}
 \BIC{$ \bot$} \RightLabel{$\imp_I^1$}
 \UIC{$\neg\nec\neg\nec \ex x. G(x)$} \dottedLine
 \UIC{$\neg\nec\pos \neg \ex x. G(x)$}
\AXC{ Axiom B} \dashedLine 
\UIC{$ \neg \ex x. G(x)\imp\nec\pos \neg \ex x. G(x)$} \doubleLine
\UIC{$ \neg\nec\pos \neg \ex x. G(x)\imp  \neg  \neg \ex x. G(x)$} \RightLabel{$\imp_E$}
 \BIC{$ \neg  \neg \ex x. G(x)$}\RightLabel{$ \neg  \neg$ E}
 \UIC{$  \ex x. G(x)$}
\end{prooftree}
\end{small}

Note that the last step is classical and we do not prove the existential statement by providing an object for which the statement holds. 
This proof makes section \ref{GodsE} superflous and the use of axiom "M" unnecessary. 

The system used contains the $\nec_E$-rule with the restriction that we have a $\nec_I$ below it. This is equivalent to modal sytem K that contains axiom K and necessitation rule N.  We aslo use axiom B ($ A\imp\nec\pos A$). No other modal axioms are needed. 


\section{God exists}\label{GodsE}

\begin{axiom}[M]
\label{M} What is necessary is the case:
$$
\all \varphi. [\nec \varphi \imp \varphi]
$$
\end{axiom}

\begin{corollary}
\label{C2} 
There exists a God:
$$
\ex x. G(x)
$$
\end{corollary}
\begin{proof}\hfill
\begin{prooftree}
\AXC{Theorem \ref{T3}} \dashedLine
\UIC{$ \nec \ex x. G(x)$}
        \AXC{$ \all \varphi. [\nec \varphi \imp \varphi]$}
        \UIC{$\nec \ex x. G(x) \imp \ex x. G(x)$}
    \BIC{$\ex x. G(x)$}
\end{prooftree}
\end{proof}

\section{Proof system equivalent to system K}

We show that the proof system with boxed parts of derivations is equivalent to the system K. The modal system K consists of the axiom K and the necessitation rule N. 

\begin{axiom}[The transitivity axiom K]
\label{K}
$$
\nec(A\imp B)\imp  (\nec A\imp \nec B)
$$
\end{axiom}

\begin{axiom}[The necessitation rule]
\label{Necessitation}
If $A$ is a theorem, then $\nec A$ is a theorem. 
\end{axiom}


\begin{lemma}
\label{systemK}
The axiom K is derivable in the system. 
\end{lemma}

\begin{small}
\begin{prooftree}
\AXC{$\nec(A\imp B)^2 $}\RightLabel{$\nec_E$}
\UIC{$A\imp B$} 
      \AXC{$\nec A ^1 $}\RightLabel{$\nec_E$}
      \UIC{$A$} \RightLabel{$\imp_E$}
   \BIC{$ B$} \RightLabel{$\nec_I$}
   \UIC{$ \nec B$} \RightLabel{$\imp_I^1$}
   \UIC{$\nec A\imp \nec B$}  \RightLabel{$\imp_I^2$}
   \UIC{$\nec(A\imp B)\imp  (\nec A\imp \nec B)$}
\end{prooftree}
\end{small}

\begin{lemma}
\label{systemK}
Assuming the axiom K and the necessitation rule N, the open formula $\nec A $ and the existence of a derivation of $ B$ from the open assumption $ A$, then we can derive $\nec B$ without the rules for boxed parts of derivations.
\end{lemma}

\begin{small}
\begin{prooftree}
\AXC{$A^1 $}\noLine
\UIC{$\vdots$}\noLine
\UIC{$ B$} \RightLabel{$\imp_I^1$}
\UIC{$A\imp B$} \RightLabel{Necessitation}
\UIC{$\nec(A\imp B)$} 
      \AXC{Axiom K}\dashedLine
      \UIC{$\nec(A\imp B)\imp  (\nec A\imp \nec B)$} \RightLabel{$\imp_E$}
  \BIC{$\nec A\imp \nec B$} 
        \AXC{$\nec A ^1 $}\RightLabel{$\imp_E$}
    \BIC{$\nec B$}
\end{prooftree}
\end{small}


\end{document}






\documentclass{article}

\usepackage{latexsym}
\usepackage{bussproofs}
\EnableBpAbbreviations
\newcommand{\rl}[1]{\RightLabel{#1}}

\newtheorem{axiom}{Axiom}
\newtheorem{definition}{Definition}
\newtheorem{theorem}{Theorem}
\newtheorem{lemma}{Lemma}
\newtheorem{corollary}{Corollary}

\newenvironment{proof}[1][Proof]{\begin{trivlist}
\item[\hskip \labelsep {\bfseries #1}]}{\end{trivlist}}


% Logical symbols
\newcommand{\imp}{\rightarrow}
\newcommand{\biimp}{\leftrightarrow}
\newcommand{\all}{\forall}
\newcommand{\ex}{\exists}
\newcommand{\seq}{\vdash}
\newcommand{\nec}{\Box} % necessarily
\newcommand{\pos}{\Diamond} % possibly

\author{Bruno Woltzenlogel Paleo, Annika Siders}

\title{G\"{o}del's Ontological Proof of God's Existence (Draft)}

\begin{document}

\maketitle

\newcommand{\ess}[2]{#1 \ \mathit{ess} \ #2}
\newcommand{\NE}{E}


\noindent
``There is a scientific (exact) philosophy and theology,
which deals with concepts of the highest abstractness; and this is also most highly fruitful for science. [\ldots] Religions are, for the most part, bad; but religion is not.'' - Kurt G\"{o}del

\section{Introduction}

\section{Natural Deduction}

ToDo: Show and explain here the rules of the calculus we are using.

ToDo: We should use a calculus for the basic modal logic K. Everything else should be stated as axioms.

ToDo: cite a paper that proves soundness and completeness for this calculus.

\section{Possibly, God Exists}

\begin{axiom}
\label{A1}
Either a property or its negation is positive, but not both:
$$
\all \varphi. [P(\neg \varphi) \biimp \neg P(\varphi)]
$$
\end{axiom}

\begin{axiom}
\label{A2}
A property necessarily implied by a positive property is positive:
$$
\all \varphi. \all \psi.[(P(\varphi) \wedge \nec \all x.[\varphi(x) \imp \psi(x)]) \imp P(\psi)]
$$
\end{axiom}


\begin{theorem}
\label{T1}
Positive properties are possibly exemplified:
$$
\all \varphi. [P(\varphi) \imp \pos \ex x.\varphi(x)]
$$
\end{theorem}
\begin{proof} \hfill
\begin{prooftree}
        \AXC{Axiom \ref{A2}} \dashedLine
        \UIC{$ \all \varphi. \all \psi.[(P(\varphi) \wedge \nec \all x.[\varphi(x) \imp \psi(x)]) \imp P(\psi)]$} \RightLabel{$\all_E$}
        \UIC{$ \all \psi.[(P(\varphi') \wedge \nec \all x.[\varphi'(x) \imp \psi(x)]) \imp P(\psi)]$} \RightLabel{$\all_E$}
        \UIC{$(P(\varphi') \wedge \nec \all x.[\varphi'(x) \imp \neg \varphi'(x)]) \imp P(\neg \varphi')$} \doubleLine
        \UIC{$(P(\varphi') \wedge \nec \all x.[\neg \varphi'(x)]) \imp P(\neg \varphi')$}
                        \AXC{Axiom \ref{A1}} \dashedLine
                        \UIC{$\all \varphi.[ P(\neg \varphi) \biimp \neg P(\varphi) ]$} \RightLabel{$\all_E$}
                        \UIC{$ P(\neg \varphi') \biimp \neg P(\varphi') $} \doubleLine
                 \BIC{$ (P(\varphi') \wedge \nec \all x.[\neg \varphi'(x)]) \imp \neg P(\varphi') $} \doubleLine
                 \UIC{$ P(\varphi') \imp \pos \ex x.\varphi'(x) $} \RightLabel{$\all_I$}
                 \UIC{$\all \varphi.[ P(\varphi) \imp \pos \ex x.\varphi(x) ] $}
\end{prooftree}
\end{proof}

\begin{definition}
\label{D1}
A \emph{God-like} being possesses all positive properties:
$$
G(x) \biimp \forall \varphi. [P(\varphi) \to \varphi(x)]
$$
\end{definition}

\begin{axiom}
\label{A3}
The property of being God-like is positive:
$$
P(G)
$$
\end{axiom}

\begin{corollary}
\label{C1}
Possibly, God exists:
$$
\pos \ex x. G(x)
$$
\end{corollary}
\begin{proof} \hfill
\begin{prooftree}
\AXC{Axiom \ref{A3}} \dashedLine
\UIC{$P(G)$}
                 \AXC{Theorem \ref{T1}} \dashedLine
                 \UIC{$\all \varphi.[ P(\varphi) \imp \pos \ex x.\varphi(x) ]$} \RightLabel{$\all_E $}
                 \UIC{$ P(G) \imp \pos \ex x.G(x) $} \RightLabel{$\imp_E$}
    \BIC{$\pos \ex x. G(x)$}
\end{prooftree}
\end{proof}


\section{Being God is an essence of any God}

\begin{axiom}
\label{A4}
Positive properties are necessarily positive:
$$
\all \varphi.[P(\varphi) \to \Box \; P(\varphi)]
$$
\end{axiom}

\begin{definition}
\label{D2}
An \emph{essence} of an individual is a property possessed by it and necessarily implying any of its properties: 
$$
\ess{\varphi}{x} \biimp \varphi(x) \wedge \all \psi. (\psi(x) \imp \nec \all x. (\varphi(x) \imp \psi(x)))
$$
\end{definition}

ToDo: instead of using the $\nec_E$ rule, we should use the $M$ axiom.


\begin{theorem}
\label{T2}
Being God-like is an essence of any God-like being:
$$
\all y.[G(y) \imp \ess{G}{y}]
$$
\end{theorem}
\begin{proof}
Let the following derivation with the open assumption $G(x)$ be $\Pi_1[G(x)]$: 

\begin{prooftree}
\AXC{$ \neg P(\psi)^1$} 
       \AXC{Axiom \ref{A1}} \dashedLine
       \UIC{$\forall \varphi.(\neg P(\varphi)\imp P(\neg\varphi))$}\RightLabel{$\forall_E$}  
       \UIC{$\neg P(\psi)\imp P(\neg\psi)$}\RightLabel{$\imp_E$}   
 \BIC{$P(\neg\psi)$}\RightLabel{} 
           	\AXC{$G(x)$} \dottedLine\RightLabel{D\ref{D1}}
		 \UIC{$\forall \varphi .(P(\varphi)\imp \varphi(x))$}\RightLabel{$\forall_E$}
              	\UIC{$P(\varphi)\imp \varphi(x)$}\RightLabel{$\imp_E$}
           \BIC{$ \neg \psi(x)$}\RightLabel{$\imp_E$}  
            \AXC{$ \psi(x)^2$} \RightLabel{$\imp_E$}  
             \BIC{$\bot $}\RightLabel{$\mathit{RAA}^1$}  
              \UIC{$ P(\psi)$}\RightLabel{$\imp_I^2$}
              \UIC{$ \psi(x)\imp P(\psi)$}
\end{prooftree}


\noindent
Let the following derivation with the open assumption $G(x)$ be $\Pi_2[G(x)]$: 
\begin{prooftree}
       \AXC{$\psi(x)^3$}
             	\AXC{$\Pi_1[G(x)] $}  \dashedLine
		\UIC{$\psi(x)\imp P(\psi)$}\RightLabel{$\imp_E$}
       		\BIC{$P(\psi)$} 
		              \AXC{Axiom \ref{A4}} \dashedLine
       	          \UIC{$\forall \psi .(P(\psi)\imp \Box P(\psi))$}\RightLabel{$\all_E$}  
		              \UIC{$P(\psi)\imp \Box P(\psi)$}\RightLabel{$\imp_E$}  
           \BIC{$\Box P(\psi)$}\RightLabel{$\imp_I^3$} 
           \UIC{$\psi(x)\imp\Box P(\psi)$} 
\end{prooftree}

\noindent
Let the following derivation without open assumptions be $\Pi_3$: 

\begin{prooftree}
       \AXC{$P(\psi)^4$}
             	\AXC{$G(x)^5$} \dottedLine\RightLabel{D\ref{D1}}
		\UIC{$\all\varphi .(P(\varphi)\imp \varphi(x))$}\RightLabel{$\all_E$}
		\UIC{$P(\psi)\imp \psi(x)$}\RightLabel{$\imp_E$}  
       		\BIC{$\psi(x)$}\RightLabel{$\imp_I^5$}  
       		\UIC{$G(x)\imp \psi(x)$}\RightLabel{$\all_I$} 
		\UIC{$\all x .(G(x)\imp \psi(x))$} \RightLabel{$\imp_I^4$}  
		\UIC{$P(\psi)\imp \forall x .(G(x)\imp \psi(x))$}
\end{prooftree}

\noindent
Let the following derivation with the open assumption $G(x)$ be $\Pi_4[G(x)]$: 
\begin{prooftree}
       \AXC{$\psi(x)^6 $} \RightLabel{}
             	\AXC{$\Pi_2$}  \dashedLine
		\UIC{$\psi(x)\imp\Box P(\psi)$}\RightLabel{$\imp_E$}
		\BIC{$\Box P(\psi)$}
		  	\AXC{$\Box P(\psi)^7 $}\RightLabel{$\Box_E$}  
       			\UIC{$P(\psi)$}
				\AXC{$\Pi_3$}  \dashedLine\RightLabel{}
       				\UIC{$P(\psi)\imp \all x .(G(x)\imp \psi(x))$}\RightLabel{$\imp_E$}  
       			\BIC{$\all x .(G(x)\imp \psi(x))$}\RightLabel{Necessitation}
			\UIC{$\Box \all x .(G(x)\imp \psi(x))$}  \RightLabel{$\imp_I^7$} 
			\UIC{$\Box P(\psi)\imp \Box \all x .(G(x)\imp \psi(x))$} \RightLabel{$\imp_E$} 
         	 \BIC{$\Box \all x .(G(x)\imp \psi(x)$}\RightLabel{$\imp_I^6$}
           \UIC{$\psi(x)\imp\Box \all x .(G(x)\imp \psi(x))$}  
\end{prooftree}

\noindent
The use of the necessitation rule above is correct, because the only open assumption $\Box P(\psi)$ is boxed. In the derivation of Theorem \ref{T2} below, the assumption $G(x)$ in the subderivation $\Pi_4[G(x)^8]$ is discharged by the rule labeled $8$.

\begin{prooftree}
       \AXC{$G(x)^8 $} \RightLabel{}
             	\AXC{$\Pi_4[G(x)^8]$}  \dashedLine
		\UIC{$\psi(x)\imp\Box \all x .(G(x)\imp \psi(x))$}\RightLabel{$\all_I$}
		\UIC{$\all \psi .(\psi(x)\imp\Box \all x .(G(x)\imp \psi(x)))$}\RightLabel{$\wedge_I$}
		\BIC{$G(x)\wedge \forall \psi .(\psi(x)\imp\Box \all x .(G(x)\imp \psi(x)))$}\dottedLine\RightLabel{D\ref{D2}} 
       			\UIC{$\ess{G}{x}$} \RightLabel{$\imp_I^8$}
			\UIC{$G(x)\imp \ess{G}{x}$}
			  \UIC{$\all y.[G(y) \imp \ess{G}{y}]$}
\end{prooftree}
\end{proof}

\section{If God's existence is possible, it is necessary}

\begin{definition}
\label{D3}
\emph{Necessary existence} of an individual is the necessary exemplification of all its essences:
$$
E(x) \biimp \all \varphi.[\ess{\varphi}{x} \imp \nec \ex x.\varphi(x)]
$$
\end{definition}

\begin{axiom}
\label{A5}
Necessary existence is a positive property:
$$
P(E)
$$
\end{axiom}

\begin{lemma}
\label{L1}
If there is a God, then necessarily there exists a God:
$$
\ex z. G(z) \imp \nec \ex x. G(x)
$$
\end{lemma}
\begin{proof} \hfill
\begin{prooftree}
\AXC{$ $} \RightLabel{1}
\UIC{$\ex z. G(z)$}
\UIC{$G(g)$}
\end{prooftree}

\begin{prooftree}
\AXC{$ $} \dashedLine
\UIC{$G(g)$}
        \AXC{Theorem \ref{T2}} \dashedLine
        \UIC{$\all y.[G(y) \imp \ess{G}{y}]$}
        \UIC{$G(g) \imp \ess{G}{g}$}
    \BIC{$\ess{G}{g}$}
                \AXC{Axiom \ref{A5}} \dashedLine
                \UIC{$P(E)$}
                        \AXC{$ $} \dashedLine
                        \UIC{$G(g)$} \dottedLine
                        \UIC{$\all \varphi.[P(\varphi) \imp \varphi(g)]$}
                        \UIC{$P(E) \imp E(g)$}
                     \BIC{$E(g)$} \dottedLine
                     \UIC{$ \all \varphi.[\ess{\varphi}{g} \imp \nec \ex x.\varphi(x)] $}
                     \UIC{$ \ess{G}{g} \imp \nec \ex x. G(x) $}
        \BIC{$\nec \ex x. G(x)$} \RightLabel{1}
        \UIC{$\ex z. G(z) \imp \nec \ex x. G(x)$}
\end{prooftree}
\end{proof}

\section{Necessarily, God exists}

ToDo: This section still needs more details. See Coq formalization for more details.

ToDo: this is proven in a way that is slightly different from G\"odel's 1970.


\begin{theorem}
\label{T3}
Necessarily, God exists:
$$
\nec \ex x. G(x)
$$
\end{theorem}
\begin{proof}\hfill
\begin{small}
\begin{prooftree}
\AXC{\textbf{S5}} \dashedLine
\UIC{$ \all \varphi. [\pos \ldots \pos \nec \varphi \biimp \nec \varphi] $}
\UIC{$ \pos \nec \ex x. G(x) \biimp \nec \ex x. G(x) $}
        \AXC{Corollary \ref{C1}} \dashedLine
        \UIC{$\pos \ex x. G(x)$}
                \AXC{Lemma \ref{L1}} \dashedLine
                \UIC{$\ex z. G(z) \imp \nec \ex x. G(x)$} \doubleLine
            \BIC{$\pos \nec \ex x. G(x)$}
    \BIC{$\nec \ex x. G(x)$}
\end{prooftree}
\end{small}
\end{proof}


\section{God exists}

\begin{axiom}[M]
\label{M} What is necessary is the case:
$$
\all \varphi. [\nec \varphi \imp \varphi]
$$
\end{axiom}

\begin{corollary}
\label{C2} 
There exists a God:
$$
\ex x. G(x)
$$
\end{corollary}
\begin{proof}\hfill
\begin{prooftree}
\AXC{Theorem \ref{T3}} \dashedLine
\UIC{$ \nec \ex x. G(x)$}
        \AXC{$ \all \varphi. [\nec \varphi \imp \varphi]$}
        \UIC{$\nec \ex x. G(x) \imp \ex x. G(x)$}
    \BIC{$\ex x. G(x)$}
\end{prooftree}
\end{proof}


\end{document}
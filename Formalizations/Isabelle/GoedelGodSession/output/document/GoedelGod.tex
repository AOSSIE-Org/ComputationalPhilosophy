%
\begin{isabellebody}%
\def\isabellecontext{GoedelGod}%
%
\isadelimtheory
%
\endisadelimtheory
%
\isatagtheory
%
\endisatagtheory
{\isafoldtheory}%
%
\isadelimtheory
%
\endisadelimtheory
%
\isamarkupsection{Introduction%
}
\isamarkuptrue%
%
\begin{isamarkuptext}%
Dana Scott's version \cite{ScottNotes}
 of Goedel's ontological argument \cite{GoedelNotes} for God's existence is here
 formalized in quantified modal logic KB (QML KB) within the proof assistant Isabelle/HOL. 
 QML KB is  modeled as a fragment of classical higher-order logic (HOL); 
 thus, the formalization is essentially a formalization in HOL. The employed embedding 
 of QML KB in HOL is adapting the work of Benzm\"uller and Paulson \cite{J23,B9}.
 Note that the QML KB formalization employs quantification over individuals and 
 quantification over sets of individuals (properties).

 The gaps in Scott's proof have been automated 
 with Sledgehammer \cite{Sledgehammer}, performing remote calls to the higher-order automated
 theorem prover LEO-II \cite{LEO-II}. Sledgehammer then suggests the 
 Metis \cite{Metis} calls. The Metis proofs are verified by Isabelle/HOL.
 For consistency checking, the model finder Nitpick \cite{Nitpick} has been employed.
 
 Isabelle is described in the textbook by Nipkow, 
 Paulson, and Wenzel \cite{Isabelle} and in tutorials available 
 at: \url{http://isabelle.in.tum.de}.
 
\subsection{Related Work}

 The formalization presented here is related to the THF \cite{J22} and 
 Coq \cite{Coq} formalizations at 
 \url{https://github.com/FormalTheology/GoedelGod/tree/master/Formalizations/}.
 
 A medieval ontological argument by Anselm was formalized in PVS by John Rushby \cite{ToDo}.
 \end{enumerate}%
\end{isamarkuptext}%
\isamarkuptrue%
%
\isamarkupsection{An Embedding of QML KB in HOL%
}
\isamarkuptrue%
%
\begin{isamarkuptext}%
The types \isa{i} for possible worlds and $\mu$ for individuals 
are introduced.%
\end{isamarkuptext}%
\isamarkuptrue%
\ \ \isacommand{typedecl}\isamarkupfalse%
\ i\ \ \ \ %
\isamarkupcmt{the type for possible worlds%
}
\ \isanewline
\ \ \isacommand{typedecl}\isamarkupfalse%
\ {\isasymmu}\ \ \ \ %
\isamarkupcmt{the type for indiviuals%
}
%
\begin{isamarkuptext}%
Possible worlds are connected by an accessibility relation \isa{r}.%
\end{isamarkuptext}%
\isamarkuptrue%
\ \ \isacommand{consts}\isamarkupfalse%
\ r\ {\isacharcolon}{\isacharcolon}\ {\isachardoublequoteopen}i\ {\isasymRightarrow}\ i\ {\isasymRightarrow}\ bool{\isachardoublequoteclose}\ {\isacharparenleft}\isakeyword{infixr}\ {\isachardoublequoteopen}r{\isachardoublequoteclose}\ {\isadigit{7}}{\isadigit{0}}{\isacharparenright}\ \ \ \ %
\isamarkupcmt{accessibility relation r%
}
%
\begin{isamarkuptext}%
The \isa{B} axiom (symmetry) for relation r is stated. \isa{B} is needed only 
for proving theorem T3.%
\end{isamarkuptext}%
\isamarkuptrue%
\ \ \isacommand{axiomatization}\isamarkupfalse%
\ \isakeyword{where}\ sym{\isacharcolon}\ {\isachardoublequoteopen}x\ r\ y\ {\isasymlongrightarrow}\ y\ r\ x{\isachardoublequoteclose}%
\begin{isamarkuptext}%
QML formulas are translated as HOL terms of type \isa{i\ {\isasymRightarrow}\ bool}. 
This type is abbreviated as \isa{{\isasymsigma}}.%
\end{isamarkuptext}%
\isamarkuptrue%
\ \ \isacommand{type{\isacharunderscore}synonym}\isamarkupfalse%
\ {\isasymsigma}\ {\isacharequal}\ {\isachardoublequoteopen}{\isacharparenleft}i\ {\isasymRightarrow}\ bool{\isacharparenright}{\isachardoublequoteclose}%
\begin{isamarkuptext}%
The classical connectives $\neg, \wedge, \rightarrow$, and $\forall$
(over individuals and over sets of individuals) and $\exists$ (over individuals) are
lifted to type $\sigma$. The lifted connectives are \isa{m{\isasymnot}}, \isa{m{\isasymand}}, \isa{m{\isasymRightarrow}},
\isa{{\isasymforall}}, \isa{{\isasymPi}}, and \isa{{\isasymexists}}. Other connectives could be introduced analogously. 
Definitions could be used instead of abbreviations.%
\end{isamarkuptext}%
\isamarkuptrue%
\ \ \isacommand{abbreviation}\isamarkupfalse%
\ mnot\ {\isacharcolon}{\isacharcolon}\ {\isachardoublequoteopen}{\isasymsigma}\ {\isasymRightarrow}\ {\isasymsigma}{\isachardoublequoteclose}\ {\isacharparenleft}{\isachardoublequoteopen}m{\isasymnot}{\isachardoublequoteclose}{\isacharparenright}\ \isakeyword{where}\ {\isachardoublequoteopen}m{\isasymnot}\ {\isasymphi}\ {\isasymequiv}\ {\isacharparenleft}{\isasymlambda}w{\isachardot}\ {\isasymnot}\ {\isasymphi}\ w{\isacharparenright}{\isachardoublequoteclose}\ \ \ \ \isanewline
\ \ \isacommand{abbreviation}\isamarkupfalse%
\ mand\ {\isacharcolon}{\isacharcolon}\ {\isachardoublequoteopen}{\isasymsigma}\ {\isasymRightarrow}\ {\isasymsigma}\ {\isasymRightarrow}\ {\isasymsigma}{\isachardoublequoteclose}\ {\isacharparenleft}\isakeyword{infixr}\ {\isachardoublequoteopen}m{\isasymand}{\isachardoublequoteclose}\ {\isadigit{7}}{\isadigit{9}}{\isacharparenright}\ \isakeyword{where}\ {\isachardoublequoteopen}{\isasymphi}\ m{\isasymand}\ {\isasympsi}\ {\isasymequiv}\ {\isacharparenleft}{\isasymlambda}w{\isachardot}\ {\isasymphi}\ w\ {\isasymand}\ {\isasympsi}\ w{\isacharparenright}{\isachardoublequoteclose}\ \ \ \isanewline
\ \ \isacommand{abbreviation}\isamarkupfalse%
\ mimplies\ {\isacharcolon}{\isacharcolon}\ {\isachardoublequoteopen}{\isasymsigma}\ {\isasymRightarrow}\ {\isasymsigma}\ {\isasymRightarrow}\ {\isasymsigma}{\isachardoublequoteclose}\ {\isacharparenleft}\isakeyword{infixr}\ {\isachardoublequoteopen}m{\isasymRightarrow}{\isachardoublequoteclose}\ {\isadigit{7}}{\isadigit{4}}{\isacharparenright}\ \isakeyword{where}\ {\isachardoublequoteopen}{\isasymphi}\ m{\isasymRightarrow}\ {\isasympsi}\ {\isasymequiv}\ {\isacharparenleft}{\isasymlambda}w{\isachardot}\ {\isasymphi}\ w\ {\isasymlongrightarrow}\ {\isasympsi}\ w{\isacharparenright}{\isachardoublequoteclose}\ \ \isanewline
\ \ \isacommand{abbreviation}\isamarkupfalse%
\ mforall{\isacharunderscore}ind\ {\isacharcolon}{\isacharcolon}\ {\isachardoublequoteopen}{\isacharparenleft}{\isasymmu}\ {\isasymRightarrow}\ {\isasymsigma}{\isacharparenright}\ {\isasymRightarrow}\ {\isasymsigma}{\isachardoublequoteclose}\ {\isacharparenleft}{\isachardoublequoteopen}{\isasymforall}{\isachardoublequoteclose}{\isacharparenright}\ \isakeyword{where}\ {\isachardoublequoteopen}{\isasymforall}\ {\isasymPhi}\ {\isasymequiv}\ {\isacharparenleft}{\isasymlambda}w{\isachardot}\ {\isasymforall}x{\isachardot}\ {\isasymPhi}\ x\ w{\isacharparenright}{\isachardoublequoteclose}\ \ \ \isanewline
\ \ \isacommand{abbreviation}\isamarkupfalse%
\ mforall{\isacharunderscore}indset\ {\isacharcolon}{\isacharcolon}\ {\isachardoublequoteopen}{\isacharparenleft}{\isacharparenleft}{\isasymmu}\ {\isasymRightarrow}\ {\isasymsigma}{\isacharparenright}\ {\isasymRightarrow}\ {\isasymsigma}{\isacharparenright}\ {\isasymRightarrow}\ {\isasymsigma}{\isachardoublequoteclose}\ {\isacharparenleft}{\isachardoublequoteopen}{\isasymPi}{\isachardoublequoteclose}{\isacharparenright}\ \isakeyword{where}\ {\isachardoublequoteopen}{\isasymPi}\ {\isasymPhi}\ {\isasymequiv}\ {\isacharparenleft}{\isasymlambda}w{\isachardot}\ {\isasymforall}x{\isachardot}\ {\isasymPhi}\ x\ w{\isacharparenright}{\isachardoublequoteclose}\isanewline
\ \ \isacommand{abbreviation}\isamarkupfalse%
\ mexists{\isacharunderscore}ind\ {\isacharcolon}{\isacharcolon}\ {\isachardoublequoteopen}{\isacharparenleft}{\isasymmu}\ {\isasymRightarrow}\ {\isasymsigma}{\isacharparenright}\ {\isasymRightarrow}\ {\isasymsigma}{\isachardoublequoteclose}\ {\isacharparenleft}{\isachardoublequoteopen}{\isasymexists}{\isachardoublequoteclose}{\isacharparenright}\ \isakeyword{where}\ {\isachardoublequoteopen}{\isasymexists}\ {\isasymPhi}\ {\isasymequiv}\ {\isacharparenleft}{\isasymlambda}w{\isachardot}\ {\isasymexists}x{\isachardot}\ {\isasymPhi}\ x\ w{\isacharparenright}{\isachardoublequoteclose}\isanewline
\ \ \isacommand{abbreviation}\isamarkupfalse%
\ mbox\ {\isacharcolon}{\isacharcolon}\ {\isachardoublequoteopen}{\isasymsigma}\ {\isasymRightarrow}\ {\isasymsigma}{\isachardoublequoteclose}\ {\isacharparenleft}{\isachardoublequoteopen}{\isasymbox}{\isachardoublequoteclose}{\isacharparenright}\ \isakeyword{where}\ {\isachardoublequoteopen}{\isasymbox}\ {\isasymphi}\ {\isasymequiv}\ {\isacharparenleft}{\isasymlambda}w{\isachardot}\ {\isasymforall}v{\isachardot}\ \ w\ r\ v\ {\isasymlongrightarrow}\ {\isasymphi}\ v{\isacharparenright}{\isachardoublequoteclose}\isanewline
\ \ \isacommand{abbreviation}\isamarkupfalse%
\ mdia\ {\isacharcolon}{\isacharcolon}\ {\isachardoublequoteopen}{\isasymsigma}\ {\isasymRightarrow}\ {\isasymsigma}{\isachardoublequoteclose}\ {\isacharparenleft}{\isachardoublequoteopen}{\isasymdiamond}{\isachardoublequoteclose}{\isacharparenright}\ \isakeyword{where}\ {\isachardoublequoteopen}{\isasymdiamond}\ {\isasymphi}\ {\isasymequiv}\ {\isacharparenleft}{\isasymlambda}w{\isachardot}\ {\isasymexists}v{\isachardot}\ w\ r\ v\ {\isasymand}\ {\isasymphi}\ v{\isacharparenright}{\isachardoublequoteclose}%
\begin{isamarkuptext}%
For grounding lifted formulas, the meta-predicate \isa{valid} is introduced.%
\end{isamarkuptext}%
\isamarkuptrue%
\ \ \isacommand{abbreviation}\isamarkupfalse%
\ valid\ {\isacharcolon}{\isacharcolon}\ {\isachardoublequoteopen}{\isasymsigma}\ {\isasymRightarrow}\ bool{\isachardoublequoteclose}\ {\isacharparenleft}{\isachardoublequoteopen}{\isacharbrackleft}{\isacharunderscore}{\isacharbrackright}{\isachardoublequoteclose}{\isacharparenright}\ \isakeyword{where}\ {\isachardoublequoteopen}{\isacharbrackleft}p{\isacharbrackright}\ {\isasymequiv}\ {\isasymforall}w{\isachardot}\ p\ w{\isachardoublequoteclose}%
\isamarkupsection{G\"odel's Ontological Argument%
}
\isamarkuptrue%
%
\begin{isamarkuptext}%
Constant symbol \isa{P} (G\"odel's `Positive') is declared.%
\end{isamarkuptext}%
\isamarkuptrue%
\ \ \isacommand{consts}\isamarkupfalse%
\ P\ {\isacharcolon}{\isacharcolon}\ {\isachardoublequoteopen}{\isacharparenleft}{\isasymmu}\ {\isasymRightarrow}\ {\isasymsigma}{\isacharparenright}\ {\isasymRightarrow}\ {\isasymsigma}{\isachardoublequoteclose}%
\begin{isamarkuptext}%
The meaning of \isa{P} is restricted by axioms \isa{A{\isadigit{1}}{\isacharparenleft}a{\isacharslash}b{\isacharparenright}}: $\all \varphi 
[P(\neg \varphi) \biimp \neg P(\varphi)]$ (Either a property or its negation is positive, but not both.) 
and \isa{A{\isadigit{2}}}: $\all \varphi \all \psi [(P(\varphi) \wedge \nec \all x [\varphi(x) \imp \psi(x)]) 
\imp P(\psi)]$ (A property necessarily implied by a positive property is positive).%
\end{isamarkuptext}%
\isamarkuptrue%
\ \ \isacommand{axiomatization}\isamarkupfalse%
\ \isakeyword{where}\isanewline
\ \ \ \ A{\isadigit{1}}a{\isacharcolon}\ {\isachardoublequoteopen}{\isacharbrackleft}{\isasymPi}\ {\isacharparenleft}{\isasymlambda}{\isasymphi}{\isachardot}\ P\ {\isacharparenleft}{\isasymlambda}x{\isachardot}\ m{\isasymnot}\ {\isacharparenleft}{\isasymphi}\ x{\isacharparenright}{\isacharparenright}\ m{\isasymRightarrow}\ m{\isasymnot}\ {\isacharparenleft}P\ {\isasymphi}{\isacharparenright}{\isacharparenright}{\isacharbrackright}{\isachardoublequoteclose}\ \isakeyword{and}\isanewline
\ \ \ \ A{\isadigit{1}}b{\isacharcolon}\ {\isachardoublequoteopen}{\isacharbrackleft}{\isasymPi}\ {\isacharparenleft}{\isasymlambda}{\isasymphi}{\isachardot}\ m{\isasymnot}\ {\isacharparenleft}P\ {\isasymphi}{\isacharparenright}\ m{\isasymRightarrow}\ P\ {\isacharparenleft}{\isasymlambda}x{\isachardot}\ m{\isasymnot}\ {\isacharparenleft}{\isasymphi}\ x{\isacharparenright}{\isacharparenright}{\isacharparenright}{\isacharbrackright}{\isachardoublequoteclose}\ \isakeyword{and}\isanewline
\ \ \ \ A{\isadigit{2}}{\isacharcolon}\ \ {\isachardoublequoteopen}{\isacharbrackleft}{\isasymPi}\ {\isacharparenleft}{\isasymlambda}{\isasymphi}{\isachardot}\ {\isasymPi}\ {\isacharparenleft}{\isasymlambda}{\isasympsi}{\isachardot}\ {\isacharparenleft}P\ {\isasymphi}\ m{\isasymand}\ {\isasymbox}\ {\isacharparenleft}{\isasymforall}\ {\isacharparenleft}{\isasymlambda}x{\isachardot}\ {\isasymphi}\ x\ m{\isasymRightarrow}\ {\isasympsi}\ x{\isacharparenright}{\isacharparenright}{\isacharparenright}\ m{\isasymRightarrow}\ P\ {\isasympsi}{\isacharparenright}{\isacharparenright}{\isacharbrackright}{\isachardoublequoteclose}%
\begin{isamarkuptext}%
We prove theorem T1: $\all \varphi [P(\varphi) \imp \pos \ex x \varphi(x)]$ (Positive 
properties are possibly exemplified). T1 is proved directly by Sledgehammer with command \isa{sledgehammer\ {\isacharbrackleft}provers\ {\isacharequal}\ remote{\isacharunderscore}leo{\isadigit{2}}{\isacharbrackright}}. 
Sledgehammer suggests to call Metis with axioms A1a and A2. 
Metis sucesfully generates a proof object 
that is verified in Isabelle/HOL's kernel.%
\end{isamarkuptext}%
\isamarkuptrue%
\ \ \isacommand{theorem}\isamarkupfalse%
\ T{\isadigit{1}}{\isacharcolon}\ {\isachardoublequoteopen}{\isacharbrackleft}{\isasymPi}\ {\isacharparenleft}{\isasymlambda}{\isasymphi}{\isachardot}\ P\ {\isasymphi}\ m{\isasymRightarrow}\ {\isasymdiamond}\ {\isacharparenleft}{\isasymexists}\ {\isasymphi}{\isacharparenright}{\isacharparenright}{\isacharbrackright}{\isachardoublequoteclose}\ \ \isanewline
\ \ \isacommand{sledgehammer}\isamarkupfalse%
\ {\isacharbrackleft}provers\ {\isacharequal}\ remote{\isacharunderscore}leo{\isadigit{2}}{\isacharbrackright}\ \isanewline
%
\isadelimproof
\ \ %
\endisadelimproof
%
\isatagproof
\isacommand{by}\isamarkupfalse%
\ {\isacharparenleft}metis\ A{\isadigit{1}}a\ A{\isadigit{2}}{\isacharparenright}%
\endisatagproof
{\isafoldproof}%
%
\isadelimproof
%
\endisadelimproof
%
\begin{isamarkuptext}%
Next, the symbol \isa{G} for `God-like'  is introduced and defined 
as $G(x) \biimp \forall \varphi [P(\phi) \to \varphi(x)]$ \\ (A God-like being possesses 
all positive properties).%
\end{isamarkuptext}%
\isamarkuptrue%
\ \ \isacommand{definition}\isamarkupfalse%
\ G\ {\isacharcolon}{\isacharcolon}\ {\isachardoublequoteopen}{\isasymmu}\ {\isasymRightarrow}\ {\isasymsigma}{\isachardoublequoteclose}\ \isakeyword{where}\ {\isachardoublequoteopen}G\ {\isacharequal}\ {\isacharparenleft}{\isasymlambda}x{\isachardot}\ {\isasymPi}\ {\isacharparenleft}{\isasymlambda}{\isasymphi}{\isachardot}\ P\ {\isasymphi}\ m{\isasymRightarrow}\ {\isasymphi}\ x{\isacharparenright}{\isacharparenright}{\isachardoublequoteclose}%
\begin{isamarkuptext}%
Axiom \isa{A{\isadigit{3}}} is added: $P(G)$ (The property of being God-like is positive).
Sledgehammer and Metis then prove corollary \isa{C}: $\pos \ex x G(x)$ 
(Possibly, God exists).%
\end{isamarkuptext}%
\isamarkuptrue%
\ \ \isacommand{axiomatization}\isamarkupfalse%
\ \isakeyword{where}\ A{\isadigit{3}}{\isacharcolon}\ \ {\isachardoublequoteopen}{\isacharbrackleft}P\ G{\isacharbrackright}{\isachardoublequoteclose}\ \isanewline
\isanewline
\ \ \isacommand{corollary}\isamarkupfalse%
\ C{\isacharcolon}\ {\isachardoublequoteopen}{\isacharbrackleft}{\isasymdiamond}\ {\isacharparenleft}{\isasymexists}\ G{\isacharparenright}{\isacharbrackright}{\isachardoublequoteclose}\ \isanewline
\ \ \isacommand{sledgehammer}\isamarkupfalse%
\ {\isacharbrackleft}provers\ {\isacharequal}\ remote{\isacharunderscore}leo{\isadigit{2}}{\isacharbrackright}%
\isadelimproof
\ %
\endisadelimproof
%
\isatagproof
\isacommand{by}\isamarkupfalse%
\ {\isacharparenleft}metis\ A{\isadigit{3}}\ T{\isadigit{1}}{\isacharparenright}%
\endisatagproof
{\isafoldproof}%
%
\isadelimproof
%
\endisadelimproof
%
\begin{isamarkuptext}%
Axiom \isa{A{\isadigit{4}}} is added: $\all \phi [P(\phi) \to \Box \; P(\phi)]$ 
(Positive properties are necessarily positive).%
\end{isamarkuptext}%
\isamarkuptrue%
\ \ \isacommand{axiomatization}\isamarkupfalse%
\ \isakeyword{where}\ A{\isadigit{4}}{\isacharcolon}\ \ {\isachardoublequoteopen}{\isacharbrackleft}{\isasymPi}\ {\isacharparenleft}{\isasymlambda}{\isasymphi}{\isachardot}\ P\ {\isasymphi}\ m{\isasymRightarrow}\ {\isasymbox}\ {\isacharparenleft}P\ {\isasymphi}{\isacharparenright}{\isacharparenright}{\isacharbrackright}{\isachardoublequoteclose}%
\begin{isamarkuptext}%
Symbol \isa{ess} for `Essence' is introduced and defined as 
$\ess{\varphi}{x} \biimp \varphi(x) \wedge \all \psi (\psi(x) \imp \nec \all y (\varphi(y) 
\imp \psi(y)))$ (An \emph{essence} of an individual is a property possessed by it 
and necessarily implying any of its properties).%
\end{isamarkuptext}%
\isamarkuptrue%
\ \ \isacommand{definition}\isamarkupfalse%
\ ess\ {\isacharcolon}{\isacharcolon}\ {\isachardoublequoteopen}{\isacharparenleft}{\isasymmu}\ {\isasymRightarrow}\ {\isasymsigma}{\isacharparenright}\ {\isasymRightarrow}\ {\isasymmu}\ {\isasymRightarrow}\ {\isasymsigma}{\isachardoublequoteclose}\ {\isacharparenleft}\isakeyword{infixr}\ {\isachardoublequoteopen}ess{\isachardoublequoteclose}\ {\isadigit{8}}{\isadigit{5}}{\isacharparenright}\ \isakeyword{where}\isanewline
\ \ \ \ {\isachardoublequoteopen}{\isasymphi}\ ess\ x\ {\isacharequal}\ {\isasymphi}\ x\ m{\isasymand}\ {\isasymPi}\ {\isacharparenleft}{\isasymlambda}{\isasympsi}{\isachardot}\ {\isasympsi}\ x\ m{\isasymRightarrow}\ {\isasymbox}\ {\isacharparenleft}{\isasymforall}\ {\isacharparenleft}{\isasymlambda}y{\isachardot}\ {\isasymphi}\ y\ m{\isasymRightarrow}\ {\isasympsi}\ y{\isacharparenright}{\isacharparenright}{\isacharparenright}{\isachardoublequoteclose}%
\begin{isamarkuptext}%
Next, Sledgehammer and Metis prove theorem \isa{T{\isadigit{2}}}: $\all x [G(x) \imp \ess{G}{x}]$ 
(Being God-like is an essence of any God-like being).%
\end{isamarkuptext}%
\isamarkuptrue%
\ \ \isacommand{theorem}\isamarkupfalse%
\ T{\isadigit{2}}{\isacharcolon}\ {\isachardoublequoteopen}{\isacharbrackleft}{\isasymforall}\ {\isacharparenleft}{\isasymlambda}x{\isachardot}\ G\ x\ m{\isasymRightarrow}\ G\ ess\ x{\isacharparenright}{\isacharbrackright}{\isachardoublequoteclose}\isanewline
\ \ \isacommand{sledgehammer}\isamarkupfalse%
\ {\isacharbrackleft}provers\ {\isacharequal}\ remote{\isacharunderscore}leo{\isadigit{2}}{\isacharbrackright}%
\isadelimproof
\ %
\endisadelimproof
%
\isatagproof
\isacommand{by}\isamarkupfalse%
\ {\isacharparenleft}metis\ A{\isadigit{1}}b\ A{\isadigit{4}}\ G{\isacharunderscore}def\ ess{\isacharunderscore}def{\isacharparenright}%
\endisatagproof
{\isafoldproof}%
%
\isadelimproof
%
\endisadelimproof
%
\begin{isamarkuptext}%
Symbol \isa{NE}, for `Necessary Existence', is introduced and
defined as $\NE(x) \biimp \all \varphi [\ess{\varphi}{x} \imp \nec \ex y \varphi(y)]$ (Necessary 
existence of an individual is the necessary exemplification of all its essences).%
\end{isamarkuptext}%
\isamarkuptrue%
\ \ \isacommand{definition}\isamarkupfalse%
\ NE\ {\isacharcolon}{\isacharcolon}\ {\isachardoublequoteopen}{\isasymmu}\ {\isasymRightarrow}\ {\isasymsigma}{\isachardoublequoteclose}\ \isakeyword{where}\ {\isachardoublequoteopen}NE\ {\isacharequal}\ {\isacharparenleft}{\isasymlambda}x{\isachardot}\ {\isasymPi}\ {\isacharparenleft}{\isasymlambda}{\isasymphi}{\isachardot}\ {\isasymphi}\ ess\ x\ m{\isasymRightarrow}\ {\isasymbox}\ {\isacharparenleft}{\isasymexists}\ {\isasymphi}{\isacharparenright}{\isacharparenright}{\isacharparenright}{\isachardoublequoteclose}%
\begin{isamarkuptext}%
Moreover, axiom \isa{A{\isadigit{5}}} is added: $P(\NE)$ (Necessary existence is a positive 
property).%
\end{isamarkuptext}%
\isamarkuptrue%
\ \ \isacommand{axiomatization}\isamarkupfalse%
\ \isakeyword{where}\ A{\isadigit{5}}{\isacharcolon}\ \ {\isachardoublequoteopen}{\isacharbrackleft}P\ NE{\isacharbrackright}{\isachardoublequoteclose}%
\begin{isamarkuptext}%
Finally, Sledgehammer and Metis prove the main theorem \isa{T{\isadigit{3}}}: $\nec \ex x G(x)$ 
(Necessarily, God exists).%
\end{isamarkuptext}%
\isamarkuptrue%
\ \ \isacommand{theorem}\isamarkupfalse%
\ T{\isadigit{3}}{\isacharcolon}\ {\isachardoublequoteopen}{\isacharbrackleft}{\isasymbox}\ {\isacharparenleft}{\isasymexists}\ G{\isacharparenright}{\isacharbrackright}{\isachardoublequoteclose}\ \isanewline
\ \ \isacommand{sledgehammer}\isamarkupfalse%
\ {\isacharbrackleft}provers\ {\isacharequal}\ remote{\isacharunderscore}leo{\isadigit{2}}{\isacharbrackright}%
\isadelimproof
\ %
\endisadelimproof
%
\isatagproof
\isacommand{by}\isamarkupfalse%
\ {\isacharparenleft}metis\ A{\isadigit{5}}\ C\ T{\isadigit{2}}\ sym\ G{\isacharunderscore}def\ NE{\isacharunderscore}def{\isacharparenright}%
\endisatagproof
{\isafoldproof}%
%
\isadelimproof
%
\endisadelimproof
\isanewline
\isanewline
\ \ \isacommand{corollary}\isamarkupfalse%
\ C{\isadigit{2}}{\isacharcolon}\ {\isachardoublequoteopen}{\isacharbrackleft}{\isasymexists}\ G{\isacharbrackright}{\isachardoublequoteclose}\ \isanewline
\ \ \isacommand{sledgehammer}\isamarkupfalse%
\ {\isacharbrackleft}provers\ {\isacharequal}\ remote{\isacharunderscore}leo{\isadigit{2}}{\isacharbrackright}{\isacharparenleft}T{\isadigit{1}}\ T{\isadigit{3}}\ G{\isacharunderscore}def\ sym{\isacharparenright}%
\isadelimproof
\ %
\endisadelimproof
%
\isatagproof
\isacommand{by}\isamarkupfalse%
\ {\isacharparenleft}metis\ T{\isadigit{1}}\ T{\isadigit{3}}\ G{\isacharunderscore}def\ sym{\isacharparenright}%
\endisatagproof
{\isafoldproof}%
%
\isadelimproof
%
\endisadelimproof
%
\begin{isamarkuptext}%
The consistency of the entire theory is checked with Nitpick.%
\end{isamarkuptext}%
\isamarkuptrue%
\ \ \isacommand{lemma}\isamarkupfalse%
\ True\ \isacommand{nitpick}\isamarkupfalse%
\ {\isacharbrackleft}satisfy{\isacharcomma}\ user{\isacharunderscore}axioms{\isacharcomma}\ expect\ {\isacharequal}\ genuine{\isacharbrackright}%
\isadelimproof
\ %
\endisadelimproof
%
\isatagproof
\isacommand{oops}\isamarkupfalse%
%
\endisatagproof
{\isafoldproof}%
%
\isadelimproof
%
\endisadelimproof
%
\begin{isamarkuptext}%
It has been critisized that G\"odel's ontological argument implies what is called the 
modal collapse. The prover Satallax \cite{Satallax} can indeed show this, but verification with 
Metis still fails.%
\end{isamarkuptext}%
\isamarkuptrue%
\ \ \isacommand{lemma}\isamarkupfalse%
\ MC{\isacharcolon}\ {\isachardoublequoteopen}{\isacharbrackleft}p\ m{\isasymRightarrow}\ {\isacharparenleft}{\isasymbox}\ p{\isacharparenright}{\isacharbrackright}{\isachardoublequoteclose}\isanewline
%
\isadelimproof
\ \ %
\endisadelimproof
%
\isatagproof
\isacommand{using}\isamarkupfalse%
\ T{\isadigit{2}}\ T{\isadigit{3}}\ ess{\isacharunderscore}def\ sym%
\endisatagproof
{\isafoldproof}%
%
\isadelimproof
%
\endisadelimproof
\ \isacommand{sledgehammer}\isamarkupfalse%
\ {\isacharbrackleft}provers\ {\isacharequal}\ remote{\isacharunderscore}satallax{\isacharbrackright}%
\isadelimproof
\ %
\endisadelimproof
%
\isatagproof
\isacommand{oops}\isamarkupfalse%
\isanewline
%
\endisatagproof
{\isafoldproof}%
%
\isadelimproof
%
\endisadelimproof
%
\isadelimtheory
%
\endisadelimtheory
%
\isatagtheory
%
\endisatagtheory
{\isafoldtheory}%
%
\isadelimtheory
%
\endisadelimtheory
\ \end{isabellebody}%
%%% Local Variables:
%%% mode: latex
%%% TeX-master: "root"
%%% End:
